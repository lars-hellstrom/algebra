% 
% \iffalse driver
%<*driver>
\documentclass{mtmtcl}
\begin{document}
\DocInput{export.dtx}
\end{document}
%</driver>
% \fi
% 
% 
% 
% \section{Data export}
% 
% 
% The basic format for exporting structure elements is as a 
% data-tree, so that further processing is easy in \Tcl. An important 
% technique for further processing is data-is-code, meaning the 
% \word{tag}s in the tree are defined as \Tcl\ commands in some 
% namespace (each performing a particular operation on \emph{its} 
% kind of node) and the data-tree is processed through |eval|uation 
% of it in that particular namespace. Among the operations which can 
% be implemented this way is conversion to XML, although if 
% readability of the result is of interest then it is probably better 
% to let the \textsf{tdom} package do this conversion.
% 
% \changes{0}{2008/10/01}{Prioritising OpenMath over MathML. 
%   Introducing the idea of \emph{temporary} conversions from 
%   OpenMath node types to specialised node types. (LH)}
% 
% The node types (\word{tag}s) used should primarily be those defined 
% in the OpenMath standard, but there are exceptions (e.g.~|integer|) 
% in which a more \Tcl-native encoding is preferable. Specific 
% processing tasks may also benefit if specialised node types are 
% used instead of the generic |OMA|~(OpenMath function application), 
% since data-is-code can only do dispatching based on the \word{tag} 
% of the node. This division is however overcome by providing 
% operations that transform data-trees between the generic OpenMath 
% format and more specialised formats; neither form is irreversible, 
% and the OpenMath one should be considered normal.
% 
% For processing to the end of rendering data, it is often 
% interesting to incorporate presentation-oriented information in the 
% data-tree. Here it is preferable to use MathML-elements to encode 
% such information, when there is one which is suitable for the 
% problem at hand (e.g.~|mfenced| for bracketing subexpressions). 
% MathML does however seem less useful as a \emph{standard} for the 
% data.
% 
% Conversions to presentation formats should \emph{not} rely on having 
% a big built-in table of OpenMath symbols and how they are 
% rendered\Dash the converter should instead expect to find all the 
% necessary information provided in the data-tree to convert 
% (e.g.~as attributions attached to the \OMSref{altenc}{LaTeX_encoding} 
% symbol). Tools should be provided for scanning trees for symbols 
% without such information and for adding such information to 
% selected symbols, but it may in many cases be better to have it 
% supplied already by the original |export| method.
% 
% \stringtypeheading{node type}{Data-tree node types:}
% \stringtypeheading{attribute}{Data-tree attributes:}
% 
% 
% \subsection{XML generation}
% 
% Since the data-tree format was taken over from \textsf{tdom}, the 
% basic approach would be to use that for generating XML, but there 
% is also a need for more lightweight generators in cases where 
% readability is not an issue but employing extensions is.
% 
% \begin{tcl}
%<*xmlgen>
namespace eval mtmtcl::support {}
% \end{tcl}
% \setnamespace{mtmtcl::support}
% 
% \begin{proc}{xml_from_tree}
%   This procedure is the toplevel generator of XML code from a 
%   data-tree. The call syntax is
%   \begin{displaysyntax}
%     |mtmtcl::support::xml_from_tree| \word{tree} 
%     \begin{regblock}[\regstar] \word{option} \word{value} 
%     \end{regblock}
%   \end{displaysyntax}
%   where \word{tree} of course is the tree to convert and the return 
%   value is the converted tree as XML code. The supported options 
%   are:
%   \begin{ttdescription}
%     \item[-nodesep]
%       Basic separator inserted between nodes of the tree; defaults 
%       to a newline. Setting it to empty reduces the output size, 
%       but may well have the effect of making the entire result a 
%       single very long line.
%     \item[-indentstep]
%       An indentation step, defaults to being empty. Some repetition 
%       (corresponding to nesting depth) of this string follows the 
%       |-nodesep|.
%     \item[-charmap]
%       A dictionary of character to entity mappings used with 
%       |string map| to quote problematic characters in XML character 
%       data. Defaults to being empty. The five syntax characters 
%       `|<|', `|>|', `|&|', `|'|', and `|"|' are automatically 
%       inserted into this dictionary if they are not already there.
%   \end{ttdescription}
%   \begin{tcl}
proc mtmtcl::support::xml_from_tree {tree args} {
   array set Opt {
      -nodesep    \n
      -indentstep ""
      -charmap    {}
   }
   array set Opt $args
   if {![dict size $Opt(-charmap)]} then {
      set Opt(-charmap) {< &lt; > &gt; & &amp; ' &apos; \" &quot;}
   } else {
      foreach char {< > & ' \"} entity {&lt; &gt; &amp; &apos; &quot;} {
         if {![dict exists $O(-charmap) $char]} then {
            dict set $O(-charmap) $char $entity
         }
      }
   }
   xml_from_treenode $tree $Opt(-nodesep) $Opt(-indentstep)\
     $Opt(-charmap)
}
%   \end{tcl}
% \end{proc}
% 
% \begin{proc}{xml_from_treenode}
%   This is a more internal procedure that generates the XML fragment 
%   for a particular branch of the tree. The call syntax is
%   \begin{displaysyntax}
%     |xml_from_treenode| \word{branch} \word{node-sep} 
%     \word{indent-step} \word{char-map}
%   \end{displaysyntax}
%   
%   Not surprisingly, it relies on recursion to convert the tree as a 
%   whole.
%   \begin{tcl}
proc mtmtcl::support::xml_from_treenode {tree sep indent map} {
   switch -- [lindex $tree 0] "#text" - "#cdata" {
      return [string map $map [lindex $tree 1]]
   } "#comment" {
      return "<!--[lindex $tree 1]-->"
   } "#pi" {
      return "<?[lindex $tree 1] [lindex $tree 2]?>"
   }
   set res "<[lindex $tree 0]"
   dict for {key val} [lindex $tree 1] {
      append res { } $key =\" [string map $map $val] \"
   }
   if {[llength [lindex $tree 2]]} then {
      set subsep $sep$indent
      append res ">"
      foreach child [lindex $tree 2] {
         append res $subsep\
           [xml_from_treenode $child $subsep $indent $map]
      }
      append res $sep "</[lindex $tree 0]>"
   } else {
      append res "/>"
   }
   return $res
}
%   \end{tcl}
% \end{proc}
% 
% 
% \begin{tcl}
%</xmlgen>
% \end{tcl}
% 
% 
% \subsection{Near-OpenMath}
% 
% The basic format of data-trees encoding \mtl\ structure elements 
% is mostly OpenMath, i.e., there are node types:
% \begin{description}
%   \item[{\describestring*+[node type]{OMA}}]
%     for function applications, where the first child is the 
%     function being applied and subsequent children are its 
%     arguments;
%   \item[{\describestring*+[node type]{OMS}}]
%     for general symbols, precisely identified through 
%     \describestring+[attribute]{name} and 
%     \describestring+[attribute]{cd} (Content Dictionary) attributes;
%   \item[{\describestring*+[node type]{OMV}}]
%     for variables, distinguished through the
%     \describestring+[attribute]{name} attribute;
%   \item[{\describestring*+[node type]{OMF}}]
%     for floating-point numbers (IEEE |double|s), with the value 
%     given by a \describestring+[attribute]{dec} or 
%     \describestring+[attribute]{hex} attribute;
% \end{description}
% and others (|OMBIND|, |OMSTR|, etc.), but one that should normally 
% not be used is \describestring+[node type]{OMI} for integers, since 
% its general structure
% \begin{displaysyntax}
%   OMI |{} {{#text| \word{string}|}}|
% \end{displaysyntax}
% is unnecessarily awkward to interpret programmatically in 
% \Tcl~(partly because the forms allowed in the \word{string} are not 
% the same as \Tcl\ understands for integers). Instead the 
% \describestring+[node type]{integer} node type should be used, 
% which has the format
% \begin{displaysyntax}
%   integer |{|value \word{integer}|} {}|
% \end{displaysyntax}
% i.e., the value is given as a \describestring+[attribute]{value} 
% attribute, whose allowed forms are precisely those that \Tcl\ 
% understands.
% 
% The |OMS|, |OMV|, |OMF|, and |OMR| node types cannot have children 
% in OpenMath, so it is convenient to sometimes\Ldash particularly 
% during rendering\Dash use the first child of these nodes as 
% ``contents'' of the node even though they are conceptually atomic.
% 
% \setnamespace{mtmtcl::openmath::stricten::helper}
% \begin{ensemble}{stricten}
%   \setnamespace{mtmtcl::openmath::stricten}
%   The |stricten| command transforms a near-OpenMath data-tree to a 
%   strictly OpenMath data-tree, i.e., nonstandard node types are 
%   converted to their standard counterparts and nonstandard contents 
%   of nodes are dropped. The external call syntax is
%   \begin{displaysyntax}
%     mtmtcl::openmath::stricten \meta{tree} \word{clean}
%   \end{displaysyntax}
%   where the \word{clean} is a boolean which controls whether 
%   nonstandard attributes should be removed as well (true is yes). 
%   The return value is the strictened data-tree.
%   
%   Instead of dropping nonstandard attributes, one could encode 
%   them using |OMATTR| nodes, but this can easily be implemented as 
%   a separate operation. This operation is primarily about getting 
%   rid of nonstandard nodes (except as children of |OMFOREIGN|).
%   
%   The organisation of this namespace and ensemble is as follows:
%   \begin{itemize}
%     \item
%       The |mtmtcl::openmath::stricten| namespace contains 
%       commands for handling specific node types. All commands in 
%       this namespace are exported and show up in the ensemble.
%     \item
%       The |helper| child of the ensemble namespace contains helpers 
%       for the various subcommands. One of these is the actual 
%       ensemble command, which is needed for processing children of 
%       a node.
%   \end{itemize}
%   
%   \begin{tcl}
%<*pkg>
namespace eval mtmtcl::openmath::stricten {
    namespace export *
    namespace eval helper {
       interp alias {} [namespace parent]\
         {} [namespace current]::stricten
    }
    namespace ensemble create -prefix 0\
      -command [namespace current]::helper::stricten
}
%   \end{tcl}
%   There is no unknown handler, since it is an error to apply this 
%   operation to an unknown node type.
%   
%   \begin{ensproc}{#text}
%     Text is returned unchanged.
%     \begin{tcl}
proc mtmtcl::openmath::stricten::#text {str clean} {list #text $str}
%     \end{tcl}
%   \end{ensproc}
%   
%   \begin{ensproc}{integer}
%     Integers are converted to |OMI| nodes.
%     \begin{tcl}
proc mtmtcl::openmath::stricten::integer {attr children clean} {
   list OMI {} [list [
      list \#text [format %lld [dict get $attr value]]
   ]]
}
%     \end{tcl}
%   \end{ensproc}
% \end{ensemble}
% 
% For the next couple of definitions, it's actually easiest to be 
% within the |mtmtcl::openmath::stricten::helper| namespace, since 
% that is where the helper procedures will be created, and one 
% anyway has to pretty much fully qualify the alias names.
% \begin{tcl}
namespace eval mtmtcl::openmath::stricten::helper {
% \end{tcl}
%   
% \begin{proc}{truncate}
%   This procedure can do what is needed for strictening 
%   several OpenMath node types; it is meant to be useful as an 
%   alias target. The call syntax is
%   \begin{displaysyntax}
%     |helper::truncate| \word{type} \word{attribute-list} 
%     \word{attributes} \word{children} \word{clean}
%   \end{displaysyntax}
%   and the return value is the cleaned-up node of type 
%   \word{type}, without children. 
%   \begin{tcl}
   proc truncate {type attrL attr children clean} {
      if {!$clean} then {return [list $type $attr {}]}
      set D [dict create]
      foreach key $attrL {
         if {[dict exists $attr $key]} then {
            dict set D $key [dict get $attr $key]
         }
      }
      return [list $type $D {}]
   }
%   \end{tcl}
% \end{proc}
%   
% \begin{ensemble}{stricten}
%   \setnamespace{mtmtcl::openmath::stricten}
%   \begin{ensproc}{OMS}
%   \begin{ensproc}{OMV}
%   \begin{ensproc}{OMF}
%   \begin{ensproc}{OMR}
%     As stated above, |OMS|, |OMV|, |OMF|, and |OMR| nodes should 
%     not have children in strict OpenMath, so they're the ones that 
%     |helper::truncate| are for.
%     
%     Symbols may carry |name|, |cd|, |cdbase|, and |id| attributes.
%     \begin{tcl}
   interp alias {} [namespace parent]::OMS {}\
     [namespace which truncate] OMS {name cd cdbase id}
%     \end{tcl}
%     Floats can carry |id|, |dec|, and |hex| attributes.
%     \begin{tcl}
   interp alias {} [namespace parent]::OMF {}\
     [namespace which truncate] OMF {dec hex id}
%     \end{tcl}
%     Variables may carry |name| and |id| attributes.
%     \begin{tcl}
   interp alias {} [namespace parent]::OMV {}\
     [namespace which truncate] OMV {name id}
%     \end{tcl}
%     References can carry |id| and |href| attributes.
%     \begin{tcl}
   interp alias {} [namespace parent]::OMR {}\
     [namespace which truncate] OMR {id href}
%     \end{tcl}
%   \end{ensproc}\end{ensproc}\end{ensproc}\end{ensproc}
% \end{ensemble}
% 
% \begin{proc}{textchild}
%   Another class of node types, from the strictening point of 
%   view, are those which can have |#text| children and an |id| 
%   attribute, so this is what this procedure checks for.
%   The call syntax is
%   \begin{displaysyntax}
%     |helper::textchild| \word{type} \word{attributes} 
%     \word{children} \word{clean}
%   \end{displaysyntax}
%   and the return value is the cleaned-up node of type 
%   \word{type}.
%   \begin{tcl}
   proc textchild {type attr children clean} {
      foreach child $children {
         if {[lindex $child 0] ne "#text"} then {
            error "Type \"[lindex $child 0]\" forbidden as child of\
              \"$type\""
         }
      }
      if {$clean} then {
         if {[dict exists $attr id]} then {
            set attr [dict filter $attr keys id]
         } else {
            set attr {}
         }
      }
      return [list $type $attr $children]
   }
%   \end{tcl}
% \end{proc}
%   
% \begin{ensemble}{stricten}
%   \setnamespace{mtmtcl::openmath::stricten}
%   \begin{ensproc}{OMI}
%   \begin{ensproc}{OMB}
%   \begin{ensproc}{OMSTR}
%     Integers, bytearrays, and strings are node types that can be 
%     handled by |textchild|.
%     \begin{tcl}
   interp alias {} [namespace parent]::OMI {}\
     [namespace which textchild] OMI
   interp alias {} [namespace parent]::OMB {}\
     [namespace which textchild] OMB
   interp alias {} [namespace parent]::OMSTR {}\
     [namespace which textchild] OMSTR
%     \end{tcl}
%   \end{ensproc}\end{ensproc}\end{ensproc}
% \end{ensemble}
% 
% \begin{proc}{recurse}
%   The final class of node types are those where we actually need 
%   to recurse over the children (and suspect they can be pretty 
%   much anything). The helper for this has the call syntax
%   \begin{displaysyntax}
%     |helper::recurse| \word{type} \word{attribute-list} 
%     \word{attributes} \word{children} \word{clean}
%   \end{displaysyntax}
%   and the return value is the cleaned-up node of type 
%   \word{type}.
%   \begin{tcl}
   proc recurse {type attrL attr children clean} {
      set L {}
      foreach child $children {
         lappend L [stricten {*}$child $clean]
      }
      if {!$clean} then {return [list $type $attr $L]}
      set D {}
      foreach key $attrL {
         if {[dict exists $attr $key]} then {
            dict set D $key [dict get $attr $key]
         }
      }
      return [list $type $D $L]
   }
%   \end{tcl}
% \end{proc}
%   
% \begin{ensemble}{stricten}
%   \setnamespace{mtmtcl::openmath::stricten}
%   \begin{ensproc}{OMA}
%   \begin{ensproc}{OMBIND}
%   \begin{ensproc}{OMATTR}
%   \begin{ensproc}{OMATP}
%     Compound elements can carry |id| and |cdbase| attributes.
%     \begin{tcl}
   interp alias {} [namespace parent]::OMA {}\
     [namespace which recurse] OMA {cdbase id}
   interp alias {} [namespace parent]::OMBIND {}\
     [namespace which recurse] OMBIND {cdbase id}
   interp alias {} [namespace parent]::OMATTR {}\
     [namespace which recurse] OMATTR {cdbase id}
   interp alias {} [namespace parent]::OMATP {}\
     [namespace which recurse] OMATP {cdbase id}
%     \end{tcl}
%   \end{ensproc}\end{ensproc}\end{ensproc}\end{ensproc}
%   
%   \begin{ensproc}{OMBVAR}
%   \begin{ensproc}{OME}
%     Bound variables and errors only carry |id| attributes.
%     \begin{tcl}
   interp alias {} [namespace parent]::OME {}\
     [namespace which recurse] OME id
   interp alias {} [namespace parent]::OMBVAR {}\
     [namespace which recurse] OMBVAR id
%     \end{tcl}
%   \end{ensproc}\end{ensproc}
%   
%   OK, that's the end of the |mtmtcl::openmath::stricten::helper| 
%   material.
%   \begin{tcl}
}
%   \end{tcl}
%   
%   \begin{ensproc}{OMFOREIGN}
%     Foreign elements can carry |id| and |encoding| attributes. 
%     Since the children are pretty arbitrary, there's no point in 
%     recursing over them.
%     \begin{tcl}
proc mtmtcl::openmath::stricten::OMFOREIGN {attr children clean} {
   if {!$clean} then {
      return [list OMFOREIGN $attr $children]
   }
   set D [dict create]
   foreach key {id encoding} {
      if {[dict exists $attr $key]} then {
         dict set D $key [dict get $attr $key]
      }
   }
   return [list OMFOREIGN $D $children]
}
%     \end{tcl}
%   \end{ensproc}
% \end{ensemble}
% 
% \begin{ensemble}{OM}
%   While it is possible to construct (near-)OpenMath objects using 
%   nothing but |list|, it is more convenient to have a specialised 
%   constructor that takes care of the double nesting arising from 
%   having both the data-is-code list and the list-of-children list; 
%   forgetting one is a common typo, especially when there is exactly 
%   one child. The |OM| ensemble has subcommands for constructing the 
%   basic types of node, with names generally formed by dropping the 
%   |OM| prefix from the corresponding OpenMath tag. This ensemble is 
%   exported by the |mtmtcl::openmath| namespace.
%   \begin{tcl}
namespace eval mtmtcl::openmath {
   namespace eval OM {
      namespace export *
      namespace ensemble create
   }
   namespace export OM
}
%   \end{tcl}
%   
%   \begin{ensproc}{S}
%     The |OM S| command is thus for constructing |OMS| nodes. It has 
%     the syntax
%     \begin{displaysyntax}
%       OM S \word{cd} \word{name} \begin{regblock}[\regstar] 
%       \word{attribute} \word{value} \end{regblock}
%     \end{displaysyntax}
%     where the content dictionary \word{cd} and symbol name 
%     \word{name} are mandatory arguments. Additional attributes 
%     (e.g. |cdbase|) can be specified in additional key--value pairs 
%     of arguments.
%     \begin{tcl}
proc mtmtcl::openmath::OM::S {cd name args} {
   list OMS [dict create cd $cd name $name {*}$args] {}
}
%     \end{tcl}
%   \end{ensproc}
%   
%   \begin{ensproc}{mS}
%     The |OM mS| command is again for constructing |OMS| nodes, but 
%     this time to conform to the \APIref*{export}{2.0} interface 
%     handling of the |mtmtcl:path| attribute. It has the syntax
%     \begin{displaysyntax}
%       OM mS \word{cd} \word{name} \word{common attributes} 
%       \word{suffix} \begin{regblock}[\regstar] 
%       \word{attribute} \word{value} \end{regblock}
%     \end{displaysyntax}
%     where the content dictionary \word{cd} and symbol name 
%     \word{name} are as for |OM S|. The \word{common attributes} is 
%     an inherited dictionary of attributes which is getting 
%     overridden, and the \word{suffix} is a list of elements to 
%     append to the |mtmtcl:path|. Further attributes (e.g. |cdbase|) 
%     can be specified in additional key--value pairs of arguments, 
%     and will override those in the \word{common attributes}.
%     \begin{tcl}
proc mtmtcl::openmath::OM::mS {cd name attr suffix args} {
   dict lappend attr mtmtcl:path {*}$suffix
   list OMS [dict replace $attr cd $cd name $name {*}$args] {}
}
%     \end{tcl}
%   \end{ensproc}
%   
%   \begin{ensproc}{A}
%     The |OM A| command is for constructing |OMA| nodes. It has 
%     the syntax
%     \begin{displaysyntax}
%       OM A \word{child}\regplus
%     \end{displaysyntax}
%     where each \word{child} should be a near-OpenMath data-tree, 
%     the first of which will be the head (``function'') of the 
%     application.
%     \begin{tcl}
proc mtmtcl::openmath::OM::A {args} {
   list OMA {} $args
}
%     \end{tcl}
%   \end{ensproc}
%   
%   \begin{ensproc}{AS}
%     The |OM AS| command is for constructing |OMA| nodes whose first 
%     childs are |OMS| nodes. It has 
%     the syntax
%     \begin{displaysyntax}
%       OM AS \word{cd} \word{name} \word{child}\regstar
%     \end{displaysyntax}
%     where each \word{child} should be a near-OpenMath data-tree to 
%     be made an argument of the |OMS| node with the specified 
%     \word{cd} and \word{name}.
%     \begin{tcl}
proc mtmtcl::openmath::OM::AS {cd name args} {
   list OMA {} [linsert $args 0 [S $cd $name]]
}
%     \end{tcl}
%     This command does not cover the case where the |OMS| child has 
%     a |cdbase| attribute, but that could be covered with a separate 
%     |OM| subcommand.
%   \end{ensproc}
%   
%   \begin{ensproc}{I}
%     The |OM I| command is for constructing |integer| nodes (rather 
%     than |OMI| nodes). It has the syntax
%     \begin{displaysyntax}
%       OM I \word{value}
%     \end{displaysyntax}
%     where the \word{value} is the integer value to encode.
%     \begin{tcl}
proc mtmtcl::openmath::OM::I {value} {
   list integer [dict create value $value] {}
}
%     \end{tcl}
%   \end{ensproc}
%   
%   \begin{ensproc}{STR}
%     The |OM STR| command is for constructing |OMSTR| nodes. It has 
%     the syntax
%     \begin{displaysyntax}
%       OM STR \word{value}
%     \end{displaysyntax}
%     where the \word{value} is the string to encode.
%     \begin{tcl}
proc mtmtcl::openmath::OM::STR {value} {
   list OMSTR {} [list [list \#text $value]]
}
%     \end{tcl}
%   \end{ensproc}
%   
%   \begin{ensproc}{V}
%     The |OM V| command is for constructing |OMV| nodes. It has 
%     the syntax
%     \begin{displaysyntax}
%       OM STR \word{name}
%     \end{displaysyntax}
%     where the \word{name} is the name of the OM variable.
%     \begin{tcl}
proc mtmtcl::openmath::OM::V {name} {
   list OMV [dict create name $name] {}
}
%     \end{tcl}
%   \end{ensproc}
%   
%   \begin{ensproc}{US}
%     The |OM US| command constructs an |OMS| element from the 
%     URI for the symbol. The call syntax is
%     \begin{displaysyntax}
%       OM US \word{uri}
%     \end{displaysyntax}
%     where the \word{uri} is one of
%     \begin{displaysyntax}
%       \meta{cd}\#\meta{name}\par
%       \meta{cdbase}/\meta{cd}\#\meta{name}
%     \end{displaysyntax}
%     depending on whether the |OMS| element should include a 
%     |cdbase| attribute.
%     \begin{tcl}
proc mtmtcl::openmath::OM::US {uri} {
   if {[
      regexp -- {^(.*)/([^/#]*)#([^/#]*)$} $uri "" base cd name
   ]} then {
      list OMS [dict create cdbase $base cd $cd name $name] {}
   } elseif {[
      regexp -- {^([^/#]*)#([^/#]*)$} $uri "" cd name
   ]} then {
      list OMS [dict create cd $cd name $name] {}
   } else {
      error "Bad symbol URI: $uri"
   }
}
%     \end{tcl}
%   \end{ensproc}
%   
%   \begin{ensproc}{UATTR}
%     The |OM UATTR| command is for attaching attributions to a given 
%     object, while specifying the attribution symbols through URIs. 
%     The syntax is
%     \begin{displaysyntax}
%       OM UATTR \word{core} \begin{regblock}[\regstar] \word{URI} 
%       \word{object} \end{regblock}
%     \end{displaysyntax}
%     where the \word{core} is the object that will be put in an 
%     |OMATTR| node. Each \word{object} is the value of an OM 
%     attribution, and the corresponding \word{URI} is the URI of the 
%     annotation symbol.
%     \begin{tcl}
proc mtmtcl::openmath::OM::UATTR {core args} {
   set L {}
   foreach {uri obj} $args {lappend L [US $uri] $obj}
   list OMATTR {} [list [list OMATP {} $L] $core]
}
%     \end{tcl}
%   \end{ensproc}
% \end{ensemble}
% 
% 
% \subsection{The \texttt{export} interface}
% 
% The key route from \mtl\ to OpenMath is the |export| interface, 
% which specifies how structure elements are converted to 
% near-OpenMath data-trees.
% 
% \begin{APIspec}{export}{2.0}
%   This interface is applicable to principal structures. It provides 
%   a method |export| for converting the internal representation of 
%   an element to an external format. The syntax for this method is:
%   \begin{APIdescription}{structure}
%     \begin{APImethod}{export}
%       \word{element} \word{common attributes}
%     \end{APImethod}
%       where \word{element} is the element to export. The return 
%       value is a data-tree that encodes the \word{element} as 
%       near-OpenMath.
%       
%       The \word{common attributes} is a dictionary of common 
%       attributes for nodes in the tree. The |export| method is 
%       however allowed to override part or all of it, so it is 
%       prefectly acceptable to ignore the \word{common attributes} 
%       entirely for some sets of nodes. The typical use is that 
%       the \word{common attributes} are used for |OMA| nodes and 
%       get |cd| and |name| entries added in |OMS| nodes.
%       
%       For |export| methods that call other |export| methods 
%       recursively, there is however the condition that they must 
%       update the \describestring+[attribute]{mtmtcl:path} entry in 
%       the \word{common attributes} dictionary. The value of this 
%       attribute is a list of strings, typically a list of method 
%       names. The purpose of this list is to establish a system for 
%       addressing subordinate structures by mirroring the command 
%       prefix that would be used to access its methods. If a 
%       structure $S_1$ makes use of the |export| method of another 
%       structure $S_2$ and $S_2$ can be accessed using the $s$ 
%       method of $S_1$, then the |mtmtcl:path| used for the 
%       \texttt{[$S_2$ export $\dotsb$]} call should have an $s$ 
%       append to its value from the \texttt{[$S_1$ export $\dotsb$]} 
%       call.
%       
%       The |mtmtcl:path| may be extended with method names also when 
%       there is no recursive call to the |export| method. This can 
%       be done e.g.~to clarify which method corresponds to a 
%       particular operation symbol.
%   \end{APIdescription}
% \end{APIspec}
% 
% \textbf{Question:} Would it be appropriate to define a 
% \emph{content dictionary} for abstract operation symbols? In the 
% long run probably yes, and this would be an appropriate place to do 
% so.
% 
% \medskip
% 
% \begin{APIspec}{import}{1.0}
%   Where there is export, this is usually also import. 
%   \begin{APIdescription}{structure}
%     \begin{APImethod}{import}
%       \word{path} \word{data-tree} 
%     \end{APImethod}
%       returns the element of the \meta{structure} that corresponds 
%       to the \word{data-tree}, or throws an error if the 
%       \word{data-tree} couldn't be interpreted as an element of the 
%       \meta{structure}. The |-errorcode| of such an error should be
%       \begin{displaysyntax}
%         API import EDOM \word{who} \word{bad node}
%       \end{displaysyntax}
%       where |EDOM| is POSIX for ``domain error'', \word{who} is the 
%       \word{path} of the party detecting the error, and \word{bad 
%       node} should be the \word{data-tree} which couldn't be 
%       interpreted.
%       
%       The \word{path} serves two purposes. One is that it may be 
%       matched against \describestring+[attribute]{mtmtcl:path} 
%       attributes in the \word{data-tree}, which can aid in 
%       interpreting abbreviated \word{data-tree}s (for example trees 
%       where unary addition or multiplication nodes have been 
%       elided). The other is to identify the substructure that 
%       has thrown an |EDOM| error. The idea is that it should be 
%       handled as |mtmtcl:path| by the |export| method, i.e., if the 
%       |import| method of structure $S_1$ makes use of the |import| 
%       method of structure $S_2$, and $S_2$ can be accessed using 
%       the $s$ method of $S_1$, then the \texttt{[$S_2$ import $P_2$ 
%       $\dotsb$]} call used by the \texttt{[$S_1$ import $P_1$ 
%       $\dotsb$]} should be such that \(P_2 = P_1 s\) (list append).
%       
%       The \word{path} should not be taken into account when 
%       interpreting a \word{data-tree} without |mtmtcl:path| 
%       attributes.
%   \end{APIdescription}
% \end{APIspec}
% 
% \medskip
% 
% \noindent Since the above is version~2.0 of |export|, there should 
% also have been a version~1 that for some reason was abandoned. It 
% was specified as follows:
% \begin{quote}
% \begin{APIspec}{export}{1.0}
%   This interface is applicable to principal structures. It provides 
%   a method |export| for converting the internal representation of 
%   an element to an external format. The syntax for this method is:
%   \begin{APIdescription}{structure}
%     \begin{APImethod}{export}
%       \word{element} \word{path} \word{options-dict}
%       \begin{regblock}[\regstar] \word{option} \word{value} 
%       \end{regblock}
%     \end{APImethod}
%       where \word{element} is the element to export. The return 
%       value should have the form of an XML-tree as described above. 
%       In want of a \mtl-native way of encoding some concept, it is 
%       recommended that MathML or OpenMath elements are used.
%       
%       The \word{path} is a list of strings, typically a list of 
%       method names. The purpose of this list is to establish a 
%       system for addressing subordinate structures by mirroring the 
%       command prefix that would be used to access its methods. If a 
%       structure $S_1$ makes use of the |export| method of another 
%       structure $S_2$ and $S_2$ can be accessed using the $s$ 
%       method of $S_1$, then the \word{path} used for the 
%       \texttt{[$S_2$ export $\dotsb$]} call should have an $s$ 
%       append to the \word{path} given in the \texttt{[$S_1$ export 
%       $\dotsb$]} call.
%       
%       The option--value pairs can be used to influence the way in 
%       which the value gets encoded, as can the \word{options-dict}. 
%       The keys in this dictionary are \word{path}s, whereas the 
%       values are option--value dictionaries.
%   \end{APIdescription}
%   Note that it is not specified how to merge options specified in 
%   the \word{option-dict} with options explicit in the call; this 
%   might well have to vary from case to case. Nor is it specified 
%   whether the options in the $p$ entry of the \word{options-dict} 
%   apply only for the structure called with $p$ as \word{path}, or 
%   e.g.~also for entries whose \word{path} has $p$ as a prefix; 
%   deciding this requires more experience.
% \end{APIspec}
% \end{quote}
% The main difference between the two are the (rather heavy) 
% machinery for options in version~1\Dash options of unclear effect 
% and without known use-cases! It turned out in all cases I've 
% considered that it is easier to achieve variations by applying a 
% transformation to the result of |export| than to have it built into 
% |export|, and moreover the useful variations are hard to predict 
% when implementing concrete structures while the overhead of 
% handling options in every |export| method is quite considerable.
% 
% Another piece of old machinery which has become deprecated is the 
% use of \mtl-specific tags for common operations:
% \begin{quotation}
%   Much of mathematics is stated in terms of binary operations, 
%   so it makes sense to have a \describestring+[tag]{binop} tag 
%   for binary operation nodes. The many binary operations which 
%   are associative are however better served by an $n$-ary 
%   operation node \describestring+[tag]{assop}, as that is the 
%   level at which they will typically be processed. In both cases, 
%   the children of a node are its operands.
% 
%   Three attributes which seem immediately relevant are:
%   \begin{description}
%     \renewcommand{\makelabel}[1]{\hspace{\labelsep}\texttt{#1}}
%     ^^A Was: \describestring*+[attribute]{#1}
%     \item[symbol]
%       The symbol used to denote the operation. A suitable 
%       default could be empty string, to denote invisible 
%       multiplication\slash juxtaposition.
%     \item[priority]
%       The priority (binding strength) associated with the 
%       symbol. Could default to $3$ for invisible multiplication, 
%       $1$ for $+$, and $2$ for everything else.
%     \item[mtmtcl-path]
%       A list of ``method names'' for accessing the operation 
%       underlying the node. The intention for this attribute is to 
%       simplify postprocessing of a raw exported structure, to 
%       e.g.~change the symbol used for some operation.
%   \end{description}
% \end{quotation}
% This was motivated by a feeling that MathML\slash OpenMath put the 
% name of an operation too far from the node containing its operands, 
% but an attribute is not very much better than an attribute of the 
% first child, and the |symbol| attribute here is \emph{way} too 
% presentation-oriented (as is the |priority| attribute, which 
% belongs fairly far on the way towards a rendering).
% 
% The |integer| tag which was introduced at the same time did however 
% prove useful, as did |mtmtcl-path| (which was renamed to sport an 
% XML namespace separator). That this was changed from a separate 
% \word{path} argument to an entry in the \word{common attributes} 
% argument is a convenience optimisation, but one that conveniently 
% opens up for e.g.~specifying a |cdbase| attribute.
% 
% 
% \StopEventually{}
% 
% 
% \subsection{Generic helpers}
% 
% \setnamespace{mtmtcl::openmath}
% 
% \begin{proc}{cdbase}
%   This procedure revises the |cdbase| attributes of all nodes 
%   in a data-tree, to make sure an |OMS| node has a |cdbase| 
%   attribute if and only if the effective |cdbase| value at that 
%   point is not the default one. The call syntax is
%   \begin{displaysyntax}
%     cdbase \word{tree-var} \word{inherited cdbase}\regopt
%   \end{displaysyntax}
%   where \word{tree-var} is the name of a variable in the calling 
%   context in which the tree is stored, and \word{inherited cdbase} 
%   is the |cdbase| value (`|http://www.openmath.org/cd|') that would 
%   be inherited from the parent. There is no particular return value.
%   
%   In order to ensure this operation is invertible, any existing 
%   attribute with a name on the form |cdbase|$($|-orig|$)^*$ gets an 
%   extra |-orig| in the name.
%   
%   \begin{tcl}
proc mtmtcl::openmath::cdbase\
  {treevar {cdbase http://www.openmath.org/cd}} {
   upvar 1 $treevar tree
%   \end{tcl}
%   The main loop in this procedure is over the nodes in the tree, in 
%   string order but with list indexing. The |path| variable points 
%   to the current node, and the last element in |stack| is the 
%   inherited cdbase value for the current node. Whenever one element 
%   is popped of the |stack|, two elements are removed from the end 
%   of the |path|.
%   \begin{tcl}
   set stack [list $cdbase]
   set path {}
   while 1 {
%   \end{tcl}
%   This outer loop is over the nodes. It is terminated by 
%   |return|ing from the entire procedure.
%   \begin{tcl}
      set cdbase [lindex $stack end]
      set type [lindex [lindex $tree $path] 0]
      if {$type ne "#text"} then {
         set apath [linsert $path end 1]
         set attr [lindex $tree $apath]
         if {[dict exists $attr cdbase]} then {
            set cdbase [dict get $attr cdbase]
         }
%   \end{tcl}
%   The following uses |dict filter| to drop |cdbase|$($|-orig|$)^*$ 
%   elements from the attribute dictionary, while collecting modified 
%   forms of them in |kvL| that are then put back.
%   \begin{tcl}
         set kvL {}
         set fattr [dict filter $attr script {k v} {
            switch -regexp -- $k {^cdbase(-orig)*$} {
               lappend kvL $k-orig $v
               return -level 0 0
            } default {
               return -level 0 1
            }
         }]
         set attr [dict replace $fattr {*}$kvL]
%   \end{tcl}
%   The attribute dictionary in the tree is only updated if it 
%   changes. The main reason for doing this is that the effective 
%   |cdbase| is not the default one, but a nonempty |kvL| also 
%   calls for an update.
%   \begin{tcl}
         if {$type eq "OMS" &&\
           $cdbase ne "http://www.openmath.org/cd"} then {
            dict set attr cdbase $cdbase
            lset tree $apath $attr
         } elseif {[llength $kvL]} then {
            lset tree $apath $attr
         }
%   \end{tcl}
%   This steps the |path| forward into the list of children, if there 
%   are any. The final path component |-1| will be incremented by the 
%   next code block before it is used as index into the tree.
%   \begin{tcl}
         if {[llength [lindex [lindex $tree $path] 2]]} then {
            lappend stack $cdbase
            lappend path 2 -1
         }
      }
%   \end{tcl}
%   The following loop handles incrementing the |path| to point to 
%   the next (in string representation order) node in the tree. There 
%   is one iteration of this loop for every tree level that needs to 
%   be popped of the |stack|.
%   \begin{tcl}
      while 1 {
         if {![llength $path]} then {return}
%   \end{tcl}
%   If the |path| is empty then we're done with the root node and can 
%   |return|. Otherwise increment the final |path| element, to point 
%   it at the next sibling of the current node.
%   \begin{tcl}
         lset path end [expr {1 + [lindex $path end]}]
         if {[llength [lindex $tree $path]]} then {break}
%   \end{tcl}
%   If we were at the final sibling, then the |path| now points to 
%   nothing, and the |lindex| returns an empty string. Any proper 
%   node would have positive list length, so if we don't then it's 
%   time to return to the parent and consider its next sibling.
%   \begin{tcl}
         set path [lreplace $path [set path end-1] end]
         set stack [lreplace $stack [set stack end] end]
      }
   }
}
%   \end{tcl}
% \end{proc}
% 
% \begin{proc}{cdbase-orig}
%   This procedure is the inverse of |cdbase|, primarily throwing 
%   away |cdbase| attributes and removing one |-orig| from the name of 
%   every |cdbase|$($|-orig|$)^+$ attribute. 
%   However, it also preserves the effective |cdbase|, so that trees 
%   which have been modified can be restored to a correct state. The 
%   call syntax is the same as for |cdbase|, i.e.,
%   \begin{displaysyntax}
%     cdbase-orig \word{tree-var} \word{inherited cdbase}\regopt
%   \end{displaysyntax}
%   with no particular return value, as the \word{tree-var} in the 
%   calling context gets modified in place.
%   
%   The overall loop structure is the same as in |cdbase|. The main 
%   difference here is that there are two |cdbase| values: |cdbase| 
%   which is the one encoded in the input, and |cdbase0| which is the 
%   one which would be effective for the node if merely an |-orig| 
%   was removed from all |cdbase|$($|-orig|$)^+$ attributes. When 
%   these are different for an |OMS| node, then the |cdbase| value 
%   is taken as value for the |cdbase| attribute.
%   \begin{tcl}
proc mtmtcl::openmath::cdbase-orig\
  {treevar {cdbase http://www.openmath.org/cd}} {
   upvar 1 $treevar tree
   set stack [list $cdbase]
   set path {}
   while 1 {
      set cdbase0 [lindex $stack end]
      set type [lindex [lindex $tree $path] 0]
      if {$type ne "#text"} then {
         set apath [linsert $path end 1]
         set attr [lindex $tree $path]
         if {[dict exists $attr cdbase-orig]} then {
            set cdbase0 [dict get $attr cdbase-orig]
         }
         if {[dict exists $attr cdbase]} then {
            set cdbase [dict get $attr cdbase]
         } else {
            set cdbase "http://www.openmath.org/cd"
         }
         set kvL {}
         set fattr [dict filter $attr script {k v} {
            switch -regexp -- $k {
               {^cdbase(-orig)+$} {
                  lappend kvL [string range $k 0 end-5] $v
                  return -level 0 0
               }
               {^cdbase$} {return -level 0 0}
               default {return -level 0 1}
            }
         }]
         if {$type eq "OMS" && $cdbase0 ne $cdbase} then {
            lappend kvL cdbase $cdbase
            set cdbase0 $cdbase
         }
         if {[llength $kvL]} then {
            lset tree $apath [dict replace $fattr {*}$kvL]
         } elseif {[dict size $attr] != [dict size $fattr]} then {
            lset tree $apath $fattr
         }
         if {[llength [lindex [lindex $tree $path] 2]]} then {
            lappend stack $cdbase0
            lappend path 2 -1
         }
      }
      while 1 {
         if {![llength $path]} then {return}
         lset path end [expr {1 + [lindex $path end]}]
         if {[llength [lindex $tree $path]]} then {break}
         set path [lreplace $path [set path end-1] end]
         set stack [lreplace $stack [set stack end] end]
      }
   }
}
%   \end{tcl}
% \end{proc}
% 
% 
% \begin{proc}{desala}
%   This procedure searches all nodes in a data-tree for |OMA| or 
%   |OMBIND| nodes whose first child is a specific |OMS| node, and 
%   changes these to some arbitrary node type. The call syntax is
%   \begin{displaysyntax}
%     desala \word{data-tree} \begin{regblock}[\regstar]
%     \word{cd} \word{name} \word{type} \word{attributes} 
%     \end{regblock}
%   \end{displaysyntax}
%   where the \word{cd}s and \word{name}s are glob-patterns matched 
%   against the |cd| and |name| attributes of the |OMS| node. 
%   \word{type} is the corresponding node type it would be converted 
%   to and \word{attributes} is a dictionary of extra attributes to 
%   add to the node. The return value is the converted data-tree.
%   
%   Depending on the original type, the modified nodes have one 
%   attribute \describestring+[attribute]{OMA} or 
%   \describestring+[attribute]{OMBIND}, and one attribute 
%   \describestring+[attribute]{OMS}, which contain the attribute 
%   dictionaries of the original |OMA|\slash |OMBIND| and |OMS| 
%   nodes. 
%   If there are attributions (which may be nested) of the symbol, 
%   then the new node will also have an 
%   \describestring+[attribute]{OMATP} attribute. The value of this 
%   is a dictionary which maps attribution symbol URIs to corresponding 
%   OpenMath objects (as data-trees); it is a flattened form of the 
%   attribution data. The URIs are constructed as
%   \begin{quote}
%     \meta{cdbase}\texttt{/}\meta{cd}\texttt{\#}\meta{name}
%   \end{quote}
%   if there is a |cdbase| attribute, and as 
%   \meta{cd}\texttt{\#}\meta{name} otherwise. Hence the presence 
%   of an alternative \LaTeX\ encoding for the |OMS| symbol can 
%   be tested by the command
%   \begin{displaysyntax}
%     dict exists \word{attributes} OMATP |altenc#LaTeX_encoding|
%   \end{displaysyntax}
%   The |OMA|, |OMBIND|, |OMS|, or |OMATP| attributes generated by 
%   this procedure will override attributes of those names provided 
%   in an \word{attributes} argument.
%   
%   The implementation is not recursive; instead an explicit |stack| 
%   of list-index paths to nodes to be converted is maintained. This 
%   preserves the Tcl\_Objs of subtrees which are unchanged.
%   \begin{tcl}
proc mtmtcl::openmath::desala {tree args} {
   set stack {{}}
   while {[llength $stack]} {
      set path [lindex $stack end]
      set stack [lreplace $stack [set stack end] end]
      set node [lindex $tree $path]
%   \end{tcl}
%   Non-element nodes (|#text| nodes) and |OMFOREIGN| nodes shouldn't 
%   be processed at all, since they either have no children or don't 
%   need to adhere to the OpenMath standard.
%   \begin{tcl}
      if {[llength $node]<3 || [lindex $node 0] eq "OMFOREIGN"}\
      then {continue}
%   \end{tcl}
%   Due to the possibility of |OMATTR|s, it is rather complicated to 
%   determine whether a particular |OMA| or |OMBIND| should be 
%   converted. The condition goes from here \dots
%   \begin{tcl}
      if {
         [lindex $node 0] in {OMA OMBIND} && [
            set cattr [dict create [lindex $node 0] [lindex $node 1]]
            set snode [lindex $node 2 0]
            while {[lindex $snode 0] eq "OMATTR"} {
               set snode [lindex $snode 2 1]
            }
            lindex $snode 0
         ] eq "OMS" && [
            dict set cattr OMS [lindex $snode 1]
            dict exists $cattr OMS cd
         ] && [dict exists $cattr OMS name] && [
            set cd [dict get $cattr OMS cd]
            set name [dict get $cattr OMS name]
            set match 0
            foreach {cdp namep type xattr} $args {
               if {[string match $cdp $cd] &&\
                 [string match $namep $name]} then {
                  set match 1
                  break
               }
            }
            set match
         ]
      } then {
%   \end{tcl}
%   \dots~to here, and it is has the side-effects of setting the 
%   |cattr|, |type|, and |xattr| variables.
%   
%   Once it has been determined that this |OMA|\slash |OMBIND| will 
%   be converted, the next step is to construct the |OMATP| entry 
%   (if there is to be one).
%   \begin{tcl}
         for {
            set snode [lindex $node 2 0]
         } {[lindex $snode 0] eq "OMATTR"} {
            set snode [lindex $snode 2 1]
         } {
            foreach {key value} [lindex $snode 2 0 2] {
               if {[lindex $key 0] ne "OMS"} then {continue}
               set sattr [dict merge {cd {} name {}}\
                 [lindex $key 1]]
               if {[dict exists $sattr cdbase]} then {
                  dict set cattr OMATP [format %s/%s#%s\
                    [dict get $sattr cdbase]\
                    [dict get $sattr cd] [dict get $sattr name]]\
                    $value
               } else {
                  dict set cattr OMATP 
                    [dict get $sattr cd]\#[dict get $sattr name]\
                    $value
               }
            }
         }
%   \end{tcl}
%   Then it is finally time to construct the new node.
%   \begin{tcl}
         set node [list $type [dict merge $xattr $cattr]\
           [lindex $node 2]]
         lset tree $path $node
      }
%   \end{tcl}
%   Regardless of whether the node was converted or not, it is 
%   necessary to process also all of its children, by pushing their 
%   |lindex|-paths on the stack.
%   \begin{tcl}
      lappend path 2
      set i 0
      foreach child [lindex $node 2] {
         lappend stack [linsert $path end $i]
         incr i
      }
   }
   return $tree
}
%   \end{tcl}
%   The original version of this procedure would drop the first child 
%   of the |OMA| node from the new node, on the grounds that all the 
%   information in it had been encoded into the attributes, but this 
%   turned out to lead to inconsistencies elsewhere. Specifically, 
%   conversions of |OMA| nodes carried out for known symbols would be 
%   different than conversions carried out for unknown symbols.
%   
%   The original version also didn't convert |OMBIND| nodes, but it 
%   makes sense to support conversion of both. The restriction to a 
%   particular symbol usually suffices for distinguishing the two.
% \end{proc}
% 
% \begin{proc}{oma}
%   This procedure implements the inverse of the |desala| operation: 
%   convert nodes with |OMA| attributes back to |OMA| nodes (having 
%   only the attributes listed in the |OMA| dictionary) and nodes 
%   with |OMBIND| attributes back to |OMBIND| nodes. The call 
%   syntax is
%   \begin{displaysyntax}
%     oma \word{data-tree} \word{pattern}\regstar
%   \end{displaysyntax}
%   where the \word{pattern}s are used to constrain the conversions: 
%   a node is only converted if it matches one of these patterns.
%   \begin{tcl}
proc mtmtcl::openmath::oma {tree args} {
   set stack {{}}
   while {[llength $stack]} {
      set path [lindex $stack end]
      set stack [lreplace $stack [set stack end] end]
      set node [lindex $tree $path]
      if {[llength $node]<3 || [lindex $node 0] eq "OMFOREIGN"}\
      then {continue}
      lassign $node type attr children
      if {[dict exists $attr OMA]} then {
         foreach pat $args {
            if {[string match $pat $type]} then {
               lset tree $path [list OMA [dict get $attr OMA] $children]
               break
            }
         }
      } elseif {[dict exists $attr OMBIND]} then {
         foreach pat $args {
            if {[string match $pat $type]} then {
               lset tree $path\
                 [list OMBIND [dict get $attr OMBIND] $children]
               break
            }
         }
      }
      lappend path 2
      set i 0
      foreach child $children {
         lappend stack [linsert $path end $i]
         incr i
      }
   }
   return $tree
}
%   \end{tcl}
% \end{proc}
% 
% \begin{proc}{gettext}
%   The |gettext| command walks through a data-tree and concatenates 
%   all the \describestring*+[node type]{#text} nodes found in it. 
%   This is typically used where a node is only supposed to contain 
%   text, and one doesn't want to mandate that all of it is in a 
%   single |#text| child. The call syntax is
%   \begin{displaysyntax}
%     gettext \word{tree} \word{join-string}\regopt
%   \end{displaysyntax}
%   where the \word{join-string} will be put between each |#text| 
%   leaf. It defaults to a space.
%   \begin{tcl}
proc mtmtcl::openmath::gettext {tree {how " "}} {
   set res {}
   set stack [list {}]
   while {[llength $stack]} {
      set path [lindex $stack end]
      set stack [lreplace $stack [set stack end] end]
      set node [lindex $tree $path]
      switch -- [lindex $node 0] "#text" {
         lappend res [lindex $node 1]
      } default {
         lappend path 2
         for {set n [llength [lindex $node 2]]} {$n>0} {} {
            lappend stack [linsert $path end [incr n -1]]
         }
      }
   }
   return [join $res $how]
}
%   \end{tcl}
% \end{proc}
% 
% \begin{tcl}
namespace eval mtmtcl::openmath {
   namespace export *
}
% \end{tcl}
% 
% 
% \section{Presenting expressions}
% 
% Although the (near-)OpenMath data-trees generated by the |export| 
% method generally do a very good job of encoding an expression (or 
% other mathematical object), they're not very readable; some way of 
% rendering the object as an ordinary formula is definitely needed. 
% One approach for that would be a conversion chain OpenMath${} 
% \longrightarrow {}$Content-MathML${}\longrightarrow
% {}$Presentation-MathML, but this is not an output format I feel 
% comfortable with; \LaTeX\ is preferable, so that one can use it in 
% one's papers. MathML furthermore tends to suggest a reliance on 
% generic XML processors, whereas I'd prefer a pure-\Tcl\ solution.
% 
% There's no reason not to borrow ideas from the big-league 
% toolchain, however. In particular, it is a reasonable first step to 
% seek a \emph{presentation form} of the data-tree to be rendered, 
% where additional material has been inserted that clarifies how the 
% expression should be rendered.
% \begin{tcl}
namespace eval mtmtcl::present {
   namespace path [list [namespace parent]]
}
% \end{tcl}
% \setnamespace{mtmtcl::present}
% 
% 
% \subsection{Typographic preprocessing}
% 
% The purpose of typographic preprocessing is to change expressions 
% to a form that is typographically optimal, even if it is not the 
% most consistent from a technical point of view. The transformations 
% that are made include:
% \begin{enumerate}
%   \item
%     Removal of pointless ones and zeroes.
%   \item
%     Expansion of numbers (particularly floating-point numbers in 
%     mantissa--exponent notation). This process is supposed to equip 
%     every number with a ``first child presentation'', which 
%     typically includes at least one |mn| node.
%   \item
%     Outward migration of minuses.
% \end{enumerate}
% Number expansion simplifies fencing for priorities, since it makes 
% explicit those operations (e.g.~minuses) that are used in the 
% standard notations for various numbers. The other transformations 
% are more a matter of cleanup.
% 
% 
% \begin{proc}{typography}
%   This is a generic simplifier of near-OpenMath objects according 
%   to typographic rules, which also takes the first steps towards 
%   generating a presentation by equipping |integer| and |OMF| nodes 
%   with a content. The call syntax is
%   \begin{displaysyntax}
%     typography \word{data-tree}
%   \end{displaysyntax}
%   and the return value is the transformed \word{data-tree}.
%   
%   Most of the work is carried out by the |tpp| ensemble.
%   \begin{tcl}
proc mtmtcl::present::typography {tree} {
   tpp::tool::tpp {*}[
      openmath::desala $tree \
        arith1 times multiply {}  arith1 power power {} \
        arith1 plus add {}  arith1 unary_minus negate {}
   ] 0
}
%   \end{tcl}
% \end{proc}
% 
% 
% The OpenMath symbols of immediate relevance are:
% \begin{description}
%   \item[\OMSref*{arith1}{times}]
%     Generally multiplication (both associative $n$-ary and scalar 
%     multiple, which could be a problem).
%   \item[\OMSref*{arith1}{power}]
%     Integer power.
%   \item[\OMSref*{arith1}{unary_minus}]
%     Negation operation, arising from eliminating a $-1$ factor.
%   \item[\OMSref*{alg1}{one}]
%     Multiplicative unit, arising e.g.~from raising something to 
%     the $0$th power, or from a nullary product.
%   \item[\OMSref*{arith1}{plus}]
%     Addition.
% \end{description}
% In processing, these are |desala|d into specialised node types:
% \begin{description}
%   \renewcommand{\makelabel}[1]{^^A
%     \hspace{\labelsep}\describestring*+[node type]{#1}^^A
%   }
%   \item[multiply]
%     Generally multiplication, both operations expressed as 
%     juxtaposition and operations with an infix operation symbol.
%   \item[power]
%     General powers.
%   \item[negate]
%     Additive negation.
%   \item[add]
%     Addition.
% \end{description}
% As a rule these |OMA|-normalises themselves upon returning, but 
% |negate| nodes may remain to denote points of soft negation. 
% Another type of node that may be created by this operation is 
% \describestring+[node type]{mn}, which surrounds a raw number. 
% The content (text) of this node type is a sequence of digits, 
% possibly including a decimal point. The reason for having such a 
% wrapper in the first place is that digits are special with respect 
% to juxtaposition; they may only occur first in a block of 
% juxtaposed factors. |mn| nodes only occur in the presentation of 
% an |integer|, |OMF|, or similar node.
% 
% 
% \setnamespace{mtmtcl::present::tpp::tool}
% \begin{ensemble}{tpp}
%   To facilitate recursion, the |tpp| ensemble command resides in 
%   the |tool| child of its actual namespace |mtmtcl::present::tpp|.
%   \begin{tcl}
namespace eval mtmtcl::present::tpp {
   namespace path [list [namespace parent [namespace parent]]]
   namespace export *
   namespace ensemble create -command tool::tpp -prefix 0
%   \end{tcl}
% \end{ensemble}
% The general syntax for this ensemble is
% \begin{displaysyntax}
%   tool::tpp \meta{node} \word{want negate}
% \end{displaysyntax}
% where \word{want negate} is a boolean specifying whether the parent 
% of the \meta{node} wants it to be a |negate| if that is 
% mathematically correct.
% 
% \begin{proc}{recurse}
%   Most node types don't require typographic preprocessing, so the 
%   handling of an unknown node type is to recurse over its children. 
%   Since |negate|s couldn't pass through such nodes, it follows that 
%   they don't want any negations from their children.
%   \begin{tcl}
   proc tool::recurse {type attr children want} {
      set res {}
      foreach child $children {lappend res [tpp {*}$child 0]}
      list $type $attr $res
   }
   namespace ensemble configure tool::tpp -unknown [
      list ::apply {{target cmd type args} {list $target $type} ::}\
        [namespace which tool::recurse]
   ] 
}
%   \end{tcl}
% \end{proc}
% 
% \setnamespace{mtmtcl::present::tpp::tool}
% \begin{ensemble}{tpp}
%   \setnamespace{mtmtcl::present::tpp}
%   \begin{ensproc}{#text}
%     As usual, |#text| nodes are exceptional and need an explicit 
%     definition.
%     \begin{tcl}
proc mtmtcl::present::tpp::#text {str want} {list #text $str}
%     \end{tcl}
%   \end{ensproc}
%   
%   \begin{ensproc}{OMFOREIGN}
%     There should not be any recursion into foreign material, so 
%     |OMFOREIGN| also needs an explicit definition.
%     \begin{tcl}
proc mtmtcl::present::tpp::OMFOREIGN {attr children want} {
   list OMFOREIGN $attr $children
}
%     \end{tcl}
%   \end{ensproc}
%   
%   Similar ``don't go there'' definitions could probably also be 
%   added for e.g.~|OME| nodes, but it does not seem essential, 
%   either.
%   
%   \begin{ensproc}{OMATTR}
%     Attributions need a special definition because they should be 
%     transparent to minus migration\Dash in particular, a |negate| 
%     should move out of it. A consequence of this is that the 
%     recursion only visits the nucleus child.
%     \begin{tcl}
proc mtmtcl::present::tpp::OMATTR {attr children want} {
   set nucleus [tool::tpp {*}[lindex $children 1] $want]
   if {$want && [lindex $nucleus 0] eq "negate"} then {
      lset nucleus 2 1 [
         list OMATTR $attr [lreplace $children 1 1\
           [lindex $nucleus 2 1]]
      ]
      return $nucleus
   } else {
      return [list OMATTR $attr [lreplace $children 1 1 $nucleus]]
   }
}
%     \end{tcl}
%     This could in principle be wrong for semantic attributions, but 
%     so far I haven't encountered any attribution that would be 
%     ill-affected by it.
%   \end{ensproc}
%   
%   \begin{ensproc}{add}
%     The primary motivator for minus migrations is addition 
%     operations, or technically the |add| nodes (which take the 
%     place of |OMA|s). Regardless of outer want, an |add| requests 
%     minus migration from all operands except the first one.
%     \begin{tcl}
proc mtmtcl::present::tpp::add {attr children want} {
   set res {}
   foreach child $children {
      lappend res [
         tool::tpp {*}$child [expr {$want || [llength $res]>1}]
      ]
   }
   if {[llength $res] == 2} then {
      return [lindex $res 1]
   } else {
      return [list OMA [dict get $attr OMA] $res]
   }
}
%     \end{tcl}
%     A variation could be to sometimes pass a |negate| on if the 
%     parent wants it, e.g. if the first term is |negate|d or if a 
%     majority of the terms are negated, but this will suffice for 
%     basic processing.
%   \end{ensproc}
%   
%   \begin{ensproc}{multiply}
%     The other main case of an operation which is permeable for 
%     minus migration is multiplication, which exists in a variety of 
%     forms. |multiply| nodes are for unital and associative 
%     multiplication operations, whereas |distributive| nodes avoid 
%     making this assumption.
%     
%     Multiplication is tricky in that it can also serve as motivator 
%     for minus migration. Concretely, a |multiply| with priority 
%     $<3$ (the default priority of a unary minus) will not motivate 
%     migration, but other |multiply|s will. The reason for this is 
%     purely to avoid fencing. If $(-a)b$ can be rendered as $-ab$ by 
%     minus migration to $-(ab)$ then this is of course good, but in 
%     $(-a) \wedge b$ the same migration would lead to the rendering 
%     $-(a \wedge b)$, whereas the original expression renders as $-a 
%     \wedge b$. The difference is that $\wedge$ has lower priority 
%     than unary minus, whereas invisible multiplication has higher 
%     priority.
%     
%     Formally, operations turned into |multiply| nodes have 
%     to satisfy the following properties:
%     \begin{itemize}
%       \item
%         A negation in any factor can be turned into a negation of 
%         the expression as a whole.
%       \item
%         A unit factor (be it an |integer| $1$ or a 
%         \OMSref{alg1}{one}) can be dropped.
%       \item
%         A unary operation can be elided. A nullary operation is 
%         a \OMSref{alg1}{one}.
%       \item
%         An |OMA| child which has a |multiply| attribute can be 
%         contracted into the node, i.e., its children (excluding the 
%         non-operand child) replace the child.
%     \end{itemize}
%     The first two properties would mostly follow if one requests 
%     multilinearity over the integers, but that would also call for 
%     combining all integer factors. It could be useful to somehow 
%     preserve the nature of a \OMSref{alg1}{one}, but that's fine 
%     detail.
%     
%     The value of a \describestring+[attribute]{multiply} 
%     attribute is a list of operations (typically |OMS| nodes or 
%     decorations thereof), concretely the operations that would have 
%     appeared between the operands in the raw data-tree. Hence it 
%     should always have one elements less than the |OMA| node has 
%     operands. The purpose of this information is to provide other 
%     processors full information about which operations have come 
%     together into a multiplication node, in order to select a 
%     dominant one. |multiply| always keeps its own operation, which 
%     in a combination of |multiply|s means it is the outermost one 
%     that wins; while natural from a computer science point of view, 
%     it is seldom mathematically optimal.
%     
%     The first step is to determine whether there is a want for 
%     negation. External want status is preserved since it will be 
%     needed later. The default for 
%     \describestring+[attribute]{priority} here is ``at least $3$''.
%     \begin{tcl}
proc mtmtcl::present::tpp::multiply {attr children want} {
   if {$want} then {
      set w 1
   } elseif {[dict exists $attr OMA priority]} then {
      set w [expr {[dict get $attr OMA priority] >= 3}]
   } else {
      set w 1
   }
%     \end{tcl}
%     The operation child is preprocessed just like all the others, 
%     just in case it has an inner structure (e.g. carries indices). 
%     The |multiply| list has an extra element at the end when it is 
%     constructed; that is dropped when putting it in the attribute.
%     \begin{tcl}
   set newchildren [list [tool::tpp {*}[lindex $children 0] 0]]
   set negate 0
   set multiply {}
   foreach child [lrange $children 1 end] {
      set child [tool::tpp {*}$child $w]
      while {[lindex $child 0] eq "negate"} {
         set negate [expr {!$negate}]
         set child [lindex $child 2 1]
      }
      switch -- [lindex $child 0] "OMS" {
         set cattr [lindex $child 1]
         switch -- [dict get $cattr cd]/[dict get $cattr name] {
            "alg1/one" {continue}
         }
      } "integer" {
         set value [dict get [lindex $child 1] value]
         if {$value == 1} then {continue}
      } "OMA" {
         set cattr [lindex $child 1]
         if {[dict exists $cattr multiply]} then {
            lappend newchildren {*}[lrange [lindex $child 2] 1 end]
            lappend multiply {*}[dict get $cattr multiply]\
              [lindex $newchildren 0]
            continue
         }
      }
      lappend newchildren $child
      lappend multiply [lindex $newchildren 0]
   }
   dict set attr OMA multiply [lrange $multiply 0 end-1]
   if {[llength $newchildren]>2} then {
      set res [list OMA [dict get $attr OMA] $newchildren]
   } elseif {[llength $newchildren] == 2} then {
      set res [lindex $newchildren 1]
   } else {
      set res [list OMS [
         dict replace $attr  cd alg1  name one
      ] {}]
   }
   if {$negate} then {
      return [tool::negate $res $want]
   } else {
      return $res
   }
}
%     \end{tcl}
%   \end{ensproc}
% \end{ensemble}
% 
% \begin{proc}{negate}
%   The |tool::negate| procedure handles the (in minus migration 
%   common) task of wrapping a node in a |negate| node. The call 
%   syntax is
%   \begin{displaysyntax}
%     tool::negate \word{node} \word{soft}
%   \end{displaysyntax}
%   where \word{node} is the node to wrap and \word{soft} is a 
%   boolean; the return value is the wrapped node.
%   
%   There are two kinds of negating: hard and soft. Hard negations 
%   are expected to show up as unary minuses, whereas soft negations 
%   are expected to be inlined into a sum as a binary minus. These 
%   are different with respect to priority, since a soft negation 
%   should have the priority level of the surrounding addition, 
%   whereas hard negation has the priority of a unary operation. 
%   Hard negations are encoded in the obvious manner (an |OMA| 
%   against \OMSref{arith1}{unary_minus}), whereas soft negations 
%   are encoded using the custom 
%   \describestring+[node type]{negate} node type. This has the form 
%   of a |desala|d hard negation, but should have 
%   |form|${}={}$|function| to ensure the argument is fenced\Ldash 
%   a fencing purely by priorities could otherwise lead to confusing 
%   results\Dash although normally it shouldn't get to handle its own 
%   formatting.
%   \begin{tcl}
proc mtmtcl::present::tpp::tool::negate {node soft} {
   set minus {OMS {cd arith1 name unary_minus} {{#text -}}}
   set children [list $minus $node]
   if {$soft} then {
      list negate {OMA {form function}} $children
   } else {
      list OMA {form prefix} $children
   }
}
%   \end{tcl}
% \end{proc}
% 
% \begin{ensemble}{tpp}
%   \setnamespace{mtmtcl::present::tpp}
%   \begin{ensproc}{negate}
%     Actual |negate| nodes in the input signify a hard negation, but 
%     these would like to cancel themselves against a soft negation. 
%     They can also turn into soft negations if the parent |want|s 
%     them to.
%     \begin{tcl}
proc mtmtcl::present::tpp::negate {attr children want} {
   set nucleus [tool::tpp {*}[lindex $children 1] 1]
   if {[lindex $nucleus 0] eq "negate"} then {
      lindex $nucleus 2 1
   } elseif {$want} then {
      tool::negate $nucleus 1
   } else {
      list OMA [dict get $attr OMA] [lreplace $children 1 1 $nucleus]]
   }
}
%     \end{tcl}
%     Both branches of the |elseif| could have been handled a single 
%     |tool::negate|, but doing the hard negation explicitly permits 
%     preserving more context.
%   \end{ensproc}
%   
%   \begin{ensproc}{distributive}
%     Not all multiplication-style operations fit the |multiply| 
%     transformation nodes (which more or less presumes the operation 
%     to be unital and associative). The |distributive| nodes 
%     only participate in minus migration, and will not force it 
%     unless they have an explicit 
%     \describestring+[attribute]{priority} setting exceeding $3$.
%     \begin{tcl}
proc mtmtcl::present::tpp::distributive {attr children want} {
   if {$want} then {
      set w 1
   } elseif {[dict exists $attr OMA priority]} then {
      set w [expr {[dict get $attr OMA priority] > 3}]
   } else {
      set w 0
   }
   set newchildren [list [tool::tpp {*}[lindex $children 0] 0]]
   set negate 0
   foreach child [lrange $children 1 end] {
      set child [tool::tpp {*}$child $w]
      while {[lindex $child 0] eq "negate"} {
         set negate [expr {!$negate}]
         set child [lindex $child 2 1]
      }
      lappend newchildren $child
   }
   if {$negate} then {
      return [tool::negate $res $want]
   } else {
      return $res
   }
}
%     \end{tcl}
%   \end{ensproc}
%   
%   \begin{ensproc}{integer}
%     The main source of minuses is negative numbers, which always 
%     split in a negation and an unsigned number.
%     \begin{tcl}
proc mtmtcl::present::tpp::integer {attr children want} {
   set value [dict get $attr value]
   set negate [expr {$value<0}]
   if {$negate} then {
      set value [expr {-$value}]
      dict set attr value $value
   }
   set res [list mn {} [list [
      list #text [format %lld $value]
   ]]]
   set res [list integer $attr [list $res]]
   if {$negate} then {
      tool::negate $res $want
   } else {
      set res
   }
}
%     \end{tcl}
%   \end{ensproc}
%   
%   \begin{ensproc}{OMI}
%     |OMI| nodes aren't supposed to occur in near-OpenMath, but in 
%     case they do then this is a suitable place to convert them to 
%     |integer| nodes.
%     \begin{tcl}
proc mtmtcl::present::tpp::OMI {attr children want} {
   set str [openmath::gettext [list OMI $attr $children]]
   regsub -all {\s} $str {} str
   if {[regsub {[Xx]} $str {} str]} then {
      scan $str %llx value
   } else {
      scan $str %lld value
   }
   integer [dict replace $attr value $value] {} $want
}
%     \end{tcl}
%     It may be remarked that this does not try (much) to cope with 
%     malformed integers; if it cannot be scanned then |value| won't 
%     get defined and an error follows. The |ll| in |scan| format 
%     specifiers is needed to get large values parsed correctly 
%     (otherwise it would be arbitrarily reduced to fit in a 
%     \Ctypeidentifier{long}).
%   \end{ensproc}
%   
%   \begin{variable}{OMF_format}
%     This variable can hold a |format| specifier string for |OMF| 
%     values, typically |%g| with some precision. When this variable 
%     is not set (as is the default) then \Tcl's string 
%     representation is used instead; this seems to do a slightly 
%     better job at choosing how many decimals to include.
%   \end{variable}
%   
%   \begin{ensproc}{OMF}
%     The expansion of a float faces several complications, the first 
%     of which is to parse the value. It is assumed that \Tcl's 
%     |double| type is IEEE-compliant, but nothing worse than loss of 
%     precision should happen if it turns out not to be.
%     \begin{tcl}
proc mtmtcl::present::tpp::OMF {attr children want} {
   if {[dict exists $attr dec]} then {
      scan [dict get $attr dec] %g val
   } else {
%     \end{tcl}
%     If there is no |dec| key, then there must be a |hex| key. It's 
%     obvious that the way to parse this value is to |binary format| 
%     the hex data and then |binary scan| it as an IEEE double. What 
%     is tricky is working out which endianness to use; the OpenMath 
%     specification seems to suggest it should be little-endian, but 
%     the combination which manages to decode the examples in it 
%     turns out to be big-endian\slash big-endian.
%     \begin{tcl}
      binary scan [binary format H* [dict get $attr hex]] Q val
   }
%     \end{tcl}
%     Once we've got the value, it's time to choose between 
%     mantissa--exponent and mantissa notation. |format %g| does this 
%     well, so the value is translate into an expression by way of 
%     the |format|ted value.
%     \begin{tcl}
   variable OMF_format
   set L [split [
      if {[info exists OMF_format]} then {format $OMF_format $val}\
        else {set val}
   ] eE]
%     \end{tcl}
%     The mantissa can be separated by |split|ting at |e|, and one 
%     can use pattern matching to check it for being negative. 
%     Infinities and not-a-numbers are handled at the same time 
%     (no |e| in those).
%     \begin{tcl}
   switch -glob -- [lindex $L 0] Inf - -Inf {
      set tree {OMS {cd nums1 name infinity} {}}
   } NaN* {
      set tree {OMS {cd nums1 name NaN} {}}
   } -* {
      set tree [list mn {} [list [
         list #text [string range [lindex $L 0] 1 end]
      ]]]
   } default {
      set tree [list mn {} [list [list #text [lindex $L 0]]]]
   }
%     \end{tcl}
%     Hard negation can at this point be applied to the mantissa 
%     (|tree|), but soft negation must to be delayed until the very 
%     end, as it needs be applied to the entire |OMF|.
%     \begin{tcl}
   set negate 0
   if {[string match -* [lindex $L 0]]} then {
      if {$want} then {set negate 1} else {
         set tree [tool::negate $tree 0]
      }
   }
%     \end{tcl}
%     Putting on an exponent is slightly roundabout, but this is how 
%     long it gets in OpenMath. The |form infix| setting tells 
%     renderers to generate an explicit symbol for this 
%     multiplication.
%     \begin{tcl}
   if {[llength $L]>1} then {
      set tree [list OMA {form infix} [list [
         list OMS {cd arith1 name times} {}
      ] $tree [
         list OMA {} [list [
            list OMS {cd arith1 name power} {}
         ] [
            integer {value 10} {} 0
         ] [
            integer [dict create value [lindex $L 1]] {} 0
         ]]
      ]]]
   }
%     \end{tcl}
%     What is tricky about soft negation is that it migrates the 
%     minus out of the |OMF|, so the numeric value of that node will 
%     change. The new |OMF| will always have its value specified in 
%     |dec| form.
%     \begin{tcl}
   if {$negate} then {
      set value [expr {-$value}]
      return [negate [
         list OMF [dict replace [dict remove $attr hex] dec $value]\
           [list $tree]
      ] 1]
   } else {
      return [list OMF $attr [list $tree]]
   }
}
%     \end{tcl}
%   \end{ensproc}
%   
%   \begin{ensproc}{power}
%     One final node type transformed by this ensemble is |power|. 
%     The two cases it is supposed to change are that the exponent 
%     is $0$ or $1$, but it turns out other integers also provides 
%     useful behaviour.
%     \begin{tcl}
proc mtmtcl::present::tpp::power {attr children want} {
%     \end{tcl}
%     If there are not two operands, or the second operand is not an 
%     |integer|, then just |recurse|.
%     \begin{tcl}
   if {[llength $children] != 3 ||\
     [lindex $children 2 0] ne "integer"} then {
      return [tool::recurse OMA [dict get $attr OMA] $children 0]
   }
%     \end{tcl}
%     The rest depends on the value of that integer.
%     \begin{tcl}
   set val [dict get [lindex $children 2 1] value]
   if {$val == 0} then {
%     \end{tcl}
%     $x^0$ is a \OMSref{alg1}{one} |OMS| node, but note that 
%     attributes of the original \OMSref{arith1}{power} are 
%     preserved, in particular the |mtmtcl:path|.
%     \begin{tcl}
      return [list OMS\
        [dict replace [dict get $attr OMS] cd alg1 name one] {}]
%     \end{tcl}
%     $x^1$ is just $x$; the |power| node disappears.
%     \begin{tcl}
   } elseif {$val == 1} then {
      return [tool::tpp {*}[lindex $children 1] $want]
   }
%     \end{tcl}
%     A third case is $x^{2n}$. Both the base and the exponent need 
%     to be preserved in this case, but since the power will kill 
%     all negations in the base, it is useful to want them migrating.
%     \begin{tcl}
   set negate 0
   if {$val % 2 == 0} then {
      set base [tool::tpp {*}[lindex $children 1] 1]
      while {[lindex $base 0] eq "negate"} {
         set base [lindex $base 2 1]
      }
%     \end{tcl}
%     The fourth and final case is $x^{2n+1}$, which minuses can 
%     migrate through.
%     \begin{tcl}
   } else {
      set base [tool::tpp {*}[lindex $children 1] $want]
      if {[lindex $base 0] eq "negate"} then {
         set negate 1
         set base [lindex $base 2 1]
      }
   }
   set res [list OMA [dict get $attr OMA] [
      lreplace $children 1 2 $base\
        [integer [lindex $children 2 1] {} 0]
   ]]
   if {$negate} then {
      return [tool::negate $res 1]
   } else {
      return $res
   }
}
%     \end{tcl}
%   \end{ensproc}
% \end{ensemble}
% 
% 
% \subsection{Data-tree decoration}
% 
% As stated above, knowledge about how various symbols are formatted 
% should not be built into processors converting to other formats, 
% but rather provided as part of the input to that processor; this 
% gives structures employing notation beyond the most common better 
% chances to control formatting. There is however also a need to 
% provide default formatting for more usual symbols, as this is 
% typically what authors would expect. This can (to a large extent) 
% be achieved by scanning a data-tree for known symbols and 
% decorating them (or their parents) with the appropriate attributes 
% or attributions, so that is what the following implements.
% 
% The rules for how to decorate when a particular symbol is 
% encountered can rightly be termed the ``phrasebook'' (at least in 
% the passive sense) in the \mtl\ rendering system. They are kept in 
% the form of an array, whose keys are short URIs for symbols and 
% whose entries are scripts specifying what should be done when that 
% symbol is encountered. These scripts will typically use only some 
% specialised commands that expect to be evaluated in a particular 
% context where they can access the tree being decorated, but they 
% are proper \Tcl\ scripts and can contain other commands if you need 
% them to. The only condition is that variable names beginning with 
% an underscore are reserved for use by the decoration context itself.
% 
% \setnamespace{mtmtcl::present}
% 
% \begin{proc}{decorate}
%   The |decorate| procedure is the chief executive in the decoration 
%   process, and also the top-level command to call. Its syntax is
%   \begin{displaysyntax}
%     decorate \word{tree-var} \word{phrasebook-var}
%   \end{displaysyntax}
%   where both arguments are names of variables in the calling 
%   context. The \word{tree-var} stores the data-tree to decorate, 
%   and will be modified in place. The \word{phrasebook-var} is an 
%   array which constitutes the phrasebook to use. The return value 
%   is a report on how the decoration went; it contains the 
%   |-errorinfo|s of all errors that occurred during the process.
%   
%   \begin{tcl}
proc mtmtcl::present::decorate {treevar phrasevar} {
   upvar 1 $treevar tree $phrasevar phraseA
%   \end{tcl}
%   The particular context for the decoration scripts is an anonymous 
%   procedure (lambda) in the |mtmtcl::present::decorate| namespace, 
%   which has |upvar|ed selected local variables from the |decorate| 
%   procedure.
%   \begin{tcl}
   set lambda [list script {
      upvar 1 tree _tree paths _paths res _res
      if 1 then $script
   } [namespace current]::decorate]
%   \end{tcl}
%   The |res| variable is a list of messages, which at the end will 
%   be |join|ed by newlines; the effect is that the result is an 
%   empty string if there aren't any messages. |paths| is a list of 
%   paths pointing to nodes in the tree. The current path when 
%   looping over the tree is kept in the |path| variable, which is 
%   used as in the |cdbase| procedure, and in the first element of 
%   the |paths| list.
%   \begin{tcl}
   set res {}
   set path {}
   set code [catch {while 1 {
%   \end{tcl}
%   Errors in the decoration scripts are being caught. Since errors 
%   in these scripts may also damage the |tree|, the entire main loop 
%   is caught as well.
%   \begin{tcl}
      if {[lindex [lindex $tree $path] 0] eq "OMS"} then {
         set attr [lindex [lindex $tree $path] 1]
         set URI ""
         if {[dict exists $attr cdbase]} then {
            append URI [dict get $attr cdbase] /
         }
         append URI [dict get $attr cd] \# [dict get $attr name]
         if {[info exists phraseA($URI)]} then {
%   \end{tcl}
%   A decoration script exists for this symbol. Run it, and record 
%   errors in |res|.
%   \begin{tcl}
            set paths [list $path]
            set code [catch {apply $lambda $phraseA($URI)} msg opt]
            set path [lindex $paths 0]
            if {$code == 1} then {
               set msg "Error \"$msg\" at"
               append msg \n [decorate::nodeheader $path\
                 [lindex $tree $path]] ":\n  "
               append msg [string map {\n "\n  "}\
                 [dict get $opt -errorinfo]]
               lappend res $msg
            }
         }
      }
%   \end{tcl}
%   A |#text| node has no list of children, but it counts as `no 
%   children' here anyway.
%   \begin{tcl}
      if {[llength [lindex [lindex $tree $path] 2]]} then {
         lappend path 2 -1
      }
      while 1 {
         if {![llength $path]} then {return}
%   \end{tcl}
%   This |return| is the normal exit from the main loop. It will be 
%   caught by the surrounding |catch|.
%   \begin{tcl}
         lset path end [expr {1 + [lindex $path end]}]
         if {[llength [lindex $tree $path]]} then {break}
         set path [lreplace $path [set path end-1] end]
      }
   }} msg opt]
   if {$code != 2} then {
%   \end{tcl}
%   All normal exits from the loop are by |return|, so anything with 
%   a code other than $2$ is abnormal.
%   \begin{tcl}
      lappend res "Abnormal termination.\n\  $path = $path\n  $opt"
   }
   return [join $res \n]
}
%   \end{tcl}
% \end{proc}
% 
% \begin{tcl}
namespace eval mtmtcl::present::decorate {}
% \end{tcl}
% \setnamespace{mtmtcl::present::decorate}
% 
% \begin{proc}{nodeheader}
%   This procedure constructs a message header giving details about a 
%   particular node in a tree, so that it is possible to identify the 
%   node that the following message is about. The call syntax is
%   \begin{displaysyntax}
%     nodeheader \word{path} \word{node}
%   \end{displaysyntax}
%   where \word{path} is the |lindex| path in the tree of the 
%   \word{node}.
%   
%   \begin{tcl}
proc mtmtcl::present::decorate::nodeheader {path node} {
   set res "node "
   set L {}
   foreach {two index} $path {lappend L $index}
   append res \{ [join $L] "\} (" [lrange $node 0 1]
   if {[lindex $node 0] ne "#text"} then {
      set len [llength [lindex $node 2]]
      if {$len == 1} then {
         append res " \{<1 child>\}"
      } else {
         append res " \{<$len children>\}"
      }
   }
   append res )
}
%   \end{tcl}
% \end{proc}
% 
% For the specialised tree-decorating commands in the 
% |mtmtcl::present::decorate| namespace, a very important concept is 
% that of the \emph{current node}, which by definition is the node 
% pointed to by the last path in the \describestring+[var.]{_paths} 
% variable. The current node starts out as the symbol which triggered 
% a decoration script, but commands can temporarily make some other 
% node the current node by appending its path to the |_paths| list; 
% typically such commands take a \word{body} script as argument, and 
% the current node is restored after that body is complete.
% 
% Changing the first path in the |_paths| list would have the effect 
% of changing the loop variable for the main loop in the 
% \describestring+[proc][mtmtcl::present]{decorate} procedure. This 
% should only be done if the corresponding node has moved in the 
% tree, which incidentally is what the next procedure will do.
% 
% 
% \begin{proc}{treeinsert}
%   This procedure inserts a node on a branch in the tree, making the 
%   node previously at the insert position a child of the inserted 
%   node. The call syntax is
%   \begin{displaysyntax}
%     treeinsert \word{path} \word{child-index} \word{node}
%   \end{displaysyntax}
%   where \word{path} is an |lindex| path into the tree stored in the 
%   \describestring+[var.]{_tree} variable in the calling context, 
%   and specifies the position in the tree where the \word{node} 
%   should be inserted. The \word{child-index} is the position among 
%   the children of the \word{node} into which the old node at 
%   position \word{path} in the tree should be inserted; children at 
%   this or higher indices will appear after the old node.
%   
%   This procedure also updates the \describestring+[var.]{_paths} 
%   variable in the calling context, so that every element in it 
%   which has \word{path} as a prefix is updated to point to the same 
%   node as before the insertion, but accounting for the new position.
%   \begin{tcl}
proc mtmtcl::present::decorate::treeinsert {path childindex node} {
   upvar 1 _tree tree _paths paths
   set old [lindex $tree $path]
   lset node 2 [linsert [lindex $node 2] $childindex $old]
   lset tree $path $node
   set end [expr {[llength $path] - 1}]
   set path [lrange $path 0 $end]
   set newpaths {}
   foreach p $paths {
      if {[lrange $p 0 $end] eq $path} then {
         lappend newpaths [linsert $p [llength $path] 2 $childindex]
      } else {
         lappend newpaths $p
      }
   }
   set paths $newpaths
}
%   \end{tcl}
% \end{proc}
% 
% \begin{proc}{OMATTR}
%   The |OMATTR| procedure inserts an |OMATTR| node as parent of the 
%   current node in the tree. The call syntax is
%   \begin{displaysyntax}
%     OMATTR \begin{regblock}[\regplus] \word{URI} \word{value} 
%     \end{regblock}
%   \end{displaysyntax}
%   where each \word{URI} is the URI for an attribution symbol and 
%   \word{value} is the data-tree of the corresponding value.
%   \begin{tcl}
proc mtmtcl::present::decorate::OMATTR {args} {
   upvar 1 _paths paths
   set L {}
   foreach {URI value} $args {
      regexp {^(?:(.*)/)?([^#]*)#(.*)$} $URI "" cdbase cd name
      set attr [dict create cd $cd name $name]
      if {$cdbase ne ""} then {dict set attr cdbase $cdbase}
      lappend L [list OMS $attr {}] $value
   }
   set node [list OMATTR {} [list [list OMATP {} $L]]]
   uplevel 1 [list [namespace which treeinsert]\
     [lindex $paths end] 1 $node]
}
%   \end{tcl}
% \end{proc}
% 
% 
% \begin{proc}{mtmtcl:path}
%   This procedure acts effectively as an alias for a |switch| on the 
%   value of the |mtmtcl:path| attribute of the current node. Like 
%   |switch|, it has the two syntaxes
%   \begin{displaysyntax}
%     mtmtcl:path \begin{regblock}[\regplus] \word{pattern} 
%     \word{body} \end{regblock}\par
%     mtmtcl:path \{ \begin{regblock}[\regplus] \word{pattern} 
%     \word{body} \end{regblock} \}
%   \end{displaysyntax}
%   and will evaluate the first \word{body} whose \word{pattern} 
%   matches the |mtmtcl:path|. The \word{pattern}s are regular 
%   expressions, and the string they are matched against is the 
%   result of joining the |mtmtcl:path| with newlines; the |(?w)| 
%   regexp mode then makes it feasible to constrain against list 
%   element boundaries using `|^|' and `|$|'.
%   
%   No \word{body} will be evaluated if the current node lacks 
%   |mtmtcl:path| attribute.
%   \begin{tcl}
proc mtmtcl::present::decorate::mtmtcl:path {args} {
   upvar 1 _tree tree _paths paths
   set attr [lindex [lindex $tree [lindex $paths end]] 1]
   if {[dict exists $attr mtmtcl:path]} then {
      uplevel 1 [list ::switch -regexp -- [
         join [dict get $attr mtmtcl:path] \n
      ]] $args
   }
}
%   \end{tcl}
% \end{proc}
% 
% \begin{proc}{attrdefault}
%   This procedure adds ``default'' values for specified keys in the 
%   attribute dictionary of the current node, but will not change the 
%   values of keys already present. The call syntax is
%   \begin{displaysyntax}
%     attrdefault \begin{regblock}[\regstar] \word{key} \word{value} 
%     \end{regblock}
%   \end{displaysyntax}
%   where the \word{key}s and \word{value}s define the new entries. 
%   There is no particular return value.
%   \begin{tcl}
proc mtmtcl::present::decorate::attrdefault {args} {
   upvar 1 _tree tree _paths paths
   set path [lindex $paths end]
   lappend path 1
   set attr [lindex $tree $path]
   set attr2 [dict merge $args $attr]
   if {[dict size $attr2] > [dict size $attr]} then {
      lset tree $path $attr2
   }
}
%   \end{tcl}
% \end{proc}
% 
% \begin{proc}{rootwards}
%   This procedure temporarily establishes a new current node further 
%   towards the root and evaluates the given \word{script} with that 
%   as the current node. The call syntax is
%   \begin{displaysyntax}
%     rootwards \begin{regblock}[\regstar] \word{directive} 
%     \word{pattern} \end{regblock} \word{script}
%   \end{displaysyntax}
%   where the \word{directive}s specify how far one should go and 
%   the \word{pattern}s constitute constraints on the type of the 
%   parent currently being considered and the child from which the 
%   procedure descended. The possible \word{directive}s are
%   \begin{description}
%     \item[-past]
%       If the \word{pattern} matches the node currently being 
%       considered, then don't stop here; instead continue to its 
%       parent.
%     \item[-not]
%       If the \word{pattern} matches the node currently being 
%       considered, then don't evaluate the \word{script} at all; 
%       the node we started from wasn't used in the context where it 
%       would be useful.
%     \item[-to]
%       We're done if the \word{pattern} matches the node currently 
%       being considered, otherwise keep descending.
%   \end{description}
%   Directives are considered left to right, and an action (try again 
%   with parent, evaluate \word{script} here, or finish without 
%   evaluating \word{script}) is taken as soon as one is found whose 
%   \word{pattern} matches the node currently under consideration. 
%   If no |-to| directive is given, then the descent continues as 
%   long as it goes via nodes matching some |-past| pattern. If a 
%   |-to| directive is given, then the search behaves as if there was 
%   an extra |-past| directive with a pattern matching anything last 
%   in the sequence.
%   
%   A \word{pattern} is a list of the form
%   \begin{displaysyntax}
%     \word{type-glob} \word{index}\regstar
%   \end{displaysyntax}
%   where the \word{type-glob} is a |string match| pattern applied 
%   against the type of the node currently being considered. The 
%   \word{index}es are integers which are compared to the child index 
%   of the previous node. The \word{pattern} matches if the 
%   \word{type-glob} matches and there either are no \word{index}es 
%   or at least one \word{index} equal to the child index.
%   
%   \begin{tcl}
proc mtmtcl::present::decorate::rootwards {args} {
   upvar 1 _tree tree _paths paths
   set path [lindex $paths end]
   if {[llength $args] % 2 != 1} then {
      return -code error\
        {wrong # args: rootwards ?dir pat ... ...? script}
   }
   set script [lindex $args end]
   set args [lreplace $args end end]
   set again 1
   while {$again && [llength $path]} {
      set again 0
      set index [lindex $path end]
      set path [lreplace $path end-1 end]
      set type [lindex [lindex $tree $path] 0]
      foreach {dir pat} $args {
         set match [string match [lindex $pat 0] $type]
         if {$match && [llength $pat]>1} then {
            set match 0
            foreach i [lrange $pat 1 end] {
               if {$i == $index} then {set match 1; break}
            }
         }
         switch -- $dir -to {
            if {$match} then {
               set again 0; break
            } else {
               set again 1
            }
         } -past {
            if {$match} then {set again 1}
         } -not {
            if {$match} then {return}
         } default {
            return -code error "unknown directive: $dir"
         }
      }
   }
   if {!$again} then {
      lappend paths $path
      uplevel 1 [list ::if 1 then $script]
      set paths [lreplace $paths [set paths end] end]
   }
}
%   \end{tcl}
% \end{proc}
% 
% 
% \begin{proc}{OMSTR}
% \begin{proc}{text}
%   The |OMSTR| and |text| procedures are helpers for forming |OMSTR| 
%   and |#text| nodes, respectively.
%   \begin{tcl}
proc mtmtcl::present::decorate::text {str} {list \#text $str}
proc mtmtcl::present::decorate::OMSTR {str} {
   list OMSTR {} [list [text $str]]
}
%   \end{tcl}
% \end{proc}\end{proc}
% 
% 
% ---
% 
% 
% 
% 
% \subsection{Fencing priorities}
% 
% Having considered expansion of |mfenced| nodes, it might be 
% appropriate to consider the process that puts them there in the 
% first place (or at least \emph{one} such process, as parentheses 
% are also part of some fixed notations): subexpression fencing for 
% overriding priorities.
% The background is well known: tree structures for data make 
% parentheses unnecessary, but traditional (linear) notation requires 
% parentheses. It is possible to insert parentheses when generating 
% linear notation, but this gets rather complicated and it is 
% generally better to have many simple operations than one complex. 
% What the fencing means is that each subexpression which requires a 
% surrounding parenthesis to be interpreted correctly will be made 
% the only child of an \describestring*+[node type]{mfenced} node, 
% taking the place of the original node. This |mfenced| will then be 
% turned into a parenthesis during the metaformula expansion phase.
% 
% The model provided here for deciding when to fence is that every 
% node receives information about its surroundings, should use this 
% to decide whether to fence, and must then pass on updated 
% information to its children so that they can decide whether to 
% fence. The information is split in three parts: left priority, 
% right priority, and blockages. Most issues are resolved based on 
% the priorities, but some (e.g.~prevention of multiple superscripts) 
% are handled by the blockages: certain types of notation may be 
% blocked by notation further out in the expression, so fencing is 
% needed in order to reenable it. The two priorities are numerical, 
% whereas the format of the blockages is yet to be decided.\footnote{
%   At first I imagined the blockages primarily as a set 
%   (implemented as the keys in a dictionary), but this didn't fit 
%   with the expected behaviour, so currently (2008-11-15) I'm more 
%   leaning towards a finite automaton for this. Even determining 
%   a suitable set of states for that is however a project in its 
%   own right.
%   
%   At a formal level, the problem is to map an ideal in the free 
%   monoid: those combinations of affixings that are not correctly 
%   interpreted. Practically this would have to be done by generating 
%   longer and longer words not known to be in the ideal, and then 
%   have a human (me) decide whether they are in or out. A suitable 
%   assistance in this task would be a program that generates \LaTeX\ 
%   code for a sample of formulae with the affixment combination 
%   under consideration, the typeset form of which could then be 
%   viewed to decide if the combination is readable. Once a decision 
%   has been reached, it would be recorded (with a comment) in the  
%   database of the generator program, after which the process is 
%   repeated with the next generation of word, until one is content.
%   
%   The automaton would then be constructed as the smallest automaton 
%   which gets all the mapped words right. See M\r{a}nsson for an 
%   algorithm.
% }
% 
% The preliminary approach used here is that the blockages can take 
% three values: \describestring+[blockage]{none}, 
% \describestring+[blockage]{exponent}, and 
% \describestring+[blockage]{all}. Superscripting operations move 
% from |none| to |exponent|, whereas subscripting operations move 
% from |none| or |exponent| to |all|. Every other combination calls 
% for a fencing, which restores |none|. This model only supports 
% exponents and indices, so it leaves out affixes above ($\hat{f}$, 
% $\overline{a+b}$), below ($\underbrace{a+b}_n$), and to the left 
% (${}_1F_2$) of a nucleus, but these still suffice rather far.
% 
% A priority is in general a floating-point number (to provide for 
% priority levels between two established priority levels), and the 
% negative infinity |-Inf| is used for ``no priority''. Besides that, 
% the following priority levels seem useful:
% \begin{enumerate}
%   \item[$-1$] Logical connectives (if the need ever arises).
%   \item[$0$] Relations.
%   \item[$1$] Addition and subtraction.
%   \item[$2$] 
%     Other types of multiplication and the like 
%     (seen by both factors in $a \times b$, $a \circ b$, or 
%     $a \cup b$). If you consider there to be a priority 
%     difference between two operations in this class, then use 
%     decimals to distinguish them.
%   \item[$3$]
%     Negation ($-a \sqcap b$ is probably $(-a) \sqcap b$) and 
%     ``math operators'' such as $\sin$. A case could however be made 
%     for putting these at $2$ as well, since this could eliminate 
%     ambiguity.
%   \item[$4$]
%     Juxtaposing multiplication (seen by both factors in $ab$).
%   \item[$5$]
%     Power (seen by $a$ in $a^b$; the exponent sees $-\infty$), 
%     function application (seen by $a$ in $a(b)$), and the factorial.
%   \item[$6$]
%     Indexing (seen by $a$ in $a_b$).
% \end{enumerate}
% In other words, the priority levels are nothing like what have been 
% suggested for MathML\,3; I don't expect there to be an easy 
% translation between the two.
% 
% Important to notice about the priorities is also that there are 
% \emph{two} of them: the left (earlier in reading order) priority 
% and the right (later in reading order) priority. The left priority 
% represents how strongly the material on the left of this 
% subexpression wants to bind to it, and the right priority 
% represents the same for the material on the right of this 
% subexpression. They are often different, simply because the 
% operation on the left side of a subexpression is not the same as 
% the operation on the right side of it. The reason it is 
% \emph{useful} to keep track of both (as opposed to keeping track of 
% just the larger of the two) is that left priority doesn't matter 
% for prefix notations and right priority doesn't matter for 
% postfix notations; priority isn't needed to parse a nesting of 
% prefix notations like $-\sin \cos x$ or postfix notations like 
% $n^2!^2$ since there is only one way that the symbols can bind! 
% Priority only comes into play when pre- and postfix notations 
% compete for a subexpression, as in $-n!$. 
% 
% Also important is that there is a difference between operation and 
% notation here. Postfix \emph{operations} are rare in standard 
% mathematical notation (the factorial being the most obvious 
% example), but postfix \emph{notation}\Ldash where something is put 
% to the right of an expression nucleus as a way of making an 
% operation on that nucleus\Dash is pretty common. The most important 
% (but also quite surpring) example is function application, where the 
% function itself is the nucleus and the parenthesisied argument(s) 
% is the postfix part of the notation. Other examples of postfix 
% notations are exponents and indices; that these really are postfix 
% can be seen from the fact that their mixing with function application 
% and the factorial is unambiguous: $f_3(n)^2!$ may conceptually be the 
% composition of three functions ($f_3$, squaring, and the factorial) 
% applied to $n$, but notationally it is rather index $3$, evaluated 
% in $n$, squaring, and the factorial applied to $f$.
% 
% Examples to illustrate the priorities:
% \begin{enumerate}
%   \item
%     The relative priorities of addition, juxtaposing multiplication, 
%     negation, and other operations have plenty of examples\Dash 
%     consider \(-ab \otimes c + d \otimes ef\).
%   \item
%     Juxtaposing multiplication versus higher priority operations is 
%     also fairly common: \(ab^c = a(b^c)\), not $(ab)^c$. \(a b(c) = 
%     a \cdot b(c)\), not $(ab)(c)$ (although one is often defined to 
%     be the other). \(2n! = 2(n!)\), not \((2n)!\).
%   \item
%     The relative priorities of power and indexing reflects the 
%     fact that $x_1^2$ is considered to be $(x_1)^2$, not $(x^2)_1$. 
%     (These two can be very different when the index selects an 
%     element in a matrix; compare $(A_{1,1})^2$ and $(A^2)_{1,1}$.)
%     In practice that doesn't make much difference however, as both 
%     are postfix notations for the base and therefore not taking 
%     note of the right priority.
% \end{enumerate}
% 
% ---
% 
% \begin{remark}
%   Ruling by priority makes sense for infix operations, but what 
%   if the operations are rather post- or prefix? Some examples:
%   \begin{enumerate}
%     \item
%       $f(x)$, $f(x)^2$, $f^2(x)$. Here it seems the relative 
%       priorities of power and function application are 
%       irrelevant; it is clear which comes first since they are 
%       both effectively postfix.
%     \item
%       $(e^x)^2$. Here the parenthesis is mandatory, despite both 
%       expressions being postfix of the same priority.
%     \item
%       $\sin(a+b)$ versus $\sin ab$. This seems to suggest that 
%       the priority of $\sin$ and friends is strictly less 
%       than~$3$, and greater than~$1$.
%     \item
%       $n!m!$ versus $n!\,m!$; $2f(x)$ versus $2\,f(x)$. There are 
%       cases in which juxtaposition should not be tight.
%       
%       One way to achieve this could be to introduce |mpadded| 
%       elements and have |juxtaposed| formatters take note of 
%       this, just like |sumnode| takes note of |negate|. But that 
%       can wait.
%     \item
%       Does $fg$ (|multiply|) need to be fenced in $(fg)(x)$? Some 
%       say no, which would indicate function application has 
%       \emph{lower} priority than multiplication. 
%   \end{enumerate}
%   The whole thing seems rather messy, even when one does not 
%   consider the arguably special cases like $\sin^2 x$.
%   
%   One conclusion \emph{could} be that there are really two kinds 
%   of juxtaposition: thight, and with thin space. The tight kind 
%   has higher priority than the thinspace kind, and the tight kind 
%   is the default, but some types of factor causes the notation to 
%   switch. Finding a reliable model for it is however likely to be 
%   difficult.
% \end{remark}
% 
% 
% ---
% 
% \setnamespace{mtmtcl::present::fence::tool}
% 
% \begin{proc}{transparent}
%   This procedure is a helper for the |fence| ensemble. It has 
%   the call syntax
%   \begin{displaysyntax}
%     tool::transparent \word{arity} \word{index} \word{type}
%     \word{attributes} \word{children} \word{left priority} 
%     \word{right priority} \word{blockages}
%   \end{displaysyntax}
%   It directly returns the `\word{type} \word{attributes} 
%   \word{children}' unless the length of \word{children} is exactly 
%   \word{arity}. If \word{children} has the requested length, then 
%   child \word{index} is processed recursively, and the result of 
%   this replaces the original child in the result. If that child 
%   comes back being |mfenced| then this fencing is moved outside 
%   the `\word{type} \word{attributes} \word{children}' node.
%   
%   The idea is to use this as a handler for nodes that are either 
%   atomic or have as only child an expression detailing their 
%   internal structure. 
%   \begin{tcl}
namespace eval mtmtcl::present::fence::tool {
   proc transparent {length index type attr children l r b} {
      if {[llength $children] != $length} then {
         return [list $type $attr $children]
      }
      set nucleus [fence {*}[lindex $children $index] $l $r $b]
      if {[lindex $nucleus 0] eq "mfenced"} then {
         lreplace $nucleus 2 2 [list [
            list $type $attr [
               lreplace $children $index $index\
                 [lindex $nucleus 2 0]
            ]
         ]]
      } else {
         list $type $attr [
            lreplace $children $index $index $nucleus
         ]
      }
   }
}
%   \end{tcl}
% \end{proc}
% 
% \begin{ensemble}{fence}
%   \setnamespace{mtmtcl::present::fence}
%   This ensemble implements the operation of inserting parentheses, 
%   encoded as |mfenced| elements with one children, into an 
%   expression where appropriate. The call syntax is
%   \begin{displaysyntax}
%     fence \meta{tree} \word{left priority} \word{right priority} 
%     \word{blockages}
%   \end{displaysyntax}
%   and the return value is the fenced version of the \word{tree}.
%   
%   The ensemble namespace is |mtmtcl::present::fence|. All commands 
%   in it are handlers for specific node types, whereas the ensemble 
%   command itself lives in the |tool| child of this namespace. 
%   Unknown node types are handled by |tool::transparent| above, in 
%   length $1$ child $0$ mode.
%   \begin{tcl}
namespace eval mtmtcl::present::fence {
   namespace export *
   namespace ensemble create -command tool::fence -prefix 0 -unknown [
      list ::apply {{cmd ensemble type args} {
         list $cmd 1 0 $type
      } ::} [namespace which tool::transparent]
   ]
}
%   \end{tcl}
%   
%   Some good examples of nodes which are fully taken care of by this 
%   |-unknown| handler are:
%   \begin{description}
%     \renewcommand{\makelabel}[1]{\hspace\labelsep
%        \texttt{#1}^^A
%        \IndexEntry{\LevelSorted{#1}{\texttt{#1} (node type)}}^^A
%          {none}{\thepage}^^A
%     }
%     \item[OMF]
%       While the presentation of a float may be subject to 
%       interpretation under the guidance of priority, and therefore 
%       need fencing in order to preserve the context, the expression 
%       from which this presentation will be generated should have 
%       been constructed during typograph preprocessing and thus 
%       appear as the only child of the |OMF| node.
%     \item[integer]
%       Similar to |OMF| nodes, although the expressions are 
%       typically simpler.
%     \item[OMS]
%       Symbols are usually atomic, but if they have a composite 
%       structure then the fencing operation should continues into 
%       the only child.
%     \item[OMV]
%       Ditto.
%     \item[OMR]
%       The usefulness of references here could be disputed, but the 
%       natural way to handle them would be to include a copy of the 
%       referenced node as only child.
%     \item[OMSTR]
%       Strings (and bytearrays) are presumed to don't need any 
%       fencing because of priority considerations, so they should 
%       just be returned unchanged. This could lead to processing of 
%       a |#text| node, but these need a separate definition anyway.
%   \end{description}
%   
%   \begin{ensproc}{#text}
%     As usual, |#text| nodes require a separate handler, even though 
%     nothing is ever done for these.
%     \begin{tcl}
proc mtmtcl::present::fence::#text {str l r b} {list #text $str}
%     \end{tcl}
%   \end{ensproc}
%   
%   
%   \begin{ensproc}{OMATTR}
%     Attributions are as a rule invisible when rendered, so 
%     recursion only continues down the second child. If that child 
%     comes back being fenced, then that fence is moved outside the 
%     |OMATTR| node, since a presentation attribution typically 
%     describes the symbol when the context has lower priority.
%     \begin{tcl}
interp alias {} mtmtcl::present::fence::OMATTR  {}\
  [namespace which mtmtcl::present::fence::tool::transparent] 2 1\
  OMATTR
%     \end{tcl}
%   \end{ensproc}
%   
%   \begin{ensproc}{mn}
%     Numbers are special, in that they can survive arbitrarily high 
%     priorities on the right but not juxtaposition on the left ($x2$ 
%     is typically interpreted as a mistyped $x^2$, $x_2$, or 
%     $\mathit{x2}$ rather than as $x \cdot 2$). Therefore the 
%     |priority| is only compared against the left priority.
%     \begin{tcl}
proc mtmtcl::present::fence::mn {attr children l r b} {
   set p [expr {[dict exists $attr priority] ?\
     [dict get $attr priority] : 3.5}]
   set res [list mn $attr $children]
   if {$l >= $p} then {
      set res [list mfenced {} [list $res]]
   }
   return $res
}
%     \end{tcl}
%     The default priority here is $3.5$, since that is less than 
%     what is common for juxtapositions and larger than what is 
%     common for prefix operations, but it is also the priority level 
%     which is most hardwired into the system, so $3.5$ is a rather 
%     strange number. Perhaps the priority levels should be 
%     renumbered?
%   \end{ensproc}
%   
%   \begin{ensproc}{negate}
%     Soft negations are transparent for fencing, as they are 
%     logically extensions of their parents. Unlike 
%     |tool::transparent| nodes however, they shouldn't let fences 
%     of their operand migrate outwards.
%     \begin{tcl}
proc mtmtcl::present::fence::negate {attr children l r b} {
   list negate $attr [lreplace $children 1 1 [
      tool::fence {*}[lindex $children 1] $l $r $b
   ]]
}
%     \end{tcl}
%   \end{ensproc}
%   
%   The main player in fencing by priority is instead the |OMA| 
%   node type, and a lesser extent |OMBIND|, but these are just 
%   further dispatchers, which redirect to more specialised node 
%   types based on the \describestring+[attribute]{form} attribute. 
%   Intuitively this specifies the kind of construction used to 
%   express this application, but technically it need only describe 
%   how the construction behaves with respect to priorities, since 
%   attributions may override the actual formatting.
%   
%   The basic |form| values are:
%   \begin{description}
%     \renewcommand{\makelabel}[1]{^^A
%        \hspace\labelsep
%        \describestring*+[\texttt{form} attribute value]{#1}^^A
%     }
%     \item[infix]
%       Infix operation (between every two neighbouring arguments, 
%       there is a copy of the operation). Example: $a \oplus b$.
%     \item[prefix]
%       Operation appears before the argument. There should be 
%       exactly two children: operation and argument. 
%       Example: $-a$.
%     \item[postfix]
%       Operation appears after the argument. There should be 
%       exactly two children: operation and argument.
%       Example: $a!$.
%     \item[function]
%       Operation appaears once before the arguments, which are in 
%       a parenthesis. Example: $f(a,b)$. This is the default for 
%       |OMA|.
%     \item[enclosed]
%       Whatever the operation is, it completely separates all 
%       children from their surroundings.
%     \item[juxtaposed]
%       The operation does not appear at all. The operands are set 
%       side to side.
%       (In MathML, the preferred presentation of this is as an 
%       |infix| operation with invisible operation, but it is 
%       convenient to have it as a separate |form|.)
%     \item[binding]
%       This is the default for |OMBIND|, and corresponds to a 
%       presentation such as $\forall x,y \mathpunct{.} z$ for
%       \begin{quote}
%         \small
%         |<OMBIND> <OMS cd="quant1" name="forall"/>|\\
%         |  <OMBVAR> <OMV name="x"/> <OMV name="y"/> </OMBVAR>|\\
%         |  <OMV name="z"/>|\\
%         |</OMBIND>|
%       \end{quote}
%     \item[exponent]
%       Operation expressed by superscripting the second operand on 
%       the first.
%     \item[index]
%       Basic index affixer, presently restricted to the subscript 
%       position. The first operand is the nucleus, remaining 
%       operands are indices
%   \end{description}
%   
%   
%   \begin{ensproc}{OMA}
%   \begin{ensproc}{OMBIND}
%     Redispatching goes through the auxilliary ensemble 
%     |tool::form|, since one could otherwise cause infinite loops by 
%     specifying |OMA| as |form|. The original attribute dictionary 
%     is preserved as the value of an |OMA| or |OMBIND| attribute, so 
%     that the original node type can be restored afterwards.
%     \begin{tcl}
proc mtmtcl::present::fence::OMA {attr children left right block} {
   if {[dict exists $attr form]} then {
      set form [dict get $attr form]
   } else {
      set form function
   }
   tool::form $form [dict replace $attr OMA $attr] $children\
     $left $right $block
}
proc mtmtcl::present::fence::OMBIND {attr children left right block} {
   if {[dict exists $attr form]} then {
      set form [dict get $attr form]
   } else {
      set form binding
   }
   tool::form $form [dict replace $attr OMBIND $attr] $children\
     $left $right $block
}
%     \end{tcl}
%   \end{ensproc}
%   \end{ensproc}
%   
%   \begin{ensproc}{OMBVAR}
%     The appropriate treatment of |OMBVAR| nodes depends heavily on 
%     what the surrounding |OMBIND| comes out as, but usually one 
%     would put commas between the variables, so a reasonable default 
%     is to let the first child see the left priority and the last 
%     child see the right priority and blockages.
%     \begin{tcl}
proc mtmtcl::present::fence::OMBVAR {attr children left right block} {
   set new {}
   foreach child $children {
      lappend new [tool::fence {*}$child [
         expr {[llength $new] ? -Inf : $left}
      ] {*}[if {[llength $children] - [llength $new] >= 1} then {
         list -Inf none
      } else {
         list $right $block
      }]]
   }
   list OMBVAR $attr $new
}
%     \end{tcl}
%   \end{ensproc}
% \end{ensemble}
% 
% \begin{proc}{reform}
%   The |reform| procedure does an |OMA|\slash |OMBIND|-normalisation 
%   of a node dispatched through the |form| ensemble. The call syntax 
%   is
%   \begin{displaysyntax}
%     tool::reform \word{fallback type} \word{attributes} 
%     \word{children} \word{fence}
%   \end{displaysyntax}
%   where \word{fallback type} is the node type to use if the 
%   \word{attributes} has neither |OMA| nor |OMBIND| entry. If 
%   \word{fence} is true then the resulting node is additionally 
%   wrapped up in an |mfenced|.
%   \begin{tcl}
proc mtmtcl::present::fence::tool::reform {type attr children fence} {
   if {[dict exists $attr OMA]} then {
      set node [list OMA [dict get $attr OMA] $children]
   } elseif {[dict exists $attr OMBIND]} then {
      set node [list OMBIND [dict get $attr OMBIND] $children]
   } else {
      set node [list $type $attr $children]
   }
   if {$fence} then {
      return [list mfenced {} [list $node]]
   } else {
      return $node
   }
}
%   \end{tcl}
% \end{proc}
% 
% 
% \begin{ensemble}{form}
%   \setnamespace{mtmtcl::present::fence}
%   The |tool::form| ensemble has the same namespace and overall 
%   syntax as |tool::fence|, but unlike the latter it has an explicit 
%   set of subcommands.
%   \begin{tcl}
namespace eval mtmtcl::present::fence {
   namespace ensemble create -command tool::form -subcommands {
      infix prefix postfix function enclosed juxtaposed binding 
      exponent index
   } -prefix 0
}
%   \end{tcl}
%   
%   
%   \begin{ensproc}{infix}
%     In the general case (two or more operands), an |infix| 
%     operation has to overcome priorities both left and right, 
%     whereas in the degenerate cases there is a fencing just for the 
%     sake of it. The blockage is passed on to the rightmost operand 
%     (but normally anything causing blockage has so high priority 
%     that it forces fencing anyway).
%     \begin{tcl}
proc mtmtcl::present::fence::infix {attr children left right block} {
   if {[dict exists $attr priority]} then {
      set priority [dict get $attr priority]
   } else {
      set priority 2
   }
   if {$left>=$priority || $priority<=$right} then {
      set fence 1
      set left [set right -Inf]
      set block none
   } else {
      set fence 0
   }
%     \end{tcl}
%     For completeness, the operation child is processed even though 
%     it will usually be atomic.
%     \begin{tcl}
   set new {}
   foreach child $children {
      if {[llength $new] != 1} then {
         set l $priority
      } else {
         set l $left
      }
      if {[llength $children]-[llength $new] == 1} then {
         set r $right
         set b $block
      } else {
         set r $priority
         set b none
      }
      lappend new [tool::fence {*}$child $l $r $b]
   }
   return [tool::reform infix $attr $new $fence]
}
%     \end{tcl}
%   \end{ensproc}
%   
%   \begin{ensproc}{function}
%     Functions have a built-in fencing of the arguments so those 
%     experience $-\infty$ at both sides and no blockages, but the 
%     function itself is another matter. From its point of view, the 
%     arguments behave like a postfix operation of priority $5$.
%     \begin{tcl}
proc mtmtcl::present::fence::function {attr children left right block} {
   set new [list [
      tool::fence {*}[lindex $children 0]\
        [expr {$left<5 ? $left : -Inf}] 5 none
   ]]
   foreach child [lrange $children 1 end] {
      lappend new [tool::fence {*}$child -Inf -Inf none]
   }
   return [tool::reform function $attr $new [expr {$left>=5}]]
}
%     \end{tcl}
%   \end{ensproc}
%   
%   
%   \begin{ensproc}{prefix}
%     Prefix operation nodes should have exactly two children: the 
%     operation and the operand; if they don't then we default to 
%     |function| notation. The |priority| has to overcome the right 
%     priority, but not the left. The priorities seen by the children 
%     are the same as for an infix operation of this priority.
%     \begin{tcl}
proc mtmtcl::present::fence::prefix {attr children left right block} {
   if {[llength $children] != 2} then {
      return [function $attr $children $left $right $block]
   }
   if {[dict exists $attr priority]} then {
      set priority [dict get $attr priority]
   } else {
      set priority 3
   }
   if {$priority <= $right} then {
      set fence 1
      set left [set right -Inf]
      set block none
   } else {
      set fence 0
   }
   set new [list [
      tool::fence {*}[lindex $children 0] $left $priority none
   ] [
      tool::fence {*}[lindex $children 1] $priority $right $block
   ]]
   return [tool::reform prefix $attr $new $fence]
}
%     \end{tcl}
%   \end{ensproc}
%   
%   \begin{ensproc}{postfix}
%     Postfix operation nodes should have exactly two children: the 
%     operation and the operand, in that order; if they don't then 
%     we default to |function| notation. Since the operand is to 
%     appear before the operation, it sees the left priority and 
%     operation priority (which defaults to $5$), whereas the 
%     operation sees the operation priority, the right priority, and 
%     the blockages.
%     \begin{tcl}
proc mtmtcl::present::fence::postfix {attr children left right block} {
   if {[llength $children] != 2} then {
      return [function $attr $children $left $right $block]
   }
   if {[dict exists $attr priority]} then {
      set priority [dict get $attr priority]
   } else {
      set priority 5
   }
   if {$priority <= $left} then {
      set fence 1
      set left [set right -Inf]
      set block none
   } else {
      set fence 0
   }
   set new [list [
      tool::fence {*}[lindex $child 0] $priority $right $block
   ] [
      tool::fence {*}[lindex $child 1] $left $priority none
   ]]
   return [tool::reform postfix $attr $new $fence]
}
%     \end{tcl}
%   \end{ensproc}
%   
%   \begin{ensproc}{enclosed}
%     Enclosed notation completely separates each operand from its 
%     surroundings. The operation child is processed as well, just 
%     for the sake of it.
%     \begin{tcl}
proc mtmtcl::present::fence::enclosed {attr children left right block} {
   set new {}
   foreach child $children {
      lappend new [tool::fence {*}$child -Inf -Inf none]
   }
   return [tool::reform enclosed $attr $new 0]
}
%     \end{tcl}
%   \end{ensproc}
%   
%   \begin{ensproc}{juxtaposed}
%     Juxtaposition is very similar to an infix operation, but the 
%     default priority is higher.
%     \begin{tcl}
proc mtmtcl::present::fence::juxtaposed {attr children left right block} {
   if {[dict exists $attr priority]} then {
      set priority [dict get $attr priority]
   } else {
      set priority 4
   }
   if {$left>=$priority || $priority<=$right} then {
      set fence 1
      set left [set right -Inf]
      set block none
   } else {
      set fence 0
   }
%     \end{tcl}
%     For completeness, the operation child is processed even though 
%     it will usually not be seen in the presentation.
%     \begin{tcl}
   set new {}
   set b none
   foreach child $children {
      if {[llength $children]-[llength $new] <= 1} then {
         set l $priority
         set r $right
         set b $block
      } elseif {[llength $new]!=1} then {
         set l [set r $priority]
      } else {
         set l $left
         set r $priority
      }
      lappend new [tool::fence {*}$child $l $r $b]
   }
   return [tool::reform infix $attr $new $fence]
}
%     \end{tcl}
%   \end{ensproc}
%   
%   \begin{ensproc}{exponent}
%     Exponentiating nodes should have three children: the operation 
%     symbol, the base operand, and the superscript operand. The 
%     superscript operand doesn't feel any surroundings ($-\infty$ 
%     priorities, no blockages), but the base operand behaves very 
%     much as the operand of a postfix operation. In addition, it 
%     requires the blockages to be |none|.
%     \begin{tcl}
proc mtmtcl::present::fence::exponent {attr children l r b} {
   if {[llength $children] != 3} then {
      return [function $attr $children $l $r $b]
   }
   if {[dict exists $attr priority]} then {
      set p [dict get $attr priority]
   } else {
      set p 5
   }
   if {$b ne "none" || $l>=$p} then {
      set l [set r -Inf]
      set b none
      set fence 1
   } else {
      set fence 0
   }
   set new [lrange $children 0 0]
   lappend new [tool::fence {*}[lindex $children 1] $l $p exponent]
   lappend new [tool::fence {*}[lindex $children 2] -Inf -Inf none]
   tool::reform exponent $attr $new $fence
}
%     \end{tcl}
%   \end{ensproc}
%   
%   \begin{ensproc}{index}
%     Indexing nodes should have three or more children: the 
%     operation symbol, the base operand, and one or more subscript 
%     operands. The subscript operands don't feel any surroundings 
%     ($-\infty$ priorities, no blockages), but the base operand 
%     behaves very much as the operand of a postfix operation. In 
%     addition, it requires the blockages to be |none| or |exponent| 
%     (i.e., not |all|).
%     \begin{tcl}
proc mtmtcl::present::fence::index {attr children l r b} {
   if {[llength $children] < 3} then {
      return [function $attr $children $l $r $b]
   }
   if {[dict exists $attr priority]} then {
      set p [dict get $attr priority]
   } else {
      set p 6
   }
   if {$b eq "all" || $l>=$p} then {
      set l [set r -Inf]
      set b none
      set fence 1
   } else {
      set fence 0
   }
   set new [lrange $children 0 0]
   lappend new [tool::fence {*}[lindex $children 1] $l $p exponent]
   foreach index [lrange $children 2 end] {
      lappend new [tool::fence {*}$index -Inf -Inf none]
   }
   tool::reform index $attr $new $fence
}
%     \end{tcl}
%   \end{ensproc}
%   
%   \begin{ensproc}{binding}
%     The default |binding| notation has four parts as far as the 
%     |OMBIND| is concerned: binding symbol (child $0$), variables 
%     (child $1$), punctuation (not a child), and nucleus (child $2$). 
%     Visually these decompose as $3+1$, so it is probably best to 
%     treat it analogously to a |prefix| operation.
%     \begin{tcl}
proc mtmtcl::present::fence::binding {attr children left right block} {
   if {[llength $children] != 3} then {
      return [function $attr $children $left $right $block]
   }
   if {[dict exists $attr priority]} then {
      set p [dict get $attr priority]
   } else {
      set p 3
   }
   if {$right>=$p} then {
      set left [set right -Inf]
      set block none
      set fence 1
   } else {
      set fence 0
   }
   set new {}
   lappend new [tool::fence {*}[lindex $children 0] $left $p none]
   lappend new [tool::fence {*}[lindex $children 1] $p -Inf none]
   lappend new [tool::fence {*}[lindex $children 2] -Inf $right $block]
   tool::reform binding $attr $new $fence
}
%     \end{tcl}
%   \end{ensproc}
% \end{ensemble}
% 
% 
% \section{Conversion to \LaTeX}
% 
% 
% \subsection{Folding \TeX\ formulae}
% 
% \begin{tcl}
namespace eval mtmtcl::latex {}
% \end{tcl}
% \setnamespace{mtmtcl::latex}
% 
% From the other side of things, the final step of generating code 
% for \AllTeX\ math formulae is to fold the token sequence to fit 
% within a sensible linewidth. This is not entirely trivial, as 
% practical use of formulae within actual papers (as opposed to 
% throwaway documents for rendering a formula in isolation) usually 
% requires that the folding respects the mathematical structure of 
% the original expression\Dash that newlines divide the formula in 
% logical chunks, and that indentation corresponds to logical nesting 
% depth. To that end, the input format for the formula folding should 
% be a data-tree with \AllTeX\ code in the leaves and a nesting 
% structure corresponding to the logical structure.
% 
% The node type used to group subexpressions while allowing 
% linebreaks between them is \describestring+[node type]{mrow}. Care 
% is taken to ensure that tokens from separate children are kept 
% separate, so that |mrow {} {{#text {\Delta}} {#text x}}| never 
% becomes |\Deltax| but rather |\Delta x|. The default way to achieve 
% this is to separate children with whitespace (which is never 
% allowed to grow so large that it gets parsed as a |\par|), but 
% an |mrow| node with a boolean true 
% \describestring+[attribute]{tight} attribute will also attempt an 
% alternative setting in which the entire branch is written without 
% separating whitespace (unless at places where that would violate 
% the token restiction).
% 
% Unknown node types in this data-tree only recurse over their 
% children, so the default node type is called 
% \describestring+[node type]{children}. This primarily means unknown 
% node types without children are ignored, so they can be used to 
% include metadata about the formula (e.g.~list packages it makes use 
% of). Secondarily this means unknown container types cannot hide 
% leaves in the tree. Just recursing over the children does however 
% have the incidental effect that linebreaks are not permitted between 
% subexpressions but might still be permitted in them, so the 
% |children| node type can in principle be used to join a prefix or 
% suffix with their base expression.
% 
% \begin{proc}{fold}
%   The |fold| ensemble implements a basic form of code folding, 
%   satisfying two conditions:
%   \begin{enumerate}
%     \item 
%       When a newline is inserted between two children, it should be 
%       followed by indentation corresponding to the nesting depth of 
%       these children.
%     \item
%       No node should appear on a line with more indentation that 
%       the nesting depth of that node.
%   \end{enumerate}
%   One way to fulfill these conditions is to put each leaf on a 
%   separate line and indent it according to how deeply it is nested, 
%   but that doesn't look very good. It is also the longest possible 
%   folding of a piece of code, whereas most human authors would 
%   rather seek a short encoding, so a reasonable approach would be 
%   to seek the shortest folding which respects these conditions. As 
%   it turns out, this is a special case of the typographic 
%   linebreaking problem, so it may be solved through the same 
%   algorithm as is used in \TeX\ (although we don't need all the 
%   bells and whistles).
%   
%   The actual implementation follows a functional style, passing a 
%   current state from node to node. This state is the list of 
%   \emph{active breakpoints}, where each breakpoint in turn is a 
%   list with the structure
%   \begin{displaysyntax}
%     \word{level} \word{demerits} \word{tight-level} 
%     \word{pre-line}\regstar\ \word{post-line}
%   \end{displaysyntax}
%   The \word{level} is the nesting level of the breakpoint. 
%   The \word{demerits} is the quantity to minimise, i.e., the total 
%   number of indentation steps made on the best path of linebreaks 
%   leading up to this breakpoint. This 
%   path consists of the \word{pre-line}s, which are the actual lines 
%   of text (with indentation) one gets by following this line. The 
%   \word{post-line} is the line of text (again with indentation) 
%   which comes \emph{after} the breakpoint but before the current 
%   position in the tree. When a new breakpoint is created by 
%   appending a line to the path of a previous breakpoint, the old
%   \word{post-line} becomes the last \word{pre-line} and a new 
%   \word{post-line} consisting only of indentation is appended. 
%   The \word{demerits} is incremented by the level of this 
%   indentation, plus one for the linebreak.
%   
%   The \word{tight-level} signals whether |#text| nodes should have 
%   tight or default formatting; if it is less than the current level, 
%   then tight formatting should be attempted. It starts out at 
%   |infinity|, and should be reset to this when the current level 
%   drops below the \word{tight-level} (since this means we've left 
%   the |mrow| that triggered tight formatting).
%   
%   The |fold| command is a conveniency wrapper around the ensemble, 
%   which handles creating and extracting a result from the state 
%   being passed around. It has the call syntax
%   \begin{displaysyntax}
%     fold \word{data-tree} \word{width} \word{level}
%   \end{displaysyntax}
%   \begin{tcl}
proc mtmtcl::latex::fold {tree width level} {
   set best {x infinity}
   foreach bkp [
      fold:: {*}$tree [list [list 0 0 infinity {}]] $level $width
   ] {
      if {[lindex $bkp 1] <= [lindex $best 1]} then {set best $bkp}
   }
   return [join [lrange $best 3 end] \n]
}
%   \end{tcl}
% \end{proc}
%   
% 
% \begin{ensemble}[mtmtcl::latex]{fold::}[fold]
%   The call syntax for nodes in this ensemble is
%   \begin{displaysyntax}
%     "" \meta{data-tree} \word{state} \word{level} \word{width}
%   \end{displaysyntax}
%   and it returns an updated \word{state}.
%   The actual ensemble is the empty string command in the |fold| 
%   namespace, pointed to by an alias that has the same name as the 
%   namespace.
%   \begin{tcl}
namespace eval mtmtcl::latex::fold {
   namespace ensemble create -command [namespace current]::\
     -prefix 0 -unknown [list ::apply\
       {{target args} {list $target} ::} [namespace current]::children]
   namespace export ?*
}
%   \end{tcl}
%   
%   \begin{ensproc}{#text}
%     Text nodes update the active breakpoints by appending their 
%     \word{string} and a space to their \word{post-line}. They also 
%     drop breakpoints which do not fit within the given line 
%     \word{width} afterwards, but if that would result in 
%     dropping all breakpoints then those are kept which attain the 
%     minimum \word{post-line} width.
%     
%     \begin{tcl}
proc mtmtcl::latex::fold::#text {str bkpL level width} {
   set sensitive [regexp {^[A-Za-z]} $str]
   set res {}
   set overlong {}
   foreach bkp $bkpL {
      set line [lindex $bkp end]
%     \end{tcl}
%     The main complication in this procedure is the handling of 
%     |tight| formatting, since this has the effect that a space is 
%     not added after the \word{string}. A space will instead be added 
%     before the \word{string} if the latter is sensitive (begins 
%     with a letter) and the previous material on the line ends with 
%     a letter-name control sequence.
%     \begin{tcl}
      if {$level > [lindex $bkp 2]} then {
         if {$sensitive &&\
           [regexp {(^|[^\\])(\\\\)*\\[A-Za-z]*$} $line]} then {
            append line { }
         }
         append line $str
      } else {
         append line $str { }
      }
      lset bkp end $line
      if {[string length $line]<=$width} then {
         lappend res $bkp
      } else {
         lappend overlong $bkp
      }
   }
   if {[llength $res]} then {return $res}
%     \end{tcl}
%     The second loop is entered when an overfull line is 
%     unavoidable, and chooses those that are least overfull.
%     \begin{tcl}
   set minlen infinity
   foreach bkp $overlong {
      set line [lindex $bkp end]
      if {[string length $line] < $minlen} then {
         set res {}
         set minlen [string length $line]
      }
      if {[string length $line] <= $minlen} then {lappend res $bkp}
   }
   return $res
}
%     \end{tcl}
%     It is assumed that no \word{string} ends with the initial 
%     backslash of a control sequence, since the results of that 
%     would be unpredictable no matter how the code folding turns 
%     out.
%   \end{ensproc}
%   
%   \begin{ensproc}{children}
%     This procedure just lets the state be modified by each of its 
%     children in turn. The effect is to join them unbreakably.
%     \begin{tcl}
proc mtmtcl::latex::fold::children {attrD childL state level width} {
   foreach child $childL {
      set state ["" {*}$child $state $level $width]
   }
   return $state
}
%     \end{tcl}
%   \end{ensproc}
%   
%   \begin{ensproc}{mrow}
%     Row nodes construct new breakpoints and remove breakpoints 
%     at higher levels. For any position between two children, a 
%     new normal breakpoint is created which minimises the demerits 
%     on the path to that position. New tight breakpoints are created 
%     at the beginning of a tight |mrow|: every normal breakpoint 
%     gets a tight duplicate.
%     \begin{tcl}
proc mtmtcl::latex::fold::mrow {attrD childL bkpL level width} {
   set tight [expr\
     {[dict exists $attrD tight] && [dict get $attrD tight]}]
   if {$tight} then {
      foreach bkp $bkpL {
         if {[lindex $bkp 2] == "infinity"} then {
            lappend bkpL [lreplace $bkp 2 2 $level]
         }
      }
   }
%     \end{tcl}
%     Row nodes construct new normal breakpoints in front of each 
%     child. It might seem more natural to construct a new breakpoint 
%     \emph{after} the child, but since the breakpoint should come at 
%     the lowest nesting level between two leaves, the logic is easier 
%     when doing it before. The breakpoint list is kept unchanged if 
%     the \word{post-line} in the last breakpoint consists entirely 
%     of whitspace; this covers both the case of a lower level |mrow| 
%     having created a breakpoint here and the case of an invisible 
%     child of the present |mrow|.
%     \begin{tcl}
   set nextlevel [expr {$level+1}]
   foreach child $childL {
      if {[regexp {^\s*$} [lindex $bkpL end end]]} then {
         set newL $bkpL
      } else {
         set newL {}
         set best {x infinity {}}
         foreach bkp $bkpL {
            if {[lindex $bkp 0] <= $level} then {lappend newL $bkp}
            if {[lindex $bkp 1] <= [lindex $best 1] &&\
              [lindex $bkp 2]=="infinity"} then {set best $bkp}
%     \end{tcl}
%     That the comparison in the previous command is |<=| means the 
%     last breakpoint attaining the minimal demerits is picked as 
%     the |best| route to the new breakpoint. This has the effect 
%     that the last line of a child |mrow| becomes the shortest. That 
%     only normal breakpoints are considered has the effect that no 
%     linebreaks are made in tight formatting.
%     \begin{tcl}
         }
%     \end{tcl}
%     The following line is what determines that one indentation step 
%     equals two spaces. It could easily be made a parameter, but I 
%     don't think there is a need for that at the moment.
%     \begin{tcl}
         lappend best [string repeat {  } $level]
         lset best 0 $level
         lset best 1 [expr {[lindex $best 1] + 1 + $level}]
         lappend newL $best
      }
      set bkpL ["" {*}$child $newL $nextlevel $width]
   }
%     \end{tcl}
%     At the end of a tight |mrow|, it is necessary to switch those 
%     breakpoints with tight formatting that it gave rise to back to 
%     ordinary formatting. This requires adding the expected space to 
%     the end of the last line.
%     \begin{tcl}
   if {$tight} then {
      set newL {}
      foreach bkp $bkpL {
         if {[lindex $bkp 2] == $level} then {
            lset bkp 2 infinity
            lset bkp end "[lindex $bkp end] "
         }
         lappend newL $bkp
      }
      set bkpL $newL
   }
   return $bkpL
}
%     \end{tcl}
%   \end{ensproc}
%   
%   \begin{ensproc}{expand}
%     In some cases, syntax forces one to return one data-tree that 
%     will be made a child of an |mrow|, but one would really want to 
%     return a sequence of such trees. As far as the folding 
%     algorithm is concerned however, it is possible to emulate such 
%     a sequence by returning the intended elements as the children 
%     of an |expand| node.
%     \begin{tcl}
proc mtmtcl::latex::fold::expand {attrD childL state level width} {
   mrow $attrD $childL $state [expr {$level-1}] $width
}
%     \end{tcl}
%     This works because an |mrow| operates on the state before each 
%     child, this operator depends only on the |level| value, and it 
%     is an idempotent.
%   \end{ensproc}
%   
%   \begin{ensproc}{unindent}
%     Carrying that idea even further, one might arrive at a node 
%     type that places its children at the level \emph{below} a 
%     surrounding |mrow|. What use could there possibly be for that? 
%     Well, it turns out to be rather good at reproducing the 
%     traditional formatting of an environment, with the |\begin| and 
%     |\end| on separate lines one indentation step lower than the 
%     environment body: the environment in full can then be wrapped 
%     up in an |mrow|, and the special indentation of some children 
%     can be achieved through an |unindent| wrapper of those 
%     children; this is much easier than making up a syntax for 
%     selecting children to unindent.
%     
%     The first step is to adjust the position in the last line of 
%     the material being unindented; this is thus also a place where 
%     the indentation step size is hardcoded. |unindent| nodes may 
%     have a \describestring[attribute]{amount} attribute, which 
%     takes an integer value and defaults to |1|. It is the number of 
%     indentation steps that are undone.
%     \begin{tcl}
proc mtmtcl::latex::fold::unindent {attrD childL state0 level width} {
   if {[dict exists $attrD amount]} then {
      set prevlevel [expr {$level-[dict get $attrD amount]-1}]
   } else {
      set prevlevel [expr {$level-2}]
   }
   if {$prevlevel<0} then {set prevlevel 0}
   set indent [string repeat {  } $prevlevel]
   set state {}
   foreach bkp $state0 {
%     \end{tcl}
%     If the last line is only whitespace, then that line can of 
%     course be replaced with one with a different amount of 
%     indentation; this is the main case. Note that the demerits and 
%     line indentation level must be adjusted accordinly.
%     \begin{tcl}
      if {[regexp {^\s*$} [lindex $bkp end]]} then {
         lset bkp end $indent
         lset bkp 1 [expr {[lindex $bkp 1] - [lindex $bkp 0] + $prevlevel}]
         lset bkp 0 $prevlevel
%     \end{tcl}
%     Discard breakpoints where the last line is nonempty and has a 
%     higher indentation level than wanted; since the parent should 
%     be an |mrow|, every breakpoint discarded by this condition has 
%     a counterpart which is adjusted by the previous branch and thus 
%     kept.
%     \begin{tcl}
      } elseif {[lindex $bkp 0] > $prevlevel} then {
         continue
%     \end{tcl}
%     Finally, there is a slight possibility that the last line is 
%     nonempty but shorter than the unindented |indent|; this could 
%     be the case if several |mrow|s are nested. Here, spaces are 
%     appended to that last line until it has be basic |indent| 
%     length, since it is typically important that a |\begin| aligns 
%     with its |\end|.
%     \begin{tcl}
      } else {
         set line [lindex $bkp end]
         while {[string length $line] < [string length $indent]} {
            append line " "
         }
         lset bkp end $line
      }
      lappend state $bkp
   }
%     \end{tcl}
%     If there are several children, then the |state| is passed 
%     verbatim from one to the next; if linebreaks are to be allowed, 
%     then wrap the children up in another |mrow|. Note that the |level| 
%     passed to the children is unchanged; this has the effect that 
%     continuations of an unindented line are still indented more 
%     than the normal siblings of the unindented node would be.
%     \begin{tcl}
   foreach child $childL {
      set state ["" {*}$child $state $level $width]
   }
%     \end{tcl}
%     |unindent| nodes can have a 
%     \describestring+[attribute]{linebreak} attribute, which takes a 
%     boolean value. When true, a linebreak is forced after the item 
%     by pretending the indentation level of the last line is greater 
%     than the pre-unindent level. The default is false, i.e., the 
%     actual indentation level shines through.
%     \begin{tcl}
   if {[dict exists $attrD linebreak] && [dict get $attrD linebreak]}\
   then {
      set nextlevel [expr {$level+1}]
      set res {}
      foreach bkp $state {
         if {[lindex $bkp 0] < $level} then {lset bkp 0 $nextlevel}
         lappend res $bkp
      }
      set state $res
   }
   return $state
}
%     \end{tcl}
%     
%     Now what does all that mean in practice? A code block such as
%\begin{verbatim}
%   f(x) = \begin{cases}
%     1 & x > 0 \\
%     0 & x \leqslant 0
%   \end{cases}
%\end{verbatim}
%     could be encoded as (in TDL, for clarity)
%\begin{verbatim}
%   mrow {
%     / f(x) 
%     / =
%     mrow {
%       unindent linebreak 1 {
%         / {\begin{cases}}
%       }
%       mrow {
%         / 1
%         / &
%         mrow {
%           / x
%           / >
%           / 0
%         }
%       }
%       unindent linebreak 1 amount 0 {
%         / {\\}
%       }
%       mrow {
%         / 0
%         / &
%         mrow {
%           / x
%           / {\leqslant}
%           / 0
%         }
%       }
%       unindent {
%         / {\end{cases}}
%       }
%     }
%   }
%\end{verbatim}
%   \end{ensproc}
% \end{ensemble}
% 
% 
% \subsection{Metaformula expansion}
% 
% A \emph{metaformula} is a data-tree node which expresses some 
% kind of formula construction that is not just the concatenation of 
% its children, and therefore needs to be ``expanded'' (in the macro 
% sense) before the folding phase. The most common node type of this 
% class is the MathML-derived |mfenced|, which expresses the fencing 
% (bracketing) of its children, but |maybewrap| (wrap contents in a 
% \TeX\ group if needed) is also useful.
% 
% Splitting it up requires an explicit representation for parentheses, 
% and as it happens, MathML has a tag 
% with precisely this meaning. MathML attributes defined for such 
% elements are
% The attributes that are of interest for 
% \describestring+[node type]{mfenced} are
% \begin{description}
%   \renewcommand{\makelabel}[1]{^^A
%      \hspace\labelsep
%      \describestring*+[attribute]{#1}^^A
%   }
%   \item[open]
%     MathML: The symbol for the opening delimiter, defaults to `|(|'.
%   \item[close]
%     MathML: The symbol for the closing delimiter, defaults to `|)|'.
%   \item[separators]
%     MathML: List of separator characters, defaults to an infinite 
%     sequence of commas (`|,|'). (Separators are inserted between 
%     children if there are more than one.) 
%   \item[expand]
%     Controls whether the fence should be |expand|ed into a 
%     surrounding formula (true), or expand to a subformula of its 
%     own (false; default).
% \end{description}
% At the moment, the MathML-attributes are ignored (hardwired to 
% default values)\Dash in part because it is not entirely obvious 
% what a ``symbol'' should be in this context. That most faithful 
% interpretation would be to really make them the relevant Unicode 
% characters, and charge |mfenced| with translating them to \LaTeX.
% 
% \begin{ensemble}{metax::}[metax]
%   The call syntax for this ensemble is
%   \begin{displaysyntax}
%     metax:: \meta{data-tree}
%   \end{displaysyntax}
%   and its return value is a list
%   \begin{displaysyntax}
%     \word{data-tree} \word{fence-height}
%   \end{displaysyntax}
%   
%   \begin{tcl}
namespace eval mtmtcl::latex::metax {
   namespace export ?*
   namespace ensemble create -unknown {
%   \end{tcl}
%   The default action for a node type it entirely straightforward, 
%   but the way it is implemented may feel uncomfortably abstract. 
%   First, the unknown handler manufactures a command prefix. What it 
%   needs for this prefix is the ensemble name and the subcommand 
%   name (node type); everything else can be ignored as the ensemble 
%   mechanism will reappend it. Hence the unknown handler is easiest 
%   done as an explicit lambda.
%   \begin{tcl}
      ::apply {{lambda ensname type args} {
         list ::apply $lambda $ensname $type
      } ::}\
%   \end{tcl}
%   The action that should be taken for unknown node types is that 
%   for nodes where noting in particular needs to be done, i.e., all 
%   the children should be processed recursively, a new node with the 
%   same type and attributes should be created with the returned 
%   \word{data-tree}s as children, and the \word{fence-height} is the 
%   maximum of the \word{fence-height}s among the children.
%   
%   Due to the fence-height processing, this is a very specific 
%   operation, so it shouldn't be put outside the |metax| namespace, 
%   but it is also not the implementation of any specific node type, 
%   so it's not right to make it a procedure in this namespace 
%   either. Therefore the prefix is |apply| on another lambda, which 
%   needs to know the ensemble name to make recursive calls and 
%   therefore takes it as an argument.
%   \begin{tcl}
      {{ensname type attr children} {
         set height 0
         set childL {}
         foreach child $children {
            lassign [$ensname {*}$child] c h
            lappend childL $c
            if {$height<$h} then {set height $h}
         }
         list [list $type $attr $childL] $height
      } ::}
   } -command [namespace current]:: -prefix 0
}
%   \end{tcl}
%   
%   \begin{ensproc}{maybewrap}
%     In some corners of \TeX\ formula syntax, most notably 
%     subscripts and superscripts, a subformula can appear raw if it 
%     is only one token\footnote{
%       Actually a little lie, because it is more about how many 
%       nodes it contributes to the math list or something like that, 
%       but counting tokens is usually good enough.
%     } but must be braced if there are more than one. Some code 
%     generators always brace in these situations, but for small 
%     exponents that usually looks awful, so I prefer an approach 
%     that only brace if necessary and that is what the 
%     \describestring*+[node type]{maybewrap} node type is about.
%     \begin{tcl}
proc mtmtcl::latex::metax::maybewrap {attr children} {
%     \end{tcl}
%     First check for the ``no wrap'' case.
%     \begin{tcl}
   if {[llength $children] == 1 && [lindex $children 0 0] eq "#text"\
     && [regexp {(?x) 
       ^( \\[A-Za-z]+ | \\[^A-Za-z] | [!-\[\]-~] )$
   } [lindex $children 0 1]]} then {
      return [list [lindex $children 0] 0]
   }
%     \end{tcl}
%     If that's not it, then put the braces in an |mrow|. In case the 
%     child already was an |mrow|, then merge it with the new one.
%     \begin{tcl}
   if {[llength $children] == 1 && [lindex $children 0 0] eq "mrow"}\
   then {
      set attr [dict merge [lindex $children 0 1] $attr]
      set children [lindex $children 0 2]
   }
   "" mrow $attr [list {{#text} \{} {*}$children {{#text} \}}]
}
%     \end{tcl}
%   \end{ensproc}
%   
%   \begin{ensproc}{mfenced}
%     Expanding |mfenced| nodes is the main task for the |metax| 
%     operation. Currently it doesn't support the MathML attributes 
%     \describestring+[attribute]{opening}, 
%     \describestring+[attribute]{closing}, and 
%     \describestring+[attribute]{separators} (hardwiring all of them 
%     to their default values `\texttt{(}', `\texttt{)}', and 
%     `\texttt{,}' respectively), but this could be changed.
%     
%     The first part is the usual recursion over children.
%     \begin{tcl}
proc mtmtcl::latex::metax::mfenced {attr children} {
   set new {}
   set height 0
   foreach child $children {
      lassign ["" {*}$child] c h
      lappend new $c
      if {$h>$height} then {set height $h}
   }
%     \end{tcl}
%     Then comes the choice of parenthesis size. This should perhaps 
%     be factored out.
%     \begin{tcl}
   switch -- $height 0 {
      set left {(}
      set right {)}
   } 1 {
      set left {\bigl(}
      set right {\bigr)}
   } 2 {
      set left {\Bigl(}
      set right {\Bigr)}
   } 3 {
      set left {\biggl(}
      set right {\biggr)}
   } default {
      set left {\Biggl(}
      set right {\Biggr)}
   }
%     \end{tcl}
%     Finally the thing is put together. There are three basic 
%     styles: included, independent, and expanded. The included style 
%     is the default when there is only one child, and that child is 
%     an |mrow|: it means the left and right delimiters are inserted 
%     into that |mrow| as the first and final respectively children. 
%     The independent style is the default for other situations (not 
%     exactly one child, or the child not a |mrow|) and means another 
%     |mrow| is created for the delimiters and separators. Finally 
%     there is the expanded style, which is selected by giving an 
%     \describestring+[attribute]{expanded} attribute with a boolean 
%     true value: it is like the independent style, except that the 
%     node will be an |expanded| rather than an |mrow| and thus 
%     contribute its elements to a surrounding |mrow|.
%     
%     The expanded style is considered appropriate for $f(x)$-type 
%     constructions.
%     \begin{tcl}
   set expand [expr {[dict exists $attr expand] &&\
     [dict get $attr expand]}]
   if {!$expand && [llength $new]==1 && [lindex $new 0 0] eq "mrow"}\
   then {
      set children [list [list {#text} $left]\
        {*}[lindex $new 0 2] [list {#text} $right]]
      set attr [dict merge [lindex $new 0 1] $attr]
      set node [list mrow $attr $children]
   } else {
      set children [list [list {#text} $left] [lindex $new 0]]
      foreach child [lrange $new 1 end] {
         lappend children {{#text} ,} $child
      }
      lappend children [list {#text} $right]
      if {$expand} then {
         set node [list expand $attr $children]
      } else {
         set node [list mrow $attr $children]
      }
   }
   if {[string is integer -strict $height]} then {incr height}
   return [list $node $height]
}
%     \end{tcl}
%   \end{ensproc}
%   
%   \begin{ensproc}{#text}
%     Not to be forgotten: |#text| nodes need special treatment.
%     \begin{tcl}
proc mtmtcl::latex::metax::#text {str} {
   list [list {#text} $str] 0
}
%     \end{tcl}
%   \end{ensproc}
%   
% \end{ensemble}
% 
% 
% \subsection{Making \TeX\ formulae}
% 
% The pivotal step in the conversion from (near-)OpenMath data-trees 
% to \LaTeX\ code is that where all the various symbols are replaced 
% by their respective \OMSref{altenc}{LaTeX_encoding}s, because this 
% is where the data-tree starts to shrink. It is not the final step 
% before code folding, since many aspects of making \LaTeX\ code 
% need to deal with the structure of the code rather than the 
% structure of the mathematics expressed, but the ``line of sight'' 
% is pretty clear.
% 
% \setnamespace{mtmtcl::latex::make::tool}
% \begin{ensemble}{make}
%   \setnamespace{mtmtcl::latex::make}
%   As usual, this conversion\Ldash making \LaTeX\ formulae 
%   (although still in the form of a data-tree)\Dash is primarily 
%   handled by an ensemble. The namespace of this ensemble is 
%   |mtmtcl::latex::make|, but the actual |make| ensemble command 
%   resides in the |tool| child of |mtmtcl::latex::make| rather than 
%   the |mtmtcl::latex| namespace (as would be the default), since 
%   the subcommands need to call the ensemble command recursively.
%   \begin{tcl}
namespace eval mtmtcl::latex::make {
   namespace export *
   namespace ensemble create -command tool::make -prefix 0
}
%   \end{tcl}
%   What this operation has to worry about is loosely the syntactic 
%   correctness of the code generated. The issues that can 
%   potentially be thorny are:
%   \begin{description}
%     \item[generalised fractions]
%       (|\over|, |\atop|, etc.): there can only be one of these in 
%       each \TeX\ math list. However, since we're generating \LaTeX, 
%       and these are there deprecated in favour of |\frac| etc., 
%       these shouldn't occur anyway and are therefore not a problem.
%     \item[growing delimiters]
%       (|\left|, |\right|, and potentially |\middle|): these need to 
%       be balanced. Balancing is however something that happens 
%       automatically, so this should be fine too.
%     \item[sub- and superscripts]
%       All atoms in a \TeX\ math list are defined to have three 
%       fields: nucleus, subscript, and superscript. The |^| and |_| 
%       operations work by modifying the last atom on the list, 
%       rather than by adding something to the list, so they are 
%       dependent on their context.
%     \item[math type]
%       \TeX\ inserts glue and penalties between items according to 
%       their math types: |\mathord|, |\mathbin|, |\mathpunct|, etc. 
%       Most of the time, the code generated automatically carries 
%       the correct math type (e.g.~a `|+|' is a |\mathbin|), but in 
%       some cases it may be good for a generator to get an advice on 
%       what is expected.
%   \end{description}
%   As it stands, it is only uses of sub- and superscripts that 
%   produce syntactic errors (even though that should be unlikely, 
%   provided operations are properly fenced), so some control of this 
%   is called for. The model used here is that the |tool::make| 
%   command has the general syntax
%   \begin{displaysyntax}
%     tool::make \meta{node} \word{subscript?} \word{superscript?} 
%     \word{expected type}
%   \end{displaysyntax}
%   where the \word{subscript?} and \word{superscript?} are booleans 
%   signalling (when true) that the context in which the \word{node} 
%   occurs has put a sub- and\slash or superscript on it. The 
%   \word{expected type} is the corresponding \TeX\ control sequence 
%   minus the initial |\math|, so: |ord|, |bin|, |rel|, |op|, |open|, 
%   |close|, |punct|, or |inner|.
%   
%   
%   \begin{ensproc}{#text}
%     To begin with, naked |#text| nodes are just returned as they 
%     are. This amounts to presuming that anything expressed in such 
%     a node will make sense as \LaTeX\ code.
%     \begin{tcl}
proc mtmtcl::latex::make::#text {str b p m} {
   if {$b||$p} then {
      list maybewrap {} [list [list #text $str]]
   } else {
      list #text $str
   }
}
%     \end{tcl}
%   \end{ensproc}
%   
%   \begin{ensproc}{mn}
%     A number can be several digits without needing a wrap.
%     \begin{tcl}
proc mtmtcl::latex::make::mn {attr children b p m} {
   if {[llength $children]!=1} then {
      list mrow {} $children
   } else {
      lindex $children 0
   }
}
%     \end{tcl}
%   \end{ensproc}
%   
%   \begin{ensproc}{mfenced}
%     |mfenced| nodes just recurse over the children, with cleared 
%     sub- and superscript statuses.
%     \begin{tcl}
proc mtmtcl::latex::make::mfenced {attr children b p m} {
   set L {}
   foreach child $children {
      lappend L [tool::make {*}$child 0 0 ord]
   }
   list mfenced $attr $L
}
%     \end{tcl}
%   \end{ensproc}
%   
%   \begin{ensproc}{OMA}
%     An |OMA| node may have been turned into something else 
%     according to its |form| attribute already when fencing 
%     subexpressions, but it might just as well have been 
%     \describestring*+[proc][mtmtcl::present::fence::tool]{reform}ed 
%     into the original |OMA|. Hence there is again a dispatch based 
%     on \describestring+[attribute]{form}.
%     \begin{tcl}
proc mtmtcl::latex::make::OMA {attr children b p m} {
   if {[dict exists $attr form]} then {
      tool::form [dict get $attr form] $attr $children $b $p $m
   } else {
      tool::form function $attr $children $b $p $m
   }
}
%     \end{tcl}
%   \end{ensproc}
% \end{ensemble}
% 
% \begin{ensemble}{form}
%   \setnamespace{mtmtcl::latex::make}
%   As in the fencing of subexpressions, the dispatch ensemble is 
%   called |tool::form|, has the same namespace as the main 
%   ensemble, and an explicitly given list of subcommands.
%   \begin{tcl}
namespace eval mtmtcl::latex::make {
   namespace ensemble create -command tool::form -subcommands {
      infix prefix postfix function enclosed juxtaposed binding 
      exponent index
   } -prefix 0
}
%   \end{tcl}
%   
%   \begin{ensproc}{infix}
%     An operation in |infix| notation usually has \(n \geqslant 
%     2\) arguments, and in that case the symbol is to appear $n-1$ 
%     times, in each case between two arguments: $a_1 \oplus a_2 
%     \oplus \dotsb \oplus a_n$. If there is $1$ argument, only that 
%     argument is shown. If there are no arguments 
%     (degenerate cases) then it appears as $(\oplus)$.
%     
%     \begin{tcl}
proc mtmtcl::latex::make::infix {attr children b p m} {
   if {![llength $children]} then {
      error "Operation missing (0 children)"
   }
   set op [tool::make {*}[lindex $children 0] 0 0 bin]
   if {[llength $children]==1} then {
      return [list mfenced {} [list $op]]
   }
   set res {}
   foreach arg [lrange $children 1 end-1] {
      lappend res [tool::make {*}$arg 0 0 ord] $op
   }
   lappend res [tool::make {*}[lindex $children end] $b $p ord]
   return [list mrow {} $res]
}
%     \end{tcl}
%   \end{ensproc}
%   
%   \begin{ensproc}{juxtaposed}
%     A juxtaposed operation is a simplified |infix|, where the 
%     operation symbol is ignored.
%     \begin{tcl}
proc mtmtcl::latex::make::juxtaposed {attr children b p m} {
   if {[llength $children]<2} then {
      return [function $attr $children $b $p $m]
   }
   set res {}
   foreach arg [lrange $children 1 end-1] {
      lappend res [tool::make {*}$arg 0 0 ord]
   }
   lappend res [tool::make {*}[lindex $children end] $b $p ord]
   return [list mrow {} $res]
}
%     \end{tcl}
%   \end{ensproc}
%   
%   \begin{ensproc}{function}
%     Function notation is realised as an |mrow|--|mfenced| 
%     combination, where the |mfenced| contains the arguments and is 
%     set to |expand| into the |mrow|.
%     \begin{tcl}
proc mtmtcl::latex::make::function {attr children b p m} {
   if {![llength $children]} then {
      error "Function missing (0 children)"
   }
   set func [tool::make {*}[lindex $children 0] 0 0 ord]
   set L {}
   foreach arg [lrange $children 1 end] {
      lappend L [tool::make {*}$arg 0 0 ord]
   }
   list mrow {} [list $func [list mfenced {expand 1} $L]]
}
%     \end{tcl}
%   \end{ensproc}
%   
%   \begin{ensproc}{prefix}
%     Operations in prefix notation, which includes ``operators'' 
%     such as $\sin$, appear first and their argument second. If 
%     there is not exactly one argument then we switch to |function| 
%     notation.
%     \begin{tcl}
proc mtmtcl::latex::make::prefix {attr children b p m} {
   if {[llength $children] != 2} then {
      function $attr $children $b $p $m
   } else {
      list mrow {} [list [
         tool::make {*}[lindex $children 0] 0 0 op
      ] [
         tool::make {*}[lindex $children 1] $b $p ord
      ]]
   }
}
%     \end{tcl}
%     The operation could alternatively have an expected mathtype of 
%     |ord|, but |op| is hopefully closer to what is expected.
%   \end{ensproc}
%   
%   \begin{ensproc}{postfix}
%     Postfix operations are almost the same as |prefix| operations, 
%     but the operand appears first, so it is to the operation that 
%     a sub- or superscript might get attached.
%     \begin{tcl}
proc mtmtcl::latex::make::postfix {attr children b p m} {
   if {[llength $children] != 2} then {
      function $attr $children $b $p $m
   } else {
      list mrow {} [list [
         tool::make {*}[lindex $children 1] 0 0 ord
      ] [
         tool::make {*}[lindex $children 0] $b $p punct
      ]]
   }
}
%     \end{tcl}
%     Labelling postfix operations a having mathtype |punct| is also 
%     slightly odd, but is inspired by the factorial.
%   \end{ensproc}
%   
%   \begin{ensproc}{exponent}
%     Exponent nodes should have three children: the 
%     superscripting operation (not shown), the base, and the 
%     exponent; otherwise we fall back to |function| formatting. 
%     If there is another superscript outside the present one then 
%     this will be wrapped up in a brace group.
%     \begin{tcl}
proc mtmtcl::latex::make::exponent {attr children b p m} {
   if {[llength $children] != 3} then {
      return [function $attr $children $b $p $m]
   }
   set L {}
   if {$p} then {lappend L {#text \{}}
   lappend L [tool::make {*}[lindex $children 1] $b 1 ord]
   lappend L {#text ^}
   lappend L [list maybewrap {} [list [
      tool::make {*}[lindex $children 2] 0 0 ord
   ]]]
   if {$p} then {lappend L {#text \}}}
   return [list mrow {tight 1} $L]
}
%     \end{tcl}
%     Exponent nodes get tight formatting if possible, since it is 
%     uncommon to see a `|^|' with surrounding whitespace.
%   \end{ensproc}
%   
%   \begin{ensproc}{index}
%     Index nodes should have three or more children: the 
%     indexing operation (not shown), the base, and the indices; 
%     otherwise we fall back to |function| formatting. 
%     If there is another subscript outside the present one then 
%     this will be wrapped up in a brace group.
%     \begin{tcl}
proc mtmtcl::latex::make::index {attr children b p m} {
   if {[llength $children] < 3} then {
      return [function $attr $children $b $p $m]
   }
   set L {}
   if {$b} then {lappend L {#text \{}}
   lappend L [tool::make {*}[lindex $children 1] 1 $p ord]
   lappend L {#text _}
   if {[llength $children] > 3} then {
      set L2 [list {#text \{}]
      foreach child [lrange $children 2 end-1] {
         lappend L2 [tool::make {*}$child 0 0 ord] {#text ,}
      }
      lappend L2 [tool::make {*}[lindex $children end] 0 0 ord]
      lappend L2 {#text \}}
      lappend L [list mrow {} $L2]
   } else {
      lappend L [list maybewrap {} [list [
         tool::make {*}[lindex $children 2] 0 0
      ]]]
   }
   if {$b} then {lappend L {#text \}}}
   return [list mrow {tight 1} $L]
}
%     \end{tcl}
%     Tightness ditto for indices.
%   \end{ensproc}
%   
% \end{ensemble}
% 
% 
% \begin{ensemble}{make}
%   \setnamespace{mtmtcl::latex::make}
%   
%   \begin{ensproc}{OMATTR}
%     The big point is the attribution node types however, because 
%     this is where most \LaTeX\ code particles are hiding: in 
%     \OMSref{altenc}{LaTeX_encoding} attribution nodes. An |OMATTR| 
%     without one of those is just equivalent to its its nucleus, 
%     but when there is a |LaTeX_encoding| then this says what 
%     should happen next.
%     \begin{tcl}
proc mtmtcl::latex::make::OMATTR {attr children b p m} {
   foreach {key value} [lindex $children 0 2] {
      if {[lindex $key 0] eq "OMS" &&\
        [dict exists [lindex $key 1] cd] &&\
        [dict get [lindex $key 1] cd] eq "altenc" &&\
        [dict exists [lindex $key 1] name] &&\
        [dict get [lindex $key 1] name] eq "LaTeX_encoding"} then {
         return [
            tool::LaTeX_encoding $value [lindex $children 1] $b $p $m
         ]
      }
   }
   tool::make {*}[lindex $children 1] $b $p $m
}
%     \end{tcl}
%   \end{ensproc}
% \end{ensemble}
% 
% There are three supported cases for the 
% \OMSref*{altenc}{LaTeX_encoding} attribution. The established case 
% (on account of being suggested in the |altenc| content dictionary) 
% is that the value is an |OMSTR| node. This string is then the exact 
% \LaTeX\ code for rendering this object, so it will simply be put in 
% a |#text| node.
% 
% The second case is that the |LaTeX_encoding| attribution is an 
% \describestring*+[node type]{OMFOREIGN} element whose 
% \describestring*+[attribute]{encoding} attribute is 
% \describestring+[encoding]{mtmtcl-LaTeX}. The children of this node 
% are then supposed to constitute the formatting of this node, wrapped 
% up in an |mrow| node if there is more than one node. This allows 
% embedded |mfenced| nodes to be taking into account when choosing 
% parenthesis sizes and also including 
% \describestring+[node type]{usepackage} nodes.
% 
% The third case also uses an \describestring*+[node type]{OMFOREIGN} 
% element, but has \describestring+[encoding]{Tcl-prefix} as 
% encoding. The element contents in this case should be a |#text| 
% element whose string constitutes a \Tcl\ command prefix which gets 
% to do the actual formatting. It will be called as
% \begin{displaysyntax}
%   \meta{prefix} \word{recursion} \word{type} \word{attributes} 
%   \word{children} \word{subscript?} \word{superscript?} 
%   \word{mathtype}
% \end{displaysyntax}
% and returns the corresponding \mtl-\LaTeX\ data-tree. The 
% `\word{type} \word{attributes} \word{children}' part is the 
% node which carries the |LaTeX_encoding| attribution. The 
% \word{recursion} is a command prefix which can be called as
% \begin{displaysyntax}
%   \meta{recursion} \meta{node} \word{subscript?} 
%   \word{superscript?} \word{mathtype}
% \end{displaysyntax}
% to convert some \word{node} (typically one of the \word{children}) 
% to a \mtl-\LaTeX\ data-tree.
% 
% \begin{tclcommand}{interpreter}{safe}
%   Since |Tcl-prefix|es allow embedding arbitrary code in 
%   near-OpenMath trees, there is a risk for injection attacks. In 
%   order to limit the security risks posed by this, the prefixes are 
%   always evaluated in a separate safe interpreter.
%   
%   \begin{tcl}
namespace eval mtmtcl::latex::make {
   interp create -safe [namespace current]::tool::safe
   tool::safe alias make [namespace which tool::make]
}
%   \end{tcl}
% \end{tclcommand}
% 
% \begin{proc}[]{sumnode}
%   One case that is not handled by the generic operations is that of 
%   plain old sums, since $a + (-b) + c$ is supposed to be rendered 
%   as $a-b+c$. As this is a very common case, it is better to 
%   install a procedure for this in the |safe| interpreter than to 
%   express it as a lambda.
%   
%   The syntax of a |Tcl-prefix| based on this command is
%   \begin{displaysyntax}
%     sumnode \word{plus} \word{minus} \word{zero}
%   \end{displaysyntax}
%   where \word{plus}, \word{minus}, and \word{zero} are \mtl-\LaTeX\ 
%   data-trees with the material for a plus between terms, a minus 
%   between terms, and a zero (used when there aren't any terms). 
%   The typical prefix constructed using this command is thus
%   \begin{quote}
%     |sumnode {#text +} {#text -} {#text 0}|
%   \end{quote}
%   but a variation that is not entirely uncommon is to want some or 
%   all of these to be ``bold'' when formatting a sum of vectors.
%   
%   What the command does is that it looks at the children (other 
%   than the first) to see whether they are |negate| nodes, and picks 
%   \word{plus} or \word{minus} as operation depending on which is 
%   found. Note that child $0$ is assumed to be the addition 
%   operation (which is ignored) and child $1$ is the first term 
%   (where a negation is left in place).
%   \begin{tcl}
mtmtcl::latex::make::tool::safe eval {
   proc sumnode\
     {plus minus zero rcall type attributes children b p m} {
      if {[llength $children] < 2} then {return $zero}
%   \end{tcl}
%   The \word{zero} case is easy to recognise and handle. After that 
%   it is known that there is at least one term, and the exceptional 
%   first term can be handled separately.
%   \begin{tcl}
      set res [list [{*}$rcall {*}[lindex $children 1] 0 0 ord]]
      foreach child [lreplace $children 0 1] {
         if {[lindex $child 0] ne "negate" ||\
           [llength [lindex $child 2]] != 2} then {
            lappend res $plus [{*}$rcall {*}$child 0 0 ord]
         } else {
            lappend res $minus\
              [{*}$rcall {*}[lindex $child 2 1] 0 0 ord]
         }
      }
%   \end{tcl}
%   If there was more than one term then wrap all of it up in an 
%   |mrow|, otherwise return the single term.
%   \begin{tcl}
      if {[llength $res]>1} then {
         set res [list mrow {} $res]
      } else {
         set res [lindex $res 0]
      }
%   \end{tcl}
%   Under normal priorities, there shouldn't any sub- or superscripts 
%   directly on a sum (a surrounding |mfence| should have cleared 
%   them), but if there still are any then fence the sum just to be 
%   on the safe side.
%   \begin{tcl}
      if {$b || $p} then {set res [list mfenced {} [list $res]]}
      return $res
   }
}
%   \end{tcl}
% \end{proc}
% 
% \begin{proc}[]{mrowjoin}
%   This is a generic handler for joining subexpressions, that should 
%   be sufficient as long as all operands follow the same syntax. The 
%   syntax of a |Tcl-prefix| based on this command is
%   \begin{displaysyntax}
%     mrowjoin \word{begin} \word{end} \word{before} \word{after} 
%     \word{between}
%   \end{displaysyntax}
%   where each argument is a list of \mtl-\LaTeX\ data-trees. The 
%   items in \word{begin} occur once, at the beginning of the entire 
%   construction, and the items in \word{end} occur once at the end 
%   of the entire construction. The items in \word{before} occur 
%   before each operand and the items in \word{after} occur after 
%   each operand, whereas the items in \word{between} occur between 
%   each pair of operands.
%   
%   In many cases, it is possible to construct equivalent pieces of 
%   \LaTeX\ in several ways with this command, e.g. 
%   `|\binom{|\meta{first}|}{|\meta{second}|}|' can be achieved both 
%   with
%   \begin{quote}
%     |mrowjoin {{#text \\binom}} {} {{#text \{}} {{#text \}}} {}|
%   \end{quote}
%   and with
%   \begin{quote}
%     |mrowjoin {{#text \\binom\{}} {{#text \}}} {} {} {{#text \}\{}}|
%   \end{quote}
%   The difference between these two lies in how the items are 
%   grouped. \word{begin}, \word{end}, and \word{between} items end 
%   up as children of the main |mrow|. \word{before} and \word{after} 
%   items appear in a subordinate |mrow| with the actual operand. The 
%   two possibilities thus lets you choose between
%   \begin{equation*}
%     \left[
%     \vcenter{\hbox{|\binom|}\hbox{|  {n}|}\hbox{|  {k}|}}
%     \right]
%     \quad\text{and}\quad
%     \left[
%     \vcenter{
%       \hbox{|\binom{|}\hbox{|  n|}\hbox{|  }{|}
%       \hbox{|  k|}\hbox{|  }|}
%     }
%     \right]
%     \text{.}
%   \end{equation*}
%   
%   \begin{tcl}
mtmtcl::latex::make::tool::safe eval {
   proc mrowjoin {begin end before after between \
     rcall type attributes children b p m} {
      set res $begin
      set n -1; foreach child $children {incr n
         if {$n==0} then {continue}
         if {$n>1} then {lappend res {*}$between}
         set subres $before
         lappend subres [{*}$rcall {*}$child 0 0 ord] {*}$after
         if {[llength $subres]>1} then {
            lappend res [list mrow {} $subres]
         } else {
            lappend res {*}$subres
         }
      }
      lappend res {*}$end
      if {[llength $res]!=1} then {
         return [list mrow {} $res]
      } else {
         return [lindex $res 0]
      }
   }
}
%   \end{tcl}
% \end{proc}
% 
% 
% 
% \begin{tclcommand}{command}{gettext}
%   The |tool::gettext| command is just an import of 
%   |mtmtcl::openmath::gettext|.
%   \begin{tcl}
namespace eval mtmtcl {
   namespace inscope latex::make::tool {namespace import}\
      [namespace which openmath::gettext]
}
%   \end{tcl}
% \end{tclcommand}
% 
% 
% \begin{proc}{LaTeX_encoding}
%   The work for the |tool::LaTeX_encoding| procedure should thus be 
%   clear. Its call syntax is
%   \begin{displaysyntax}
%     |tool::LaTeX_encoding| \word{value} \word{nucleus} 
%     \word{subscript?} \word{superscript?} \word{mathtype}
%   \end{displaysyntax}
%   where the \word{value} is the value of the 
%   \OMSref{altenc}{LaTeX_encoding} attribution (an OM object) and 
%   the \word{nucleus} is the node to which it was applied. The 
%   return value is the \mtl-\LaTeX\ data-tree.
%   
%   If the \word{value} format is not recognised then this defaults 
%   to formatting the \word{nucleus} instead. The result is not 
%   explicitly |return|ed, as |switch| and |if| passes it on anyway.
%   \begin{tcl}
proc mtmtcl::latex::make::tool::LaTeX_encoding {value nucleus b p m} {
   switch -- [lindex $value 0] "OMSTR" {
%   \end{tcl}
%   The practical work of generating the result data-tree for an 
%   |OMSTR| is done as if the whole thing was a single |#text| node.
%   \begin{tcl}
      make \#text [gettext $value {}] $b $p $m
   } "OMFOREIGN" {
%   \end{tcl}
%   Getting the |encoding| is slightly awkward.
%   \begin{tcl}
      switch -- [
         dict get [dict merge {encoding ""} [lindex $value 1]]\
           encoding
      ] "mtmtcl-LaTeX" {
         if {[llength [lindex $value 2]]==1} then {
            lindex $value 2 0
         } else {
            lreplace $value 0 1 mrow {}
         }
      } "Tcl-prefix" {
%   \end{tcl}
%   The |make| here is the name of an alias in the |tool::safe| 
%   interpreter which points back to the |tool::make| ensemble.
%   \begin{tcl}
         safe eval [linsert [lindex $value 2 0 1] end make\
           {*}$nucleus $b $p $m]
      } default {
%   \end{tcl}
%   What remains are the cases of defaulting to the \word{nucleus}.
%   \begin{tcl}
         make {*}$nucleus $b $p $m
      }
   } default {
      make {*}$nucleus $b $p $m
   }
}
%   \end{tcl}
% \end{proc}
% 
% \begin{proc}{transparent}
%   Some node types behave mostly as containers for a child. These 
%   can be implemented as aliases to |tool::transparent|, whose call 
%   syntax is
%   \begin{displaysyntax}
%     tool::transparent \word{arity} \word{index} \word{node type} 
%     \word{attributes} \word{children} \word{superscript?} 
%     \word{subscript?} \word{mathtype}
%   \end{displaysyntax}
%   where \word{arity} is the requested number of children and 
%   \word{index} is the index of the child which will be processed 
%   further. The \word{node type} is used in error messages (if there 
%   isn't the specified number of children).
%   \begin{tcl}
proc mtmtcl::latex::make::tool::transparent\
  {arity index type attr children p b m} {
   if {[llength $children] != $arity} then {
      error "$type nodes must have $arity child(ren)"
   }
   make {*}[lindex $children $index] $p $b $m
}
%   \end{tcl}
% \end{proc}
% 
% 
% For other node types, the default is mostly to do the same thing as 
% MathML would (via the correspondence 
% OpenMath${}\mapsto{}$Content-MathML${}\mapsto{}$Presentation-MathML). 
% This frequently involves translating (Unicode) strings to \LaTeX, 
% so some tools for this are useful.
% 
% \begin{variable}{licr_map}
%   One basic problem is that of characters for which one would 
%   normally use a control sequence. The relationship between 
%   character and command are pretty arbitrary, so it is probably 
%   best done as a |string map|. This |licr_map| variable holds a 
%   mapping from characters to \LaTeX\ text counterparts.
%   
%   At the moment, this only does something for characters in the 
%   visible ASCII range, but one could consider extending it to 
%   larger character sets (\texttt{utf8enc.dfu} would be a good 
%   starting point for data about this mapping).
%   \begin{tcl}
namespace eval mtmtcl::latex::make::tool {
   variable licr_map {
      { } ~
%   \end{tcl}
%   Expressing spaces as |~| is primarily good for ensuring proper 
%   parsing, particularly avoiding that spaces are lost. It also 
%   prevents linebreaks, but that is probably enforced by a 
%   surrounding |\mbox| or |\fbox| anyway.
%   \begin{tcl}
      #   {\#}
      $   {\textdollar }
      %   {\%}
      &   {\&}
%   \end{tcl}
%   There might be a point in transforming the quotes as well, namely 
%   to prevent unintended ligatures, but that is not an urgent issue.
%   \begin{tcl}
      <   {\textless }
      >   {\textgreater }
      \\  {\textbackslash }
      ^   {\textasciicircum }
      _   {\textunderscore }
      \{  {\textbraceleft }
      |   {\textbar }
      \}  {\textbraceright }
      ~   {\textasciitilde }
   }
}
%   \end{tcl}
%   The map is automatically extended by |str2licr| whenever it 
%   encounters a character outside the visible ASCII range which the 
%   map does not cover. This automatic extension takes the form 
%   `\ensuremath{\langle}U+03a0\ensuremath{\rangle}' for $\Pi$.
% \end{variable}
% 
% \begin{proc}{str2licr}
%   This command takes an arbitrary string as argument and generate 
%   \LaTeX\ code for that string in a text context. The call syntax is
%   \begin{displaysyntax}
%     str2licr \word{string}
%   \end{displaysyntax}
%   and the return value is the \LaTeX\ code.
%   
%   \begin{tcl}
proc mtmtcl::latex::make::tool::str2licr {str} {
   variable licr_map
   set conv [string map $licr_map $str]
   if {![regexp {[^ -~]} $str]} then {return $conv}
%   \end{tcl}
%   Since a \LaTeX\ Internal Character Representation only employs 
%   ASCII characters, we can detect unprocessed characters as those 
%   outside this character range and return the result early 
%   otherwise.
%   
%   The character parser converts surrogate pairs to simple non-BMP 
%   codepoints, since this is how those characters presently have to 
%   be encoded in \Tcl, but it will accept explicit characters beyond 
%   |\uFFFF| as well if \Tcl\ is extended to support them.
%   \begin{tcl}
   regsub -all {[ -~]} $conv "" chars
   foreach char [lsort -unique [
      regexp -inline -all\
        {[^\uD800-\uDFFF]|[\uD800-\uDBFF][\uDC00-\uDFFF]} $chars
   ]] {
      set latex {\ensuremath{\langle}U+}
      if {[scan $char %c%c hi lo]==1} then {
         append latex [format %04x $hi]
      } else {
         append latex [format %x [expr\
           {1024*($hi-0xD800) + ($lo-0xDC00) + 65536}]]
      }
      append latex {\ensuremath{\rangle}}
      lappend licr_map $char $latex
   }
   return [string map $licr_map $str]
}
%   \end{tcl}
% \end{proc}
% 
% 
% \begin{variable}{mathchar_map}
%   This variable holds a |string map| that translates Unicode 
%   characters to \LaTeX\ math character counterparts, primarily for 
%   the purpose of rendering variable names. This implies that the 
%   \LaTeX\ math characters should not get any space around them, so 
%   characters of |\mathbin|, |\mathop|, |\mathrel|, and |\mathpunct| 
%   class need to be wrapped up in a brace group.
%   
%   The first bunch of characters would be those which map to 
%   themselves, but since my initial code for initialising this 
%   variable encountered \Tcl\ bug \#2251175, I've moved that (which 
%   just does a bunch of |lappend mathchar_map $c $c|) to the end so 
%   that it gets the side-effect of purifying the list. That way, it 
%   can be built up by doing |append mathchar_map {|\dots\ instead of 
%   |lappend mathchar_map {*}{|\dots\ as was originally the case.
%   \begin{tcl}
namespace eval mtmtcl::latex::make::tool {
   variable mathchar_map {
%   \end{tcl}
%   ASCII characters missing from this map are space and quotes, 
%   since they don't have a clear role in mathematics. `|#|'--`|&|' 
%   can occur, but need to be ``escaped'' since the plain forms have 
%   syntactic roles in \LaTeX:
%   \begin{tcl}
      #  {\#}
      $  {\$}
      %  {\%}
      &  {\&}
%   \end{tcl}
%   The apostrophe is another quote. `|*|'--`|-|' need to be 
%   |\mathord|ed, and the same holds for `|:|'--`|>|':
%   \begin{tcl}
      *  {{*}}
      +  {{+}}
      ,  {{,}}
      -  {{-}}
      :  {{:}}
      ;  {{;}}
      <  {{<}}
      =  {{=}}
      >  {{>}}
%   \end{tcl}
%   The backslash is perhaps not a character one would expect in math 
%   identifiers, but |\backslash| is an ord character. A circumflex 
%   can be constructed as |\hat{}|. Underscore is available as |\_|.
%   \begin{tcl}
      \\ {\backslash }
      ^  {\hat{}}
      _  {\_}
%   \end{tcl}
%   The grave accent is yet another quote.
%   \begin{tcl}
      \{ {\{}
      |  {\vert }
      \} {\}}
      ~  {\tilde{}}
   }
}
%   \end{tcl}
%   Now for the latin-1 range. `{`!}' doesn't normally seem to be 
%   available as a math character, and the same holds for the cent, 
%   currency, Yen, and broken bar signs.
%   \begin{tcl}
append mtmtcl::latex::make::tool::mathchar_map {
   \u00A3 {\pounds }
   \u00A7 {\S }
   \u00A9 {\copyright }
   \u00AC {\neg }
%   \end{tcl}
%   The following is a degree sign.
%   \begin{tcl}
   \u00B0 {{}^\circ }
   \u00B1 {{\pm}}
%   \end{tcl}
%   Superscript $2$, superscript $3$, and the micro sign could be 
%   done (as `$^2$', `$^3$', and $\mu$), but are probably better left 
%   out. The acute accent is not particularly distinct.
%   \begin{tcl}
   \u00B6 {\P }
   \u00B7 {{\cdot}}
%   \end{tcl}
%   Remaining characters in the |\u00B?| range are cedilla, 
%   superscript $1$, masculine ordinal, right-pointing guillement, 
%   the $\frac{1}{4}$, $\frac{1}{2}$, and $\frac{3}{4}$ fractions, 
%   and `\textexclamdown', but none of these make much sense here.
%   
%   If accented letters are to be supported, we've got a ton of work 
%   ahead; it is probably best handled through a dynamic extension 
%   mechanism (generate decomposed unicode, then translate combining 
%   characters to math accents).
%   \begin{tcl}
   \u00D7 {{\times}}
   \u00F7 {{\div}}
}
%   \end{tcl}
%   It could well be argued that none of |\neg|, |\pm|, |\cdot|, 
%   |\times|, and |\div| should be allowed here as they are all 
%   established as operation symbols, but for now I'll leave them in.
%   
%   The next group of any size is the Greek letters.
%   \begin{tcl}
append mtmtcl::latex::make::tool::mathchar_map {
   \u0393 {\Gamma }
   \u0394 {\Delta }
   \u0398 {\Theta }
   \u039B {\Lambda }
   \u039E {\Xi }
   \u03A0 {\Pi }
   \u03A3 {\Sigma }
   \u03A5 {\Upsilon }
   \u03A6 {\Phi }
   \u03A7 {\Chi }
   \u03A8 {\Psi }
   \u03A9 {\Omega }
   \u03B1 {\alpha }
   \u03B2 {\beta }
   \u03B3 {\gamma }
   \u03B4 {\delta }
   \u03B5 {\varepsilon }
   \u03B6 {\zeta }
   \u03B7 {\eta }
   \u03B8 {\theta }
   \u03B9 {\iota }
   \u03BA {\kappa }
   \u03BB {\lambda }
   \u03BC {\mu }
   \u03BD {\nu }
   \u03BE {\xi }
%   \end{tcl}
%   |\u03BF| is omikron, which is classically indistinguishable from 
%   $o$ anyway.
%   \begin{tcl}
   \u03C0 {\pi }
   \u03C1 {\rho }
   \u03C2 {\varsigma }
   \u03C3 {\sigma }
   \u03C4 {\tau }
   \u03C5 {\upsilon }
   \u03C6 {\varphi }
   \u03C7 {\chi }
   \u03C8 {\psi }
   \u03C9 {\omega }
   \u03D1 {\vartheta }
   \u03D5 {\phi }
   \u03D6 {\varpi }
   \u03F0 {\varkappa }
   \u03F1 {\varrho }
   \u03F5 {\epsilon }
}
%   \end{tcl}
%   Fast forward to |\u2???| block.
%   \begin{tcl}
append mtmtcl::latex::make::tool::mathchar_map {
   \u2020 {\dag }
   \u2021 {\ddag }
   \u2032 '
   \u2033 ''
   \u2034 '''
%   \end{tcl}
%   Letterlike symbols group.
%   \begin{tcl}
   \u2102 {\mathbb{C}}
   \u210B {\mathcal{H}}
   \u210C {\mathfrak{H}}
   \u210D {\mathbb{H}}
   \u210F {\hslash }
   \u2110 {\mathcal{I}}
   \u2111 {\Im }
   \u2112 {\mathcal{L}}
   \u2113 {\ell }
   \u2115 {\mathbb{N}}
   \u2118 {\wp }
   \u2119 {\mathbb{P}}
   \u211A {\mathbb{Q}}
   \u211B {\mathcal{R}}
   \u211C {\Re }
   \u211D {\mathbb{R}}
   \u2124 {\mathbb{Z}}
   \u2127 {\mho }
   \u2128 {\mathfrak{Z}}
   \u212C {\mathcal{B}}
   \u212D {\mathfrak{C}}
   \u2130 {\mathcal{E}}
   \u2131 {\mathcal{F}}
   \u2133 {\mathcal{M}}
   \u2135 {\aleph }
   \u2136 {\beth }
   \u2137 {\gimel }
   \u2138 {\daleth }
   \u2141 {\Game }
}
%   \end{tcl}
%   \begin{tcl}
%   \end{tcl}
%   Final group of characters: Those that map to themselves.
%   \begin{tcl}
apply {{} {
   variable mathchar_map
   foreach c {
      ! ( ) . / 0 1 2 3 4 5 6 7 8 9 ? 
      @ A B C D E F G H I J K L M N O P Q R S T U V W X Y Z [ ]
      a b c d e f g h i j k l m n o p q r s t u v w x y z
   } {lappend mathchar_map $c $c}
} mtmtcl::latex::make::tool}
%   \end{tcl}
% \end{variable}
% 
% 
% \begin{ensemble}{make}
%   \setnamespace{mtmtcl::latex::make}
%   \begin{ensproc}{OMS}
%     Symbols should typically have been given an explicit markup 
%     (either through some |OMATTR|, or as an embedded 
%     subexpression), but when there hasn't been any of that then the 
%     fallback is to render the symbol as a box containing the 
%     \describestring*+[attribute]{name} and 
%     \describestring*+[attribute]{cd} attributes.
%     \begin{tcl}
proc mtmtcl::latex::make::OMS {attr children b p m} {
   if {[llength $children]} then {
      tool::make {*}[lindex $children 0] $b $p $m
   } else {
      set text [format {\texttt{%s}@\texttt{%s}}\
        [tool::str2licr [dict get $attr name]]\
        [tool::str2licr [dict get $attr cd]]]
      set text "\\text{\\fbox{\\normalfont $text}}"
      switch -- $m "bin" - "op" {
         set text "\\math$m\{$text\}"
      }
      list #text $text
   }
}
%     \end{tcl}
%     The MathML approach in this situation would instead be to 
%     render the |name| as if it was a variable. This would (mostly) 
%     do the right thing for symbols such as \OMSref{transc1}{cos}, 
%     \OMSref{arith1}{gcd}, and \OMSref{veccalc1}{curl}, but would 
%     perhaps invite to not defining them, which on the whole seems 
%     worse.
%   \end{ensproc}
%   
%   \begin{ensproc}{OMV}
%     Variables are more faithfully formatted according to MathML 
%     conventions\Ldash by applying |mathchar_map| to the |name|\Dash 
%     but often the actual formatting is rather given by an 
%     attribution or child.
%     \begin{tcl}
proc mtmtcl::latex::make::OMV {attr children b p m} {
   if {[llength $children]} then {
      tool::make {*}[lindex $children 0] $b $p $m
   } else {
      set name [dict get $attr name]
      regsub -all {[^ -~]} [string map $tool::mathchar_map $name]\
        {} text
      if {[string length $name]>1} then {set text "\\mathrm{$text}"}
      switch -- $m "bin" - "op" {
         set text "\\math$m{$text}"
      }
      list #text $text
   }
}
%     \end{tcl}
%   \end{ensproc}
%   
%   \begin{ensproc}{OMSTR}
%     Strings are typeset in typewriter font, with roman quotes 
%     around them.
%     \begin{tcl}
proc mtmtcl::latex::make::OMSTR {attr children b p m} {
   set text [tool::str2licr [
      tool::gettext [list OMSTR $attr $children] ""
   ]]
   set text "\\text{\\normalfont``\\texttt{$text}''}"
   switch -- $m "bin" - "op" {
      set text "\\math$m\{$text\}"
   }
   list #text $text
}
%     \end{tcl}
%   \end{ensproc}
%   
%   \begin{ensproc}{OMF}
%   \begin{ensproc}{integer}
%   \begin{ensproc}{OMR}
%     Floats, integers, and references are at this stage mere 
%     containers for their first and only children.
%     \begin{tcl}
interp alias {} mtmtcl::latex::make::OMF  {}\
  [namespace which mtmtcl::latex::make::tool::transparent] 1 0 OMF
interp alias {} mtmtcl::latex::make::integer  {}\
  [namespace which mtmtcl::latex::make::tool::transparent] 1 0 integer
interp alias {} mtmtcl::latex::make::OMR  {}\
  [namespace which mtmtcl::latex::make::tool::transparent] 1 0 OMR
%     \end{tcl}
%   \end{ensproc}\end{ensproc}\end{ensproc}
%   
%   \begin{ensproc}{negate}
%     |negate| nodes generated by the typographic preprocessing are 
%     aliased to |prefix|.
%     \begin{tcl}
interp alias {} mtmtcl::latex::make::negate  {}\
  [namespace which mtmtcl::latex::make::prefix]
%     \end{tcl}
%     An alternative would be to alias them to |OMA|, but that turned 
%     out to have the disadvantage of putting a seemingly nonsensical 
%     parenthesis around the first term of a sum when it was 
%     negative. Combining |OMA| with a |form| attribute could however 
%     be better than this.
%   \end{ensproc}
%   
%   Still not implemented are:
%   \begin{description}
%     \renewcommand{\makelabel}[1]{\hspace\labelsep\texttt{#1},}
%     \item[OMATP]
%       but those should only appear inside |OMATTR| anyway, so 
%       that's covered;
%     \item[OMB]
%       but that's probably not wanted anyway\Dash in case of 
%       emergency it could be made an alias to |OMSTR|;
%     \item[OMBIND]
%       which really should get implemented;
%     \item[OMBVAR]
%       but that's dependent on |OMBIND|;
%     \item[OME]
%       but for the moment I'm not wrapping any errors;
%     \item[OMFOREIGN]
%       but that is almost by definition not possible to implement 
%       generically.
%   \end{description}
% \end{ensemble}
% 
% 
% 
% \subsection{Putting it all together}
% 
% The following combines the above into a basic ``\LaTeX\ formula 
% from \mtl\ near-OpenMath generator''.
% 
% \setnamespace{mtmtcl::latex}
% 
% \begin{proc}{convertto}
%   Preliminary syntax
%   \begin{displaysyntax}
%     mtmtcl::latex::convertto \word{tree}
%   \end{displaysyntax}
%   \begin{tcl}
proc mtmtcl::latex::convertto {tree} {
   set typified [[namespace parent]::present::typography $tree]
   variable phrasebook
   [namespace parent]::present::decorate typified phrasebook
   fold [
      lindex [metax:: {*}[
         make::tool::make {*}[
            [namespace parent]::present::fence::tool::fence\
              {*}$typified -Inf -Inf none
         ] 0 0 ord
      ]] 0
   ] 70 2 ; # width, level
}
%   \end{tcl}
%   Some timing results for formatting a polynomial with ?? terms 
%   (66585~characters):
%   \begin{center}
%     \begin{tabular}{l r@{}l}
%       \textbf{Operation}& \multicolumn{2}{c}{\textbf{Time} (s)}\\
%       \meta{ring}| export|& 6&222223\\
%       \texttt{mtmtcl::present::typography}& 23&.990825\\
%       \texttt{mtmtcl::present::decorate}& 8&.300743\\
%       \texttt{mtmtcl::present::fence::tool::fence}&
%         7&084628\\
%       \texttt{mtmtcl::latex::make::tool::make}& 17&.474231\\
%       \texttt{mtmtcl::latex::metax::}& 9&.936303\\
%       \texttt{mtmtcl::latex::fold}& $>5000$
%     \end{tabular}
%   \end{center}
%   Apparently it is folding that is the bottleneck here. For massive 
%   data generation, it might be good to have a simpler alternative, 
%   but it is also not clear \emph{why} it is so slow---for fixed 
%   line width, its runtime should be linear!
% \end{proc}
% 
% \begin{arrayvar}{phrasebook}
%   The following array holds basic directions on how to decorate a 
%   near-OpenMath tree before converting it to a \LaTeX\ formula.
%   \begin{tcl}
namespace eval mtmtcl::latex {
   variable phrasebook
   array unset phrasebook
}
%   \end{tcl}
%   Two easy symbols are \OMSref{alg1}{one} and \OMSref{alg1}{zero}.
%   \begin{tcl}
set mtmtcl::latex::phrasebook(alg1#one) {
   OMATTR altenc#LaTeX_encoding [OMSTR 1]
}
set mtmtcl::latex::phrasebook(alg1#zero) {
   OMATTR altenc#LaTeX_encoding [OMSTR 0]
}
%   \end{tcl}
%   
%   The first operation handled is addition. Its formatting is 
%   normally controlled by decorations at the |OMA| node; the 
%   attributes affect fencing and the attribution with a |sumnode| 
%   prefix the actual formatting.
%   \begin{tcl}
set mtmtcl::latex::phrasebook(arith1#plus) {
   rootwards -past {OMATTR 1} -to {OMA 0} -not OM* {
      attrdefault priority 1 form infix
      OMATTR altenc#LaTeX_encoding {
         OMFOREIGN {encoding Tcl-prefix} {
           {#text {sumnode {#text +} {#text -} {#text 0}}}
         }
      }
   }
%   \end{tcl}
%   But in case the symbol occurs in an other capacity than as an 
%   operation acting on operands, a direct symbol is available as 
%   well.
%   \begin{tcl}
   OMATTR altenc#LaTeX_encoding {
      OMSTR {} { {#text +} }
   }
}
%   \end{tcl}
%   The second operation handled is ordinary multiplication. This 
%   is slightly tricky, since there are three different cases to 
%   take into account. First there is the basic associative 
%   multiplication where factors are merely written next to each 
%   other, so the default \describestring+[attribute]{form} should 
%   be \describestring+{juxtaposed}. Second there is the 
%   multiplication of numbers, which has an explicit infix symbol 
%   and which I prefer to render as a |\cdot|; this form of 
%   multiplication has |form|${}={}$\describestring+{infix}. Finally 
%   there is the scalar multiple multiplication, which should perhaps 
%   be encoded using a different symbol altogether, but which for now 
%   is recognised by having an 
%   \describestring+[attribute]{mtmtcl:path} ending in `|.|'; this 
%   gets \describestring+[attribute]{priority} $3.9$ rather than the 
%   juxtaposition default of $4$, to avoid a (usually unnecessary) 
%   fencing of the right operand.
%   \begin{tcl}
set mtmtcl::latex::phrasebook(arith1#times) {
   mtmtcl:path {(?w)^\.\Z} {
      rootwards -past {OMATTR 1} -to {OMA 0} -not OM* {
         attrdefault form juxtaposed priority 3.9
      }
   } default {
      rootwards -past {OMATTR 1} -to {OMA 0} -not OM* {
         attrdefault form juxtaposed
      }
   }
   OMATTR altenc#LaTeX_encoding {
      OMSTR {} { {#text {\cdot}} }
   }
}
%   \end{tcl}
%   The third operation is of course the power. It is virtually 
%   always set as superscripting, so no fallback symbol is provided, 
%   but if one is required then |\mathbin{\arrowup}| is probably the 
%   best alternative.
%   \begin{tcl}
set mtmtcl::latex::phrasebook(arith1#power) {
   rootwards -past {OMATTR 1} -to {OMA 0} -not OM* {
      attrdefault form exponent
   }
}
%   \end{tcl}
%   A symbol that needs rendering, because it can be generated by 
%   preprocessing an |OMF| node, is \OMSref{nums1}{infinity}. 
%   \OMSref{nums1}{NaN} can also be generated that way, but it is 
%   probably better off being rendered in its fallback form anyway.
%   \begin{tcl}
set mtmtcl::latex::phrasebook(nums1#infinity) {
   OMATTR altenc#LaTeX_encoding {
      OMSTR {} { {#text {\infty}} }
   }
}
%   \end{tcl}
%   Negation can also occur as an operation, although it would 
%   typically have been regurgitated by the typographic preprocessing 
%   already.
%   \begin{tcl}
set mtmtcl::latex::phrasebook(arith1#unary_minus) {
   rootwards -past {OMATTR 1} -to {OMA 0} -not OM* {
      attrdefault form prefix
   }
   OMATTR altenc#LaTeX_encoding {
      OMSTR {} { {#text -} }
   }
}
%   \end{tcl}
%   Matrices can be used to test another piece of code.
%   \begin{tcl}
set mtmtcl::latex::phrasebook(linalg2#matrix) {
   rootwards -past {OMATTR 1} -to {OMA 0} -not OM* {
      OMATTR altenc#LaTeX_encoding {
         OMFOREIGN {encoding Tcl-prefix} {
           {#text {mrowjoin {{#text {\begin{pmatrix}}}}\
             {{#text {\end{pmatrix}}}} {} {} {{#text {\\}}}}}
         }
      }
      attrdefault form enclosed
   }
}
set mtmtcl::latex::phrasebook(linalg2#matrixrow) {
   rootwards -past {OMATTR 1} -to {OMA 0} -not OM* {
      OMATTR altenc#LaTeX_encoding {
         OMFOREIGN {encoding Tcl-prefix} {
           {#text {mrowjoin {} {} {} {} {{#text &}}}}
         }
      }
      attrdefault form enclosed
   }
}
%   \end{tcl}
%   \begin{tcl}
%   \end{tcl}
% \end{arrayvar}
% 
% 
% 
% \begin{tcl}
%</pkg>
% \end{tcl}
% 
% 
\endinput

