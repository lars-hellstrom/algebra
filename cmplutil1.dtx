% 
% \iffalse (driver)
%<*driver>
\documentclass[a4paper]{tclldoc}[2007/04/01]
\usepackage{longtable}
\usepackage{amsmath,amssymb,amsfonts,amsthm}

\newtheorem{lemma}{Lemma}[section]
\theoremstyle{definition}
\newtheorem{definition}[lemma]{Definition}
\newtheorem{example}[lemma]{Example}
\theoremstyle{remark}
\newtheorem*{remark}{Remark}
\newtheorem*{algsketch}{Algorithm sketch}

\PageIndex
\CodelineNumbered
\setcounter{IndexColumns}{2}

\newenvironment{ttdescription}{%
  \description
  \def\makelabel##1{\hspace\labelsep\normalfont\ttfamily ##1}%
}{\enddescription}


\makeatletter
\DeclareRobustCommand\SMC{%
  \ifx\@currsize\normalsize\small\else
   \ifx\@currsize\small\footnotesize\else
    \ifx\@currsize\footnotesize\scriptsize\else
     \ifx\@currsize\large\normalsize\else
      \ifx\@currsize\Large\large\else
       \ifx\@currsize\LARGE\Large\else
        \ifx\@currsize\scriptsize\tiny\else
         \ifx\@currsize\tiny\tiny\else
          \ifx\@currsize\huge\LARGE\else
           \ifx\@currsize\Huge\huge\else
            \small\SMC@unknown@warning
 \fi\fi\fi\fi\fi\fi\fi\fi\fi\fi
}
\newcommand{\SMC@unknown@warning}{\TBWarning{\string\SMC: unrecognised
    text font size command -- using \string\small}}
\makeatother

\newcommand{\PROP}{{\SMC PROP}}
\newcommand{\PROPs}{{\SMC PROP}s}
\newcommand{\PRO}{{\SMC PRO}}
\newcommand{\ISBN}{{\SMC ISBN}}
\newcommand{\ISBNX}{{\SMC X}}
\providecommand*{\Dash}{%
   \hspace*{0.166667em}\textemdash\hspace{0.166667em}%
}
\providecommand*{\Ldash}{%
   \hspace{0.166667em}\textemdash\hspace*{0.166667em}%
}
\providecommand*{\dash}{\textendash\hspace{0pt}}

\newcommand{\package}[1]{\textsf{#1}}

\newcommand{\Tcl}{\Tcllogo}
\newcommand*{\DefOrd}[2][]{\textbf{#2}}
\newcommand*{\emDefOrd}[2][]{\emph{#2}}

\newcommand{\mc}{\mathcal}
\newcommand{\N}{\mathbb{N}}
\newcommand{\Zp}{\mathbb{Z}_{>0}}
\newcommand{\Fpil}{\longrightarrow}
\newcommand{\cross}[2]{{}^{#1}\mathsf{X}^{#2}}
\newcommand{\same}[1]{\mathsf{I}^{#1}}

\newcommand{\arity}{\mathrm{ar}}
\newcommand{\coarity}{\mathrm{ca}}
\newcommand{\id}{\mathrm{id}}

\newcommand{\GV}{\mathrm{V}}
\newcommand{\GE}{\mathrm{E}}

\makeatletter
\newcommand*{\setOf}[3][\@gobble]{%
   \left\{ \, #2 \,\,\vrule\relax#1.\,\, #3 \, \right\}%
}
\newcommand*{\card}[2][\@gobble]{\left|#1. #2 \right|}
\makeatother

\newcommand*{\pin}[1]{%
   \mathchoice{%
      \mathrel{\mathrm{in}}%
   }{%
      \mathrel{\mathrm{in}}%
   }{%
      \mathop{\mathrm{in}}%
   }{%
      \mathop{\mathrm{in}}%
   }#1%
}
\newcommand{\restr}[2]{#1_{\lvert #2}}


\begin{document}
\DocInput{cmplutil1.dtx}
\end{document}
%</driver>
% \fi
% 
% \title{Completing a set of \PROP py identities}
% \author{Lars Hellstr\"om}
% \maketitle
% 
% \begin{abstract}
%   This is a simple utility for completing sets of \PROP\ 
%   identities, utilising the \textsf{network} package version 2. 
%   Hopf algebra identities are provided as examples.
% \end{abstract}
% 
% \tableofcontents
% 
% 
% \section{A completion utility}
% \label{Sec:Tillslutning}
% 
% This section contains the implementation of a small utility for 
% computing the completion of a system of substitution rules. In 
% preparation for expected developments, the rules are lists
% \begin{quote}
%   \word{left hand side (wfb)} \word{left hand side (rich)}\
%   \word{right hand side} \word{derivation}\regopt
% \end{quote}
% where the first left hand side is in the form of a network with 
% feedback\Ldash that determines the sort of the entire rule\Dash and 
% the second left hand side is the same as a rich network. The 
% \word{right hand side} is going to be a linear combination of 
% networks, but currently it defaults to being a pair
% \begin{quote}
%   \word{canonical} \word{rich}
% \end{quote}
% where \word{canonical} is the canonical form of the right hand 
% side network and \word{rich} is a corresponding rich network. The 
% utility can thus only handle the special case that every element 
% considered has one coefficient equal to $1$ and all other 
% coefficients equal to $0$.
% 
% If included, the \word{derivation} is a recording of the steps in 
% which this rule was derived (rules given as axioms don't have 
% derivations). The format of a derivation is as a list
% \begin{quote}
%   \begin{regblock}[\regstar]\word{rule spec}
%   \word{rich network}\end{regblock}
% \end{quote}
% where each \word{rich network} is a step in the derivation; the 
% last one is the \word{right hand side (rich)} whereas the 
% \word{left hand side (rich)} is omitted since it is readily 
% available elsewhere.\footnote{For more serious work, it might 
% actually be better to record only the pure networks, but right now 
% rich networks are easier.} A \word{rule spec} is one of
% \begin{quote}
%   |backward| \word{rule no.}\\
%   |forward| \word{rule no.}
% \end{quote}
% where |backward| items mean that the right hand side of the rule 
% is being replaced by the left hand side, and |forward| items is the 
% normal way around. It is all |backward| until one reaches the site 
% of the ambiguity, and after that it is all |forward|.
% 
% \setnamespace{}
% \begin{tcl}
%<*util1>
package require Tk
package require network 2
% \end{tcl}
% 
% The following deals with a problem in the Aqua versions of Tk: 
% short curves drawn as many small line segments become horribly 
% jagged.
% \begin{tcl}
namespace eval tk::mac {variable CGAntialiasLimit 1}
% \end{tcl}
% 
% 
% \subsection{User interface}
% 
% The previous incarnation of this completion procedure was only meant 
% to be semi-automatic, so user interaction was an important part of 
% the program. This incarnation should do much better, but it still 
% retains all the GUI elements of the predecessor.
% 
% 
% \begin{proc}{new_ambiguity_window}
%   The |new_ambiguity_window| procedure creates a new window in which 
%   one can show an ambiguity and its resolution. There syntax is
%   \begin{quote}
%     |new_ambiguity_window| \word{title}\regopt
%   \end{quote}
%   where \word{title} is a title for the window. The return value is 
%   the name of the |toplevel| created.
%   
%   \begin{variable}{last_ambiguity_window}
%     The |last_ambiguity_window| variable is a number that was used in 
%     the name of the last ambiguity window. It is incremented with 
%     every creation of an ambiguity resolution window.
%     \begin{tcl}
set last_ambiguity_window 0
%     \end{tcl}
%   \end{variable}
%   The layout of the window created is roughly
%   \begin{center}
%     \begin{tabular}{ccc}
%       \textbf{Left}& \textbf{Top}& \textbf{Right}\\
%       \framebox[0.25\linewidth]{Network $1$\vrule width 0pt
%         height 10ex depth 9ex}&
%         \framebox[0.25\linewidth]{Network $0$\vrule width 0pt 
%         height 10ex depth 9ex}&
%         \framebox[0.25\linewidth]{Network $2$\vrule width 0pt 
%         height 10ex depth 9ex}\\
%       \meta{button}& \meta{message}&\meta{button}
%     \end{tabular}
%   \end{center}
%   where the middle network $0$ is the ambiguity and the flank networks 
%   $1$ and $2$ are things to which it can be reduced. The buttons are 
%   for choosing which of the two that will be made the left hand side 
%   of a new rule.
%   
%   This procedure also creates a ``middle button'' |bm| which is 
%   used for cases where neither the left nor the right button would 
%   do the right thing. This button is initially unmapped, but so 
%   anyone wanting to make it visible should supply the relevant 
%   |grid| command.
%   \begin{tcl}
proc new_ambiguity_window {{title {}}} {
   variable last_ambiguity_window
   incr last_ambiguity_window
   set t [toplevel .ambiguity$last_ambiguity_window]
   if {![string length $title]} then {
      set title "Ambiguity $last_ambiguity_window"
   }
   wm title $t $title
   bindtags $t [linsert [bindtags $t] 1 Ambiguity]
   label $t.hl -text "Left"
   label $t.hm -text "Top"
   label $t.hr -text "Right"
   grid $t.hl -row 0 -column 0 -sticky s
   grid $t.hm -row 0 -column 1 -sticky s
   grid $t.hr -row 0 -column 2 -sticky s
   foreach c [list $t.cl $t.cm $t.cr] {
      canvas $c -width 200 -height 400 -relief solid -borderwidth 1\
        -xscrollincrement 1 -yscrollincrement 1
   }
   grid $t.cl -row 1 -column 0 -sticky nswe
   grid $t.cm -row 1 -column 1 -sticky nswe
   grid $t.cr -row 1 -column 2 -sticky nswe
   button $t.bl -text "Reduce" -state disabled
   button $t.bm -text "Change this text"
   button $t.br -text "Reduce" -state disabled
   grid $t.bl -row 2 -column 0 -sticky n
   grid $t.br -row 2 -column 2 -sticky n
   label $t.lm -justify center -text {}
   grid $t.lm -row 2 -column 1 -sticky nwe
   grid rowconfigure $t 1 -weight 1
   grid columnconfigure $t 0 -weight 1
   grid columnconfigure $t 1 -weight 1
   grid columnconfigure $t 2 -weight 1
   foreach c [list $t.cl $t.cm $t.cr] {
      bindtags $c [linsert [bindtags $c] 1 Network]
   }
%<*test3>
   bind $t <FocusIn> [list puts stdout "$t got focus."]
   bind $t <FocusOut> [list puts stdout "$t lost focus."]
%</test3>
   return $t
}
%   \end{tcl}
% \end{proc}
% 
% The \describestring[bindtag]{Network}|Network| bindtag is used for 
% all canvases displaying networks. It offers bindings for grabbing the 
% network with the mouse and moving it around. The |-cursor| 
% man\oe uvres are meant to isolate this use of the mouse from that 
% of clicking hyperlinks, which is supported by the rule browser.
% \begin{tcl}
bind Network <ButtonPress-1> {
   if {[%W cget -cursor] eq ""} then {
      %W configure -cursor fleur
      %W scan mark %x %y
   }
}
bind Network <B1-Motion> {
   if {[%W cget -cursor] eq "fleur"} then {%W scan dragto %x %y 1}
}
bind Network <ButtonRelease-1> {
   if {[%W cget -cursor] eq "fleur"} then {
      %W configure -cursor ""
   }
}
bind Network <B1-Leave> {%W configure -cursor ""}
% \end{tcl}
% 
% \begin{proc}{center_network}
%   This procedure scrolls a canvas so that the contents appear in 
%   the center of the visible region. The call syntax is
%   \begin{quote}
%     |center_network| \word{canvas} \word{itemOrId}\regopt
%   \end{quote}
%   where \word{itemOrId} selects the items to center. If not 
%   specified, then this defaults to |all|.
%   \begin{tcl}
proc center_network {c {id all}} {
   foreach {left top right bottom} {0 0 0 0} break
   foreach {left top right bottom} [$c bbox $id] break
%<debug>   puts "In $c center $id:($left,$top,$right,$bottom)"
   $c scan mark [expr {($right+$left)/2}] [expr {($bottom+$top)/2}]
   set w [winfo width $c]
   if {$w < 10} then {set w [winfo reqwidth $c]}
   set h [winfo height $c]
   if {$h < 10} then {set h [winfo reqheight $c]}
   $c scan dragto [
      expr {round([$c canvasx [expr {$w/2}]])}
   ] [
      expr {round([$c canvasy [expr {$h/2}]])}
   ] 1
%<debug>   puts "TopLeft is ([$c canvasx 0],[$c canvasy 0])."
}
%   \end{tcl}
% \end{proc}
% 
% \begin{arrayvar}{after_id}
%   The |after_id| array is used for storing |after| identifiers which 
%   are related to a particular ambiguity. Every index should have the 
%   name of the toplevel window to which they belong as prefix. It is 
%   recommended that further qualifications use widget names as far as 
%   these are able to distinguish matters, and then use |'| as 
%   delimiter for further separation.
%   
%   See |idle_reschedule| for an example of a procedure that uses 
%   this array.
% \end{arrayvar}
% 
% \begin{proc}{ambiguity_cancel}
%   The |ambiguity_cancel| procedure cancels the |after| script(s) for a 
%   toplevel, if there are any. The syntax is
%   \begin{quote}
%     |ambiguity_cancel| \word{toplevel}
%   \end{quote}
%   It is meant to be called from an envent binding, e.g.~|FocusOut|.
%   \changes{2}{2007/02/14}{Modified to handle the case that a 
%      toplevel name is a prefix of another toplevel name. (LH)}
%   \begin{tcl}
proc ambiguity_cancel {t} {
   global after_id
   foreach i [
     concat [list $t] [array names after_id "${t}.*"]\
       [array names after_id "${t}'*"]
   ] {
      after cancel $after_id($i)
      unset after_id($i)
   }
}
%   \end{tcl}
% \end{proc}
% 
% The \describestring[bindtag]{Ambiguity}|Ambiguity| bindtag is used for 
% toplevels displaying ambiguities. It is used for controlling which 
% ambiguity is being processed. An ambiguity whose window loses focus 
% should seize all processing.
% \begin{tcl}
bind Ambiguity <FocusOut> {ambiguity_cancel %W}
% \end{tcl}
% It seems toplevels have a tendency to wander off the screen, so the 
% ambiguity windows can do with a |<Configure>| binding to correct that.
% \begin{tcl}
bind Ambiguity <Configure> {ensure_on_screen %W}
% \end{tcl}
% 
% \begin{proc}{ensure_on_screen}
%   This procedure moves a toplevel to ensure that it is completely on 
%   its screen. The argument is the name of the toplevel.
%   \begin{tcl}
proc ensure_on_screen {w} {
   if {$w ne [winfo toplevel $w]} then {return}
   set extra_width\
     [expr {[winfo vrootwidth $w] - [winfo reqwidth $w]}]
   set extra_height\
     [expr {[winfo vrootheight $w] - [winfo reqheight $w]}]
   set x [winfo rootx $w]
   set y [winfo rooty $w]
   set moved 0
   if {$extra_width > 0 && $x > $extra_width} then {
      set x [expr {$x - $extra_width}]
      set moved 1
   }
   if {$extra_height > 0 && $y > $extra_height} then {
      set y [expr {$y - $extra_height}]
      set moved 1
   }
   if {$moved} then {
      wm geometry $w +${x}+${y}
   }
}
%   \end{tcl}
% \end{proc}
% 
% \begin{proc}{create_controller}
%   The |create_controller| procedure creates a controller window, for 
%   controlling how the processing of ambiguities are run. The syntax is
%   \begin{quote}
%     |create_controller| \word{major-var} \word{minor-var}
%   \end{quote}
%   where \word{major-var} and \word{minor-var} are names of two global 
%   variables which are linked to |entry| widgets.
%   \begin{tcl}
proc create_controller {major_var minor_var} {
   set t [toplevel .controller]
   label $t.ll -text "Left rule:"
   label $t.lr -text "Right rule:"
   entry $t.el -textvariable $minor_var
   entry $t.er -textvariable $major_var
   foreach c [list $t.cl $t.cr] {
      canvas $c -width 200 -height 400 -relief solid -borderwidth 1
      bindtags $c [linsert [bindtags $c] 1 Network]
   }
   grid $t.cl - -row 0 -column 0 -sticky nsew
   grid $t.cr - -row 0 -column 2 -sticky nsew
   grid rowconfigure $t 0 -weight 1
   grid $t.ll -row 1 -column 0 -sticky e
   grid $t.el -row 1 -column 1 -sticky w
   grid $t.lr -row 1 -column 2 -sticky e
   grid $t.er -row 1 -column 3 -sticky w
   foreach i {0 1 2 3} {grid columnconfigure $t $i -weight 1}
   button $t.b_go -text Go
   grid $t.b_go -row 2 -column 3
   button $t.b_next -text Next
   grid $t.b_next -row 2 -column 0
   checkbutton $t.check_auto -text "Auto Next Go"\
     -variable $t.check_auto
   grid $t.check_auto - -row 2 -column 1 -sticky e
}
%   \end{tcl}
% \end{proc}
% 
% \begin{variable}{.controller.check_auto}
%   This is the value variable of the \textsf{Auto Next Go} 
%   checkbutton.
% \end{variable}
% 
% \begin{arrayvar}{Network}
%   The |Network| array is used for storing networks directly associated 
%   with some widget. The indices are widget names (paths). The entries 
%   are rich networks.
% \end{arrayvar}
% 
% \begin{variable}{vertex_appearances}
% \begin{variable}{vertex_sizes}
%   The |vertex_appearances| variable holds the dictionary of 
%   appearances for the vertex types currently occuring in the rule 
%   networks. It is used by those procedures that draw or 
%   redraw networks. The |vertex_sizes| variable is a list patterns 
%   and vertex sizes for use with |make_level-layout-Tk|.
%   \begin{tcl}
set vertex_sizes {* {20 20}}
%   \end{tcl}
% \end{variable}\end{variable}
% 
% 
% \begin{proc}{show_sparsest_cut}
%   This procedure exists to examine the capabilities of the 
%   |network::pure::sparsest_cut| procedure. The call syntax is
%   \begin{quote}
%     |show_sparsest_cut| \word{canvas}
%   \end{quote}
%   and this does the following: (i)~look up the network shown in the 
%   \word{canvas}, (ii)~compute its sparsest cut region, (iii)~update 
%   the display of the network with this region, and (iv)~report the 
%   time taken to |stdout|. Nothing is done if the \word{canvas} does 
%   not have a |Network| entry.
%   \begin{tcl}
proc show_sparsest_cut {c} {
   global Network vertex_appearances
   if {![info exists Network($c)]} then {return}
   array set NW $Network($c)
   set time [time {set region [
      network::pure::sparsest_cut $NW(pure)
   ]}]
   if {[llength $region]} then {
      network::rich::drawit $c $vertex_appearances {} -update 1\
        -var NW -region $region
   }
   puts [format {Vertices: %d/%d. Boundary: %d. %s}\
     [llength [lindex $region 0]] [llength [lindex $NW(pure) 0]]\
       [llength [concat [lindex $region 1] [lindex $region 2]]] $time]
}
%<showcut>bind Network <Command-ButtonPress-1> {show_sparsest_cut %W}
%   \end{tcl}
% \end{proc}
% 
% 
% 
% \subsection{Reduction}
% 
% \begin{variable}{rulesL}
%   The |rulesL| variable contains the current list of reduction rules. 
%   New rules are appended at the end. Rules are never removed.
% \end{variable}
% 
% \begin{variable}{reduce_stats}
%   For the purpose of speeding up the reduction process, the normal 
%   reduction procedure requires an auxilliary table of how often the 
%   various rules have been used just after another rule. This table is 
%   implemented as a list with one elment per rule, where each list 
%   element is a list with the structure
%   \begin{quote}
%     \word{rule no.} \word{after $0$} \word{after $1$} \word{after $2$} 
%     \dots
%   \end{quote}
%   where each \word{after $i$} is the number of times this rule could be 
%   applied following an application of rule $i$. These numbers would be 
%   updated whenever a reduction was made.
%   
%   The table is stored in the variable |reduce_stats|.
% \end{variable}
% 
% \begin{proc}{repair_reduce_stats}
%   When importing a dumped set of rules, the |reduce_stats| and 
%   |rulesL| variables can get out of sync. This procedure extends 
%   |reduce_stats| to be consistent with |rulesL|. There are no 
%   variables, nor any particular return value.
%   \changes{2.1}{2007/12/19}{Procedure added. (LH)}
%   \begin{tcl}
proc repair_reduce_stats {} {
   global reduce_stats rulesL
   set newL {}
   foreach line $reduce_stats {
      set Seen([lindex $line 0]) ""
      while {[llength $line] <= [llength $rulesL]} {lappend line 0}
      lappend newL $line
   }
   for {set n 0} {$n < [llength $rulesL]} {incr n} {
      if {[info exists Seen($n)]} then {continue}
      set line [list $n]
      while {[llength $line] <= [llength $rulesL]} {lappend line 0}
      lappend newL $line
   }
   set reduce_stats $newL
   return
}
%   \end{tcl}
% \end{proc}
% 
% \begin{proc}{add_rule}
%   The |add_rule| procedure adds a rule to the list of rules and 
%   extends the table of reduction statistics accordingly. The syntax 
%   is
%   \begin{displaysyntax}
%     |add_rule| \word{feedbacks} 
%     \begin{regblock}|pure|\regalt|rich|\end{regblock}
%     \word{left hand side}
%     \begin{regblock}|pure|\regalt|rich|\end{regblock}
%     \word{right hand side}
%     \word{target-var}\regopt
%   \end{displaysyntax}
%   where the \word{left hand side} and \word{right hand side} are 
%   pure or rich networks, according to the indication in the 
%   preceeding argument. \word{feedbacks} is the list of feedbacks 
%   for this rule. The return value is the number of the new rule.
%   
%   If a \word{target-var} is given, then it is the name in the 
%   calling context of a variable to which the rule will be 
%   |lappend|ed instead of |rulesL|. In this case, the reduction 
%   statistics tables are left alone.
%   \begin{tcl}
proc add_rule {feedbackL type_left left type_right right {tovar ""}} {
   global rulesL rule_queue reduce_stats vertex_appearances\
     vertex_sizes
   switch -- $type_left "pure" {
      set NW1(pure) $left
   } "rich" {
      array set NW1 $left
   } default {
      error "type must be 'pure' or 'rich'"
   }
   if {![info exists NW1(vpos-Tk)] || ![info exists NW1(ecurve-Tk-tt)]}\
   then {
      if {![info exists NW1(level-layout-Tk)]} then {
         network::rich::make_level-layout-Tk $vertex_sizes -var NW1
      }
      if {![info exists NW1(vpos-Tk)]} then {
         network::rich::vpos-Tk_by_level -var NW1
      }
      if {![info exists NW1(ecurve-Tk-tt)]} then {
         network::rich::ecurve-Tk-tt_by_level $vertex_appearances\
           -var NW1
      }
   }
   if {![info exists NW1(canonical)]} then {
      set NW1(canonical) [network::pure::canonise $NW1(pure)]
   }
   switch -- $type_right "pure" {
      set NW2(pure) $right
   } "rich" {
      array set NW2 $right
   } default {
      error "type must be 'pure' or 'rich'"
   }
   if {![info exists NW2(vpos-Tk)] || ![info exists NW2(ecurve-Tk-tt)]}\
   then {
      if {![info exists NW2(level-layout-Tk)]} then {
         network::rich::make_level-layout-Tk $vertex_sizes -var NW2
      }
      if {![info exists NW2(vpos-Tk)]} then {
         network::rich::vpos-Tk_by_level -var NW2
      }
      if {![info exists NW2(ecurve-Tk-tt)]} then {
         network::rich::ecurve-Tk-tt_by_level $vertex_appearances\
           -var NW2
      }
   }
   if {![info exists NW2(canonical)]} then {
      set NW2(canonical) [network::pure::canonise $NW2(pure)]
   }
   set rule [list [
      list $NW1(pure) $feedbackL
   ] [
      array get NW1
   ] [
      list $NW2(canonical) [array get NW2]
   ]]
   if {$tovar ne ""} then {
      upvar 1 $tovar toL
      set index [llength $toL]
      lappend toL $rule
      return $index
   }
   cull_rules [list $NW1(pure) $feedbackL]
   set index [llength $rulesL]
   lappend rulesL $rule
   set new_stats {}
   foreach row $reduce_stats {
      lappend row 0
      lappend new_stats $row
   }
   lappend new_stats\
     [linsert [string repeat " 0" [llength $rulesL]] 0 $index]
   set reduce_stats $new_stats
   lappend rule_queue [list $index\
     [score_network $NW1(pure) $feedbackL]]
   return $index
}
%   \end{tcl}
% \end{proc}
% 
% \begin{proc}{score_network}
%   This procedure computes the sort key used in the |rule_queue|. 
%   The call syntax is
%   \begin{displaysyntax}
%     |score_network| \word{network} \word{feedbacks}
%   \end{displaysyntax}
%   The default definition just returns the number of vertices, but 
%   it can be overridden for specific problems.
%   \begin{tcl}
proc score_network {NW fbL} {llength [lindex $NW 0]}
%   \end{tcl}
% \end{proc}
% 
% 
% \begin{proc}{make_rule_derivation}
%   This procedure constructs a \word{derivation} section to for a 
%   rule, but doesn't modify the |rulesL|. The call syntax is
%   \begin{quote}
%     |make_rule_derivation| \word{backwards} \word{top} 
%     \word{forwards}
%   \end{quote}
%   where the \word{backwards} argument are the steps in the 
%   derivation which will be |backward|s, the \word{top} is the rich 
%   network that serves as turning point, and the \word{forwards} 
%   is the list of |forward| steps in the derivation.
%   \begin{tcl}
proc make_rule_derivation {bwL top fwL} {
   set res {}
   for {set n [llength $bwL]} {$n>0} {} {
      set step [lindex $bwL [incr n -1]]
      set rule [lindex $bwL [incr n -1]]
      lappend res $step [list backward $rule]
   }
   lappend res $top
   foreach {rule step} $fwL {
      lappend res [list forward $rule] $step
   }
   return [lrange $res 1 end]
}
%   \end{tcl}
% \end{proc}
% 
% \begin{proc}{add_rule_derivation}
%   This procedure adds a \word{derivation} section to an existing 
%   rule. The call syntax is
%   \begin{quote}
%     |add_rule_derivation| \word{to-no.} \word{backwards} \word{top} 
%     \word{forwards}
%   \end{quote}
%   where \word{to-no.} is the number of the rule to modify. This 
%   number is also the return value. The \word{backwards} argument 
%   are the steps in the derivation which will be |backward|s, the 
%   \word{top} is the rich network that serves as turning point, and 
%   the \word{forwards} is the list of |forward| steps in the 
%   derivation.
%   \begin{tcl}
proc add_rule_derivation {target bwL top fwL} {
   global rulesL
   set L [lrange [lindex $rulesL $target] 0 2]
   lappend L [make_rule_derivation $bwL $top $fwL]
   lset rulesL $target $L
   return $target
}
%   \end{tcl}
% \end{proc}
% 
% \begin{proc}{unmake_rule_derivation}
%   This procedure does the opposite of |make_rule_derivation|, i.e., 
%   it reconstructs the
%   \begin{quote}
%     \word{backwards} \word{top} \word{forwards}
%   \end{quote}
%   from a rule with derivation. The call syntax is
%   \begin{quote}
%     |unmake_rule_derivation| \word{rule}
%   \end{quote}
%   and the return value is the three element list shown above.
%   \begin{tcl}
proc unmake_rule_derivation {rule} {
   set res {}
   set fwL {}
   set n -1
   foreach {rel step} [lindex $rule 3] {
      if {[lindex $rel 0] eq "backward"} then {incr n 2} else {
         lappend fwL [lindex $rel 1] $step
      }
   }
   set bwL {}
   for {} {$n>0} {incr n -2} {
      lappend bwL [lindex $rule 3 $n] [lindex $rule 3 [expr {$n-1}] 1]
   }
   lappend bwL [lindex $rule 1]
   return [list [lrange $bwL 1 end] [lindex $bwL 0] $fwL]
}
%   \end{tcl}
% \end{proc}
% 
% 
% \begin{variable}{equalitiesL}
%   This variable has the same structure as |rulesL|, although it is 
%   for equalities that could not be turned into rules. The choice of 
%   what to label as the ``left'' or ``right'' hand side is arbitrary.
% \end{variable}
% 
% \begin{arrayvar}{equality_side}
%   A problem with the equalities is that there are usually several 
%   ways to derive them, so if one simply adds everything derived 
%   that cannot be turned into a rule to |equalitiesL| then there 
%   will be duplicate entries. The |equality_side| array maps sides 
%   of an equality to the location in |equalitiesL| where they occur, 
%   so that such rederivation can be detected. The indices are lists 
%   with the structure
%   \begin{quote}
%     \word{canonical pure network} \word{feedback-list}
%   \end{quote}
%   where the \word{canonical pure network} is what it says it is, 
%   and the \word{feedback-list} is the |lsort -dictionary| sorted 
%   list of feedbacks. The entries are lists, where each element is a 
%   pair
%   \begin{quote}
%     \word{equality number} \word{side}
%   \end{quote}
%   where the \word{side} is |0| for the left hand side and |1| for 
%   the right hand side. The reason the entries are lists is the 
%   chance that more than two networks can be found equal.
% \end{arrayvar}
% 
% \begin{variable}{remaining_equalities}
%   Since |equality_side| maps networks to equality numbers, the 
%   equality numbers should be fixed, and hence it is not possible to 
%   remove elements from |equalitiesL|. Since equalities may still be 
%   dropped if a new rule applies to them, the |remaining_equalities| 
%   list stores the numbers of those equalities which remain to 
%   resolve.
% \end{variable}
% 
% \begin{proc}{add_equality}
%   This procedure adds an entry to |equalitiesL|. The call syntax is
%   \begin{quote}
%     |add_equality| \word{feedbacks} \word{backwards} \word{top}
%     \word{forwards} \word{reason}
%   \end{quote}
%   The \word{feedbacks} is the feedback type for this equality. 
%   The \word{backwards} argument are the steps in the derivation 
%   which will be |backward|s, the \word{top} is the rich network that 
%   serves as turning point, and the \word{forwards} is the list of 
%   |forward| steps in the derivation. The \word{reason} is the 
%   reason this was turned into an equality, typically the result of 
%   comparing the two sides of it.
%   \changes{2.2}{2009/02/10}{Modified to notice duplicate 
%     equalities. Moved \texttt{stdout} messages here. (LH)}
%   \begin{tcl}
proc add_equality {feedbackL bwL top fwL reason} {
   global equalitiesL equality_side remaining_equalities
   array set NW1 [lindex $bwL end]
   array set NW2 [lindex $fwL end]
   set NW1c [network::pure::canonise $NW1(pure)]
   set NW2c [network::pure::canonise $NW2(pure)]
   set feedbackLc [lsort -dictionary $feedbackL]
   set idx1 [list $NW1c $feedbackLc]
   set idx2 [list $NW2c $feedbackLc]
   if {[info exists equality_side($idx1)] &&\
     [info exists equality_side($idx2)]} then {
      foreach pair $equality_side($idx1) {
         set S([lindex $pair 0]) [lindex $pair 1]
      }
      foreach pair $equality_side($idx2) {
         if {[info exists S([lindex $pair 0])]} then {
%<*stdout>
            puts stdout "Rediscovered equality [lindex $pair 0]."
%</stdout>
            return
         }
      }
   }
   set num [llength $equalitiesL]
   lappend equality_side($idx1) [list $num 0]
   lappend equality_side($idx2) [list $num 1]
   set eq [list [list $NW1(pure) $feedbackL] [lindex $bwL end]]
   lappend eq [list $NW2c [lindex $fwL end]]
   lappend eq [make_rule_derivation $bwL $top $fwL]
   lappend equalitiesL $eq
   lappend remaining_equalities $num
%<*stdout>
   puts stdout "Added equality $num ($reason)."
%</stdout>
   return $num
}
%   \end{tcl}
% \end{proc}
% 
% \begin{proc}{rule_children}
%   This procedure lists the numbers and equalities that are direct 
%   children of a specified rule, i.e., the given rule was one of 
%   those that gave rise to the underlying ambiguity. The call syntax 
%   is
%   \begin{displaysyntax}
%     |rule_children| \word{rule number}
%   \end{displaysyntax}
%   and the result is a string, with one line per child.
%   \begin{tcl}
proc rule_children {parent} {
   global rulesL equalitiesL
   set res {}
   foreach list [list $rulesL $equalitiesL] prefix {"" e} {
      set n -1; foreach rule $list {incr n
         set line $prefix$n
         set bwL {}; set fwL {}
         foreach {spec NW} [lindex $rule 3] {
            switch -- [lindex $spec 0] "forward" {
               lappend fwL [lindex $spec 1]
            } "backward" {
               lappend bwL [lindex $spec 1]
            }
         }
         if {[lindex $fwL 0] == $parent} then {
            append line " with [lindex $bwL end]"
            lappend res $line
         } elseif {[lindex $bwL end] == $parent} then {
            append line " with [lindex $fwL 0]"
            lappend res $line
         }
      }
   }
   return [join $res \n]
}
%   \end{tcl}
% \end{proc}
% 
% \begin{proc}{clear_rules}
%   The |clear_rules| procedure clears the lists of rules and statistics 
%   of rules. It takes no arguments.
%   \begin{tcl}
proc clear_rules {} {
   global rulesL reduce_stats rule_base rule_queue equalitiesL\
     remaining_equalities browser_fwL browser_bwL
   set rulesL [list]
   set reduce_stats [list]
   set rule_base [list]
   set rule_queue [list]
   set equalitiesL {}
   set remaining_equalities {}
   set browser_fwL {}
   set browser_bwL {}
}
%   \end{tcl}
% \end{proc}
% 
% \begin{proc}{cull_rules}
%   The |cull_rules| procedure searches through |reduce_stats| for rules 
%   that are should be removed from active use because their principal 
%   parts can be reduced by another rule. The syntax is
%   \begin{quote}
%     |cull_rules| \word{network with feedback}
%   \end{quote}
%   \begin{tcl}
proc cull_rules {NW} {
   global reduce_stats rulesL
   set new_statsL [list]
   foreach row $reduce_stats {
      if {[llength [
         network::wfb::instances [lindex $rulesL [lindex $row 0] 0] $NW
      ]]} then {
%<stdout>         puts stdout "Dropped rule [lindex $row 0]."
      } else {
         lappend new_statsL $row
      }
   }
   set reduce_stats $new_statsL
}
%   \end{tcl}
% \end{proc}
% 
% \begin{arrayvar}{Steps}
%   This array records the reduction steps that have been carried out 
%   in a |Network| canvas. The index is the name of the canvas, 
%   whereas an entry is a list
%   \begin{quote}
%     \begin{regblock}[\regstar]\word{rule no.} 
%     \word{rich network}\end{regblock}
%   \end{quote}
%   where the \word{rich network} is what appeared in the canvas 
%   after applying the corresponding rule. Later results are appended 
%   to the end of this list.
% \end{arrayvar}
% 
% 
% \begin{proc}{graphic_reduce}
%   The |graphic_reduce| procedure reduces a network one step, if 
%   possible, while updating its graphical appearance on a canvas. The 
%   syntax is
%   \begin{quote}
%     |graphic_reduce| \word{network-var} \word{feedbacks}
%     \word{canvas} \word{last rule}
%   \end{quote}
%   where \word{network-var} is the name of a variable in the local 
%   context of the caller. The \word{network-var} is assumed to contain 
%   a rich network (as a dict, not unrolled into an array) whose 
%   feedbacks are \word{feedbacks}, which is currently drawn on the 
%   specified \word{canvas}. The procedure will try to reduce the 
%   network, starting with the assumption that \word{last rule} is the 
%   number of the most recently applied rule, and if it succeeds update 
%   the |reduce_stats| variable. In the case of success it will also 
%   replace the network on the canvas by the reduced form.
%   
%   The return value is the number of the rule that was applied, or |-1| 
%   if no rule could be applied. The \word{last rule} upon calling may 
%   be |-1|.
%   
%   \begin{tcl}
proc graphic_reduce {NWvar fbL canvas last} {
   global rulesL reduce_stats vertex_appearances vertex_sizes Steps
   array set NW [uplevel 1 [list ::set $NWvar]]
   set reduce_stats\
     [lsort -index [expr {$last+1}] -decreasing -integer $reduce_stats]
   set sites {}
   set i -1; foreach t $reduce_stats {incr i
      set n [lindex $t 0]
      set sites [
         network::wfb::instances [list $NW(pure) $fbL]\
           [lindex $rulesL $n 0] 1
      ]
      if {[llength $sites]} then {break}
   }
   if {![llength $sites]} then {return -1}
   if {$last>=0} then {
      lset reduce_stats $i [expr {$last+1}]\
        [expr {[lindex $reduce_stats $i [expr {$last+1}]] +1}]
   }
   set NW(pure) [network::pure::replace $NW(pure) [lindex $sites 0]\
     [lindex $rulesL $n 2 0]]
   network::rich::make_level-layout-Tk $vertex_sizes -var NW
   network::rich::vpos-Tk_by_level -var NW
   network::rich::ecurve-Tk-tt_by_level $vertex_appearances -var NW
   $canvas delete all
   network::rich::drawit $canvas $vertex_appearances {} -var NW
   center_network $canvas
   lappend Steps($canvas) $n [array get NW]
   uplevel 1 [list ::set $NWvar [lindex $Steps($canvas) end]]
   return $n
}
%   \end{tcl}
% \end{proc}
% 
% \begin{arrayvar}{lastrule}
%   The |lastrule| array stores the number of the last reduction rule 
%   that was applied to the network in a particular canvas. The index 
%   is the canvas name (widget path).
% \end{arrayvar}
% 
% \begin{arrayvar}{feedbacks}
%   The |feedbacks| array stores the sorts (feedbacks-lists) of 
%   networks that appear in windows. The index is the name of a 
%   toplevel.
% \end{arrayvar}
% 
% \begin{proc}{draw_network_feedbacks}
%   This procedure draws the feedbacks for a network and optionally 
%   recenters it in its canvas. The call syntax is
%   \begin{quote}
%     |draw_network_feedbacks| \word{canvas} \word{feedback-list} 
%     \word{rich network} \word{center?} 
%     \begin{regblock}[\regstar]\word{option} 
%     \word{value}\end{regblock}
%   \end{quote}
%   where \word{center?} is a boolean for whether to center while the 
%   options and values are passed on to 
%   |network::rich::draw_feedbacks|. As a special case, the |-tags| 
%   option is interpreted here (as it is a separate argument to the 
%   underlying procedure).
%   
%   \begin{tcl}
proc draw_network_feedbacks {c fbL NWval center args} {
   array set NW $NWval
   set inL {}
   foreach item [lindex $NW(level-layout-Tk) 0 end-2] {
      if {[lindex $item 0] eq "edge"} then {
         lappend inL [lindex $item 2]
      }
   }
   set outL {}
   foreach item [lindex $NW(level-layout-Tk) 0 0] {
      if {[lindex $item 0] eq "edge"} then {
         lappend outL [lindex $item 2]
      }
   }
   set Opt(-tags) ""
   array set Opt $args
   eval [list network::rich::draw_feedbacks $c $Opt(-tags) $fbL\
     -inx $inL -outx $outL -bbox [lindex $NW(level-layout-Tk) 1]]\
     $args
   if {$center} then {center_network $c}
}
%   \end{tcl}
% \end{proc}
% 
% 
% \begin{variable}{count_down}
%   The |count_down| variable is used for counting down to zero, in the 
%   delay before automatically closing an ambiguity window.
% \end{variable}
% 
% \begin{proc}{run_reduction}
%   The |run_reduction| procedure runs the step-wise reduction of an 
%   ambiguity, and is meant to be run as an |after| script. The syntax is
%   \begin{quote}
%     |run_reduction| \word{toplevel} \word{mode}
%   \end{quote}
%   where \word{toplevel} is the name of the toplevel window of the 
%   ambiguity. \word{mode} is one of:
%   \begin{enumerate}
%     \item[\texttt{cl}]
%       Reduce network in the left canvas.
%     \item[\texttt{cr}]
%       Reduce network in the right canvas.
%     \item[\texttt{equal}]
%       The left and right networks have been reduced to equal networks. 
%       When |count_down| reaches $0$, the toplevel will be destroyed.
%     \item[\texttt{autoleft}]
%       The left and right networks have been fully reduced, but are not 
%       equal. When |count_down| reaches zero, the left network will be 
%       chosen as the principal part of the new rule manufactured from 
%       this ambiguity.
%     \item[\texttt{autoright}]
%       The left and right networks have been fully reduced, but are not 
%       equal. When |count_down| reaches zero, the right network will be 
%       chosen as the principal part of the new rule manufactured from 
%       this ambiguity.
%     \item[\texttt{automiddle}]
%       The left and right networks have been fully reduced, but 
%       turned out to be incomparable. In this case, there will be a 
%       middle button which manufactures an equality from the left 
%       and right hand sides.
%       Previously, this case was called \texttt{incomparable} and 
%       asked the user to select one of three buttons.
%   \end{enumerate}
%   
%   \begin{tcl}
proc run_reduction {t mode} {
   global after_id lastrule feedbacks Network count_down
   if {$mode eq "cl"} then {
      set c $t.cl
      if {![info exists Network($c)]} then {
         puts "Expected network for $c, but found none (hickup?)."
         return
      }
      set rule [
         graphic_reduce Network($c) $feedbacks($t) $c $lastrule($c)
      ]
      if {$rule >= 0} then {
         set lastrule($c) $rule
         set after_id($t) [after 1000 [list run_reduction $t cl]]
         return
      }
      set mode "cr"
      draw_network_feedbacks $c $feedbacks($t) $Network($c) 1
   }
   if {$mode eq "cr"} then {
      set c $t.cr
      if {![info exists Network($c)]} then {
         puts "Expected network for $c, but found none (hickup?)."
         return
      }
      set rule [
         graphic_reduce Network($c) $feedbacks($t) $c $lastrule($c)
      ]
      if {$rule >= 0} then {
         set lastrule($c) $rule
         set after_id($t) [after 1000 [list run_reduction $t cr]]
         return
      }
      set mode "compare"
      draw_network_feedbacks $c $feedbacks($t) $Network($c) 1
   }
   if {$mode eq "compare"} then {
%   \end{tcl}
%   The networks have now reached their final form, so it ``only'' 
%   remains to compare them. The first test is for equality (the 
%   least troublesome possibility).
%   \begin{tcl}
      array set NW1 $Network($t.cl)
      array set NW2 $Network($t.cr)
      set NW1(canonical) [network::pure::canonise $NW1(pure)]
      set NW2(canonical) [network::pure::canonise $NW2(pure)]
      if {$NW1(canonical) eq $NW2(canonical)} then {
         set mode "equal"
         set count_down 2
      } else {
         $t.bl configure -state normal
         $t.br configure -state normal
         set cmp [compare_networks $NW1(canonical) $NW2(canonical)]
         switch -- $cmp "<" {
            set count_down 5
            set mode "autoright"
         } ">" {
            set count_down 5
            set mode "autoleft"
%<*manualequal>
         } "=" {
            $t.bm configure -text "Equal" -state normal
            grid remove $t.lm
            grid configure $t.bm -row 2 -column 1 -sticky n
            bell -displayof $t
            return
%</manualequal>
         } default {
            grid remove $t.bl $t.br
            grid configure $t.lm -row 2 -column 0 -sticky n
            grid configure $t.bm -row 2 -column 1 -sticky n
            $t.bm configure -text $cmp
            set count_down 5
            set mode "automiddle"
         }
      }
   }
   switch -- $mode "equal" {
      $t.lm configure -text "Equal! $count_down"
      if {$count_down <= 0} then {
         rule_ambiguity $t equal
         return
      }
   } "autoleft" {
      $t.lm configure -text "<- in $count_down"
      if {$count_down <= 0} then {
         $t.bl invoke
         return
      }
   } "autoright" {
      $t.lm configure -text "-> in $count_down"
      if {$count_down <= 0} then {
         $t.br invoke
         return
      }
   } "automiddle" {
      $t.lm configure -text "-> in $count_down"
      if {$count_down <= 0} then {
         $t.bm invoke
         return
      }
   } default {
      error "This shouldn't happen!"
   }
   incr count_down -1
   set after_id($t) [after 1000 [list run_reduction $t $mode]]
}
%   \end{tcl}
% \end{proc}
% 
% \begin{proc}{compare_networks}
%   The |compare_networks| command has the call syntax
%   \begin{quote}
%     |compare_networks| \word{left} \word{right}
%   \end{quote}
%   where \word{left} and \word{right} are pure networks. It should 
%   return |<| if the left is less than the right, |>| if the left is 
%   greater than the right, and something else otherwise. The 
%   following definition is a placeholder that should be overridden 
%   for particular problems, if an ordering is known.
%   \begin{tcl}
proc compare_networks {left right} {return "don't know"}
%   \end{tcl}
% \end{proc}
% 
% \begin{proc}{check_rule_compatibility}
%   This procedure goes through all rules in |rulesL| and checks that 
%   they are comparable with the order implemented in 
%   |compare_networks|. The call syntax is
%   \begin{quote}
%     |check_rule_compatibility| 
%     \begin{regblock}[\regstar]\word{option} 
%     \word{value}\end{regblock}
%   \end{quote}
%   where the currently supported \word{option}s are:
%   \begin{ttdescription}
%     \item[-print]
%       The \word{value} is a channel opened for writing. Supplying 
%       this value causes information about incompatible rules to be 
%       written to this channel.
%   \end{ttdescription}
%   The return value is the list of numbers of rules which are not 
%   compatible.
%   \begin{tcl}
proc check_rule_compatibility {args} {
   global rulesL
   array set Opt $args
   set res {}
   set n -1; foreach rule $rulesL {incr n
      set cmp [compare_networks [lindex $rule 0 0] [lindex $rule 2 0]]
      if {$cmp ne ">"} then {
         lappend res $n
         if {[info exists Opt(-print)]} then {
            puts $Opt(-print) "Rule $n is incompatible ($cmp)!"
         }
      }
   }
   return $res
}
%   \end{tcl}
% \end{proc}
% 
% \begin{proc}{try_comparison_order}
%   This procedure goes through all elements of |rulesL| and 
%   |equalitiesL| to compare them according to a particular 
%   comparison order. The call syntax is
%   \begin{quote}
%     |try_comparison_order| 
%     \begin{regblock}[\regstar]\word{option} 
%     \word{value}\end{regblock}
%   \end{quote}
%   where the currently supported \word{option}s are:
%   \begin{ttdescription}
%     \item[-keycmd]
%       A command to generate a comparison key for each network. Keys 
%       are then compared using |network::pure::grouplex_compare|.
%     \item[-cmpcmd]
%       A command to directly compare two networks. Is overridden by 
%       |-keycmd|. Defaults to |compare_networks|.
%     \item[-print]
%       The \word{value} is a channel opened for writing. Supplying 
%       this value causes information about incompatible rules to be 
%       written to this channel.
%   \end{ttdescription}
%   The return value is the list of numbers of rules which are not 
%   compatible.
%   \begin{tcl}
proc try_comparison_order {args} {
   array set Opt {-cmpcmd compare_networks}
   array set Opt $args
   if {[info exists Opt(-keycmd)]} then {
      set comparecmd [list network::pure::grouplex_compare\
        $Opt(-keycmd)]
   } else {
      set comparecmd $Opt(-cmpcmd)
   }
   set res {}
   foreach {var trim expect format} {
      ::rulesL "" > %d  
      ::equalitiesL <> "" e%d
   } {
      set n -1; foreach rule [set $var] {incr n
         set call $comparecmd
         lappend call [lindex $rule 0 0] [lindex $rule 2 0]
         set cmp [eval $call]
         if {$cmp ne $expect} then {
            if {[string trim $cmp $trim] ne $expect} then {
               lappend res [format $format $n]
            }
            if {[info exists Opt(-print)]} then {
               puts $Opt(-print) [format "$format is %s" $n $cmp]
            }
         }
      }
   }
   return $res
}
%   \end{tcl}
% \end{proc}
% 
% \begin{proc}{idle_reschedule}
%   The |idle_reschedule| procedure schedules a command for being 
%   executed later, after an |after 0| and |after idle| loop. The 
%   syntax is
%   \begin{quote}
%     |idle_reschedule| \word{index} \word{script}
%   \end{quote}
%   where \word{script} is the script to execute and \word{index} is 
%   the entry in the |after_id| array where the relevant |after| 
%   identifiers will be kept.
%   \begin{tcl}
proc idle_reschedule {index script} {
   global after_id
   set after_id($index) [after 0 "[list set after_id($index)]\
     \[[list after idle $script]\]"]
}
%   \end{tcl}
% \end{proc}
% 
% 
% 
% \subsection{Ambiguities}
% 
% \begin{proc}{make_ambiguities}
%   The |make_ambiguities| procedure computes all ambiguities for a 
%   pair of rules and creates windows displaying these and their two 
%   main descendants. The syntax is
%   \begin{quote}
%     |make_ambiguities| \word{rule no.~$1$} \word{rule no.~$2$} 
%     \word{base window}
%   \end{quote}
%   where the two \emph{rule} arguments are rule numbers. The 
%   \word{base window} is used in positioning the new windows.
%   The return value is the list of names of toplevels that were created.
%   
%   The way that the \word{base window}, which should better be visible 
%   when this procedure is called, is used is the following: a 
%   displacement is computed by taking the maximal of the vertical and 
%   horizontal differences between the |winfo root|[|xy|] and the |wm\
%   geometry| coordinates of that \word{base window}. Then the 
%   geometries of the new windows are positioned some multiple of that 
%   displacement from the geometry of the \word{base window}.
%   \begin{tcl}
proc make_ambiguities {rule1 rule2 base_w} {
   global rulesL Network feedbacks lastrule vertex_appearances\
     vertex_sizes Steps
   upvar #0 $vertex_appearances TA
   foreach {wm_wh wm_x wm_y} [split [wm geometry $base_w] +] {break}
   set delta_x [expr {abs( [winfo rootx $base_w] - $wm_x )}]
   set delta_y [expr {abs( [winfo rooty $base_w] - $wm_y )}]
   set delta [expr {$delta_x>$delta_y ? $delta_x : $delta_y}]
   set res [list]
   foreach ambiguity [
      network::wfb::groomed_ambiguities [lindex $rulesL $rule1 0]\
        [lindex $rulesL $rule2 0]
   ] {
%   \end{tcl}
%   Skip the trivial ambiguity of a rule against itself. (The 
%   |ambiguities| procedure cannot do that, since it cannot tell 
%   whether the right hand sides are equal.)
%   \begin{tcl}
      if {$rule1==$rule2 && 
         [lindex $ambiguity 1 1] eq [lindex $ambiguity 2 1] &&
         [lindex $ambiguity 1 2] eq [lindex $ambiguity 2 2] &&
         [lsort -integer [lindex $ambiguity 1 0]] eq\
           [lsort -integer [lindex $ambiguity 2 0]]
      } then {continue}
      set t [new_ambiguity_window]
      set wm_x [expr {$wm_x + $delta}]
      set wm_y [expr {$wm_y + $delta}]
      wm geometry $t +${wm_x}+${wm_y}
      set Network($t.cm) [lindex $ambiguity 0 0]
      set feedbacks($t) [lindex $ambiguity 0 1]
      set Network($t.cl) [
         network::pure::replace [lindex $ambiguity 0 0]\
           [lindex $ambiguity 1] [lindex $rulesL $rule1 2 0]
      ]
      set lastrule($t.cl) $rule1
      set Network($t.cr) [
         network::pure::replace [lindex $ambiguity 0 0]\
           [lindex $ambiguity 2] [lindex $rulesL $rule2 2 0]
      ]
      set lastrule($t.cr) $rule2
      foreach c [list $t.cm $t.cl $t.cr] {
         set NW(pure) $Network($c)
         network::rich::make_level-layout-Tk $vertex_sizes -var NW
         network::rich::vpos-Tk_by_level -var NW
         network::rich::ecurve-Tk-tt_by_level $vertex_appearances\
           -var NW
         network::rich::drawit $c $vertex_appearances {} -var NW
         center_network $c
         set Network($c) [array get NW]
         array unset NW
      }
      draw_network_feedbacks $t.cm $feedbacks($t) $Network($t.cm) 1
      set Steps($t.cl) [list $rule1 $Network($t.cl)]
      set Steps($t.cr) [list $rule2 $Network($t.cr)]
      $t.bl configure -command [list rule_ambiguity $t left]
      $t.br configure -command [list rule_ambiguity $t right]
      $t.bm configure -command [list rule_ambiguity $t neither]
      lappend res $t
   }
   return $res
}
%   \end{tcl}
% \end{proc}
% 
% The processing to resolve an ambiguity is initiated by a |<FocusIn>| 
% binding for the \describestring[bindtag]{Ambiguity}|Ambiguity| 
% bindtag. This has the effect that the processing for a network begins 
% anew whenever one switches to an ambiguity window. The delay is 
% needed to allow observing the initial state of the window.
% \begin{tcl}
bind Ambiguity <FocusIn> {
   set after_id(%W) [after 1000 {run_reduction %W cl}]
}
% \end{tcl}
% 
% 
% 
% \begin{proc}{rule_ambiguity}
%   This procedure is called to end an ambiguity by turning it into a 
%   new rule or equality. The syntax is
%   \begin{quote}
%     |rule_ambiguity| \word{toplevel} \word{choice}
%   \end{quote}
%   where \word{toplevel} is the toplevel of the ambiguity and 
%   \word{choice} is |left|, |right|, or |neither| depending on which 
%   network was chosen for the principal part in the rule. The procedure 
%   |destroy|s the ambiguity toplevel.
%   
%   \begin{tcl}
proc rule_ambiguity {t choice} {
   global Network feedbacks rule_queue Steps
   if {$choice eq "left"} then {
%<stdout>      puts stdout "Added rule [
      add_rule_derivation [
         add_rule $feedbacks($t) rich $Network($t.cl)\
           rich $Network($t.cr)
      ] $Steps($t.cl) $Network($t.cm) $Steps($t.cr)
%<stdout>      ] (left)"
   } elseif {$choice eq "right"} then {
%<stdout>      puts stdout "Added rule [
      add_rule_derivation [
         add_rule $feedbacks($t) rich $Network($t.cr)\
           rich $Network($t.cl)
      ] $Steps($t.cr) $Network($t.cm) $Steps($t.cl)
%<stdout>      ] (right)"
   } elseif {$choice eq "neither"} {
      add_equality $feedbacks($t) $Steps($t.cl) $Network($t.cm)\
        $Steps($t.cr) [$t.bm cget -text]
   }
   ambiguity_cancel $t
%<*showcalc>
   show_calculation $feedbacks($t) $Steps($t.cl) $Network($t.cm)\
     $Steps($t.cr)
%</showcalc>
   array unset Network $t.c\[lmr\]
   array unset Steps $t.c\[lr\]
   array unset feedbacks $t
   destroy $t
}
%   \end{tcl}
% \end{proc}
% 
% \begin{proc}{show_calculation}
%   This is a debugging utility procedure which displays a 
%   calculation in the rule browser window, if there is one. The call 
%   syntax is
%   \begin{displaysyntax}
%     |show_calculation| word{feedbacks} \word{steps left} 
%     \word{middle network} \word{steps right}
%   \end{displaysyntax}
%   and there is no particular return value.
%   \begin{tcl}
proc show_calculation {fbL left middle right} {
   if {![winfo exists .rulebrowser]} then {return}
   array set NW1 [lindex $left end]
   array set NW2 [lindex $right end]
   set rule [list [list $NW1(pure) $fbL] [lindex $left end]\
     [list $NW2(pure) [lindex $right end]]\
       [make_rule_derivation $left $middle $right]]
   display_rule .rulebrowser $rule
}
%   \end{tcl}
% \end{proc}
% 
% \begin{variable}{rule_queue}
% \begin{variable}{rule_base}
%   The |rule_queue| and |rule_base| are two lists of rule numbers that 
%   are used to select which ambiguities to process next. When rules 
%   are added to the |rulesL|, also an element
%   \begin{quote}
%     \word{rule number} \word{sort key}
%   \end{quote}
%   is appended to the |rule_queue|. When all ambiguities that can be 
%   formed from the rules in the |rule_base| have been examined, the 
%   rule in |rule_queue| that has the smallest \word{sort key} is 
%   removed and that \word{rule number} is instead appended to the 
%   |rule_base|. The remaining ambiguities are then those which have 
%   the new rule on one side and one of the elements in the |rule_base| 
%   on the other.
% \end{variable}\end{variable}
% 
% 
% \begin{variable}{minor_index}
%   The |minor_index| variable stores the index in |rule_base| of the 
%   next rule to form ambiguities by.
% \end{variable}
% 
% 
% \begin{proc}{run_ambiguities}
%   The |run_ambiguities| procedure is the body of an interation in 
%   which the |minor_index| variable is incremented until it reaches the 
%   length of the |rule_base| and then stops, while simultaneously 
%   computing all ambiguities for the current pair of these. The 
%   increment part happens in |minor_step_ambiguities|. What happens here 
%   is that the ambiguity windows (if any) are created and another call 
%   to |minor_step_ambiguity| is arranged for (either via a binding or as 
%   a direct call).
%   \begin{tcl}
proc run_ambiguities {} {
   global rule_base minor_index
   if {[llength [
      make_ambiguities [lindex $rule_base $minor_index]\
        [lindex $rule_base end] .controller
   ]]} then {
      bind .controller <FocusIn> {minor_step_ambiguity}
   } else {
      minor_step_ambiguity
   }
}
%   \end{tcl}
% \end{proc}
% 
% \begin{proc}{minor_step_ambiguity}
%   The |minor_step_ambiguity| procedure increments the |minor_rule| 
%   variable, updates the left network in the controller window, and 
%   arranges for |run_ambiguities| to be called a second later. There 
%   are no arguments, nor any particular return value.
%   \begin{tcl}
proc minor_step_ambiguity {} {
   global minor_index rule_base rulesL minor_rule_no Network\
     vertex_appearances
   bind .controller <FocusIn> {}
   if {[incr minor_index] >= [llength $rule_base]} then {
      after 1000 {
         if {${.controller.check_auto}} then {
            .controller.b_next invoke
         }
      }
      return
   }
   set minor_rule_no [lindex $rule_base $minor_index]
   set Network(.controller.cl) [lindex $rulesL $minor_rule_no 1]
   .controller.cl delete all
   network::rich::drawit .controller.cl $vertex_appearances {}\
     -val $Network(.controller.cl)
   center_network .controller.cl
   after 1000 {run_ambiguities}
}
%   \end{tcl}
%   This procedure should perhaps be rewritten to not have variables 
%   and widgets hardwired, but on the other hand the |rulesL| is such 
%   a heavily entrenched piece of global state anyway, so it probably 
%   isn't worth it.
% \end{proc}
% 
% \begin{proc}{inclusion_ambiguity}
%   This procedure creates an ambiguity window for a given instance 
%   of the left hand side of one rule in the left hand side of 
%   another. The call syntax is
%   \begin{quote}
%     |inclusion_ambiguity| \word{outer rule no.} 
%       \word{inner rule no.} \word{region}
%   \end{quote}
%   where the \word{region} is for the left hand side of the inner 
%   rules within the left hand side of the outer rule. The return 
%   value is the name of the toplevel created for the ambiguity.
%   
%   \begin{tcl}
proc inclusion_ambiguity {rule1 rule2 region} {
   global feedbacks Network lastrule rulesL vertex_appearances\
     vertex_sizes Steps
   set t [new_ambiguity_window]
   set feedbacks($t) [lindex $rulesL $rule1 0 1]
   set Network($t.cm) [lindex $rulesL $rule1 1]
   network::rich::drawit $t.cm $vertex_appearances {}\
     -val $Network($t.cm)
   draw_network_feedbacks $t.cm $feedbacks($t) $Network($t.cm) 1
   set Network($t.cl) [lindex $rulesL $rule1 2 1]
   set lastrule($t.cl) $rule1
   network::rich::drawit $t.cl $vertex_appearances {}\
     -val $Network($t.cl)
   center_network $t.cl
   set NW(pure) [
      network::pure::replace [lindex $rulesL $rule1 0 0] $region\
        [lindex $rulesL $rule2 2 0]
   ]
   network::rich::make_level-layout-Tk $vertex_sizes -var NW
   network::rich::vpos-Tk_by_level -var NW
   network::rich::ecurve-Tk-tt_by_level $vertex_appearances\
     -var NW
   network::rich::drawit $t.cr $vertex_appearances {} -var NW
   center_network $t.cr
   set Network($t.cr) [array get NW]
   set lastrule($t.cr) $rule2
   set Steps($t.cl) [list $rule1 $Network($t.cl)]
   set Steps($t.cr) [list $rule2 $Network($t.cr)]
   $t.bl configure -command [list rule_ambiguity $t left]
   $t.br configure -command [list rule_ambiguity $t right]
   $t.bm configure -command [list rule_ambiguity $t neither]
   return $t
}
%   \end{tcl}
% \end{proc}
% 
% \begin{proc}{equality_ambiguity}
%   This procedure creates an ambiguity window which will resume 
%   reduction of an equality. The call syntax is
%   \begin{quote}
%     |equality_ambiguity| \word{equality}
%   \end{quote}
%   where the \word{equality} is an |equalitiesL| element. 
%   The return value is the name of the toplevel created for the 
%   ambiguity.
%   
%   \begin{tcl}
proc equality_ambiguity {equality} {
   global feedbacks Network lastrule vertex_appearances\
     vertex_sizes Steps
   set t [new_ambiguity_window]
   set feedbacks($t) [lindex $equality 0 1]
   set bwL [list [lindex $equality 1]]
   set fwL {}
   foreach {spec NW} [lindex $equality 3] {
      switch -- [lindex $spec 0] "backward" {
         lappend bwL [lindex $spec 1] $NW
      } "forward" {
         lappend fwL [lindex $spec 1] $NW
      }
   }
   set Network($t.cm) [lindex $bwL end]
   network::rich::drawit $t.cm $vertex_appearances {}\
     -val $Network($t.cm)
   draw_network_feedbacks $t.cm $feedbacks($t) $Network($t.cm) 1
   set Network($t.cl) [lindex $bwL 0]
   set lastrule($t.cl) [lindex $bwL 1]
   network::rich::drawit $t.cl $vertex_appearances {}\
     -val $Network($t.cl)
   center_network $t.cl
   set Network($t.cr) [lindex $fwL end]
   set lastrule($t.cr) [lindex $fwL end-1]
   network::rich::drawit $t.cr $vertex_appearances {}\
     -val $Network($t.cr)
   center_network $t.cr
   set Steps($t.cl) [lreverse [lrange $bwL 0 end-1]]
   set Steps($t.cr) $fwL
   $t.bl configure -command [list rule_ambiguity $t left]
   $t.br configure -command [list rule_ambiguity $t right]
   $t.bm configure -command [list rule_ambiguity $t neither]
   return $t
}
%   \end{tcl}
% \end{proc}
% 
% \begin{proc}{major_step_ambiguity}
%   The |major_step_ambiguity| procedure moves one rule from the 
%   |rule_queue| to the |rule_base|, and then resets the |minor_index|. 
%   As part of this, the other rules in the base and queue, and 
%   undirected equalities in the |equalitiesL|, are checked for whether 
%   their principal parts contain the new rule, and if they 
%   do then they are dropped from these lists and corresponding 
%   ambiguities are prepared to be checked.
%   \changes{2.2}{2008/12/5}{Checking and reopening equalities. (LH)}
%   
%   There are no arguments, nor any particular return value.
%   \begin{tcl}
proc major_step_ambiguity {base_w} {
   global minor_index rule_base rule_queue rulesL Network lastrule\
     vertex_appearances minor_rule_no major_rule_no feedbacks\
     equalitiesL remaining_equalities
   if {![llength $rule_queue]} then {return}
   set rule_queue [lsort -dictionary -index 1 $rule_queue]
   set new [lindex $rule_queue 0 0]
   foreach {wm_wh wm_x wm_y} [split [wm geometry $base_w] +] {break}
   set delta_x [expr {abs( [winfo rootx $base_w] - $wm_x )}]
   set delta_y [expr {abs( [winfo rooty $base_w] - $wm_y )}]
   set delta [expr {$delta_x>$delta_y ? $delta_x : $delta_y}]
% \smallskip
   set new_base [list]
   set lost_focus 0
   foreach n $rule_base {
      set L [network::wfb::instances [lindex $rulesL $n 0]\
        [lindex $rulesL $new 0] 1]
      if {[llength $L]} then {
         set wm_x [expr {$wm_x + $delta}]
         set wm_y [expr {$wm_y + $delta}]
         wm geometry [
            inclusion_ambiguity $n $new [lindex $L 0]
         ] +${wm_x}+${wm_y}
%<stdout>         puts stdout "Removed rule $n from base."
         set lost_focus 1
      } else {
         lappend new_base $n
      }
   }
   set rule_base $new_base
   lappend rule_base $new
% \smallskip
   set new_queue [list]
   foreach pair [lrange $rule_queue 1 end] {
      set n [lindex $pair 0]
      set L [network::wfb::instances [lindex $rulesL $n 0]\
        [lindex $rulesL $new 0] 1]
      if {[llength $L]} then {
         set wm_x [expr {$wm_x + $delta}]
         set wm_y [expr {$wm_y + $delta}]
         wm geometry [
            inclusion_ambiguity $n $new [lindex $L 0]
         ] +${wm_x}+${wm_y}
%<stdout>         puts stdout "Removed rule $n from base."
         set lost_focus 1
      } else {
         lappend new_queue $pair
      }
   }
   set rule_queue $new_queue
% \smallskip
   set new_equalities [list]
   foreach num $remaining_equalities {
      set rule [lindex $equalitiesL $num]
      if {[llength [
         network::wfb::instances [lindex $rule 0]\
           [lindex $rulesL $new 0] 1
      ]] || [llength [
         network::wfb::instances\
           [list [lindex $rule 2 0] [lindex $rule 0 1]]\
           [lindex $rulesL $new 0] 1
      ]]} then {
         set wm_x [expr {$wm_x + $delta}]
         set wm_y [expr {$wm_y + $delta}]
         wm geometry [
            equality_ambiguity $rule
         ] +${wm_x}+${wm_y}
%<stdout>         puts stdout "Reopened equality $num."
         set lost_focus 1
      } else {
         lappend new_equalities $num
      }
   }
   set remaining_equalities $new_equalities
% \smallskip
   set minor_index 0
   set minor_rule_no [lindex $rule_base 0]
   $base_w.cl delete all
   network::rich::drawit $base_w.cl $vertex_appearances {} -val [
      set Network($base_w.cl) [lindex $rulesL $minor_rule_no 1]
   ]
   center_network $base_w.cl
   set major_rule_no [lindex $rule_base end]
   $base_w.cr delete all
   network::rich::drawit $base_w.cr $vertex_appearances {} -val [
      set Network($base_w.cr) [lindex $rulesL $major_rule_no 1]
   ]
   center_network $base_w.cr
   if {[set ::$base_w.check_auto]} then {
      if {$lost_focus} then {
         bind .controller <FocusIn> {
            bind .controller <FocusIn> {}
            .controller.b_go invoke
         }
      } else {
         after 1000 {.controller.b_go invoke}
      }
   }
}
%   \end{tcl}
% \end{proc}
% 
% \begin{variable}{major_rule_no}
% \begin{variable}{minor_rule_no}
%   These two variables are displayed in the two entry widgets in the 
%   controller window. They should be set when the networks displayed 
%   there change.
%   \begin{tcl}
create_controller major_rule_no minor_rule_no
%   \end{tcl}
% \end{variable}\end{variable}
% 
% 
% Any pending |run_ambiguities| script is cancelled when the controller 
% window loses focus.
% \begin{tcl}
bind .controller <FocusOut> {after cancel {run_ambiguities}}
% \end{tcl}
% 
% Pressing the \textsf{Next} button performs a 
% |major_step_ambiguity|, whereas pressing the \textsf{Go} button 
% performs a |run_ambiguities|.
% \begin{tcl}
.controller.b_next configure -command {
   major_step_ambiguity .controller
}
.controller.b_go configure -command {run_ambiguities}
% \end{tcl}
% 
% 
% \subsection{A rule browser}
% 
% The following implements a window that allows one to browse the list 
% of rules.
% 
% \changes{2}{2007/03/17}{Redesigning rule browser layout. (LH)}
% 
% \begin{proc}{rule_browser_layout}
%   This procedure creates a window for the rule browser and sets up 
%   the widgets that should reside in it. At the top there is bar of 
%   controls, below that is a pair of canvases displaying the left 
%   and right hand sides of the rule, and below that is a canvas 
%   displaying the rule derivation.
%   All canvases have vertical and horizontal scroll bars. 
%   
%   The control bar has fixed height, but the heights of the other 
%   parts is under the control of a |panedwindow| widget. Each of the 
%   three parts of the window are found inside frames called 
%   |.rulebrowser.fbar|, |.rulebrowser.frule|, and 
%   |.rulebrowser.fproof| respectively.
%   \begin{tcl}
proc rule_browser_layout {} {
   set t [toplevel .rulebrowser]
   wm title $t "Rule browser"
   pack [
      frame $t.fbar -background grey90
   ] -expand true -fill x -side top
   pack [
      panedwindow $t.panes -orient vertical -borderwidth 0
   ] -expand true -fill both
   frame $t.frule
   frame $t.fproof
%   \end{tcl}
%   The two frames will eventually be made panes of the 
%   |panedwindow|, but delaying adding them until the grids inside 
%   them are complete makes it less likely to run into Tk 
%   bug~\#1714535. 
%   The order of widget creations here is important, as the frames 
%   need to be above the panedwindow in the stacking order.
%   
%   Creating the network canvases and their associated scrollbars 
%   is a highly uniform task, so all three of them are handled by the 
%   same loop. The widget names all have the form 
%   \meta{toplevel}|.|\meta{midpart}\meta{suffix} where \meta{suffix} 
%   is |c|, |h|, or |v| depending on whether this is a 
%   \texttt{c}anvas, \texttt{h}orizontal scrollbar, or 
%   \texttt{v}ertical scrollbar, but the \meta{midpart} is 
%   syntactically different in the rule and derivation parts: in the 
%   former case it is |frule.left_| or |frule.right_|, but in the 
%   latter it is |fproof.|.
%   \begin{tcl}
   foreach c_ [list $t.frule.left_ $t.frule.right_ $t.fproof.] {
      scrollbar ${c_}h -orient horiz -command [list ${c_}c xview]
      scrollbar ${c_}v -command [list ${c_}c yview]
      canvas ${c_}c -width 200 -height 300 -relief solid -borderwidth 1\
        -xscrollcommand [list ${c_}h set]\
        -yscrollcommand [list ${c_}v set]
      bindtags ${c_}c [linsert [bindtags ${c_}c] 1 Network]
   }
%   \end{tcl}
%   
%   Geometry inside the the |frule| and |fproof| frames is |grid|ded, 
%   since this automatically leaves a square of the right size in the 
%   corner. The |fproof| arrangement is slightly simpler, but the 
%   |frule| is mostly just a repetition.
%   \begin{tcl}
   grid $t.fproof.c -row 0 -column 0 -sticky nsew
   grid $t.fproof.v -row 0 -column 1 -sticky ns
   grid $t.fproof.h -row 1 -column 0 -sticky ew
   grid rowconfigure $t.fproof 0 -weight 1
   grid columnconfigure $t.fproof 0 -weight 1
%
   grid $t.frule.left_c  -row 0 -column 0 -sticky nsew
   grid $t.frule.left_v  -row 0 -column 1 -sticky ns
   grid $t.frule.left_h  -row 1 -column 0 -sticky ew
   grid $t.frule.right_c -row 0 -column 3 -sticky nsew
   grid $t.frule.right_v -row 0 -column 4 -sticky ns
   grid $t.frule.right_h -row 1 -column 3 -sticky ew
   grid rowconfigure $t.frule 0 -weight 1
   grid columnconfigure $t.frule 0 -weight 1
   grid columnconfigure $t.frule 3 -weight 1
%   \end{tcl}
%   Column 2 is reserved for a canvas with an arrow pointing right.
%   \begin{tcl}
   canvas $t.frule.arrow_c -width 20 -height 20
   $t.frule.arrow_c create line {0 10 20 10} -arrow last -width 2
   grid $t.frule.arrow_c -row 0 -column 2
%   \end{tcl}
%   Now the frame contents are complete, so they are added as panes.
%   \begin{tcl}
   $t.panes add $t.frule -sticky nsew -minsize 50
   $t.panes add $t.fproof -sticky nsew -minsize 50
%   \end{tcl}
%   The control bar |fbar| is filled in by a separate procedure, 
%   since the details of these controls are closely associated with 
%   the actions that they will perform. Similarly with the bindings 
%   for hyperlinks.
%   \begin{tcl}
   rule_browser_fbar $t
   hyper_references $t.fproof.c
   return $t
}
%   \end{tcl}
% \end{proc}
% 
% 
% 
% \begin{proc}{display_rule}
%   The |display_rule| procedure updates the rule browser window to 
%   show the specified rule or equality. The syntax is
%   \begin{quote}
%     |display_rule| \word{toplevel} \word{rule}
%   \end{quote}
%   where \word{toplevel} is the name of the browser window and 
%   \word{rule} is the actual rule (element of |rulesL|, not 
%   just an index into it) to display.
%   
%   \begin{tcl}
proc display_rule {t rule} {
   global Network vertex_appearances
%
   set tr $t.frule
   $tr.left_c delete all
   network::rich::drawit $tr.left_c $vertex_appearances {} -val [
      set Network($tr.left_c) [lindex $rule 1]
   ]
   draw_network_feedbacks $tr.left_c [lindex $rule 0 1]\
     [lindex $rule 1] 1
   $tr.left_c configure -scrollregion [$tr.left_c bbox all]
%
   $tr.right_c delete all
   network::rich::drawit $tr.right_c $vertex_appearances {} -val [
      set Network($tr.right_c) [lindex $rule 2 1]
   ]
   center_network $tr.right_c
   $tr.right_c configure -scrollregion [$tr.right_c bbox all]
%   \end{tcl}
%   The tricky part is to display the proof. The first step is a bit 
%   of a special case, since it includes brackets.
%   \begin{tcl}
   $t.fproof.c delete all
   display_rule_proof $t.fproof.c $rule
   $t.fproof.c configure -scrollregion [$t.fproof.c bbox all]
   center_network $t.fproof.c
}
%   \end{tcl}
% \end{proc}
% 
% 
% \begin{proc}{flatten_bboxes}
%   This procedure has the call syntax
%   \begin{quote}
%     |flatten_bboxes| \word{bbox}\regplus
%   \end{quote}
%   where each \word{bbox} is a list of an even number (typically 
%   four or zero) of real numbers. It returns the bounding box
%   \begin{quote}
%     \word{left} \word{top} \word{right} \word{bottom}
%   \end{quote}
%   for the total of the coordinates in these arguments.
%   \begin{tcl}
proc flatten_bboxes {args} {
   set xL {}; set yL {}
   foreach box $args {
      foreach {x y} $box {
         lappend xL $x; lappend yL $y
      }
   }
   set xL [lsort -real $xL]
   set yL [lsort -real $yL]
   return [list [lindex $xL 0] [lindex $yL 0]\
     [lindex $xL end] [lindex $yL end]]
}
%   \end{tcl}
% \end{proc}
% 
% 
% \begin{proc}{display_rule_proof}
%   This procedure displays the proof of a rule in a |canvas| (or 
%   approximation thereof; only the |create| subcommand is needed). 
%   The call syntax is
%   \begin{displaysyntax}
%     |display_rule_proof| \word{canvas-cmd} \word{rule} 
%     \begin{regblock}[\regstar] \word{option} \word{value}
%     \end{regblock}
%   \end{displaysyntax}
%   where \word{rule} is the actual rule data, i.e., a list with the 
%   structure
%   \begin{displaysyntax}
%     \word{left hand side (wfb)} \word{left hand side (rich)} 
%     \word{right hand side} \word{derivation}\regopt
%   \end{displaysyntax}
%   The return value is the bounding box of all the things drawn, or 
%   an empty list if there wasn't a \word{derivation} (in which case 
%   nothing is drawn).
%   
%   The \word{option}s are\dots
%   
%   \begin{tcl}
proc display_rule_proof {prefix rule args} {
   if {[llength $rule] < 4} then {return}
   array set Opt {
      -tags "" 
      -eqtext "="
      -eqfont {Times 24}
      -eqwidth 36
      -eqhalfheight 12
      -reffont {Courier 10 underline}
      -refcolour blue
   }
   array set Opt $args
   global vertex_appearances
%   \end{tcl}
%   The first step is to draw the left hand side, which is the only 
%   network for which feedbacks are drawn.
%   \begin{tcl}
   array set NW [lindex $rule 1]
   set stepTagsL [linsert $Opt(-tags) end step0]
   network::rich::drawit $prefix $vertex_appearances $stepTagsL -var NW
   set inL {}
   foreach item [lindex $NW(level-layout-Tk) 0 end-2] {
      if {[lindex $item 0] eq "edge"} then {
         lappend inL [lindex $item 2]
      }
   }
   set outL {}
   foreach item [lindex $NW(level-layout-Tk) 0 0] {
      if {[lindex $item 0] eq "edge"} then {
         lappend outL [lindex $item 2]
      }
   }
   set call [list flatten_bboxes]
   foreach {component bbox} $NW(level-layout-Tk) {lappend call $bbox}
   lappend call [
      network::rich::draw_feedbacks $prefix $stepTagsL\
        [lindex $rule 0 1] -inx $inL -outx $outL\
        -bbox [lindex $NW(level-layout-Tk) 1]
   ]
   set bbox [
      network::rich::draw_brackets $prefix $stepTagsL -bbox [eval $call]
   ]
   set yeq [expr {0.5*([lindex $bbox 1] + [lindex $bbox 3])}]
%   \end{tcl}
%   Remaining steps are drawn in a loop over the \word{derivation}.
%   \begin{tcl}
   set stepno 1
   foreach {rspec RNW} [lindex $rule 3] {
      set stepTagsL [linsert $Opt(-tags) end step${stepno}]
      set xeq [expr {[lindex $bbox 2] + 0.5*$Opt(-eqwidth)}]
      eval [linsert $prefix end create text $xeq $yeq -anchor center\
        -text $Opt(-eqtext) -font $Opt(-eqfont) -tags $Opt(-tags)]
      eval [linsert $prefix end create text\
        $xeq [expr {$yeq-$Opt(-eqhalfheight)}]\
        -anchor s -font $Opt(-reffont) -fill $Opt(-refcolour)\
        -tags [linsert $stepTagsL end reference] -text [
           switch -- [lindex $rspec 0] "forward" {
              format "(%d)>" [lindex $rspec 1]
           } "backward" {
              format "<(%d)" [lindex $rspec 1]
           }
        ]]
      array unset NW
      array set NW $RNW
      set call [list flatten_bboxes]
      foreach {component bbox2} $NW(level-layout-Tk) {
         lappend call $bbox2
      }
      set bbox2 [eval $call]
      set w [network::rich::draw_brackets $prefix {} -bbox $bbox2\
        -operation width]
      set xyofs [list [expr {[lindex $bbox 2] + $Opt(-eqwidth) + $w}]\
        [expr {$yeq - 0.5*([lindex $bbox2 1]+[lindex $bbox2 3])}]]
      network::rich::drawit $prefix $vertex_appearances $stepTagsL\
        -var NW -origin $xyofs
      set bbox2 {}
      foreach x [eval $call] dx [concat $xyofs $xyofs] {
         lappend bbox2 [expr {$x+$dx}]
      }
      lappend call [
         network::rich::draw_brackets $prefix $stepTagsL -bbox $bbox2
      ] $bbox
      set bbox [eval $call]
      incr stepno
   }
   return $bbox
}
%   \end{tcl}
% \end{proc}
% 
% 
% \begin{variable}{browser_bwL}
% \begin{variable}{browser_fwL}
%   The browser backward and forward lists are lists of rule numbers, 
%   serving as backward and forward stacks. Moving backward means the 
%   top (last) element is popped off |browser_bwL| and pushed onto 
%   |browser_fwL|, and vice versa. The top (last) element of 
%   |browser_fwL| is the current rule.
%   
%   The format of a rule number is either an integer or 
%   |e|\meta{integer}, where the latter refers to an equality.
% \end{variable}\end{variable}
% 
% \begin{proc}{parse_ruleno}
%   The |parse_ruleno| procedure splits a rule number into a pair
%   \begin{quote}
%     \word{variable} \word{index}
%   \end{quote}
%   where \word{variable} is the name of the variable in which the 
%   rule is kept and \word{index} is the index into the list that 
%   this variable should be of the rule in question. The call syntax 
%   is
%   \begin{quote}
%     |parse_ruleno| \word{ruleno} \word{bad-script}
%   \end{quote}
%   where the \word{bad-script} is evaluated in the calling context 
%   if the \word{ruleno} could not be parsed. Its return code and 
%   value will then be the return code and value of |parse_ruleno|.
%   \begin{tcl}
proc parse_ruleno {ruleno script} {
   switch -regexp -- $ruleno {^\d+$} {
      scan $ruleno %d num
      return [list rulesL $num]
   } {^[eE]\d+$} {
      scan $ruleno e%d num
      return [list equalitiesL $num]
   }
   return -code [catch {uplevel 1 $script} res] $res
}
%   \end{tcl}
% \end{proc}
% 
% \begin{proc}{unparse_ruleno}
%   This procedure those the opposite of |parse_ruleno|. Its call 
%   syntax is
%   \begin{quote}
%     |unparse_ruleno| \word{variable} \word{index}
%   \end{quote}
%   \begin{tcl}
proc unparse_ruleno {var num} {
   switch -- $var "rulesL" {
      format %d $num
   } "equalitiesL" {
      format e%d $num
   } default {
      error "Bad variable name"
   }
}
%   \end{tcl}
% \end{proc}
% 
% \begin{proc}{go_to_rule}
%   The |go_to_rule| procedure is the high level handler of going 
%   from one rule to another. It changes what is displayed in the 
%   browser window canvases, the rule number displayed in the control 
%   bar, and updates the backwards\slash forwards lists accordingly. 
%   The call syntaxes are
%   \begin{displaysyntax}
%     |go_to_rule| absolute \word{rule number}\par
%     |go_to_rule| relative \word{offset}\par
%     |go_to_rule| backward\par
%     |go_to_rule| forward
%   \end{displaysyntax}
%   where \word{rule number} is the number of the item in |rulesL| to 
%   display, and \word{style} is the style of movement this is. 
%   |forward| or |backward| as \word{style} means this is navigation 
%   using the forward and backward buttons, whereas |normal| means 
%   this is a normal move.
%   
%   \begin{tcl}
proc go_to_rule {style {data ""}} {
   global browser_fwL browser_bwL rule_number_var
   switch -- $style "absolute" {
      foreach {var num} [
         parse_ruleno $data {bell; return}
      ] break
   } "relative" {
      foreach {var num} [
         parse_ruleno [lindex $browser_fwL end] {bell; return}
      ] break
      incr num $data
   } "backward" {
      if {![llength $browser_bwL]} then {bell; return}
      foreach {var num} [
         parse_ruleno [lindex $browser_bwL end] {bell; return}
      ] break
   } "forward" {
      if {[llength $browser_fwL]<2} then {bell; return}
      foreach {var num} [
         parse_ruleno [lindex $browser_fwL end-1] {bell; return}
      ] break
   }
   upvar #0 $var rulesL
   if {$num >= [llength $rulesL]} then {bell; return}
   display_rule .rulebrowser [lindex $rulesL $num]
   set rule_number_var [unparse_ruleno $var $num]
   switch -- $style "backward" {
      lappend browser_fwL $rule_number_var
      set browser_bwL [lreplace $browser_bwL\
        [set browser_bwL end] end]
   } "forward" {
      lappend browser_bwL [lindex $browser_fwL end]
      set browser_fwL [lreplace $browser_fwL\
        [set browser_fwL end] end]
   } default {
      if {[llength $browser_fwL]} then {
         lappend browser_bwL [lindex $browser_fwL end]
      }
      set browser_fwL [list $rule_number_var]
   }
   .rulebrowser.fbar.back configure -state [lindex {disabled normal}\
     [expr {[llength $browser_bwL]>0}]]
   .rulebrowser.fbar.forward configure -state [lindex {disabled normal}\
     [expr {[llength $browser_fwL]>1}]]
   .rulebrowser.fbar.next configure -state [lindex {disabled normal}\
     [expr {$num+1 < [llength $rulesL]}]]
   .rulebrowser.fbar.prev configure -state [lindex {disabled normal}\
     [expr {$num>0}]]
}
%   \end{tcl}
% \end{proc}
% 
% 
% \begin{proc}{hyper_references}
%   This procedure makes bindings in a canvas so that all items 
%   tagged |reference| become hyperlinks. The call syntax is
%   \begin{quote}
%     |hyper_references| \word{canvas}
%   \end{quote}
%   \begin{tcl}
proc hyper_references {c} {
   $c bind reference <Enter> [list $c configure -cursor hand2]
   $c bind reference <Leave> [list $c configure -cursor {}]
   $c bind reference <ButtonPress-1> [list\
     $c itemconfigure current -fill red]
   $c bind reference <B1-Leave> "
      [list $c configure -cursor {}]
      [list $c itemconfigure current -fill blue]
   "
   $c bind reference <ButtonRelease-1> [list hyper_jump $c red]
}
%   \end{tcl}
% \end{proc}
% 
% \begin{proc}{hyper_jump}
%   This procedure does the jumping associated with clicking on a 
%   hyperlink. The call syntax is
%   \begin{quote}
%     |hyper_jump| \word{canvas} \word{hot fill}
%   \end{quote}
%   It only does anything if the |-fill| of the item is equal to 
%   \word{hot fill}, since a different value of |-fill| would indicate 
%   that the mouse was moved in over the hyperlink while being pressed, 
%   which should not count as clicking the hyperlink.
%   \begin{tcl}
proc hyper_jump {c hot} {
   if {[$c itemcget current -fill] ne $hot} then {return}
%   \end{tcl}
%   The number of the rule to jump to is parsed from the text of the 
%   link.
%   \begin{tcl}
   regsub -all -- {[<>()]} [$c itemcget current -text] {} ruleno
   go_to_rule absolute $ruleno
}
%   \end{tcl}
% \end{proc}
%
% 
% \begin{proc}{rule_browser_fbar}
%   This procedure fills in the |fbar| child of a specified toplevel 
%   with the controls for a rule browser window. The call syntax is
%   \begin{quote}
%     |rule_browser_fbar| \word{toplevel}
%   \end{quote}
%   There is no particular return value.
%   
%   ---
%   
%   The control bar is a frame, into which the individual controls are 
%   packed. Its background has a colour.
%   \begin{tcl}
proc rule_browser_fbar {t} {
   set tf $t.fbar
   set bg [$tf cget -background]
   pack [
      button $tf.back -text "Back" -highlightbackground $bg\
        -command {go_to_rule backward}
   ] -side left
   pack [
      button $tf.forward -text "Forward" -highlightbackground $bg\
        -command {go_to_rule forward}
   ] -side left
   
   pack [
      button $tf.go -text "Go" -highlightbackground $bg -command {
         go_to_rule absolute $rule_number_var
      }
   ] -side right
   pack [
      entry $tf.go_e -width 4 -textvariable rule_number_var
   ] -side right
   bind $tf.go_e <Return> [list $tf.go invoke]
   pack [label $tf.go_l -text "Rule" -background $bg] -side right
   pack [
      button $tf.next -text "+1" -highlightbackground $bg\
        -command {go_to_rule relative 1}
   ] -side right
   pack [
      button $tf.prev -text "-1" -highlightbackground $bg\
        -command {go_to_rule relative -1}
   ] -side right
}
%   \end{tcl}
% \end{proc}
% 
% Now that this code is completed, it is fitting to also create the 
% corresponding window.
% \begin{tcl}
rule_browser_layout
% \end{tcl}
% 
% 
% \subsection{Dumping results as PDF}
% 
% The following code dumps the rules and equalities, as well as 
% their derivations, to a PDF file. There is one rule per page. The 
% page dimensions are determined by the rules.
% 
% \begin{proc}{shipout_rule}
%   This procedure does most of the work. It outputs one rule, 
%   producing one page and adding one entry to the document outline. 
%   The call syntax is
%   \begin{displaysyntax}
%     |shipout_rule| \word{file} \word{rule} \word{title}
%   \end{displaysyntax}
%   and the return value is the label given to the page object.
%   
%   \begin{tcl}
proc shipout_rule {F rule title} {
   global vertex_appearances
   set label [pdf::auto_label $F $title]
%
   set c [pdf::begin_scanvas_xobject $F "$label LHS" -anchor e]
   network::rich::drawit $c $vertex_appearances {}\
     -val [lindex $rule 1]
   draw_network_feedbacks $c [lindex $rule 0 1] [lindex $rule 1] 0
   set rect [pdf::scanvas_bbox $F]
   network::rich::draw_brackets $c {} -bbox\
     [list [lindex $rect 0] [expr {-[lindex $rect 1]}]\
       [lindex $rect 2] [expr {-[lindex $rect 3]}]]
   set rect_l [pdf::inset_rect [pdf::end_scanvas_xobject $F] -5 mm]
%
   set c [pdf::begin_scanvas_xobject $F "$label RHS" -anchor w]
   network::rich::drawit $c $vertex_appearances {}\
     -val [lindex $rule 2 1]
   draw_network_feedbacks $c [lindex $rule 0 1] [lindex $rule 2 1] 0
   set rect [pdf::scanvas_bbox $F]
   network::rich::draw_brackets $c {} -bbox\
     [list [lindex $rect 0] [expr {-[lindex $rect 1]}]\
       [lindex $rect 2] [expr {-[lindex $rect 3]}]]
   set rect_r [pdf::inset_rect [pdf::end_scanvas_xobject $F] -5 mm]
%
   if {[llength $rule] > 3} then {
      set c [pdf::begin_scanvas_xobject $F "$label proof" -anchor s]
      display_rule_proof $c $rule
      set rect_p [pdf::end_scanvas_xobject $F]
   } else {
      set rect_p {0 0 0 0}
   }
%
   pdf::begin_contents "" $F "$label contents"
%
   if {[llength $rule] > 3} then {
      pdf::name_resource X1 $F XObject\
        [pdf::obj_ref $F "$label proof"]
      pdf::printf $F {q 1 0 0 1 %l2 cm %o Do Q} 0 cm 1 cm $X1
      pdf::printf $F {%r 0 m %r 0 l S} [lindex $rect_p 0]\
        [lindex $rect_p 2]
   }
%
   set rect_lr [pdf::covering_rect [pdf::offset_rect $rect_l -1 0 cm]\
     [pdf::offset_rect $rect_r 1 0 cm]]
   set yofs [expr {[pdf::length_obj -1 cm] - [lindex $rect_lr 1]}]
   pdf::printf $F {1 0 0 1 0 %r cm} $yofs
   pdf::name_resource X2 $F XObject [pdf::obj_ref $F "$label LHS"]
   pdf::printf $F {q 1 0 0 1 %l 0 cm %o Do Q} -1 cm $X2
   pdf::name_resource X3 $F XObject [pdf::obj_ref $F "$label RHS"]
   pdf::printf $F {q 1 0 0 1 %l 0 cm %o Do Q} 1 cm $X3
   puts $F {-15 -2 m -15 2 l 9 2 l 9 6 l 15 0 l 9 -6 l 9 -2 l f}
%
   pdf::end_contents Res $F
   set rect [pdf::covering_rect [pdf::offset_rect $rect_lr 0 $yofs]\
     [pdf::offset_rect $rect_p 0 1 cm]]
   pdf::shipout $F $label /Contents [pdf::obj_ref $F "$label contents"]\
     /Resources [pdf::resource_dict_obj Res]\
     /MediaBox [pdf::rect_obj [pdf::inset_rect $rect -2 cm]]
   pdf::outline_heading $F 1 $title /Dest\
     [pdf::array_obj [pdf::obj_ref $F $label] /FitB]
   return $label
}
%   \end{tcl}
% \end{proc}
% 
% 
% \begin{proc}{dump_to_PDF}
%   This procedure dumps a list of rules or equalities to a PDF 
%   file. The call syntax is
%   \begin{quote}
%     |dump_to_PDF| \word{file name} \begin{regblock}[\regstar] 
%     \word{heading} \word{rule-list} \end{regblock}
%   \end{quote}
%   and there is no particular return value. Any previous file with 
%   the name \word{file name} will be overwritten. 
%   
%   Each \word{heading} is made a top-level (level $0$) heading with 
%   one subheading for each element in the list of rules following 
%   it. Not specifying any heading or list is equivalent to
%   \begin{quote}
%     |Rules| \word{\$rulesL} |Equalities| \word{\$equalitiesL}
%   \end{quote}
%   \begin{tcl}
proc dump_to_PDF {fname args} {
   package require pdf
   set F [pdf::rewrite_pdf $fname]
   pdf::begin_pages $F "Pages"
   pdf::begin_outline $F "Headings"
   if {![llength $args]} then {
      set args [list Rules $::rulesL Equalities $::equalitiesL]
   }
   foreach {heading itemL} $args {
      if {![llength $itemL]} then {continue}
      pdf::outline_heading $F 0 $heading /Dest\
        [pdf::obj_ref $F [set label [pdf::auto_label $F]]]
      set page [shipout_rule $F [lindex $itemL 0] 0]
      pdf::put_obj $F $label [pdf::array_obj\
        [pdf::obj_ref $F $page] /Fit]
      set n 0; foreach item [lrange $itemL 1 end] {incr n
         shipout_rule $F $item $n
      }
   }
   pdf::put_obj $F "Catalog" [pdf::dict_obj \
     /Type /Catalog \
     /Pages [pdf::obj_ref $F [pdf::end_pages $F]] \
     /Outlines [pdf::obj_ref $F [pdf::end_outline $F]]]
   pdf::close_pdf $F "Catalog"
}
%   \end{tcl}
% \end{proc}
% 
% 
% \begin{tcl}
%</util1>
% \end{tcl}
% 
% 
% \section{Frobenius algebra}
% 
% The following is a demonstration problem, backported from version~2 
% of the completion utility. It sets up rules for a Frobenius algebra.
% 
% \subsection{Generators and rules}
% 
% \begin{tcl}
%<*Frobenius>
clear_rules
set vertex_types {
  Delta {1 2}  epsilon {1 0}  cap {0 2}  m {2 1}  unit {0 1}
}
set vertex_appearances {
   m {
      {oval {square 8} {}} 
      {{0 8}}
      {{-5.657 -5.657 -1 -1} {5.657 -5.657 1 -1}}
   }
   unit {
      {oval {square 8} {}} 
      {{0 8}}
      {}
   }
   Delta {
      {oval {square 8} {}} 
      {{-5.657 5.657 -1 1} {5.657 5.657 1 1}}
      {{0 -8}}
   }
   epsilon {
      {oval {square 8} {}}
      {}
      {{0 -8}}
   }
   cap {
      {oval {square 8} {}} 
      {{-5.657 5.657 -1 1} {5.657 5.657 1 1}}
      {}
   }
}
% \end{tcl}
% 
% First, there are the rules for an algebra.
% \begin{tcl}
add_rule {} pure [
   network::pure::construct $vertex_types {unit . \r m}
] pure [
   network::pure::construct $vertex_types {.}
]
add_rule {} pure [
   network::pure::construct $vertex_types {. unit \r m}
] pure [
   network::pure::construct $vertex_types {.}
]
add_rule {} pure [
   network::pure::construct $vertex_types {. m \r m}
] pure [
   network::pure::construct $vertex_types {m . \r m}
]
% \end{tcl}
% 
% Then, the rules for bends (zig--zag identities, such as they are).
% \begin{tcl}
add_rule {} pure [
   network::pure::construct $vertex_types {. cap \n m . \n epsilon .}
] pure [
   network::pure::construct $vertex_types {.}
]
add_rule {} pure [
   network::pure::construct $vertex_types {cap . \n . m \n . epsilon}
] pure [
   network::pure::construct $vertex_types {.}
]
% \end{tcl}
% Finally, the rule defining the coproduct.
% \begin{tcl}
add_rule {} pure [
   network::pure::construct $vertex_types {cap . \n . m}
] pure [
   network::pure::construct $vertex_types {Delta}
]
% \end{tcl}
% 
% 
% \subsection{Order}
% 
% \begin{proc}{compare_networks}
%   In this case, |compare_networks| is really an alias for 
%   |network::pure::grouplex_compare| with a suitable helper.
%   \begin{tcl}
interp alias {} compare_networks {}\
  network::pure::grouplex_compare Frobenius_cmpkey
%   \end{tcl}
% \end{proc}
% 
% \begin{proc}{Frobenius_cmpkey}
%   This procedure computes comparison keys that should be suitable 
%   for a Frobenious algebra, so it has the syntax
%   \begin{displaysyntax}
%     |Frobenius_cmpkey| \word{network}
%   \end{displaysyntax}
%   and returns a list of comparison groups. (The version~2 form of 
%   this procedure returns the flattening of that list.)
%   \begin{tcl}
proc Frobenius_cmpkey {NW} {
   set groups {}
%   \end{tcl}
%   The first group has one item: a basic count of vertices.
%   \begin{tcl}
   lappend groups [list [llength [lindex $NW 0]]]
%   \end{tcl}
%   The next group consists of path counts, so essentially it is what 
%   one gets from evaluating the network in the biaffine \PROP. A 
%   slight twist is however that the signs are reversed, to make the 
%   symmetric term (with every input connected to every output) in the 
%   Frobenius identities the smallest of the three.
%   \begin{tcl}
   set pathMat [network::pure::list_paths_turning $NW]
   set group {}
   foreach row $pathMat {
      foreach cell $row {
         lappend group [expr {-[llength $cell]}]
      }
   }
   lappend groups $group
%   \end{tcl}
%   The final group consists of weighted path counts, and could 
%   similarly be obtained by evaluating the network in the biaffine 
%   \PROP\ (with a slightly different assignment of values to 
%   vertices). Here, the weight of a path is doubled every time it 
%   goes through the right branch of a product, coproduct, or cap. 
%   This has the effect of orienting the associativity and 
%   coassociativity identities.
%   \begin{tcl}
   set group {}
   foreach row $pathMat {
      foreach cell $row {
         set sum 0
         foreach path $cell {
            set w 1
            foreach {tail annot head} $path {
               if {$head > 0} then {set w [expr {2*$w}]}
               if {$tail > 0} then {set w [expr {2*$w}]}
            }
            set sum [expr {$sum + $w}]
         }
         lappend group $sum
      }
   }
   lappend groups $group
   return $groups
}
%</Frobenius>
%   \end{tcl}
% \end{proc}
% 
% 
% \section{Hopf algebra}
% 
% The following is part research, part test of the utility. It sets 
% up the rules for a Hopf algebra.
% 
% \subsection{Generators and rules}
% 
% \begin{tcl}
%<*Hopf>
clear_rules
set vertex_types {
  Delta {1 2}  epsilon {1 0}  S {1 1}  m {2 1}  unit {0 1}
%<twist>  twist {2 2}
}
set vertex_appearances {
   m {
      {oval {square 8} {}} 
      {{0 8}}
      {{-5.657 -5.657 -1 -1} {5.657 -5.657 1 -1}}
   }
   unit {
      {oval {square 8} {}} 
      {{0 8}}
      {}
   }
   Delta {
      {oval {square 8} {}} 
      {{-5.657 5.657 -1 1} {5.657 5.657 1 1}}
      {{0 -8}}
   }
   epsilon {
      {oval {square 8} {}}
      {}
      {{0 -8}}
   }
   S {
      {rectangle {square 8} {}} 
      {{0 8}}
      {{0 -8}}
   }
   twist {
      {oval {square 8} {}  line {square 5.657} {} \
        line {offsets {5.657 -5.657} {-5.657 5.657}} {}} 
      {{-5.657 5.657 -1 1} {5.657 5.657 1 1}}
      {{-5.657 -5.657 -1 -1} {5.657 -5.657 1 -1}}
   }
}
% \end{tcl}
% 
% 
% \begin{proc}{score_network}
%   The basic scoring method has a tendency to spend a lot of time 
%   with rules that on first sight don't seem all that interesting: 
%   containing only products, coproducts, and twists, in one of a 
%   gazillion combinations. Hence this customised scoring procedure 
%   tries to encourage the use of ``productive'' (meaning involved in 
%   the derivation of interesting identities) rules and discourage 
%   the use of ``disruptive'' (producing lots of ambiguities, but very 
%   little of interest) rules.
%   
%   The basic score is again the number of vertices, but times $10$ 
%   to allow a ``decimal''.
%   \begin{tcl}
proc score_network {NW feedbacks} {
   set res [expr {10*[llength [lindex $NW 0]]}]
%   \end{tcl}
%   Adjustments to the basic figure are made according to the types 
%   of vertices found.
%   \begin{tcl}
   foreach v [lindex $NW 0] {
      switch -- [lindex $v 0] {
%   \end{tcl}
%   Antipodes are considered interesting, so they are only ``half as 
%   bad'' as a normal vertex.
%   \begin{tcl}
         "S" {incr res -5}
%   \end{tcl}
%   Products and coproducts with both the left and right terminal 
%   external to the network are highly disruptive, as they form 
%   ambiguities with almost everything, which only rarely lead to 
%   anything interesting. Hence networks with this feature are 
%   heavily penalised.
%   \begin{tcl}
         "m" {
            set ext 0
            foreach e [lindex $v 2] {
               if {[lindex $NW 1 $e 2]==1} then {incr ext}
            }
            if {$ext>1} then {incr res 40}
         }
         "Delta" {
            set ext 0
            foreach e [lindex $v 1] {
               if {[lindex $NW 1 $e 0]==0} then {incr ext}
            }
            if {$ext>1} then {incr res 40}
         }
%   \end{tcl}
%   Twists are on the other hand usually buried deep in the networks, 
%   so rules that expose them are interesting.
%   \begin{tcl}
         "twist" {
            foreach e [lindex $v 1] {
               if {[lindex $NW 1 $e 0]==0} then {incr res -2}
            }
            foreach e [lindex $v 2] {
               if {[lindex $NW 1 $e 2]==1} then {incr res -2}
            }
         }
      }
   }
%   \end{tcl}
%   Another thing that warrants interest are networks with 
%   ``cavities''---a pair of neighbouring vertices between which an 
%   unrelated network may be inserted---since these figure 
%   prominently in almost all basic derivations. In this case, it is 
%   the feedbacks that are important.
%   \begin{tcl}
   foreach fb $feedbacks {
      set e0 [lindex $NW 0 0 2 [lindex $fb 0]]
      set e1 [lindex $NW 0 1 1 [lindex $fb 1]]
      set v0 [lindex $NW 1 $e0 2]; if {$v0 < 2} then {continue}
      set v1 [lindex $NW 1 $e1 0]; if {$v1 < 2} then {continue}
      set L0 [lindex $NW 0 $v0 1]
      set L1 [lindex $NW 0 $v1 2]
      if {[llength [lsort -integer -unique [concat $L0 $L1]]] <\
        [llength $L0] + [llength $L1]} then {incr res -20}
   }
%   \end{tcl}
%   Well, that's it for now.
%   \begin{tcl}
   return $res
}
%   \end{tcl}
% \end{proc}
% 
% The following is a bialgebra rule, but putting it first may help 
% avoid creating stray edges.
% \begin{tcl}
add_rule {} pure [
   network::pure::construct $vertex_types {unit \r epsilon}
] pure [
   network::pure::construct $vertex_types {}
]
% \end{tcl}
% Then comes the rules for an algebra.
% \begin{tcl}
add_rule {} pure [
   network::pure::construct $vertex_types {unit . \r m}
] pure [
   network::pure::construct $vertex_types {.}
]
add_rule {} pure [
   network::pure::construct $vertex_types {. unit \r m}
] pure [
   network::pure::construct $vertex_types {.}
]
add_rule {} pure [
   network::pure::construct $vertex_types {. m \r m}
] pure [
   network::pure::construct $vertex_types {m . \r m}
]
% \end{tcl}
% The rules for a coalgebra.
% \begin{tcl}
add_rule {} pure [
   network::pure::construct $vertex_types {Delta \r epsilon .}
] pure [
   network::pure::construct $vertex_types {.}
]
add_rule {} pure [
   network::pure::construct $vertex_types {Delta \r . epsilon}
] pure [
   network::pure::construct $vertex_types {.}
]
%<*!rightCoass>
add_rule {} pure [
   network::pure::construct $vertex_types {Delta \r . Delta}
] pure [
   network::pure::construct $vertex_types {Delta \r Delta .}
]
%</!rightCoass>
%<*rightCoass>
add_rule {} pure [
   network::pure::construct $vertex_types {Delta \r Delta .}
] pure [
   network::pure::construct $vertex_types {Delta \r . Delta}
]
%</rightCoass>
% \end{tcl}
% The additional rules for a bialgebra.
% \begin{tcl}
add_rule {} pure [
   network::pure::construct $vertex_types {unit \r Delta}
] pure [
   network::pure::construct $vertex_types {unit unit}
]
add_rule {} pure [
   network::pure::construct $vertex_types {m \r epsilon}
] pure [
   network::pure::construct $vertex_types {epsilon epsilon}
]
add_rule {} pure [
   network::pure::construct $vertex_types {m \r Delta}
] pure [
   network::pure::construct $vertex_types {
%<!twist>      Delta Delta \r . X . \r m m
%<twist>      Delta Delta \r . twist . \r m m
   }
]
% \end{tcl}
% And finally the rules for the antipode.
% \begin{tcl}
add_rule {} pure [
   network::pure::construct $vertex_types {Delta \r . S \r m}
] pure [
   network::pure::construct $vertex_types {epsilon unit}
]
add_rule {} pure [
   network::pure::construct $vertex_types {Delta \r S . \r m}
] pure [
   network::pure::construct $vertex_types {epsilon unit}
]
%</Hopf>
% \end{tcl}
% 
% As a variant, one can add the rule that the antipode has order $2$ 
% or $4$.
% \begin{tcl}
%<*Hopf2>
add_rule {} pure [
   network::pure::construct $vertex_types {S \r S}
] pure [
   network::pure::construct $vertex_types {.}
]
%</Hopf2>
%<*Hopf4>
add_rule {} pure [
   network::pure::construct $vertex_types {S \r S \r S \r S}
] pure [
   network::pure::construct $vertex_types {.}
]
%</Hopf4>
% \end{tcl}
% 
% 
% \subsection{\PROP-based ordering}
% 
% The trickiest part of running a completion on the rules above is to 
% conceive of a well-founded ordering which is compatible with the 
% rules. Particular stumbling blocks are:
% \begin{enumerate}
%   \item
%     The rule turning $\Delta \circ m$ into \(m \otimes m \circ 
%     (2\:3) \circ \Delta \otimes \Delta\) \emph{increases} the 
%     number of vertices in the network.
%   \item
%     The associativity and coassociativity rules makes some paths 
%     longer.
%   \item
%     The rules (which will be derived after a while) that the 
%     antipode is an antihomomorphism have the property of exchanging 
%     left and right branches of products and coproducts, which 
%     severely complicates ordering associativity by minimising the 
%     number of right-branches on paths. The natural direction of 
%     these rules also increases the number of vertices (there is a 
%     separate antipode for each branch).
% \end{enumerate}
% Strange as it sounds, these problems are actually clues to the 
% solution. The first solution found was \emph{very} corny, but 
% nonetheless worked:
% \begin{enumerate}
%   \item
%     Compare the number of paths connecting each pair of input and 
%     output leg: more paths means the network is larger.
%   \item
%     If that is the same, then compare the number of antipodes on 
%     the paths: more antipodes means the network is larger.
%   \item
%     If that is also the same, then compare the total number of 
%     vertices: more vertices means the network is \emph{smaller!}
%   \item
%     If there is still no difference, then compare the number of 
%     times each path takes a right branch in or out of a vertex.
% \end{enumerate}
% A noticeable feature of the above is that there is no lexicographic 
% ordering at work, even though many rules certainly suggest using 
% a path-lexicographic order. The catch which prevents that is that 
% associativity can increase path length, which seriously messes up 
% compatibility with composition (pure lexicographic order in the 
% free monoid is not compatibile with composition\Dash it is 
% necessary to first ensure that items compared are of equal length).
% 
% Late 2008, I had advanced other parts of the theory enough that I 
% could device a more orderly solution to the problem, namely that of 
% importing an order from the biaffine \PROP\ over the natural numbers. 
% This is still fairly close to being a path-counting order, but 
% avoids some of the technical difficulties inherent in proving that 
% the above is DCC.
% 
% \begin{proc}{compare_networks}
%   The form of |compare_networks| that will be used for comparing 
%   networks in the Hopf setting is a lexicographic composition of 
%   orders obtained by mapping the networks to a specific ordered 
%   \PROP\ and comparing them there. The target \PROP, the array of 
%   elements to map the original value to, and the command to use for 
%   comparing them, are all provided as lists.
%   
%   For reference, the call syntax is
%   \begin{quote}
%     |compare_networks| \word{left} \word{right}
%   \end{quote}
%   as this is not immediately visible from the implementation.
%   \begin{tcl}
%<*Hopf&!HopfObsolete>
proc compare_networks {args} {
   foreach PROP {
      integer_biaffine_PROP 
      integer_biaffine_PROP
%<*twist>
      realmatrix_biaffine_PROP
      integer_biaffine_PROP
%</twist>
   } array {
      ::Hopf_value_1
      ::Hopf_value_2
%<*twist>
      ::Hopf_value_3
      ::Hopf_value_4
%</twist>
   } cmpcmd {
      integer_biaffine_PROP_cmp
      integer_biaffine_PROP_cmp
%<*twist>
      realmatrix_PROP_cmp
      integer_biaffine_PROP_cmp
%</twist>
   } {
      foreach NW $args {
         set call $PROP
         lappend call fuse [lindex $NW 0 0 2]
         foreach v [lrange [lindex $NW 0] 2 end] {
            lappend call [lindex $v 1]\
              [set ${array}([lindex $v 0])] [lindex $v 2]
         }
         lappend call [lindex $NW 0 1 1]
         lappend cmpcmd [eval $call]
      }
      set res [eval $cmpcmd]
      if {$res ne "="} then {return $res}
   }
   return $res
}
%   \end{tcl}
% \end{proc}
% 
% \begin{tclcommand}{alias}{integer_biaffine_PROP}
%   The primary ordered \PROP\ used is the biaffine \PROP\ over the 
%   integers (primarily the natural numbers, to ensure DCC).
%   \begin{tcl}
interp alias {} integer_biaffine_PROP {}\
  mtmtcl::matprop::biaffine ::mtmtcl::rings::integers::all
%   \end{tcl}
% \end{tclcommand}
% 
% \begin{proc}{integer_biaffine_PROP_cmp}
%   The natural order to use in an biaffine \PROP\ is the partial order 
%   of element-wise comparisons.
%   \begin{tcl}
proc integer_biaffine_PROP_cmp {left right} {
   set diffL {}
   lappend diffL [expr {[integer_biaffine_PROP part closed $left] -\
     [integer_biaffine_PROP part closed $right]}]
   foreach l [integer_biaffine_PROP part begin $left]\
           r [integer_biaffine_PROP part begin $right] {
      lappend diffL [expr {$l-$r}]
   }
   foreach l [integer_biaffine_PROP part end $left]\
           r [integer_biaffine_PROP part end $right] {
      lappend diffL [expr {$l-$r}]
   }
   foreach L [integer_biaffine_PROP part open $left]\
           R [integer_biaffine_PROP part open $right] {
      foreach l $L r $R {lappend diffL [expr {$l-$r}]}
   }
   if {[lsearch -glob $diffL {-*}] >= 0} then {
      if {[lsearch -glob $diffL {[1-9]*}] >= 0} then {
         return "incomparable"
      } else {
         return "<"
      }
   } else {
      if {[lsearch -glob $diffL {[1-9]*}] >= 0} then {
         return ">"
      } else {
         return "="
      }
   }
}
%   \end{tcl}
% \end{proc}
% 
% \begin{arrayvar}{Hopf_value_1}
%   The primary comparison order attains a subtle balance between 
%   growing and shrinking details in networks. The basic idea in it 
%   (to the extent that there is one) is to move |m| and |unit| 
%   towards the output, but |Delta| and |epsilon| towards the input, 
%   by counting \emph{inversions} where a |Delta| is applied to 
%   something coming from an |m|. This amounts to putting a positive 
%   termination weight on the simple side leg of these, while keeping 
%   the other side at weight $0$.
%   \begin{tcl}
set Hopf_value_1(m)\
  [integer_biaffine_PROP fromparts 0 1 {0 0} {{1 1}}]
set Hopf_value_1(unit)\
  [integer_biaffine_PROP fromparts 0 1 {} {}]
set Hopf_value_1(Delta)\
  [integer_biaffine_PROP fromparts 0 {0 0} 1 {{1} {1}}]
set Hopf_value_1(epsilon)\
  [integer_biaffine_PROP fromparts 0 {} 1 {}]
set Hopf_value_1(S)\
  [integer_biaffine_PROP fromparts 0 0 0 2]
set Hopf_value_1(twist)\
  [integer_biaffine_PROP fromparts 0 {0 0} {0 0} {{0 1} {1 0}}]
%   \end{tcl}
% \end{arrayvar}
% 
% \begin{arrayvar}{Hopf_value_2}
%   The secondary comparison order is similar, but places a positive 
%   weight on terminating in the right branch of a product or 
%   coproduct, to distinguish left-branching from right-branching: a 
%   right--right branch sees $2$ terminations, but a left--right, 
%   right--left, or singly right branch sees only $1$. The oppositely 
%   directed midpoint termination also contributes, by $1$ in the 
%   closed component.
%   \begin{tcl}
array set Hopf_value_2 [array get Hopf_value_1]
set Hopf_value_2(m)     [integer_biaffine_PROP fromparts 0 1 {0 1} {{1 1}}]
set Hopf_value_2(Delta) [integer_biaffine_PROP fromparts 0 {0 1} 1 {{1} {1}}]
%   \end{tcl}
% \end{arrayvar}
% 
% \begin{tclcommand}{alias}{realmatrix_biaffine_PROP}
%   The secondary ordered \PROP\ used is the biaffine \PROP\ over 
%   $\mathrm{M}_2(\mathbb{R})$, i.e., over real $2 \times 2$ 
%   matrices. Rational numbers would work too, but I currently don't 
%   have an implementation of these.
%   \begin{tcl}
interp alias {} realmatrix_biaffine_PROP {}\
  mtmtcl::matprop::biaffine {
     ::mtmtcl::matprop::ring 2 ::mtmtcl::rings::float
  }
%   \end{tcl}
% \end{tclcommand}
% 
% \begin{arrayvar}{Hopf_value_3}
%   This tertiary comparison order is focused on making it count 
%   which side of a product or coproduct that a crossing in abstract 
%   twist happens on.
%   \begin{tcl}
set one     {{1.0  0.0} {0.0 1.0 }}
set zero    {{0.0  0.0} {0.0 0.0 }}
set quarter {{0.25 0.0} {0.0 0.25}}
set double  {{2.0  0.0} {0.0 2.0 }}
set induce  {{0.0  1.0} {0.0 0.0 }}
set Hopf_value_3(m) [realmatrix_biaffine_PROP fromparts\
    $zero [list $zero] [list $zero $zero]\
    [list [list $double $double]]]
set Hopf_value_3(unit) [realmatrix_biaffine_PROP fromparts\
    $zero [list $one] {} {}]
set Hopf_value_3(Delta) [realmatrix_biaffine_PROP fromparts\
    $zero [list $zero $zero] [list $zero]\
    [list [list $quarter] [list $quarter]]]
set Hopf_value_3(epsilon) [realmatrix_biaffine_PROP fromparts\
    $zero {} [list $one] {}]
set Hopf_value_3(S) [realmatrix_biaffine_PROP fromparts\
    $zero [list $zero] [list $zero] [list [list $one]]]
set Hopf_value_3(twist) [realmatrix_biaffine_PROP fromparts\
    $zero [list $zero $zero] [list $zero $zero]\
    [list [list $induce $one] [list $one $induce]]]
%   \end{tcl}
% \end{arrayvar}
% 
% \begin{proc}{compare_matrix}
%   This procedure performs an element-wise comparison of two 
%   $2 \times 2$ matrics, returning |=|, |<|, |>|, or |incomparable| 
%   accordingly.
%   \begin{tcl}
proc compare_matrix {left right} {
   set greater 0
   set less 0
   foreach L $left R $right {
      foreach l $L r $R {
         if {$l<$r} then {
            set less 1
         } elseif {$l>$r} then {
            set greater 1
         }
      }
   }
   if {$less} then {
      if {$greater} then {return incomparable} else {return <}
   } else {
      if {$greater} then {return >} else {return =}
   }
}
%   \end{tcl}
% \end{proc}
% 
% \begin{proc}{realmatrix_PROP_cmp}
%   The natural order to use in an biaffine \PROP\ is the partial order 
%   of element-wise comparisons.
%   \begin{tcl}
proc realmatrix_PROP_cmp {left right} {
   array set A {= 0 < 0 > 0 incomparable 0}
   incr A([compare_matrix\
     [realmatrix_biaffine_PROP part closed $left]\
     [realmatrix_biaffine_PROP part closed $right]])
   foreach l [realmatrix_biaffine_PROP part begin $left]\
           r [realmatrix_biaffine_PROP part begin $right] {
      incr A([compare_matrix $l $r])
   }
   foreach l [realmatrix_biaffine_PROP part end $left]\
           r [realmatrix_biaffine_PROP part end $right] {
      incr A([compare_matrix $l $r])
   }
   foreach L [realmatrix_biaffine_PROP part open $left]\
           R [realmatrix_biaffine_PROP part open $right] {
      foreach l $L r $R {incr A([compare_matrix $l $r])}
   }
   if {$A(incomparable)} then {return "incomparable (element)"}
   if {$A(<)} then {
      if {$A(>)} then {return incomparable} else {return <}
   } else {
      if {$A(>)} then {return >} else {return =}
   }
}
%   \end{tcl}
% \end{proc}
% 
% \begin{arrayvar}{Hopf_value_4}
%   The fourth comparison level is about ordering the sides of the 
%   ``Yang--Baxter equation'':
%   |. twist \n twist . \n . twist| is less than 
%   |twist . \n . twist \n twist .|. The difference comes out for the 
%   middle strand in this braid, as it goes first right and then left 
%   in the former case, but first left and then right in the latter.
%   \begin{tcl}
set Hopf_value_4(m)\
  [integer_biaffine_PROP fromparts 0 0 {0 0} {{1 1}}]
set Hopf_value_4(unit)\
  [integer_biaffine_PROP fromparts 0 1 {} {}]
set Hopf_value_4(Delta)\
  [integer_biaffine_PROP fromparts 0 {0 0} 0 {{1} {1}}]
set Hopf_value_4(epsilon)\
  [integer_biaffine_PROP fromparts 0 {} 1 {}]
set Hopf_value_4(S)\
  [integer_biaffine_PROP fromparts 0 0 0 1]
set Hopf_value_4(twist)\
  [integer_biaffine_PROP fromparts 0 {0 0} {1 0} {{0 2} {1 0}}]
%   \end{tcl}
% \end{arrayvar}
% 
% \begin{tcl}
%</Hopf&!HopfObsolete>
% \end{tcl}
% 
% \subsection{Obsolete ordering schemes}
%   
% \begin{proc}{Hopf_comparison_key}
%   The |Hopf_comparison_key| procedure takes a pure network as 
%   argument and returns a corresponding comparison key, as a list of 
%   comparison groups. The call syntax is thus
%   \begin{quote}
%     |Hopf_comparison_key| \word{network}
%   \end{quote}
%   and the comparison groups will be
%   \begin{quote}
%     \word{\#paths through} \word{\#paths ending} \word{\#inner paths} 
%     \word{\#antipodes per path} \word{$-$\#vertices} 
%     \word{\#right-turns per path}
%   \end{quote}
%   Notes on why things are done this way are given throughout the 
%   implementation.
%   \begin{tcl}
%<*HopfObsolete>
proc Hopf_comparison_key {NW} {
   set pathMat [network::pure::list_paths_turning $NW]
%   \end{tcl}
%   The first comparison group only counts paths between an input 
%   and an output. The reason for this is that the axioms for the 
%   antipode turn two paths between input and output into two paths 
%   with an internal end. If both types of paths were part of the 
%   same comparison groups, the left and right hand sides of these 
%   rules would be incomparable.
%   
%   Counting paths with one end in an internal vertex orients the 
%   defining indentities of units and counits. Counting paths with 
%   both ends in an internal vertex orients the unit--counit 
%   compatibility identity.
%   \begin{tcl}
   set L1 {}; set L2 {}; set L3 {}
   foreach row $pathMat eo [lindex $NW 0 0 2] {
      foreach cell $row ei [lindex $NW 0 1 1] {
         lappend L[
            expr {1 + ($eo eq "") + ($ei eq "")}
         ] [llength $cell]
      }
   }
   set res [list $L1 $L2 $L3]
%   \end{tcl}
%   Counting antipodes is not directly needed for any of the standard 
%   axioms, but it is needed if one adds an axiom about the order of 
%   the antipode. It is also used by the derived identities on how 
%   the antipode interacts with the unit and counit. It is not 
%   necessary to distinguish between classes of paths because of the 
%   kinds of ends they have in this case, so all cells in the path 
%   matrix are treated the same, but the comparison is carried out on 
%   a per-cell basis.
%   \begin{tcl}
   set L1 {}
   foreach row $pathMat {
      foreach cell $row {
         set L2 {}
         foreach path $cell {
            set S 0
            foreach {s a g} $path {
               if {$a eq "S"} then {incr S}
            }
            lappend L2 $S
         }
         eval [list lappend L1] [lsort -decreasing -integer $L2]
      }
   }
   lappend res $L1
%   \end{tcl}
%   Straightforwardly counting the number of vertices is easy, and it 
%   doesn't matter whether the input and output vertices are included 
%   or not (since it's the same for all networks), but the number is 
%   more natural if they are not included in the count. This takes 
%   care of the coproduct--product compatibility axiom and the 
%   antipode--coproduct\slash product compatibility identities.
%   \begin{tcl}
   lappend res [expr {2-[llength [lindex $NW 0]]}]
%   \end{tcl}
%   Counting vertices can be generalised, if necessary, by making the 
%   weight of a vertex depend on its annotation. This turned out to 
%   not be necessary here, however (even though it is a close shave 
%   for the unit--coproduct and product--counit compatibility axioms).
%   
%   Counting right-brances is primarily a need for the associativity 
%   and coassociativity relations, but it turns out to take care of 
%   the unit--coproduct and product--counit compatibility axioms as 
%   well.
%   \begin{tcl}
   set L1 {}
   foreach row $pathMat {
      foreach cell $row {
         set L2 {}
         foreach path $cell {
            set r 0
            foreach {s a g} $path {
               if {$s > 0} then {incr r $s}
               if {$g > 0} then {incr r $g}
            }
            lappend L2 $r
         }
         eval [list lappend L1] [lsort -decreasing -integer $L2]
      }
   }
   lappend res $L1
}
proc Hopf_comparison_key {NW} {
   set key [network::pure::compkey_paths-wdeg-lex weight {
      S 1
   } weight {
      m {0 0 {0 1}}  
%<!rightCoass>      Delta {0 {0 1} 0}
%<rightCoass>      Delta {0 {1 0} 0}
   } $NW]
   linsert $key 2 [list [expr {-[llength [lindex $NW 0]]}]]
}
%   \end{tcl}
%   The difficult thing about the ordering obtained from doing a 
%   group-lexicographic comparison of the keys this procedure returns 
%   is to prove that it satisfies the descending chain condition. The 
%   tricky part is \word{$-$\#vertices}, since this is not obviously 
%   bounded from below. It turns out, however, that the parts that go 
%   before really give an upper bound on the number of vertices in 
%   the network and thus on how low this quantity can become.
%   
%   The key property to use is that the number of paths in the 
%   network is known (although an upper bound on the number would 
%   suffice). For starters, every vertex with arity zero must be an 
%   endpoint of a path, and no two such vertices can be endpoints of 
%   the same path, hence the total number of paths is an upper bound 
%   on the number of arity zero vertices. By symmetry, the total 
%   number of paths is also an upper bound on the number of vertices 
%   with coarity zero.
%   
%   When bounding the number of vertices with arity greater than two, 
%   it is useful to first view the network as a DAG rooted on the 
%   output side (i.e., ignore the meaning of tail indices) and then 
%   consider the forest obtained by unsharing common subgraphs of 
%   this DAG. In that forest, every path begins in a unique leaf 
%   (most of them are probably images of the input vertex |1|, but 
%   that is irrelevant here) and every vertex with arity greater than 
%   one is the root of a subtree with a strictly larger number of 
%   leafs than either child has on its own. Hence the number of 
%   vertices with arity greater than one is also bounded by the 
%   number of paths, and by symmetry the same holds for the number of 
%   vertices with coarity greater than one.
%   
%   All that remains to bound is thus the number of vertices with 
%   arity and coarity equal to one. These vertices are all antipodes, 
%   and therefore an upper bound for their number is known from the 
%   per-path numbers of antipodes. Hence there are only finitely many 
%   values possible for \word{$-$\#vertices} given values for earlier 
%   comparison keys and a grouplex comparison using 
%   |Hopf_comparison_key| is DCC.
% \end{proc}
% 
% 
% \subsection{Abstract twists}
% 
% For Hopf algebras without abstract twist, the above procedure can 
% be used straight off.
% \begin{tcl}
%<*!twist>
interp alias {} compare_networks {} network::pure::grouplex_compare\
  Hopf_comparison_key
%</!twist>
% \end{tcl}
% However, when there is a |twist| vertex, the above doesn't work 
% because there are more paths through \(m \otimes m \circ \id 
% \otimes T \otimes \id \circ \Delta \circ \Delta\) than there are 
% through \(\Delta \circ m\). A first stab at a way around that is 
% to first do a comparison such as above for the network where 
% crossings have been substituted for all twists, and then refine 
% this by additional comparisons (e.g.~of the total number of twists).
% \begin{tcl}
proc Hopf_twist_comparison_key {NW} {
   set NWX [
      network::pure::substitute $NW {
         twist {{{{} {} {1 0}} {{} {0 1} {}}} {{0 1 1 0} {0 0 1 1}}}
      }
   ]
   set key [Hopf_comparison_key $NWX]
   lappend key [list [llength [lindex $NW 0]]]
   return $key
}
% \end{tcl}
% Unfortunately this turns out to get stuck on the 
% (co)associativity-resolving comparison group in many cases, as 
% there are identities which can exchange a right-branching 
% product--twist--product sequence on one path with a similar 
% sequence on another path. This need not be disastrous, however, as 
% the comparisons for resolving (co)associativity can always be 
% placed in the final group; everything is still OK if one can find 
% some other criterion distinguishing the two networks.
% 
% One such criterion seems to be the orientation of the twist part in 
% the above sequence: on one branch it goes `$/$', on the other it 
% goes `$\backslash$'. The latter seems preferable, but it should be 
% the other way around for coproducts. The way to achieve that would 
% be to could product-right and coproduct-right separately, with 
% `$\backslash$' cancelling one product-right and `$/$' cancelling 
% one coproduct-right.
% 
% An even trickier identity is
% \[
%   \id \otimes m \circ
%   T \otimes \id \circ
%   \id \otimes m \otimes \id \circ
%   \id \otimes \id \otimes T \circ 
%   \id \otimes \Delta \otimes \id =
%   m \otimes \id \circ
%   \id \otimes T \circ
%   \id \otimes m \otimes \id \circ
%   T \otimes \id \otimes \id \circ
%   \id \otimes m \otimes \id
% \]
% A way to resolve this turns out to be to give different weights to 
% the slash- and backslash routes through a twist, since there on one 
% side is a path that has two slash passages and on the other side a 
% path that has two backslash passages.
% 
% \begin{tcl}
proc Hopf_twist_comparison_key {NW} {
   set NWX [
      network::pure::substitute $NW {
         twist {
            {{{} {} {3 2}} {{} {0 1} {}} {slash 3 1} {backslash 2 0}} 
            {{3 0 1 0} {2 0 1 1} {0 1 3 0} {0 0 2 0}}
         }
      }
   ]
   set key [network::pure::compkey_paths-wdeg-lex weight {
      S 1
   } weight {
      m {0 0 {0 1}}  Delta {0 {0 1} 0}
   } weight {
      slash 1  backslash -1
   } $NWX]
   linsert $key 2 [list [
      expr {-2*[llength [lindex $NW 0]] + [llength [lindex $NWX 0]]}
   ]] [list [llength [lindex $NWX 0]]]
}
% \end{tcl}
% 
% 2007-05-29: The above is not perfect, but not bad either. 
% Could use a run from the axioms.
% 
% 2007-12-19: It turns out the above also exhibits the strange ``rule 
% reversal when one adds a coproduct to an output leg'' as seems to 
% plauge the older |compare_networks| found below. However, at least 
% one case would rather be consistent if the first inserted item 
% above was dropped---and even if it was inserted as element 3 
% instead! The reason to insert as element 2 seems to be a desire to 
% make \(S(ab) > S(b)S(a)\) (the more antipodes, the smaller), but 
% it is not necessary to look at all vertex types to make that happen.
% 
% Thus we're at\dots
% \begin{tcl}
proc Hopf_twist_comparison_key2 {NW} {
   set NWX [
      network::pure::substitute $NW {
         twist {
            {{{} {} {3 2}} {{} {0 1} {}} {slash 3 1} {backslash 2 0}} 
            {{3 0 1 0} {2 0 1 1} {0 1 3 0} {0 0 2 0}}
         }
      }
   ]
   set key [network::pure::compkey_paths-wdeg-lex weight {
      S 1
   } weight {
      m {0 0 {0 1}}  Delta {0 {0 1} 0}
   } weight {
      slash 1  backslash -1
   } $NWX]
   array set C {m 0 unit 0 Delta 0 epsilon 0 S 0 twist 0 "" 0}
   foreach v [lindex $NW 0] {incr C([lindex $v 0])}
% \end{tcl}
% This cannot appear earlier, because then $S^n$ gets smaller as $n$ 
% increases. Nor can it appear later, if $S$ is some sort of 
% antihomomorphism.
% \begin{tcl}
   set key [linsert $key 2 [list [expr {-$C(S)}]]]
% \end{tcl}
% The main reason for the next criterion is to order the $\Delta 
% \circ m$ rule.
% \begin{tcl}
   linsert $key 4 [list [expr {-$C(Delta)-$C(m)}]]\
     [list [llength [lindex $NWX 0]]]
}
% \end{tcl}
% 
% 2007-12-20: Experimented with replacing |twist| by something 
% terminal (units and counits instead of a |slash| and/or 
% |backslash|) as a way of giving higher priority to twist 
% restructurings, but without success. The main problem is with the 
% product--coproduct axiom, since some kinds of path decrease in 
% number whereas others increase, in total leading to 
% incomparability. It is looking increasingly likely that this axiom 
% must be run the other way.
% 
% ---
% 
% 
% 
% \begin{proc}{compare_networks}
%   The way that networks for a Hopf algebra with abstract twist are 
%   compared does not seem to fall into the general 
%   |grouplex_compare| framework, so there has to be an explicit 
%   |compare_networks| procedure. The first part still follows the 
%   usual structure, however; the only difference is that the final 
%   rightness group has been replaced by a count of the total number 
%   of vertices (meaning fewer twists are preferred to more twists).
%   \begin{tcl}
%<*twist>
proc compare_networks {NW1 NW2} {
   set key1 [Hopf_comparison_key [
      network::pure::substitute $NW1 {
         twist {{{{} {} {1 0}} {{} {0 1} {}}} {{0 1 1 0} {0 0 1 1}}}
      }
   ]]
   lset key1 end [list [llength [lindex $NW1 0]]]
   set key2 [Hopf_comparison_key [
      network::pure::substitute $NW2 {
         twist {{{{} {} {1 0}} {{} {0 1} {}}} {{0 1 1 0} {0 0 1 1}}}
      }
   ]]
   lset key2 end [list [llength [lindex $NW2 0]]]
   set res [network::pure::grouplex_comparison $key1 $key2]
   if {$res ne "="} then {return $res}
%   \end{tcl}
%   The idea for the orienting |m| associativity is something quite 
%   different. The basic quantity $q$ being compared is the number of 
%   subnetworks on the form `product right branch -- some number of 
%   $/$-part twists -- product', the fewer of which that can be found 
%   the better. There is however the complication that a prolongation 
%   may affect different networks differently, so just counting 
%   occurrences of that subnetwork will not produce a compatible 
%   order. Instead two networks $\mu$ and $\nu$ are considered to 
%   satisfy \(\mu \geqslant \nu\) iff, for all prolongations $v$, it 
%   holds that \(q\bigl( v(\mu) \bigr) \geqslant q\bigl( v(\nu) 
%   \bigr)\). This may seem difficult to test, but is actually 
%   manageable.
%   
%   The information one needs about a network $\mu$ in order to 
%   compute $q\bigl(v(\mu)\bigr)$ for arbitrary prolongations $v$ is 
%   first of all $q(\mu)$ itself, and second information about 
%   whether the part of the network which is adjacent to a leg looks 
%   like the beginning or end of a sequence counted by $q$. There is 
%   also the possibility of such a sequence which passes through the 
%   network, so there is essentially a matrix of possibilities to 
%   count. Since the matrix is sparse, it is implemented as an array 
%   whose entries are strings. A |+| is appended to the entry if 
%   there is a hit in |NW1| and a |-| if there is a hit in |NW2|. 
%   Indices are on the form \meta{i}\texttt{,}\meta{j}, where 
%   \meta{i} and \meta{j} are either output\slash input indices or 
%   the string |internal|.
%   \begin{tcl}
   foreach NW [list $NW1 $NW2] sign {+ -} {
      set L {}
      foreach v [lindex $NW 0] {
         if {[lindex $v 0] eq "m"} then {
            lappend L [lindex $v 2 1] internal
         }
      }
      foreach e [lindex $NW 0 0 2] {lappend L $e [lindex $NW 1 $e 1]}
      foreach {e i} $L {
         while 1 {
            if {[lindex $NW 1 $e 2] < 2} then {
               append A($i,[lindex $NW 1 $e 3]) $sign
            } else {
               set v [lindex $NW 0 [lindex $NW 1 $e 2]]
               switch -- [lindex $v 0] "m" {
                  append A($i,internal) $sign
               } "twist" {
                  if {[lindex $NW 1 $e 3]==0} then {
                     set e [lindex $v 2 1]
                     continue
                  }
               }
            }
            break
         }
      }
   }
%   \end{tcl}
%   Given this matrix, how can one then make the comparison? By 
%   computing bounds for $q\bigl(v(\mu)\bigr) - q\bigl(v(\nu)\bigr)$, 
%   were $\mu$ is |NW1|, $\nu$ is |NW2|, and $v$ ranges over all 
%   prolongations. Because of the way the subnetworks being counted 
%   look, there can be at most one matrix entry for each combination 
%   of leg and network. Furthermore if one network has a ``middle'' 
%   part connecting an input to an output then the other network 
%   cannot have a different kind of part at either leg, because 
%   having a middle part implies that the substituted networks for 
%   which comparison keys were computed above must agree upon having 
%   one path between input and output which furthermore must not have 
%   any branches (or else we wouldn't have gotten this far in this 
%   procedure). Thus it always possible to choose a prolongation $v$ 
%   so that a particular set of nonzero matrix elements get counted 
%   by $q$. This means that the lower bound can be found by counting 
%   the negative non-internal entries and the upper bound can be 
%   found by counting the positive non-internal entries!
%   \begin{tcl}
   set max 0; set min 0
   foreach item [array names A] {
      set sum [expr [string map {+ +1 - -1} $A($item)]]
      if {$item eq "internal,internal"} then {
         incr max $sum;  incr min $sum
      } elseif {$sum>0} then {
         incr max $sum
      } elseif {$sum<0} then {
         incr min $sum
      }
   }
   if {$min<0} then {
      if {$max>0} then {
         return "incomparable m"
      } else {
         return "<"
      }
   } elseif {$max>0} then {
      return ">"
   }
%   \end{tcl}
%   Then the whole thing has to be done again for the coproducts. 
%   Combining the two into a loop is quite possible, but perhaps not 
%   very clear. Note that the input and output indices switch places 
%   in the array indices.
%   \begin{tcl}
   array unset A
   foreach NW [list $NW1 $NW2] sign {+ -} {
      set L {}
      foreach v [lindex $NW 0] {
         if {[lindex $v 0] eq "Delta"} then {
            lappend L [lindex $v 1 1] internal
         }
      }
      foreach e [lindex $NW 0 1 1] {lappend L $e [lindex $NW 1 $e 3]}
      foreach {e i} $L {
         while 1 {
            if {[lindex $NW 1 $e 0] < 2} then {
               append A($i,[lindex $NW 1 $e 1]) $sign
            } else {
               set v [lindex $NW 0 [lindex $NW 1 $e 0]]
               switch -- [lindex $v 0] "Delta" {
                  append A($i,internal) $sign
               } "twist" {
                  if {[lindex $NW 1 $e 1]==0} then {
                     set e [lindex $v 1 1]
                     continue
                  }
               }
            }
            break
         }
      }
   }
   set max 0; set min 0
   foreach item [array names A] {
      set sum [expr [string map {+ +1 - -1} $A($item)]]
      if {$item eq "internal,internal"} then {
         incr max $sum;  incr min $sum
      } elseif {$sum>0} then {
         incr max $sum
      } elseif {$sum<0} then {
         incr min $sum
      }
   }
   if {$min<0} then {
      if {$max>0} then {
         return "incomparable Delta"
      } else {
         return "<"
      }
   } elseif {$max>0} then {
      return ">"
   }
   return "="
}
%</twist>
%   \end{tcl}
% \end{proc}
% 
% 2007-12-19: A reexamination of results 
% (\texttt{Hopf+twist2.dump-tcl}) that were probably computed using 
% the above has shown an impractical instability of this comparison 
% order: the right hand side for one rule becomes the left hand side 
% for another simply by adding an extra coproduct on a leg. This is 
% quite odd since said coproduct can be removed again by putting a 
% counit on either of its outputs. This is not a counterexample to 
% correlation since one gets a strictly $P$-smaller network by removing 
% the coproduct and counit as well, but it seems rather close.
% 
% \begin{tcl}
%</HopfObsolete>
% \end{tcl}
% 
% 
% \section{Hom-algebras}
% 
% A \emph{Hom-associative} algebra has a binary multiplication 
% operation and a unary operation $\alpha$ called ``the homomorphism'' 
% which satisfy the identity \(\alpha(x) \cdot (y \cdot z) = (x \cdot 
% y) \cdot \alpha(z)\) for all $x$, $y$, and $z$. This identity has a 
% tendency to collapse in the presence of a multiplicative identity 
% however, so it seems best to not assume the existance of that. It 
% may on the other hand be convenient to sometimes assume the 
% existence of $\alpha^{-1}$.
% 
% \begin{tcl}
%<*HomAlg>
clear_rules
set vertex_types {
  m {2 1}  alpha {1 1}  alpha^-1 {1 1}
}
% \end{tcl}
% Multiplication is represented as a circle, the homomorphism and its 
% inverse are represented as squares.
% \begin{tcl}
set vertex_appearances {
   m {
      {oval {square 8} {}} 
      {{0 8}}
      {{-5.657 -5.657 -1 -1} {5.657 -5.657 1 -1}}
   }
   alpha {
      {rectangle {square 8} {}} 
      {{0 8}}
      {{0 -8}}
   }
   alpha^-1 {
      {rectangle {square 8} {}
       text {offsets {0 0}} {-text "-"}} 
      {{0 8}}
      {{0 -8}}
   }
}
% \end{tcl}
% There are only four rules in this type of algebra: 
% hom-associativity, two rules for the unit, and the definition of 
% |alpha(1)|.
% \begin{tcl}
%<*unit>
add_rule {} pure [
   network::pure::construct $vertex_types {unit . \r m}
] pure [
   network::pure::construct $vertex_types {.}
]
add_rule {} pure [
   network::pure::construct $vertex_types {. unit \r m}
] pure [
   network::pure::construct $vertex_types {.}
]
%</unit>
add_rule {} pure [
   network::pure::construct $vertex_types {alpha m \r m}
] pure [
   network::pure::construct $vertex_types {m alpha \r m}
]
%<*alghom>
add_rule {} pure [
   network::pure::construct $vertex_types {m \r alpha}
] pure [
   network::pure::construct $vertex_types {alpha alpha \r m}
]
%</alghom>
% \end{tcl}
% 
% Most of the work with running a completion on these rules is 
% about finding a suitable ordering of elements. One that seems 
% appropriate is to order by the following four criteria:
% \begin{enumerate}
%   \item
%     Number of |alpha| vertices.
%   \item
%     Per input, the number of right-turns on the path between there 
%     an the output.
%   \item
%     The number of constant vertices.
%   \item
%     The number of right-turns on paths to constant vertices. This 
%     is compared as a multiset.
% \end{enumerate}
% In each case, the larger the number, the larger the network.
% 
% \textbf{Unfortunately,} the above doesn't work! It considers \(m 
% \circ \mathrm{id} \otimes \alpha(1)\) to be strictly smaller than 
% \(m \circ \alpha(1) \otimes \mathrm{id}\), but they're both 
% subnetworks of \(m \circ \alpha(1) \otimes \alpha(1)\), which is 
% invariant if one is replaced by the other!!
% 
% \begin{proc}{go_down_count_right}
%   This procedure counts the number of times it passes through the 
%   \emph{right} input to a vertex while following a path \emph{down} 
%   to the network output. The call syntax is
%   \begin{quote}
%     |go_down_count_right| \word{network} \word{start}
%   \end{quote}
%   where \word{network} is a pure network and \word{start} is the 
%   label of the edge in which the path starts. The return value is 
%   the number of rights counted.
%   \begin{tcl}
proc go_down_count_right {NW e} {
   set res 0; set res2 0
   while {[set v [lindex $NW 1 $e 0]] != 0} {
      incr res [lindex $NW 1 $e 1]
      if {[lindex $NW 0 $v 0] ne "m"} then {incr res2}
      set e [lindex $NW 0 $v 1 0]
   }
   return [list $res $res2]
}
%   \end{tcl}
% \end{proc}
% 
% \begin{proc}{HomAlg_comparison_key}
%   The |HomAlg_comparison_key| procedure takes a pure network as 
%   argument and returns a corresponding comparison key, as a list of 
%   comparison groups. The call syntax is thus
%   \begin{quote}
%     |HomAlg_comparison_key| \word{network}
%   \end{quote}
%   
%   First a loop over all vertices, which counts |alpha|'s, 
%   constants, and right-turns on paths from constants.
%   \begin{tcl}
proc HomAlg_comparison_key {NW} {
   set no_alpha 0
   set rightL {}
   foreach v [lindex $NW 0] {
      switch -- [lindex $v 0] "alpha" {
         incr no_alpha
      } "alpha(1)" - "unit" {
         lappend rightL [go_down_count_right $NW [lindex $v 1 0]]
      }
   }
%   \end{tcl}
%   Then a loop over inputs, in which the remaining information is 
%   computed.
%   \begin{tcl}
   set inrL {}
   set acL {}
   foreach e [lindex $NW 0 1 1] {
      foreach {r a} [go_down_count_right $NW $e] break
      lappend inrL $r
      lappend acL $a
   }
%   \end{tcl}
%   Finally, the result is constructed.
%   \begin{tcl}
   return [concat [list $no_alpha $inrL [llength $rightL]\
     [lsort -integer -decreasing $rightL]] $acL]
}
%   \end{tcl}
% \end{proc}
% 
% 
% \begin{tcl}
interp alias {} compare_networks {} network::pure::grouplex_compare\
  HomAlg_comparison_key
interp alias {} compare_networks {} network::pure::tree_up_comparison\
  {m alpha} 0 0
% \end{tcl}
% 
% \begin{tcl}
%</HomAlg>
% \end{tcl}
% 
% 
% \section{GWS-example}
% 
% This is really about the traditional diamond lemma; an example for 
% doing completion in class.
% 
% \begin{tcl}
%<*GWS>
clear_rules
set vertex_types {
  a {1 1}  b {1 1}  c {1 1}
}
% \end{tcl}
% All vertex types are squares with a letter inside.
% \begin{tcl}
set vertex_appearances {}
foreach label {a b c} {
   set L [list \
     text {offsets {0 0}} [list -text $label -font "Helvetica 24"]]
   lappend vertex_appearances $label [list $L {{0 16}} {{0 -16}}]
}
% \end{tcl}
% 
% \begin{tcl}
add_rule {} pure [
   network::pure::construct $vertex_types {b \r a}
] pure [
   network::pure::construct $vertex_types {c}
]
add_rule {} pure [
   network::pure::construct $vertex_types {a \r b}
] pure [
   network::pure::construct $vertex_types {c \r c}
]
add_rule {} pure [
   network::pure::construct $vertex_types {a \r c}
] pure [
   network::pure::construct $vertex_types {c \r c \r a}
]
add_rule {} pure [
   network::pure::construct $vertex_types {c \r b}
] pure [
   network::pure::construct $vertex_types {b \r c \r c}
]
proc compare_networks {NW1 NW2} {
   if {[llength [lindex $NW1 0]] > [llength [lindex $NW2 0]]} then {
      return ">"
   } elseif {[llength [lindex $NW1 0]] < [llength [lindex $NW2 0]]}\
   then {
      return "<"
   } else {
      return "="
   }
}
%</GWS>
% \end{tcl}
% 
% 
% \section{Transposition example}
% 
% What happens if one only has a |twist| vertex of arity and coarity 
% $2$, that is its own inverse and satisfies the Yang--Baxter 
% equation? 
% 
% \begin{tcl}
%<*Transposition>
clear_rules
set vertex_types {
  twist {2 2}
}
% \end{tcl}
% Twists are circles with a cross.
% \begin{tcl}
set vertex_appearances {
   twist {
      {oval {square 8} {}  line {square 5.657} {} \
        line {offsets {5.657 -5.657} {-5.657 5.657}} {}} 
      {{-5.657 5.657 -1 1} {5.657 5.657 1 1}}
      {{-5.657 -5.657 -1 -1} {5.657 -5.657 1 -1}}
   }
}
% \end{tcl}
% 
% \begin{tcl}
add_rule {} pure [
   network::pure::construct $vertex_types {twist \r twist}
] pure [
   network::pure::construct $vertex_types {. .}
]
add_rule {} pure [
   network::pure::construct $vertex_types {twist . \n . twist \n twist .}
] pure [
   network::pure::construct $vertex_types {. twist \n twist . \n . twist}
]
proc compare_networks {NW1 NW2} {
   if {[llength [lindex $NW1 0]] > [llength [lindex $NW2 0]]} then {
      return ">"
   } elseif {[llength [lindex $NW1 0]] < [llength [lindex $NW2 0]]}\
   then {
      return "<"
   } else {
      return "="
   }
}
%</Transposition>
% \end{tcl}



% \section{Sleep deprivation}
% 
% When running this application on a larger problem, the screen saver 
% often activates. This stops the completion process, as it steals 
% focus from all its windows. A way around that would be to have a 
% background loop periodically nudging the system so that the timer 
% for activating the screen saver is reset.
% 
% \begin{proc}{nudge_nudge}
%   Nudging the system requires a system call that \Tcl\ doesn't 
%   provide, so we need an extension for this. The |nudge_nudge| 
%   package provides the single command |nudge_nudge| which does this.
%   \begin{tcl}
%<*insomnia>
%<require.nudge>package require nudge_nudge
%   \end{tcl}
%   The Cri\Tcl\ implementation of this package is so short that it 
%   can be given in full here.
%   \begin{tcl}
%<*!require.nudge>
package require critcl 2
package provide nudge_nudge 1
% \smallskip
critcl::ccode {#include <CoreServices/CoreServices.h>}
critcl::framework CoreServices
critcl::cproc nudge_nudge {} void {
    UpdateSystemActivity(OverallAct);
}
critcl::cbuild
%</!require.nudge>
%   \end{tcl}
%   Why not do that in-line, you may ask? Well, Cri\Tcl\ is very 
%   interested in what file its commands are called from, and 
%   docstripping can confuse it. It's safer to let it see a 
%   straightforward \texttt{.tcl} file being sourced.
% \end{proc}
% 
% \begin{proc}{zeitgeber}
%   A \emph{zeitgeber} is, in psycology, an external stimulus 
%   (e.g.~light) that synchronises the biological clock with the 
%   day\slash night cycle. The |zeitgeber| procedure plays a similar 
%   (but more active) role for the screen saver in that it periodically 
%   provides sufficient stimuli that the system will stay awake and 
%   never go to sleep. The call syntax is
%   \begin{quote}
%     |zeitgeber| \word{variable} \word{period}
%   \end{quote}
%   where \word{variable} is the name (in the global context) of the 
%   variable controlling the zeitgeber and \word{period} is the period 
%   (in milliseconds) between stimuli.
%   \begin{tcl}
proc zeitgeber {varname period} {
   upvar #0 $varname control
   if {!$control} then {return}
   nudge_nudge
   after $period [list zeitgeber $varname $period]
}
%   \end{tcl}
% \end{proc}
% 
% \begin{proc}{create_insomnia}
%   This procedure creates a toplevel window with a checkbox to control 
%   |zeitgeber|-induced insomnia. There are no arguments, nor any 
%   particular return value.
%   \begin{tcl}
proc create_insomnia {} {
   toplevel .insomnia
   if {0 && [tk windowingsystem] eq "aqua"} then {
      ::tk::unsupported::MacWindowStyle plain .insomnia closeBox
   }
   wm title .insomnia "Sleep control"
   pack [
      checkbutton .insomnia.cb -variable insomnia_state -command {
         if {$insomnia_state} then {
            zeitgeber insomnia_state 30000
         } else {
            after cancel {zeitgeber insomnia_state 30000}
         }
      } -text "Insomnia"
   ]
   wm resizable .insomnia 0 0
   bind .insomnia.cb <Destroy> {
      after cancel {zeitgeber insomnia_state 30000}
      unset -nocomplain insomnia_state
   }
}
%</insomnia>
%   \end{tcl}
% \end{proc}
% 
% 
% \begin{thebibliography}{8}
% 
% \bibitem{Colley}
%    \textsc{Wesley N. Colley}: 
%    \textit{Colley’s Bias Free College Football Ranking Method: 
%    The Colley Matrix Explained} (2003), 23~pp. 
%    Retrieved 18 November 2006 from \textsc{http:}/\slash
%    \texttt{www.colleyrankings.com}\slash \texttt{matrate.pdf}.
%
% \bibitem{MacLane}
%    \textsc{S. MacLane}:
%    Categorical Algebra,
%    \textit{Bull. Amer. Math. Soc.} \textbf{71} (1965), 40--106.
%    
% \bibitem{skiplist}
%    \textsc{William Pugh}: 
%    Skip lists: a probabilistic alternative to balanced trees,
%    \textit{Communications of the ACM}, June 1990, \textbf{33} (6), 
%    668-676.
% 
% \bibitem{Stefanescu}
%    \textsc{Gheorghe \c{S}tef\u{a}nescu}:
%    \textit{Network Algebra},
%    Springer, 2000; \ISBN~1-85233-195-\ISBNX.
% 
% \end{thebibliography}
% 
% 
\endinput
