\documentclass{article}

\usepackage[swedish]{babel}
\usepackage[T1]{fontenc}
\usepackage{times}

\hyphenation{acyc-li-ci-teten}

\begin{document}

\title{Specifikation f\"or ett examensarbete i datavetenskap}
\author{Lars Hellstr\"om}
\maketitle

\section{Bakgrund}

Bakgrunden till detta arbete \"ar ett p\r{a}g\r{a}ende 
forskningsprojekt i gr\"anslandet mellan matematik och datavetenskap, 
vilket under 2011 bland annat har resulterat i en VR-ans\"okan med 
titeln \emph{Kretsomskrivning} (eng. \emph{Circuit Rewriting}) och 
huvuds\"okande Frank Drewes. I detta projekt representeras till 
exempel algebraiska uttryck eller programfragment som kretsar, vilka 
\"ar riktat acykliska men i \"ovrigt saknar grafstrukturella 
begr\"ansningar; i synnerhet \"ar de i allm\"anhet inte tr\"ad, utan 
tenderar att uppvisa f\"orgreningar, sammanfl\"oden och korsningar 
p\r{a} ett s\"att som g\"or det icke-trivialt att konstruera en 
\"oversk\r{a}dlig presentation. Dessutom producerar 
omskrivningsprocessen stora m\"angder av dessa kretsar, s\r{a} det 
\"ar n\"odv\"andigt att presentationer kan genereras automatiskt.

En mycket naturlig form f\"or presentation \"ar den grafiska, med 
noder i kretsen som grindar i ett kopplingsschema och 
f\"orbindelser dememellan som kanter i en graf. Acycliciteten 
inneb\"ar att grindarna kan placeras in i distinkta lager s\r{a} att 
de mottar indata endast fr\r{a}n tidigare lager och producerar utdata 
endast f\"or senare lager; problemet att d\"arvidlag s\"oka en s\r{a} 
kompakt indelning i lager som m\"ojligt kan formuleras som ett 
linj\"arprogram och denna l\"osning finns implementerad sedan 
n\r{a}gra \r{a}r. D\"aremot finns det inget uppenbart svar p\r{a} hur 
grindarna skall ordnas inom ett lager; man vill helst att logiska 
grannar fr\r{a}n n\"arliggande lager skall vara n\"ara varandra 
\"aven visuellt, men det finns varken n\r{a}gon uppenbar formulering 
av eller l\"osning p\r{a} detta layoutproblem. En heuristisk metod 
\"ar f\"or n\"arvarande i bruk, men dess l\"osningar inneh\r{a}ller 
ibland kanter som synes g\"ora m\"arkliga omv\"agar, s\r{a} 
uppenbarligen finns det utrymme f\"or f\"orb\"attringar.

Vad som g\"or problemet besv\"arligt \"ar att dess dominanta 
frihetsgrader \"ar huvudsakligen diskreta (i vilken ordning placerar 
man elementen inom de olika lagren?) och d\"armed delar upp 
s\"okrummet i ett otal fall (i en krets med $5$ lager om $7$ element i 
varje finns det \(7!^5 \approx 3{,}25 \cdot 10^{18}\) s\"att att 
ordna elementen). \r{A} andra sidan s\r{a} tenderar naturliga 
m\r{a}lfunktioner (till exempel sammanlagd l\"angd av alla kanter) 
snarast att vara kontinuerliga och sv\r{a}ra att uppskatta 
utifr\r{a}n partiella konfigurationer (vilket \"ar vad 
branch-and-bound skulle beh\"ova). Slutligen \"ar det f\"ormodligen 
inte n\"ov\"andigt att alltid hitta den b\"asta l\"osningen, s\r{a} 
l\"ange som den man hittar \"ar bra.

\section{M\r{a}l och metoder}

M\r{a}let med detta arbete \"ar att utv\"ardera en specifik 
l\"osningsheuristik -- som faller inom den klass av metoder som 
relaxerar diskreta variabler till att vara kontinuerliga, l\"oser 
det erh\r{a}llna kontinuerliga optimeringsproblemet, och sedan s\"oker 
en diskret l\"osning n\"ara den kontinuerliga -- f\"or layoutproblemet. 
N\"armare best\"amt skall ordningsf\"oljderna kodas som 
permutationer och permutationerna ses som matriser i en l\"amplig 
matrisgrupp. Det kontinuerliga minimat kan s\"okas med en vanlig 
gradientmetod och en n\"araliggande diskret l\"osning kan sedan 
s\"okas genom att gradvis p\r{a}f\"ora en straffpotential som tvingar 
matriselementen mot $0$ eller $1$.

Arbetets metod \"ar att f\"orst implementera den ovan beskrivna 
heuristiken och sedan pr\"ova densamma. Exempel p\r{a} s\r{a}dana 
prov \"ar:
\begin{enumerate}
  \item
    \"Ar det kontinuerliga optima som metoden hittar unikt, eller 
    bara ett av m\r{a}nga? Detta kan unders\"okas genom att pr\"ova 
    ett stort antal startpunkter f\"or s\"okningen; om alla hittar 
    samma optimum \"ar det en god chans att det \"ar unikt. Om alla 
    optima visar sig ha samma v\"arde p\r{a} m\r{a}lfunktionen \"ar 
    de f\"ormodligen lika goda l\"osningar (detta kan ibland h\"anda 
    till f\"oljd av symmetrier i den underliggande kreten), men 
    skulle v\"ardet variera starkt s\r{a} har metoden ett problem.
  \item
    \"Ar de diskreta l\"osningar som erh\r{a}lls i n\"arheten av 
    motsvarande kontinuerliga, eller kan de hamna l\r{a}ngt 
    d\"arifr\r{a}n? H\"ar kan man bed\"omma b\r{a}de avst\r{a}nd 
    i s\"okrummet och skillnad i m\r{a}lfunktionen. F\"or att 
    heuristiken skall kunna betraktas som lyckad s\r{a} borde det 
    inte vara n\r{a}gra avsev\"arda skillnader mellan kontinuerliga 
    och diskreta l\"osningar p\r{a} optimeringsproblemet.
  \item
    Upplever en m\"anniska att de layouter som metoden producerar 
    \"ar naturliga? H\"ar kan man utg\r{a} ifr\r{a}n en corpus av 
    kretsar och fr\r{a}ga i hur h\"og andel av dessa man skulle vilja 
    \"andra p\r{a} den layout heuristiken har producerat.
  \item
    \"Ar heuristiken tillr\"ackligt snabb f\"or att kunna anv\"andas 
    praktiskt?
\end{enumerate}
F\"or att arbetet skall vara genomf\"ort skall en implementation ha 
tagits fram och de ovanst\r{a}ende fr\r{a}gorna besvarats. En 
brasklapp kan dock vara att om implementationen visar sig vara 
opraktiskt l\r{a}ngsam s\r{a} kan man kanske n\"oja sig med ett 
mindre underlag f\"or unders\"okningarna av l\"osningskvalitet.


\section{Avancerade kurser p\r{a} vilka arbetet baseras}

\begin{description}
  \item[Differentialgeometri D (MATD04)]
    M\r{a}nga variabler i den kontinuerliga relaxionen \"ar 
    specificerade att vara matriser ur en viss grupp. Detta inneb\"ar 
    att de s\"okrum som gradientmetoden r\"or sig i inte \"ar 
    n\r{a}got affint eller vektorrum, utan en m\r{a}ngfald, med allt 
    vad det inneb\"ar av tangentrum som \"andrar sig fr\r{a}n punkt 
    till punkt och Lie-teori.
  \item[Kvantinformation D (5FY052)]
    Det jag tar med mig h\"arifr\r{a}n \"ar v\"al mest en vana vid 
    unit\"ara matriser och de knep man kan utf\"ora med dessa.
\end{description}


\section{Resurser}

\begin{itemize}
  \item
    M\"ojliga handledare har vidtalats \fbox{\strut\kern5cm}
  \item
    Matematikinstitutionens bibliotek \"ar tillg\"angligt f\"or mig.
\end{itemize}


\section{Sekretess}

Ingen del av arbetet kommer att vara hemlig.


\section{Vetenskapligt djup}

F\"or att s\"akerst\"alla vetenskapligt djup skall f\"oljande 
\r{a}tg\"arder vidtagas:
\begin{itemize}
  \item
    Implementationen skall utf\"oras meddelst literat programmering. 
  \item
    N\"ar abstrakta formler skall oms\"attas i kod kommer en 
    h\"arledning att vara inkluderad, alternativt en extern referens 
    ges till l\"amplig k\"alla i den vetenskapliga litteraturen. 
  \item
    Fr\r{a}gor om stabilitet relativt brus i den underliggande 
    numeriken kommer att beaktas.
  \item
    Den utv\"ardering av den implementerade heuristikens resultat som 
    beskrivits ovan har som syfte att pr\"ova dess kvalitet.
\end{itemize}

\end{document}