% 
% \iffalse
%<*driver>
\documentclass{mtmtcl}
\begin{document}
\DocInput{rings.dtx}
\end{document}
%</driver>
% \fi
% 
% 
% \part{Rings and modules}
% 
% \section{Rings}
% 
% The basic rings are put in the |mtmtcl::rings| namespace. The main 
% |mtmtcl| namespace is put in the path of that namespace to 
% facilitate using features from other parts of the system (at the 
% time of writing, particularly |mtmtcl::openmath::gettext|).
% \begin{tcl}
namespace eval mtmtcl::rings {
   namespace path [list [namespace parent]]
}
% \end{tcl}
% \setnamespace{mtmtcl::rings}
% 
% \begin{APIspec}{ring}{2.0}
%   A |ring| is neither required to be unital nor associative, but it 
%   is easy to impose such conditions by claiming support for |monoid| 
%   and |semigroup| respectively.
%   
%   \begin{APIdescription}{ring}
%     \begin{APImethod}{=}
%       \word{element} \word{element}
%     \end{APImethod}
%       The |=| method satisfies the |equality|~v\,1.0 interface.
%       
%     \begin{APImethod}{+}
%       \word{element} \word{element}
%     \end{APImethod}
%     \begin{APImethod}{0}
%     \end{APImethod}
%     \begin{APImethod}{neg}
%       \word{element}
%     \end{APImethod}
%       The |+|, |0|, and |neg| methods are as required by the 
%       \APIref{additive group}{2.0} interface version 2.0, and in 
%       addition the |+| method must be commutative, i.e., for any 
%       two elements $a$ and $b$ of a \meta{ring} $R$, the two 
%       expressions
%       \begin{displaysyntax}
%         [$R$ + $a$ $b$]\par
%         [$R$ + $b$ $a$]
%       \end{displaysyntax}
%       are |=|-equal.
%       
%     \begin{APImethod}{*}
%       \word{element} \word{element}
%     \end{APImethod}
%       Returns the product of the two \word{element}s. The |*| 
%       operation must be distributive over |+|, i.e., for any three 
%       elements $a$, $b$, and $c$ of a \meta{ring} $R$ the two 
%       expressions
%       \begin{displaysyntax}
%         [$R$ * [$R$ + $a$ $b$] $c$]\par
%         [$R$ + [$R$ * $a$ $c$] [$R$ * $b$ $c$]]
%       \end{displaysyntax}
%       are |=|-equal, as are the two expressions
%       \begin{displaysyntax}
%         [$R$ * $a$ [$R$ + $b$ $c$]]\par
%         [$R$ + [$R$ * $a$ $b$] [$R$ * $a$ $c$]]
%       \end{displaysyntax}
%   \end{APIdescription}
%   All methods are congruent.
% \end{APIspec}
% 
% \begin{APIspec}{ring}{1.0}
%   Version~1.0 of |ring| was the same as version~2.0, except that it 
%   built upon \APIref{additive group}{1.0} verion~1.0 rather than 
%   ditto version~2.0.
% \end{APIspec}
% 
% \begin{tclcommand}{ensemble}{integer}
%   A structure for integers can be implemented directly as an 
%   ensemble.
%   \begin{tcl}
namespace eval ::tcl::mathop {
   namespace ensemble create -subcommands {
      API 
      = 
      + neg 0 - iszero integer.
      * 
      1 ^ 
      canonise
      /
      export
      div mod
   } -map {
      API {
         ::APIutil::API {
            equality 1.1
            "additive group" {2.2 1.2}
            ring {2.0 1.0}
            semigroup 1.2  monoid 1.2
            canonise 1.1
            division 1.0
            export 2.0
            import 1.0
         }
      }
      = ==
      neg -
      0 {::return -level 0 0}
      iszero {== 0}
      integer. *
      1 *
      ^ **
      canonise +
%   \end{tcl}
%   I originally tried to use |::tcl::mathfunc::entier| for 
%   canonising integers, but it turned out that \emph{didn't} 
%   canonise them, whereas |::tcl::mathop::+| did.
%   \begin{tcl}
      / ::mtmtcl::rings::integer_divide
      div /
      mod %
      export ::mtmtcl::rings::integer_export
      import ::mtmtcl::rings::integer_import
   } -command ::mtmtcl::rings::integer
}
%   \end{tcl}
%   
%   \begin{proc}{integer_divide}
%     This procedure implements strict division for integers.
%     \begin{tcl}
proc ::mtmtcl::rings::integer_divide {numer denom} {
   if {$numer % $denom == 0} then {
      return [expr {$numer / $denom}]
   } else {
      return -code error -errorcode {API division nosolution}\
        "Does not divide"
   }
}
%     \end{tcl}
%   \end{proc}
%   
%   \begin{proc}{integer_export}
%     This procedure implements the \APIref+{export}{2.0} interface 
%     for integers. The call syntax is
%     \begin{displaysyntax}
%       |::mtmtcl::rings::integer_export| \word{element} 
%       \word{attributes}
%     \end{displaysyntax}
%     
%     \begin{tcl}
proc ::mtmtcl::rings::integer_export {elem attr} {
   list integer [dict create value $elem] {}
}
%     \end{tcl}
%   \end{proc}
%   
%   \begin{proc}{integer_import}
%     This procedure implements the \APIref+{import}{1.0} interface 
%     for integers. The call syntax is
%     \begin{displaysyntax}
%       |::mtmtcl::rings::integer_import| \word{path} \word{data-tree}
%     \end{displaysyntax}
%     It understands both \describestring*+[node type]{integer} and 
%     \describestring*+[node type]{OMI} nodes, but the former is 
%     preferred. It also skips any |OMATTR| nodes that may surround 
%     the element.
%     
%     \begin{tcl}
proc ::mtmtcl::rings::integer_import {path tree} {
   while {[lindex $tree 0] eq "OMATTR"} {
      set tree [lindex $tree 2 1]
   }
   switch -- [lindex $tree 0] "integer" {
      set value [dict get [lindex $tree 1] value]
      if {[string is integer -strict $value]} then {return $value}
   } "OMI" {
      set str [openmath::gettext $tree]
      regsub -all {\s} $str {} str
      if {[regsub {[Xx]} $str {} str] ?\
        [scan $str %llx value] : [scan $str %lld value]} then {
         return $value
      }
   }
   return -code error -errorcode [list API import EDOM $path $tree]\
     "Not an integer"
}
%     \end{tcl}
%   \end{proc}
% \end{tclcommand}
% 
% \begin{proc}{integer_modulo}
%   This procedure implements the algebraic structure of `integers 
%   modulo $n$' (denoted $\mathbb{Z}_n$, $\mathbb{Z}/n$, or 
%   $\mathbb{Z}/n\mathbb{Z}$) for positive integers $n$. The call 
%   syntax is
%   \begin{displaysyntax}
%     |::mtmtcl::rings::integer_modulo| \word{$n$} \word{method} 
%     \word{argument}\regstar
%   \end{displaysyntax}
%   so the actual \meta{ring} is the two word prefix 
%   |::mtmtcl::rings::integer_modulo |\word{$n$}.
%   
%   The canonical representation is as an integer $k$ satisfying \(0 
%   \leqslant k < n\), but all integers are valid representations for 
%   their cosets. It is assumed that $1$ is a canonical 
%   representation for the unit, so it is required that 
%   \(n \geqslant 2\).
%   
%   \begin{tcl}
proc ::mtmtcl::rings::integer_modulo {n method args} {
   switch -- $method {
%   \end{tcl}
%   
%   \begin{procmethod}{=}
%     A variadic |=| seems more bother than it's worth, so the binary 
%     should do.
%     \begin{tcl}
      = {expr {([lindex $args 0] - [lindex $args 1]) % $n == 0}}
%     \end{tcl}
%   \end{procmethod}
%   
%   \begin{procmethod}{+}
%     Variadic addition is however pretty straightforward\Dash just a 
%     |foreach| loop.
%     \begin{tcl}
      + {
         set res 0
         foreach term $args {set res [expr {($res+$term)%$n}]}
         return $res
      }
%     \end{tcl}
%   \end{procmethod}
%   
%   \begin{procmethod}{0}
%     The zero method is trivial.
%     \begin{tcl}
      0 {return 0}
%     \end{tcl}
%   \end{procmethod}
%   
%   \begin{procmethod}{neg}
%     The |neg| method is also pretty easy.
%     \begin{tcl}
      neg {
         expr {(-[lindex $args 0]) % $n}
      }
%     \end{tcl}
%   \end{procmethod}
%   
%   \begin{procmethod}{-}
%     As is the |-| method, since it only has one form.
%     \changes{0.1}{2010/03/25}{Dropping unary form of \texttt{-}, 
%        and thus \texttt{ring} 1.0 support. (LH)}
%     \begin{tcl}
      - {
         expr {([lindex $args 0] - [lindex $args 1]) % $n}
      }
%     \end{tcl}
%   \end{procmethod}
%   
%   \begin{procmethod}{*}
%     Variadic multiplication is just like addition.
%     \begin{tcl}
      * {
         set res 1
         foreach factor $args {set res [expr {($res*$factor) % $n}]}
         return $res
      }
%     \end{tcl}
%   \end{procmethod}
%   
%   That completes the set of methods required by the |ring| 
%   interface.
%   
%   \begin{procmethod}{1}
%     The |1| method has the two forms
%     \begin{displaysyntax}
%       ::mtmtcl::rings::integer 1\par
%       ::mtmtcl::rings::integer 1 \word{integer}
%     \end{displaysyntax}
%     where the first is the unit constant required for a |monoid|. 
%     The second form is the canonical homomorphism of 
%     unital rings from integers into this ring: the \word{integer}th 
%     multiple of the ring unit.
%     \begin{tcl}
      1 {
         if {[llength $args]} then {
            return [expr {[lindex $args 0] % $n}]
         } else {
            return 1
         }
      }
%     \end{tcl}
%      This completes support for the |monoid| interface.
%   \end{procmethod}
%   
%   \begin{procmethod}{/}
%     The |/| method needed to support |division|~(v\,1.0) is the 
%     only one whose algorithm is nontrivial.
%     \begin{tcl}
      / {
%     \end{tcl}
%     Division \texttt{/ $a$ $b$} starts with applying Euclid's 
%     algorithm to the denominator $b$ and the modulo number $n$. 
%     One starts with \(b_0 = b\) and \(c_0 = n\). Then one 
%     recursively sets
%     \begin{align*}
%       q_k :={}& \lfloor c_{k-1}/b_{k-1}\rfloor \\
%       b_k :={}& c_{k-1} - q_k b_{k-1}\\
%       c_k :={}& b_{k-1}
%     \end{align*}
%     until \(b_k=0\), at which point \(c_k = \gcd(b_0,c_0)\). This 
%     is very standard. Slighly less standard is the need to find a 
%     number $r$ such that \(r b \equiv c_k \pmod{n}\). This is done 
%     by maintaining $r_k$ and $s_k$ such that \(r_k b \equiv c_k 
%     \pmod{n}\) and \(s_k b \equiv b_k \pmod{n}\). Clearly \(r_0=0\) 
%     and \(s_0=1\) satisfy these conditions, as do
%     \begin{align*}
%       s_k :={}& r_{k-1} - q_k s_{k-1}\\
%       r_k :={}& s_{k-1}
%     \end{align*}
%     Thus the wanted $r$ is $r_k$ when \(b_k=0\).
%     \begin{tcl}
         set b [lindex $args 1]
         set c $n
         set r 0
         set s 1
         while {$b} {
            set q [expr {$c / $b}]
            set b [expr {$c - $q*[set c $b]}]
            set s [expr {$r - $q*[set r $s]}]
         }
%     \end{tcl}
%     If at this point \(c=c_k \mid a\) then \(x = ra/c = 
%     r_k a / c_k\), otherwise there was no solution.
%     \begin{tcl}
         if {[lindex $args 0] % $c == 0} then {
            return [expr {$r * [lindex $args 0] / $c % $n}]
         } else {
            return -code error -errorcode {API division nosolution}\
              "No such quotient"
         }
      }
%     \end{tcl}
%   \end{procmethod}
%   
%   \begin{procmethod}{integer.}
%     Another way of importing integers is to form integer 
%     multiples of existing elements.
%     \begin{tcl}
      integer. {
         expr {([lindex $args 0] * [lindex $args 1]) % $n}
      }
%     \end{tcl}
%   \end{procmethod}
%   
%   \begin{procmethod}{iszero}
%     The final method needed for v\,1.2 of \APIref{additive 
%     group}{1.2} is the unary test for being zero.
%     \begin{tcl}
      iszero { expr {[lindex $args 0] % $n == 0} }
%     \end{tcl}
%   \end{procmethod}
%   
%   Another useful interface to support is \texttt{generated additive 
%   group}, since that provides a syntactically pure way of 
%   extracting the integer representing a group element.
%   
%   \begin{procmethod}{+ decomposition}
%     Since the additive group $\mathbb{Z}_n$ is cyclic, it can 
%     always be expressed as a multiple of one generator. The easiest 
%     generator is $1$.
%     \begin{tcl}
      "+ decomposition" {
         return [list 1 [expr {[lindex $args 0] % $n}]]
      }
%     \end{tcl}
%   \end{procmethod}
%   
%   \begin{procmethod}{named}
%     To satisfy |named element| interface, it is necessary to 
%     provide translation of |1| (the name) to |1| (the element).
%     \begin{tcl}
      named {
         switch -- [lindex $args 0] "1" {
            return 1
         } default {
            return -code error "\"[lindex $args 0]\" is not an\
              element name"
         }
      }
%     \end{tcl}
%   \end{procmethod}
%   
%   \begin{procmethod}{canonise}
%     The operations always return canonical representations, but 
%     adaptors sometimes find it useful to have an explicit 
%     canonisation method.
%     \begin{tcl}
      canonise { expr {[lindex $args 0] % $n} }
%     \end{tcl}
%   \end{procmethod}
%   
%   \begin{procmethod}{export}
%     This method implements the \APIref+{export}{2.0} interface, 
%     so the call syntax is
%     \begin{displaysyntax}
%       \meta{ring} export \word{element} \word{attributes}
%     \end{displaysyntax}
%     The export format is as an integer $k$ with \(2\lvert k\rvert 
%     \leqslant n\), since this is usually more appropriate for 
%     presentations.
%     
%     \changes{0.1}{2009/11/30}{Added \texttt{export} method. (LH)}
%     \begin{tcl}
      export {
        set elem [expr {[lindex $args 0] % $n}]
        list integer [dict replace [lindex $args 1] value [
           expr {2*$elem > $n ? $elem-$n : $elem}
        ]] {}
        
      }
%     \end{tcl}
%   \end{procmethod}
%   
%   \begin{procmethod}{API}
%     Finally, there is the |API| method, whose implementation is 
%     mostly an enumeration of all the supported interfaces.
%     \begin{tcl}
      API {
         ::APIutil::API {
            equality 1.0
            canonise 1.0
            autocanonical 1.0
            semigroup 1.1
            monoid 1.1
            "additive group" 2.2
            ring 2.0
            division 1.0
            "generated additive group" 2.0
            "named element" 1.0
            export 2.0
         } {*}$args
      }
%     \end{tcl}
%   \end{procmethod}
%   
%   \begin{tcl}
      default {
         return -code error "Unknown method \"$method\";\
           must be =, +, neg, -, *, /, 0, 1, API, integer.,\
           + decomposition, or named"
      }
   }
}
%   \end{tcl}
% \end{proc}
% 
% 
% \begin{ensemble}{float}
%   \setnamespace{mtmtcl::rings}
%   Like integers, a structure for floating-point numbers can be 
%   implemented directly as an ensemble. The format of a |float| 
%   element is that of \Tcl's native |double| numbers.
%   \begin{tcl}
namespace eval ::tcl::mathop {
   namespace ensemble create -subcommands {
      API 
      = 
      + neg - 0 iszero integer.
      * 
      1 ^ 
      canonise
      /
      export import
   } -map {
      API {
         ::APIutil::API {
            equality 1.1
            "additive group" {2.2 1.2}
            ring {2.0 1.0}
            semigroup 1.2  monoid 1.2  group 1.2
%   \end{tcl}
%   Unlike the case with |integer|, it is reasonable for |float| to 
%   claim it is a \APIref{group}{1.2}, even though it doesn't quite 
%   fit the description when $0$ is involved. Supporting |group| is 
%   however no reason not to also support \APIref{division}{1.0}.
%   \begin{tcl}
            division 1.0
            canonise 1.1
            export 2.0
            export 1.0
         }
      }
      =   ==
      0   {::return -level 0 0.0}
      neg -
      iszero {== 0}
      integer. *
      1   {::return -level 0 1.0}
      inverse /
      ^   **
      canonise +
      export ::mtmtcl::rings::float_export
      import ::mtmtcl::rings::float_import
   } -command ::mtmtcl::rings::float
}
%   \end{tcl}
%   The issue of \emph{which} |double| values are legal as elements 
%   of this ring is one that gets surprisingly technical. One might 
%   expect that $1/0$ (technically |[float / 1.0 0.0]|) should not be 
%   defined, and thus require a custom proc to convert \Tcl's native 
%   |ARITH DIVZERO| error to an |API division| error, but 
%   |[::tcl::mathfunc::/ 1.0 0.0]| returns |Inf|, so by including 
%   this (and |-Inf|) in the ``ring'' the issue solves itself. It is 
%   not possible to e.g.~add the two infinities together (an 
%   |ARITH DOMAIN| error will be thrown), but that feels reasonable.
%   
%   Similarly including |NaN| might seem like a good idea, but since 
%   not even |::tcl::mathfunc::double| will take that as input without 
%   throwing an error it seems more practical to leave it out.
%   
%   What turns out to be a thorny issue is however that of what to do 
%   with |-0.0|: this is in IEEE floating-point arithmetic a value 
%   which is distinct from the ordinary zero |0.0| in having inverse 
%   |-Inf| rather than |Inf|. Since |::tcl::mathfunc::==| sees the 
%   two zeroes as equal, one could argue that |-0.0| is merely a 
%   noncanonical representation for |0.0|, and in order to satisfy 
%   \APIref{canonise}{1.1} it becomes necessary to choose a 
%   |canonise| implementation which transform one into the other; the 
%   |+| command happens to fulfill this request. The only operation 
%   which can tell |0.0| and |-0.0| apart is |/|, and since this is 
%   not required to be congruent, we actually manage to get by 
%   without violating that part of any interface! (As is well known, 
%   floating-point arithmetic violates most axioms of a ring in one 
%   way or another, but these violations are generally overlooked as 
%   they are usually small.) An alternative could be to treat |0.0| 
%   and |-0.0| as distinct values, with |tcl::mathfunc::double| as 
%   |canonise|, but that would only satisfy version~1.0 of 
%   \APIref{canonise}{1.0}.
%   
%   
%   \begin{proc}{float_export}
%     This procedure implements the \APIref+{export}{2.0} interface 
%     for |float|s. The call syntax is
%     \begin{displaysyntax}
%       |::mtmtcl::rings::float_export| \word{element} 
%       \word{attributes}
%     \end{displaysyntax}
%     The default implementation is to use \Tcl's string 
%     representation of the canonical value as 
%     \describestring+[attribute]{dec} attribute, since that is much 
%     better than the default for e.g.~|format %g|, but just in case 
%     the syntax doesn't wholly match that of OpenMath, I've also 
%     included the \Module{hexfloat} variant which encodes the exact 
%     IEEE value in the \describestring+[attribute]{hex} attribute
%     \begin{tcl}
proc ::mtmtcl::rings::float_export {elem attr} {
%<*hexfloat>
   binary scan [binary format Q $elem] H* hex
   list OMF [dict create hex $hex] {}
%</hexfloat>
%<!hexfloat>   list OMF [dict create dec [expr {double($elem)}] {}
}
%     \end{tcl}
%   \end{proc}
%   
%   \begin{proc}{float_import}
%     Conversely, this procedure implements the \APIref+{import}{1.0} 
%     interface for |float|s. The call syntax is thus
%     \begin{displaysyntax}
%       |::mtmtcl::rings::float::import| \word{path} \word{data-tree} 
%     \end{displaysyntax}
%     and it accepts not only |OMF| elements, but also |integer| and 
%     |OMI| elements, and skips |OMATTR| wrappers around the number 
%     in question.
%     \begin{tcl}
proc ::mtmtcl::rings::float_import {path tree} {
   while {[lindex $tree 0] eq "OMATTR"} {
      set tree [lindex $tree 2 1]
   }
   switch -- [lindex $tree 0] "OMF" {
%     \end{tcl}
%     See |mtmtcl::present::tpp::OMF| for the concerns encountered 
%     when parsing an |OMF| element. It's possible though that it 
%     should be refactored into a separate procedure somewhere in 
%     the |mtmtcl::openmath| namespace.
%     \begin{tcl}
      if {[dict exists $attr dec]} then {
         scan [dict get $attr dec] %g value
      } else {
         binary scan [binary format H* [dict get $attr hex]] Q value
      }
      return $value
   } "integer" - "OMI" {
      if {![catch {
         integer_import $path $tree
      } value]} then {
         return [expr {double($value)}]
      }
   }
   return -code error -errorcode [list API import EDOM $path $tree]\
     "Not a float"
}
%     \end{tcl}
%   \end{proc}
% \end{ensemble}
% 
% 
% 
% \section{Modules}
% 
% \begin{APIspec}{ring-module}{2.1}
%   The |ring-module| interface is for left modules over a ring 
%   (modules over e.g.~groups are not unheard of).
%   
%   \begin{remark}
%     The reason this interface has version number 2.1 (rather than 
%     2.0) is an expectation that concepts obtained by dropping some 
%     of the axioms might prove useful on their own.
%   \end{remark}
%   
%   \begin{APIdescription}{module}
%     \begin{APImethod}{=}
%       \word{element} \word{element}
%     \end{APImethod}
%       The |=| method satisfies the |equality|~v\,1.0 interface.
%       
%     \begin{APImethod}{+}
%       \word{element} \word{element}
%     \end{APImethod}
%     \begin{APImethod}{0}
%     \end{APImethod}
%     \begin{APImethod}{neg}
%       \word{element}
%     \end{APImethod}
%       The |+|, |0|, and |neg| methods are as required by the 
%       \APIref{additive group}{2.0} interface version 2.0. In 
%       addition the |+| method must be commutative, i.e., for any 
%       two elements $a$ and $b$ of a \meta{ring} $R$, the two 
%       expressions
%       \begin{displaysyntax}
%         [$R$ + $a$ $b$]\par
%         [$R$ + $b$ $a$]
%       \end{displaysyntax}
%       are |=|-equal.
%     
%     \begin{APImethod}{scalar}
%       \word{submethod} \word{argument}\regstar
%     \end{APImethod}
%       The |scalar| method accesses a structure which is both a 
%       |ring|~v\,2.0 and a |monoid|~v\,1.0.
%       
%     \begin{APImethod}{.}
%       \word{scalar} \word{element}
%     \end{APImethod}
%       The |.| method is multiplication (on the left) by a scalar, 
%       i.e.,  the \word{scalar} is an element of the 
%       \meta{module}| scalar| ring and the \word{element} is an 
%       element of the \meta{module}, whereas the result is an 
%       element of the \meta{module}.
%       
%       For every element $a$ of a \meta{module} $M$, the expression
%       \begin{displaysyntax}
%         [$M$ . [$M$ scalar 1] $a$]
%       \end{displaysyntax}
%       is |=|-equal to $a$. Furthermore for all elements $a$ 
%       and $b$ of $M$ and element $r$ of $M$| scalar|, the two 
%       expressions
%       \begin{displaysyntax}
%         [$M$ . $r$ [$M$ + $a$ $b$]]\par
%         [$M$ + [$M$ . $r$ $a$] [$M$ . $r$ $b$]]
%       \end{displaysyntax}
%       are |=|-equal. Finally for every element $a$ of $M$ and 
%       all elements $r$ and $s$ of $M$| scalar|, the two 
%       expressions
%       \begin{displaysyntax}
%         [$M$ . [$M$ scalar + $r$ $s$] $a$]\par
%         [$M$ + [$M$ . $r$ $a$] [$M$ . $s$ $a$]]
%       \end{displaysyntax}
%       are |=|-equal, as are the two expressions
%       \begin{displaysyntax}
%         [$M$ . [$M$ scalar * $r$ $s$] $a$]\par
%         [$M$ . $r$ [$M$ . $s$ $a$]]
%       \end{displaysyntax}
%   \end{APIdescription}
%   All methods are congruent.
% \end{APIspec}
% 
% 
% \begin{APIspec}{free ring-module}{2.1}
%   A free module over a ring is a module which has a basis. 
%   Version~2.1 of this interface satisfies version~2.1 of the 
%   |ring-module| interface.
%   
%   \begin{APIdescription}{module}
%     \begin{APImethod}{basis}
%       \word{submethod} \word{argument}\regstar
%     \end{APImethod}
%       The |basis| method accesses a principal structure which 
%       constitutes the set of (indices of) basis elements. This 
%       structure must satisfy v\,1.0 of the |equality| interface. 
%       \word{index}es below are elements of this structure, and 
%       need not be equal to their counterparts in the module (nor 
%       need they be integers).
%       
%     \begin{APImethod}{basiselement}
%       \word{index}
%     \end{APImethod}
%       Returns the \meta{module} basis element whose index is 
%       \word{index}. Every element of the \meta{module} must be 
%       possible to express in exactly one way as a linear 
%       combination of values of this method.
%       
%     \begin{APImethod}{coeff}
%       \word{index} \word{element}
%     \end{APImethod}
%       Returns the coefficient (an element of 
%       \meta{module}| scalar|) of the \word{index}th basis element 
%       for the \word{element}, i.e., if $M$ is the \meta{module}, then
%       \begin{displaysyntax}
%         $a \mapsto {}$[$M$ coeff $i$ $a$]
%       \end{displaysyntax}
%       is the $R$-module homomorphism for which
%       \begin{displaysyntax}
%         [$M$ coeff $i$ [$M$ basiselement $j$]]
%       \end{displaysyntax}
%       is |=|-equal to |[|$M$| scalar 1]| if $i$ is |basis =|-equal 
%       to $j$, and |=|-equal to |[|$M$| scalar 0]| otherwise.
%       
%     \begin{APImethod}{support}
%       \word{element}
%     \end{APImethod}
%       This method returns the \emph{support} of the \word{element}, 
%       i.e., a list of those basis indices $i$ such that
%       \begin{displaysyntax}
%         [$M$ coeff $i$ \word{element}]
%       \end{displaysyntax}
%       is nonzero.
%   \end{APIdescription}
%   The |basiselement| and |coeff| methods above are congruent.
%   
%   An identity linking the methods defined above is that $a$ is 
%   |=|-equal to 
%   \begin{displaysyntax}
%     $\displaystyle\sum_{i \in \text{[$M$ support $a$]}}$
%     [$M$ . [$M$ coeff $i$ $a$] [$M$ basiselement $i$]]
%   \end{displaysyntax}
%   for every element $a$ of the \meta{module}~$M$. 
%   That there exist methods satisfying this 
%   identity is not itself a sufficient condition for the 
%   \meta{module} to be free (it is in fact a necessary and 
%   sufficient condition for the \meta{module} to be 
%   \emph{projective}, which is slightly weaker), but combining it  
%   with the conditions on |coeff| above produces a sufficient 
%   condition for freeness.
% \end{APIspec}
% 
% 
% \begin{proc}{free_rank_n_module}
%   This procedure implements $R^n$ (the free $R$-module of rank $n$) 
%   for a given associative ring with unit $R$ and natural number 
%   $n$. The call syntax is
%   \begin{displaysyntax}
%     |::mtmtcl::rings::free_rank_n_module| \word{$R$} \word{$n$} 
%     \word{method} \word{argument}\regstar
%   \end{displaysyntax}
%   The element representation is as a length $n$ list of elements of 
%   $R$. Most methods are simply pointwise operations on these lists.
%   
%   \begin{tcl}
proc ::mtmtcl::rings::free_rank_n_module {R n method args} {
   switch -- $method {
%   \end{tcl}
%   
%   \begin{procmethod}{.}
%     Multiplication by scalar is in many ways the characteristic 
%     operation for a module.
%     \begin{tcl}
      . {
         set res [list]
         set r [lindex $args 0]
         foreach a [lindex $args 1] {lappend res [{*}$R * $r $a]}
         return $res
      }
%     \end{tcl}
%   \end{procmethod}
%   
%   \begin{procmethod}{=}
%     Equality is made variadic from the start, since the extra loop 
%     needed for this is a minor overhead.
%     \begin{tcl}
      = {
         if {[llength $args]<2} then {return 1}
         set leftL [lindex $args 0]
         foreach rightL [lrange $args 1 end] {
            foreach l $leftL r $rightL {
               if {![{*}$R = $l $r]} then {return 0}
            }
         }
         return 1
      }
%     \end{tcl}
%   \end{procmethod}
%   
%   \begin{procmethod}{iszero}
%     Testing for being zero is not as important in a module as it is 
%     in a ring, but it is needed to support 
%     \APIref+{additive group}{1.1}.
%     \begin{tcl}
      iszero {
         if {[{*}$R API "additive group" 1.1]} then {
            foreach a [lindex $args 0] {
               if {![{*}$R iszero $a]} then {return 0}
            }
         } else {
            set zero [{*}$R 0]
            foreach a [lindex $args 0] {
               if {![{*}$R = $zero $a]} then {return 0}
            }
         }
         return 1
      }
%     \end{tcl}
%   \end{procmethod}
%   
%   \begin{procmethod}{+}
%     Variadic addition is similar, although this simple 
%     implementation could be considered wasteful if ring addition is 
%     expensive.
%     \begin{tcl}
      + {
         set res [lrepeat $n [{*}$R 0]]
         foreach term $args {
            set res2 [list]
            foreach l $res r $term {
               lappend res2 [{*}$R + $l $r]
            }
            set res $res2
         }
         return $res
      }
%     \end{tcl}
%   \end{procmethod}
%   
%   \begin{procmethod}{0}
%     The implementation of |0| is implicit in the above.
%     \begin{tcl}
      0 {return [lrepeat $n [{*}$R 0]]}
%     \end{tcl}
%   \end{procmethod}
%   
%   \begin{procmethod}{-}
%     Subtraction is straightforward.
%     \begin{tcl}
      - {
         set res [list]
         foreach t [lindex $args 0] u [lindex $args 1] {
            lappend res [{*}$R + $t [{*}$R neg $u]]
         }
%     \end{tcl}
%     It might be faster to use binary subtraction in the ring 
%     instead, but checking whether that is available isn't free 
%     either.
%     \begin{tcl}
         return $res
      }
%     \end{tcl}
%   \end{procmethod}
%   
%   \begin{procmethod}{neg}
%     Negation is even easier.
%     \begin{tcl}
      neg {
         set res [list]
         foreach t [lindex $args 0] {
            lappend res [{*}$R neg $t]
         }
         return $res
      }
%     \end{tcl}
%   \end{procmethod}
%   
%   \begin{procmethod}{scalar}
%     The only nontrivial thing about the |scalar| method is that the 
%     call needs to be |uplevel|ed, in case the structure has some 
%     method that does an |upvar| or similar.
%     \begin{tcl}
      scalar {uplevel 1 $R $args}
%     \end{tcl}
%   \end{procmethod}
%   
%   This completes the support for the |ring-module| v\,2.1 
%   interface. The next part is to support \tclverb*|free ring-module|.
%   
%   \begin{procmethod}{basis}
%     The index set for the basis is taken to be $\{0,\dotsc,n 
%     -\nobreak 1\}$.
%     \begin{tcl}
      basis {::mtmtcl::sets::finite_ordinal $n {*}$args}
%     \end{tcl}
%   \end{procmethod}
%   
%   \begin{procmethod}{basiselement}
%     The built-in basis is the standard basis where the $i$th basis 
%     element has the $i$th coordinate equal to $1$ and all others 
%     equal to $0$.
%     \begin{tcl}
      basiselement {
         set res [lrepeat $n [{*}$R 0]]
         lset res [lindex $args 0] [{*}$R 1]
         return $res
      }
%     \end{tcl}
%   \end{procmethod}
%   
%   \begin{procmethod}{support}
%     Using the standard basis makes it easy to determine the support 
%     for an element (but this could probably be done more 
%     efficiently if the ring had an |iszero| method).
%     \begin{tcl}
      support {
         set zero [{*}$R 0]
         set res [list]
         set i -1; foreach part [lindex $args 0] {incr i
            if {[{*}$R = $zero $part]} then {lappend res $i}
         }
         return $res
      }
%     \end{tcl}
%   \end{procmethod}
%   
%   \begin{procmethod}{coord}
%   \begin{procmethod}{index}
%     The |coord| mathod of \tclverb*|free ring-module| and the 
%     |index| method of \tclverb*|direct product| turn out to be the 
%     same in this case.
%     \begin{tcl}
      coord - index {
         return [lindex [lindex $args 1] [lindex $args 0]]
      }
%     \end{tcl}
%     Thus endeth one interface, and beginneth yet another.
%   \end{procmethod}\end{procmethod}
%   
%   \begin{procmethod}{tuple}
%     Building a tuple is straightforward as long as one doesn't 
%     bother to enforce canonicity.
%     \begin{tcl}
      tuple {
         if {[llength $args] != $n} then {
            return -code error "Wrong \# arguments: must be $n\
              elements in tuple"
         } else {
            return $args
         }
      }
%     \end{tcl}
%   \end{procmethod}
%   
%   \begin{procmethod}{no.components}
%     The number of components in the direct product is $n$.
%     \begin{tcl}
      no.components {return $n}
%     \end{tcl}
%   \end{procmethod}
%   
%   \begin{procmethod}{component}
%     Access to components is almost the same as |scalar|, but it has 
%     an extra argument that needs to be ignored.
%     \begin{tcl}
      component {
         uplevel 1 $R [lrange $args 1 end]
      }
%     \end{tcl}
%   \end{procmethod}
%   
%   \begin{procmethod}{canonise}
%     Lacking automatic canonisation, there should at least be a 
%     manual canonisation operation.
%     \begin{tcl}
      canonise {
         set res [list]
         foreach part [lindex $args 0] {
            lappend res [{*}$R canonise $part]
         }
         return $res
      }
%     \end{tcl}
%     \textbf{ToDo:} declare support of this when available.
%   \end{procmethod}
%   
%   \begin{procmethod}{API}
%     The support for interfaces is as usual an all-or-nothing 
%     affair: everything follows if only the base structure is an 
%     associative unital ring, otherwise we promise nothing.
%     \begin{tcl}
      API {
         if {[{*}$R API ring 1.0] && [{*}$R API monoid 1.0]} then {
            ::APIutil::API {
               equality 1.1
               ring-module 2.1
               "additive group" 2.1
               "free ring-module" 1.1
               "direct product" 1.0
            } {*}$args
         } else {
            ::APIutil::API {} {*}$args
         }
      }
%     \end{tcl}
%   \end{procmethod}
%   
%   \begin{tcl}
      default {
         return -code error "Unsupported method \"$method\""
      }
   }
}
%   \end{tcl}
% \end{proc}
% 
% 
% \begin{APIspec}{ring-module}{1.1}
%   Version~1.1 of the \APIref{ring-module}{2.1} interface is the 
%   same as version~2.1, except that it supports version~1.0 of 
%   \APIref{additive group}{1.0} rather than version~2.0, and 
%   similarly the |scalar|s support version~1.0 of \APIref{ring}{1.0} 
%   rather than version~2.0, i.e., the various |neg| methods are 
%   called |-| instead.
% \end{APIspec}
% 
% \iffalse
% \begin{APIspec}{ring-module}{1.1}
%   The |ring-module| interface is for left modules over a ring 
%   (modules over e.g.~groups are not unheard of).
%   
%   \begin{remark}
%     The reason this interface has version number 1.1 (rather than 
%     1.0) is that concepts obtained by dropping some of the 
%     axioms might prove useful on their own.
%   \end{remark}
%   
%   \begin{APIdescription}{module}
%     \begin{APImethod}{=}
%       \word{element} \word{element}
%     \end{APImethod}
%       The |=| method satisfies the |equality|~v\,1.0 interface.
%       
%     \begin{APImethod}{+}
%       \word{element} \word{element}
%     \end{APImethod}
%     \begin{APImethod}{0}
%     \end{APImethod}
%     \begin{APImethod}{-}
%       \word{element}
%     \end{APImethod}
%       The |+|, |0|, and |-| methods are as required by the 
%       \APIref{additive group}{1.0} interface version 1.0. In 
%       addition the |+| method must be commutative, i.e., for any 
%       two elements $a$ and $b$ of a \meta{ring} $R$, the two 
%       expressions
%       \begin{displaysyntax}
%         [$R$ + $a$ $b$]\par
%         [$R$ + $b$ $a$]
%       \end{displaysyntax}
%       are |=|-equal.
%     
%     \begin{APImethod}{scalar}
%       \word{submethod} \word{argument}\regstar
%     \end{APImethod}
%       The |scalar| method accesses a structure which is both a 
%       |ring|~v\,1.0 and a |monoid|~v\,1.0.
%       
%     \begin{APImethod}{.}
%       \word{scalar} \word{element}
%     \end{APImethod}
%       The |.| method is multiplication (on the left) by a scalar, 
%       i.e.,  the \word{scalar} is an element of the 
%       \meta{module}| scalar| ring and the \word{element} is an 
%       element of the \meta{module}, whereas the result is an 
%       element of the \meta{module}.
%       
%       For every element $a$ of a \meta{module} $M$, the expression
%       \begin{displaysyntax}
%         [$M$ . [$M$ scalar 1] $a$]
%       \end{displaysyntax}
%       is |=|-equal to $a$. Furthermore for all elements $a$ 
%       and $b$ of $M$ and element $r$ of $M$| scalar|, the two 
%       expressions
%       \begin{displaysyntax}
%         [$M$ . $r$ [$M$ + $a$ $b$]]\par
%         [$M$ + [$M$ . $r$ $a$] [$M$ . $r$ $b$]]
%       \end{displaysyntax}
%       are |=|-equal. Finally for every element $a$ of $M$ and 
%       all elements $r$ and $s$ of $M$| scalar|, the two 
%       expressions
%       \begin{displaysyntax}
%         [$M$ . [$M$ scalar + $r$ $s$] $a$]\par
%         [$M$ + [$M$ . $r$ $a$] [$M$ . $s$ $a$]]
%       \end{displaysyntax}
%       are |=|-equal, as are the two expressions
%       \begin{displaysyntax}
%         [$M$ . [$M$ scalar * $r$ $s$] $a$]\par
%         [$M$ . $r$ [$M$ . $s$ $a$]]
%       \end{displaysyntax}
%   \end{APIdescription}
%   All methods are congruent.
% \end{APIspec}
% \fi
% 
% \begin{APIspec}{free ring-module}{1.1}
%   Version~1.1 of \APIref{free ring-module}{2.1} is the same as 
%   version~2.1 except that it implies version~1.1 of 
%   \APIref{ring-module}{1.1} rather than version~2.1, i.e., the 
%   |neg| method is called |-| instead.
% \end{APIspec}
% 
% \iffalse
% \begin{APIspec}{free ring-module}{1.1}
%   A free module over a ring is a module which has a basis. 
%   Version~1.1 of this interface satisfies version~1.1 of the 
%   |ring-module| interface.
%   
%   \begin{APIdescription}{module}
%     \begin{APImethod}{basis}
%       \word{submethod} \word{argument}\regstar
%     \end{APImethod}
%       The |basis| method accesses a principal structure which 
%       constitutes the set of (indices of) basis elements. This 
%       structure must satisfy v\,1.0 of the |equality| interface. 
%       \word{index}es below are elements of this structure, and 
%       need not be equal to their counterparts in the module.
%       
%     \begin{APImethod}{basiselement}
%       \word{index}
%     \end{APImethod}
%       Returns the \meta{module} basis element whose index is 
%       \word{index}. Every element of the \meta{module} must be 
%       possible to express in exactly one way as a linear 
%       combination of values of this method.
%       
%     \begin{APImethod}{coeff}
%       \word{index} \word{element}
%     \end{APImethod}
%       Returns the coefficient (an element of 
%       \meta{module}| scalar|) of the \word{index}th basis element 
%       for the \word{element}, i.e., if $M$ is the \meta{module}, then
%       \begin{displaysyntax}
%         $a \mapsto {}$[$M$ coeff $i$ $a$]
%       \end{displaysyntax}
%       is the $R$-module homomorphism for which
%       \begin{displaysyntax}
%         [$M$ coeff $i$ [$M$ basiselement $j$]]
%       \end{displaysyntax}
%       is |=|-equal to |[|$M$| scalar 1]| if $i$ is |basis =|-equal 
%       to $j$, and |=|-equal to |[|$M$| scalar 0]| otherwise.
%       
%     \begin{APImethod}{support}
%       \word{element}
%     \end{APImethod}
%       This method returns the \emph{support} of the \word{element}, 
%       i.e., a list of those basis indices $i$ such that
%       \begin{displaysyntax}
%         [$M$ coeff $i$ \word{element}]
%       \end{displaysyntax}
%       is nonzero.
%   \end{APIdescription}
%   The |basiselement| and |coeff| methods above are congruent.
%   
%   An identity linking the methods defined above is that $a$ is 
%   |=|-equal to 
%   \begin{displaysyntax}
%     $\displaystyle\sum_{i \in \text{[$M$ support $a$]}}$
%     [$M$ . [$M$ coeff $i$ $a$] [$M$ basiselement $i$]]
%   \end{displaysyntax}
%   for every element $a$ of the \meta{module}~$M$. 
%   That there exist methods satisfying this 
%   identity is not itself a sufficient condition for the 
%   \meta{module} to be free (it is in fact a necessary and 
%   sufficient condition for the \meta{module} to be 
%   \emph{projective}, which is slightly weaker), but combining it  
%   with the conditions on |coeff| above produces a sufficient 
%   condition for freeness.
% \end{APIspec}
% \fi
% 
% 
% \section{Algebras}
% 
% \begin{APIspec}{ring-algebra}{2.0}
%   An algebra over a ring is a module with a bilinear multiplication 
%   operation.
%   
%   \begin{APIdescription}{algebra}
%     \begin{APImethod}{=}
%       \word{element} \word{element}
%     \end{APImethod}
%       The |=| method satisfies the |equality|~v\,1.0 interface.
%       
%     \begin{APImethod}{+}
%       \word{element} \word{element}
%     \end{APImethod}
%     \begin{APImethod}{0}
%     \end{APImethod}
%     \begin{APImethod}{neg}
%       \word{element}
%     \end{APImethod}
%     \begin{APImethod}{scalar}
%       \word{submethod} \word{argument}\regstar
%     \end{APImethod}
%     \begin{APImethod}{.}
%       \word{scalar} \word{element}
%     \end{APImethod}
%       These methods satisfy the \APIref+{ring-module}{2.1} interface.
%       
%     \begin{APImethod}{*}
%       \word{element} \word{element}
%     \end{APImethod}
%       The |*| method satisfies the \APIref+{ring}{2.0} interface. In 
%       addition, the three expressions
%       \begin{displaysyntax}
%         [$R$ * [$R$ scalar . $r$ $a$] $b$]\par
%         [$R$ * $a$ [$R$ scalar . $r$ $b$]]\par
%         [$R$ scalar . $r$ [$R$ * $a$ $b$]]
%       \end{displaysyntax}
%       are |=|-equal for all elements $a$ and $b$ of the 
%       \meta{algebra} $R$ and every element $r$ of $R$| scalar|.
%   \end{APIdescription}
% \end{APIspec}
% 
% 
% \begin{proc}{semigroup_algebra}
%   This procedure implements a semigroup algebra structure, with 
%   given ring of coefficients. The basic call syntax is
%   \begin{displaysyntax}
%     |mtmtcl::rings::semigroup_algebra| \word{coefficient ring} 
%     \word{semigroup} \word{method} \word{argument}\regstar
%   \end{displaysyntax}
%   whereas the return value depends on the \word{method}. The 
%   \meta{coefficient ring} must support \APIref+{additive 
%   group}{2.1}, since distinguishing zero coefficients is a very 
%   common operation.
%   
%   The representation format is a dictionary where the keys are 
%   semigroup elements and the values are coefficients. Values that 
%   are zero should be dropped. The keys must be on canonical format.
%   \begin{tcl}
proc mtmtcl::rings::semigroup_algebra {R S method args} {
   switch -- $method {
%   \end{tcl}
%   
%   \begin{procmethod}{=}
%     Equality is tested by comparing entry by entry. It is forgiving 
%     for entries with an explicit zero coefficient, since the 
%     overhead of this for canonical representations is at most one 
%     ring comparison.
%     \begin{tcl}
      = {
         lassign $args lD rD
         dict for {k v} $lD {
            if {![dict exists $rD $k]} then {
               if {![{*}$R iszero $v]} then {return 0}
            } elseif {[{*}$R = $v [dict get $rD $k]]} then {
               dict unset rD $k
            } else {
               return 0
            }
         }
         dict for {k v} $rD {
            if {![{*}$R iszero $v]} then {return 0}
         }
         return 1
      }
%     \end{tcl}
%   \end{procmethod}
%   
%   \begin{procmethod}{+}
%     Addition is variadic.
%     \changes{0.1}{2010/03/30}{Remove zero elements at the end, 
%       rather than checking after each coefficient addition. (LH)}
%     \begin{tcl}
      + {
         if {![llength $args]} then {return {}}
         foreach term [lassign $args sum] {
            dict for {k v} $term {
               if {[dict exists $sum $k]} then {
                  dict set sum $k [{*}$R + [dict get $sum $k] $v]
               } else {
                  dict set sum $k $v
               }
            }
         }
         foreach k [dict keys $sum] {
            if {[{*}$R iszero [dict get $sum $k]]} then {
               dict unset sum $k
            }
         }
         return $sum
      }
%     \end{tcl}
%   \end{procmethod}
%   
%   \begin{procmethod}{neg}
%     Negation is easy, since one only has to switch the sign of 
%     every coefficient.
%     \begin{tcl}
      neg {
         set res [dict create]
         dict for {k v} [lindex $args 0] {
            dict set res $k [{*}$R neg $v]
         }
         return $res
      }
%     \end{tcl}
%   \end{procmethod}
%   
%   \begin{procmethod}{-}
%     Subtraction is more like addition, but here it makes sense to 
%     immediately check coefficients for being zero, since the 
%     operation is strictly binary.
%     \begin{tcl}
      - {
         lassign $args res term
         dict for {k v} $term {
            if {[dict exists $res $k]} then {
               dict update res $k s {
                  set s [{*}$R - $s $v]
                  if {[{*}$R iszero $s]} then {unset s}
               }
            } else {
               dict set res $k [{*}$R neg $v]
            }
         }
         return $res
      }
%     \end{tcl}
%   \end{procmethod}
%   
%   \begin{procmethod}{0}
%     The zero value is an empty dictionary.
%     \begin{tcl}
      0 {return {}}
%     \end{tcl}
%   \end{procmethod}
%   
%   \begin{procmethod}{.}
%     Multiplication by scalar is straightforward.
%     \begin{tcl}
      . {
         lassign $args r a
         set res [dict create]
         dict for {k v} $a {
            set rv [{*}$R * $r $v]
            if {![{*}$R iszero $rv]} then {dict set res $k $rv}
         }
         return $res
      }
%     \end{tcl}
%   \end{procmethod}
%   
%   \begin{procmethod}{*}
%     The algebra multiplication operation is conceptually done in 
%     the trivial way, by expanding both factors:
%     \begin{equation*}
%       \biggl( \sum_{\mu \in S_1} a_\mu \mu \biggr)
%       \biggl( \sum_{\nu \in S_2} b_\nu \nu \biggr) =
%       \sum_{\mu \in S_1} \sum_{\nu \in S_2} (a_\mu b_\nu) (\mu\nu)
%       \text{.}
%     \end{equation*}
%     Since this gives rise to one nested loop per factor, it would 
%     be somewhat awkward to make variadic.
%     \begin{tcl}
      * {
         set res [dict create]
         dict for {lk lv} [lindex $args 0] {
            dict for {rk rv} [lindex $args 1] {
               set lrv [{*}$R * $lv $rv]
               set lrk [{*}$S canonise [{*}$S * $lk $rk]]
               if {[dict exists $res $lrk]} then {
                  dict set res $lrk [{*}$R + [dict get $res $lrk] $lrv]
               } elseif {![{*}$R iszero $lrv]} then {
                  dict set res $lrk $lrv
               }
               
            }
         }
         foreach k [dict keys $res] {
            if {[{*}$R iszero [dict get $res $k]]} then {
               dict unset res $k
            }
         }
         return $res
      }
%     \end{tcl}
%   \end{procmethod}
%   
%   \begin{procmethod}{1}
%     The semigroup algebra has a unit iff the semigroup is a monoid. 
%     When this unit exists, it consists of the coefficient unit 
%     times the semigroup unit.
%     \begin{tcl}
      1 {
         if {[{*}$S API monoid 1.0]} then {
            return [
               dict create [{*}$S canonise [{*}$S 1]] [{*}$R 1]
            ]
         } else {
            return -code error "This is not a unital algebra"
         }
      }
%     \end{tcl}
%   \end{procmethod}
%   
%   \begin{procmethod}{scalar}
%     The |scalar| method gives access to the coefficient ring.
%     \begin{tcl}
      scalar {uplevel 1 $R $args}
%     \end{tcl}
%     This completes support for \APIref+{ring-module}{1.1}.
%   \end{procmethod}
%   
%   \begin{procmethod}{basis}
%     On to \APIref{free ring-module}{2.1}, where the |basis| method 
%     naturally provides access to the underlying semigroup.
%     \begin{tcl}
      basis {uplevel 1 $S $args}
%     \end{tcl}
%   \end{procmethod}
%   
%   \begin{procmethod}{support}
%     The support for an element is a native |dict| subcommand.
%     \begin{tcl}
      support {dict keys [lindex $args 0]}
%     \end{tcl}
%   \end{procmethod}
%   
%   \begin{procmethod}{coeff}
%     The same is almost true for getting coefficients, but zero 
%     coefficients need to be provided explicitly.
%     \begin{tcl}
      coeff {
         set k [{*}$S canonise [lindex $args 0]]
         if {[dict exists [lindex $args 1] $k]} then {
            return [dict get [lindex $args 1] $k]
         } else {
            return [{*}$R 0]
         }
      }
%     \end{tcl}
%   \end{procmethod}
%   
%   \begin{procmethod}{basiselement}
%     Converting a semigroup element to the corresponding algebra 
%     element is a matter of equipping it with a unit coefficient.
%     \begin{tcl}
      basiselement {
         dict create [{*}$S canonise [lindex $args 0]] [{*}$R 1]
      }
%     \end{tcl}
%   \end{procmethod}
%   
%   \begin{procmethod}{iszero}
%     The thing missing to satisfy \APIref+{additive group}{2.1} is 
%     the |iszero| method. This is made tolerant for zero 
%     coefficients.
%     \begin{tcl}
      iszero {
         dict for {k v} [lindex $args 0] {
            if {![{*}$R iszero $v]} then {return 0}
         }
         return 1
      }
%     \end{tcl}
%   \end{procmethod}
%   
%   \begin{procmethod}{named}
%     The |named| method is a shorthand for applying |basiselement| 
%     to the result of the |named| method of the semigroup.
%     \begin{tcl}
      named {
         dict create [{*}$S canonise [{*}$S named [lindex $args 0]]]\
           [{*}$R 1]
      }
%     \end{tcl}
%   \end{procmethod}
%   
%   \begin{procmethod}{export}
%     The part of the internal representation that this structure 
%     needs to convert for export is the formal linear combination, 
%     since that is what is represented. The sum is addition 
%     throughout, even if the coefficients are negative, since a 
%     general coefficient ring doesn't have a concept of negative; 
%     if the minuses are to be extracted, then that should probably 
%     be done at a later stage. For now the multiplication by scalar 
%     is represented by the \texttt{arith1} \texttt{times} symbol, 
%     but this isn't quite right as it isn't really an associative 
%     operation.
%     \begin{tcl}
      export {
         lassign $args elemD attr
         set battr $attr
         set cattr $attr
         set mattr [dict replace $attr  cd arith1  name times]
         set aattr [dict replace $attr  cd arith1  name plus]
         dict lappend battr mtmtcl:path basis
         dict lappend cattr mtmtcl:path scalar
         dict lappend mattr mtmtcl:path .
         dict lappend aattr mtmtcl:path +
         set mnode        [list OMS $mattr {}]
         set childL [list [list OMS $aattr {}]]
         dict for {k v} $elemD {
            lappend childL [
               list OMA $attr [list $mnode\
                 [{*}$R export $v $cattr] [{*}$S export $k $battr]]
            ]
         }
         return [list OMA $attr $childL]
      }
%     \end{tcl}
%     
%     It would probably be a good idea to present the terms in some 
%     predictable order rather than the hash table order, but again 
%     it is hard to express this generally. It might be necessary to 
%     introduce a \tclverb*|totally ordered semigroup| interface to 
%     get that sorted out (although technically it suffices with a 
%     total order).
%   \end{procmethod}
%   
%   \begin{procmethod}{/}
%     Having a division-by-monomial operation is convenient for 
%     factoring.
%     \begin{tcl}
      / {
         lassign $args numer denom
         if {[dict size $denom] != 1} then {
            return -code error -errorcode {API division unimplemented}\
              "Sorry, I only know how to divide by monomials"
         }
%     \end{tcl}
%     This relies heavily on having division in the underlying ring 
%     and semigroup, and will in particular error out if any 
%     coefficient or monomial errors out. Each term in the numerator 
%     gives rise to a separate term in the result, since only 
%     $k = {}$\texttt{[$S$ * $\mathit{sd}$ $x$]} can give rise to a 
%     result term with key $x$.
%     \begin{tcl}
         lassign [dict get $denom] sd rd
         set res {}
         dict for {k v} $numer {
            dict set res [{*}$S canonise [{*}$S / $k $sd]]\
              [{*}$R / $v $rd]
         }
         return $res
      }
%     \end{tcl}
%   \end{procmethod}
%   
%   \begin{procmethod}{API}
%     \changes{0.1}{2010/01/28}{Formalised usability as a free 
%        module even when not over a semigroup. (LH)}
%     \begin{tcl}
      API {
         if {![{*}$R API ring 2.0] || ![{*}$R API monoid 1.0] ||\
           ![{*}$R API "additive group" 2.1]} then {
            return [::APIutil::API {} {*}$args]
         }
         set D {
            "additive group" 2.1
            ring-module 2.1
            "free ring-module" 2.1
            equality 1.0
         }
         if {[{*}$S API semigroup 1.0]} then {
            set D [dict merge $D {
               ring-algebra 2.0
               ring 2.0
               semigroup 1.0
            }]
            if {[{*}$S API monoid 1.0]} then {dict lappend D monoid 1.0}
         }
         if {[{*}$R API export 2.0] && [{*}$S API export 2.0]} then {
            dict lappend D export 2.0
         }
         if {[{*}$R API division 1.0] && [{*}$S API division 1.0]} then {
            dict lappend D division 1.0
         }
         return [::APIutil::API $D {*}$args]
      }
%     \end{tcl}
%   \end{procmethod}
%   
%   \begin{tcl}
      default {
         return -code error "Unsupported method \"$method\""
      }
%   \end{tcl}
%   \begin{tcl}
   }
}
%   \end{tcl}
% \end{proc}
% 
% \begin{APIspec}{ring-algebra}{1.0}
%   Version~1.0 of \APIref{ring-algebra}{2.0} is the same as 
%   version~2.0, except that it supports version~1.1 of 
%   \APIref{ring-module}{1.1} rather than version~2.1 and version~1.0 
%   of \APIref{ring}{1.0} rather than version~2.0, i.e., the |neg| 
%   method is called |-| instead.
% \end{APIspec}
% 
% \iffalse
% \begin{APIspec}{ring-algebra}{1.0}
%   An algebra over a ring is a module with a bilinear multiplication 
%   operation.
%   
%   \begin{APIdescription}{algebra}
%     \begin{APImethod}{=}
%       \word{element} \word{element}
%     \end{APImethod}
%       The |=| method satisfies the |equality|~v\,1.0 interface.
%       
%     \begin{APImethod}{+}
%       \word{element} \word{element}
%     \end{APImethod}
%     \begin{APImethod}{0}
%     \end{APImethod}
%     \begin{APImethod}{-}
%       \word{element}
%     \end{APImethod}
%     \begin{APImethod}{scalar}
%       \word{submethod} \word{argument}\regstar
%     \end{APImethod}
%     \begin{APImethod}{.}
%       \word{scalar} \word{element}
%     \end{APImethod}
%       These methods satisfy the |ring-module| (v\,1.1) interface.
%       
%     \begin{APImethod}{*}
%       \word{element} \word{element}
%     \end{APImethod}
%       The |*| method satisfies the |ring| (v\,1.0) interface. In 
%       addition, the three expressions
%       \begin{displaysyntax}
%         [$R$ * [$R$ scalar . $r$ $a$] $b$]\par
%         [$R$ * $a$ [$R$ scalar . $r$ $b$]]\par
%         [$R$ scalar . $r$ [$R$ * $a$ $b$]]
%       \end{displaysyntax}
%       are |=|-equal for all elements $a$ and $b$ of the 
%       \meta{algebra} $R$ and every element $r$ of $R$| scalar|.
%   \end{APIdescription}
% \end{APIspec}
% \fi
%
\endinput
